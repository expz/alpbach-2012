\chapter{Motivic structure: Tannakian categories by Konrad V\"olkel}
%\addcontentsline{toc}{chapter}{**Motivic structure on the fundamental group: Tannakian categories by Konrad V\"olker}

Konrad V\"olkel on September 6th, 2012.

% Was \semidirect supposed to be \rtimes or \ltimes? I tried \rtimes.
\medskip
\medskip

\indent After reminding you of the basic properties of Tannakian categories, along some examples, we will work out a strategy to prove upper bounds on periods of mixed Tate motives, which will be used by the following talk, and which will finally lead to the theorem of Goncharov-Terasoma. The strategy consists of defining a weigth filtration on the real periods, obtained as fixed points of all periods under complex conjugation, and then to show that it's a quotient of a certain explicit graded $\Q$-subalgebra $\mathcal{O}(I(\omega,\eta))^\epsilon_+ \subseteq \mathcal{O}(I(\omega,\eta))$, whose Poincar\'e series can be computed. In the proof, we exhibit the structure of graded k-algebra
\[\mathcal{O}(I(\omega,\eta))^\epsilon_+ \simeq \mathcal{O}(\Gm \rtimes U) = k[t,t^{-1}] \otimes_k T\left(\bigoplus_{n>0}Ext^1_{MTM(\Z)}\left(\Q(0),\Q(n)\right)\right).\]
Since we don't have mixed Tate motives over $\Z$ at hand at this time, we will do this in a very abstract setting, using mixed Tate Hodge structures as main example. We will have to make several assumptions for the proofs, all of which will be proven in the case of mixed Tate motives over $\Z$ in the next talk. Mixed Tate Hodge structures don't suffice for the proof of Goncharov-Terasoma, since the Ext-groups are too large.

% Basic notions:
% - def neutral tannakian cat over a field
% - def fiber functor
% - example G-rep for G linear pro-algebraic
% - def fundamental group
% - def isomorphism scheme
% - example tensor generated by one object

\section{Basic notions}

\begin{defn}
 A \emph{tensor category} over $k$ is an abelian category $\mathcal{C}$ enriched over $k$-vector spaces, equiped with a ``tensor'' product $k$-bifunctor $\otimes : \mathcal{C} \times \mathcal{C} \to \mathcal{C}$ and an ``identity'' object $1 \in \mathcal{C}$ together with natural isomorphisms $(\cdot) \otimes 1 \cong \mathrm{id}$ and $1 \otimes (\cdot) \cong \mathrm{id}$ that make the identity object worth its name, and natural isomorphisms $\alpha_{A,B,C} : (A \otimes B) \otimes C \cong A \otimes (B \otimes C)$, that are called \emph{associators}, which have to obey the pentagon and the triangle identites. From MacLane's coherence theorem, this implies all further combinations of associators and identities that can be isomorphic, are isomorphic. It is called \emph{symmetric} if there are symmetry isomorphisms $A \otimes B \cong B \otimes A$. These conditions are often abbreviated to ACU (for associativity, commutativity, unit). We require furthermore that a tensor category is \emph{rigid}, i.e. internal Homs exist, and 
$End(1)\simeq k$.

 A neutral Tannakian category over $k$ is a rigid abelian tensor category over $k$, together with a $k$-tensor functor to $k$-vector spaces, that is exact and faithful. A Tannakian category is a category which admits a functor to $k$-vector spaces that makes it neutral Tannakian (in other words: we don't fix the fiber functor).
\end{defn}

One can do the same not only for $k$-vector spaces but for quasicoherent sheaves over a $k$-scheme $S$. Deligne discusses this in detail, but we won't need the more general theory right now.

\begin{exam}
 Let $G$ be a linear pro-algebraic group over $k$, e.g. $\SL_{\infty} = \lim \SL_n$ over $\Q$, then the category of all finite-dimensional $k$-rational representations is a rigid abelian tensor category over $k$,
 and the forgetful functor to $k$-vector spaces is exact and faithful, making $\mathrm{Rep}(G)$ a neutral Tannakian category.
\end{exam}

\begin{prop}
 Let $\omega : \mathrm{Rep}(G) \to \Vect(k)$ be the forgetful functor. Then the tensor-automorphisms $G_\omega := \mathcal{A}ut^{\otimes}_k(\omega)$ form a linear pro-algebraic group, called \emph{fundamental group}. There is a canonical isomorphism $G \to G_\omega$.
\end{prop}
\begin{proof}
 Let $R$ be a $k$-algebra. The $R$-points of $G_\omega$ are the automorphisms $\mathcal{A}ut^{\otimes}_k(\omega)(R)$, i.e.,
\[
G_\omega(R) = \left\{ (\lambda_X)_{X \in \mathrm{Rep}(G)} 
\left| \begin{array}{c}
 \lambda_X \in \Aut(X \otimes R) \qquad \lambda_X\ R\textrm{-linear} \qquad \lambda_{X\otimes Y} = \lambda_X \otimes \lambda_Y \\ 
 \lambda_1 = \id_R \textrm{ and for all } G\textrm{-equivariant } \alpha : X \to Y \\ 
 \lambda_Y \circ (\alpha \otimes 1) = (\alpha \otimes 1) \circ \lambda_X : X \otimes R \to Y \otimes R
\end{array} \right.
 \right\}
 \]
so we can map $G(R) \to G_\omega(R)$, since every $g \in G(R)$ acts on every $G$-representation $X$ tensored with $R$.
This gives $G \to G_\omega$ and is in fact an isomorphism of functors of $k$-algebras:

We restrict to the full subcategory $\mathcal{C}_X$ of subquotients of any sum of tensor powers of $X$ and $X^\vee$ for a fixed object $X \in \mathcal{C}$. Then $\Aut^{\otimes}(\omega|_{\mathcal{C}_X})(R)$ can be considered as a subgroup of $\GL(X \otimes R)$ by $\lambda \mapsto \lambda_X$. Let $G_X$ be the image of $G$ in $\GL_X$, which is a closed algebraic subgroup, then we have
\[G_X(R) \subseteq \Aut^{\otimes}(\omega|_{\mathcal{C}_X})(R) \subseteq \GL_X(R) = \GL(X \otimes R).\]
If $V \in \mathcal{C}_X$ and $t \in V^G$, then the 1-parameter group $\alpha : k \to V,\ a \mapsto at$ is $G$-equivariant, and so $\lambda_V(t \otimes 1) = t \otimes 1$.
Thus $\Aut^{\otimes}(\omega|_{\mathcal{C}_X})$ is the subgroup of $\GL_X$ fixing all tensors in representations of $G_X$ fixed by $G_X$, which implies that $G_X = \Aut^{\otimes}(\omega|_{\mathcal{C}_X})$.

Now this works for all $X$, and we can take a limit construction to get the general result.
\end{proof}



\begin{thm}[Tannakian Reconstruction]
 Let $\mathcal{C}$ be a neutral Tannakian category with fiber functor $\omega$. Then the representation category of the fundamental group $G_\omega := \mathcal{A}ut^{\otimes}_k(\omega)$, as a neutral Tannakian category $(\mathrm{Rep}(G_\omega),\omega_\textrm{forget})$, is canonically equivalent to $(\mathcal{C},\omega)$.
\end{thm}

 Vague proof idea: We can write down a functor $(\mathcal{C},\omega) \to (\mathrm{Rep}(G_\omega),\omega_\textrm{forget})$ by mapping any $S \in \mathcal{C}$ to $\omega(S)$ with the $G_\omega$-action given by
\[
\mathrm{Aut}_k^{\otimes}(\omega) \times \omega(S) \to \omega(S),\qquad (\alpha,v) \mapsto \alpha_{S}(v).
\]


\begin{defn}
 Let $I(\omega,\eta) := \mathrm{Iso}^{\otimes}_k(\omega,\eta)$ be the isomorphism scheme from one fiber functor $\omega : \mathcal{C} \to \Vect(k)$ to another $\eta$. The $S$-points of this scheme (for $u : S \to k$ a $k$-scheme) consists of the set of isomorphisms of the fiber functor $u^\ast \omega$ with $u^\ast \eta$.

The scheme $I(\omega,\eta)$ is a right torsor under $G_\omega$ and a left torsor under $G_\eta$.
If $\mathcal{C}$ is tensor generated by a single object $S$, then $I(\omega,\eta)$ is a closed subscheme of $\mathrm{Iso}_k(\omega(S),\eta(S))$, the relations corresponding to the coherence constraints on the tensor product.
\end{defn}

















% Examples
% - ex graded vector spaces, Gm
% - ex loc sys and fibers
% - ex unipot completion of abstract group
% - ex MTHS/Q with deRham and Betti fiber functors and comparison

\section{Examples}

\begin{exam}
 The category of $\Z$-graded vector spaces over a field $k$ is neutral Tannakian with fiber functor the forgetful functor to ungraded $k$-vector spaces. A $\Z$-grading can be thought of as the weight grading of a $\Gm$-representation, where $\lambda \in \Gm(k)$ acts as multiplication with $\lambda^{-n}$ on the $n$th graded part. This shows that the category of $\Z$-graded vector spaces over a field $k$ is equivalent to the category of $k$-rational $\Gm$-representations, with forgetful functors on both sides corresponding to each other.
\end{exam}


\begin{exam}
 Take an abstract group $G$ and the subcategory of all finite-dimensional representations in $k$-vector spaces where $G$ acts by unipotent matrices. This is a Tannakian subcategory (sums and tensor products of unipotent representations are still unipotent). Its fundamental group wrt. the forgetful functor is then the solution to the universal problem of a group over which unipotent representations factor, and it is called the unipotent completion of $G$. It coincides with the Malcev completion because it satisfies the same universal property. In short:
\[\mathrm{Rep}^{un}(G) \simeq \mathrm{Rep}(G^{un}).\]
\end{exam}

\begin{exam}
 The category $MTH(\Q)$ carries at least two interesting fiber functors: deRham realization
\[\omega_{dR} : MTH(\Q) \to \Vect(\Q),\qquad (H,H_{dR},W_\bullet,F^\bullet,\alpha) \mapsto H_{dR})\]
 and Betti realization
\[\omega_B : MTH(\Q) \to \Vect(\Q),\qquad (H,H_{dR},W_\bullet,F^\bullet,\alpha) \mapsto H.\]
There is a comparison isomorphism over $\C$, i.e. a $\C$-point of $I(\omega_{dR},\omega_B)$,
given by $\alpha : H_{dR} \otimes \C \cong H \otimes \C$.
\end{exam}


% \begin{exam} %TODO not sure if explicit enough. Maybe should elaborate the topological case.
%  It is a classical result that for $(X,x)$ a topological space with basepoint $x \in X$, the category of local systems is equivalent to the category of $\pi_1(X;x)$-representations.
%  The same can be said for $X$ a scheme with $x \in X(k)$ a rational point, then local systems $E$ can be reconstructed from their monodromy representations, i.e. $E_x$ as $\pi_1^{et}(X,x)$-set.
% 
% %  Let $X$ be a smooth $k$-scheme and $Loc(X)$ the category of local systems (that is, locally constant \'etale sheaves). From any local system $E \in Loc(X)$ and any element $\gamma \in \pi_1^{et}(X,x)$ we get a chain of isomorphisms of fibers
% % \[E_{\gamma(0)} \cong E_{\gamma(t_1)} \cong \cdots \cong E_{\gamma(t_{n-1})} \cong E_{\gamma(0)}\]
% % for some segmentation $[0,1] = [t_0,t_1] \sqcup [t_1,t_2] \sqcup \cdots \sqcup [t_{n-1},t_n]$, where $t_0=0$, $t_1 = 1$ and $\gamma(t_i)$ are points in ... ahrg
% % 
% %  The category of local systems on a topological space $X$, name it $Loc(X)$, has a forgetful functor to $k$-vector spaces which gives just the fiber at a fixed basepoint $x \in X$. The $\otimes$-automorphisms are precisely the monodromy of all local systems, i.e. $\pi_1(X;x)$.
% \end{exam}










% Upper bounds on periods
% - def periods wrt field extension K/k and comparison iso and object S in Tannakian cat C
% - proposition periods as subquotient of certain Hopf algebra
% - proof
% - def involution action on periods and Hopf algebra


\section{Upper bounds on periods}

\begin{defn}
 Let $K/k$ be a field extension, $S \in \mathcal{C}$ an ind-object in a Tannakian category $\mathcal{C}$ with two fiber functors $\omega, \eta : \mathcal{C} \to \Vect(k)$ and a point $p \in I(\omega,\eta)(K)$. Then a \emph{period} of this data is an element of the $k$-vector space $P \subset K$ of \emph{periods} generated by the numbers $\langle \alpha,p^\vee \beta\rangle$ for all $\alpha \in \omega(S)$ and all $\beta \in \eta(S)^\vee$.
\end{defn}

We can use this to define various maps, in particular the main actor of this talk:
\begin{defn}
 Let $S$ be an object in $\mathcal{C}$.
 We define a $k$-linear map
\[\psi: \omega(S) \otimes_k \eta(S)^\vee \to \mathcal{O}(I(\omega,\eta))\]
which assigns to every $\gamma \otimes \sigma \in \omega(S) \otimes_k \eta(S)^\vee$ the function on $I(\omega,\eta)$ that sends a point $p \in I(\omega,\eta)(K)$ to the value of $\gamma \otimes p^\vee(\sigma) \in \omega(S)_K \otimes_K \omega(S)^\vee_K$ under the canonical pairing (i.e. evaluation map).
\end{defn}

\begin{prop}
 Periods $P \subset K$ for fixed $(k,K,\mathcal{C},S,\omega,\eta,p)$ are a quotient of the subset of the Hopf algebra $\mathcal{O}(I(\omega,\eta))$ which is generated by the image of $\omega(S) \otimes_k \eta(S)^\vee$ under $\psi$.
\end{prop}
\begin{proof}
 With the point $p \in I(\omega,\eta)(K)$ we can define an evaluation map $p^\ast : \mathcal{O}(I(\omega,\eta)) \to K$, whose concatenation with $\psi$ gives a generating set for the period $k$-vector space $P \subset K$ of $(k,K,\mathcal{C},S,\omega,\eta,p)$. The evaluation map is $k$-linear and the map $\mathrm{Span}_k\psi\left(\omega(S) \otimes_k \eta(S)^\vee\right) \to P$ is surjective by definition of $P$.
\end{proof}

\begin{lemma}
 Let $c : K \to K$ be a field involution over $k$, assume $\mathrm{char~} k \neq 2$. Suppose $c$ extends (not necessarily uniquely) to an involution $\tilde{c}$ of $I(\omega,\eta)$ over $k$ that commutes with $p : \spec(K) \to I(\omega,\eta)$, i.e. $c \circ p^\ast = p^\ast \circ \tilde{c}$.
Then we have not only the $c$-fixed periods $P^c$, but also the $\tilde{c}$-fixed space $\mathcal{O}(I(\omega,\eta))^{\tilde{c}}$,
and $\mathcal{O}(I(\omega,\eta))^{\tilde{c}} \to P^c$ is still surjective.

In particular,
The $c$-fixed periods $P^c$ are a subquotient of $\mathcal{O}(I(\omega,\eta))^{\tilde{c}}$.
\end{lemma}
\begin{proof}
 Let $x \in P^c$, then there is a preimage $y \in \mathcal{O}(I(\omega,\eta))$, and $(y+\tilde{c}(y))/2 \in \mathcal{O}(I(\omega,\eta))^{\tilde{c}}$ is a preimage of $x$ (which fails for characteristics $2$).
\end{proof}


 In the application, the involution $c$ of $I(\omega,\eta)$ will be complex conjugation on the $\C$-points of varieties, which is a natural choice.


























% =========================================================================================================


% Pro-unipotent group
% - now char k = 0
% - lemma fundamental group pro-unipotent iff all objects filtered with unit object subquotients
% - proposition given some cohomology finiteness and vanishing assumptions, the Hopf algebra of the fundamental group is a tensor algebra over some Ext-group
% - proof

\section{Pro-unipotent groups}

Now we work only in characteristics $0$ (since unipotent groups in positive characteristics behave worse than in characteristics $0$).
Furthermore, we assume in this section the fundamental group $G_\omega$ of our Tannakian category $(\mathcal{C},\omega)$ to be pro-unipotent.

\begin{lemma}
 The fundamental group $G_\omega$ of a neutral Tannakian category $(\mathcal{C},\omega)$ is a pro-unipotent algebraic group if and only if every object $S \in \mathcal{C}$ has a filtration such that the subquotients are isomorphic to the unit object $1 \in \mathcal{C}$.
\end{lemma}
\begin{proof}
If every object has such a filtration, then in particular $\mathcal{O}(G_\omega)$ has one, and therefore is pro-unipotent.

If, on the other hand, $G_\omega$ is pro-unipotent, then every $S$ has a unipotent filtration, and this can be refined to one with the properties of the lemma.
\end{proof}


\begin{lemma}\label{lem:prounisurj}
 Whenever a morphism of pro-unipotent groups $f : G' \to G$ is surjective on $H_1(-,k)$, it is already surjective.
\end{lemma}
\begin{proof}
Since $H_1(G,k)^\vee \simeq H^1(G,k)$, the fact that $H_1 f$ is surjective implies that $H^1 f$ is injective.
Also, surjectivity of $f$ is equivalent to injectivity of $f^\natural : \mathcal{O}(G) \to \mathcal{O}(G')$.

We know from a previous talk that $V := H^1(G,k) \simeq \gr_1^N \mathcal{O}(G)$, since $G$ is pro-unipotent.
There is a canonical morphism $\gr_\bullet^N \mathcal{O}(G) \inj T(\gr_\bullet^1 \mathcal{O}(G)) = T(V)$,
which commutes with the morphisms $\gr_\bullet f$ and $T(\gr_\bullet^1 f^\natural)$, so the former is injective:
\[
\xymatrix{
\gr_{\bullet}^N \cO(G) \ar[d]^{\mathrm{can}} \ar[r]^{\gr_{\bullet} f^\natural} & \gr_{\bullet}^N \cO(G') \ar[d]^{\mathrm{can}} \\
T(V) \ar[r]_{T(\gr_1 f^\natural)} & T(V')
}
\]
The result follows from the fact that a morphism which is injective on $\gr_\bullet$ is already injective,
since the kernel has $\gr_\bullet = 0$, so it must vanish as well.
\end{proof}



\begin{prop}
 Suppose $V := \Ext^1_{\mathcal{C}}(1,1)$ has $k$-dimension $r < \infty$ and $\Ext^2_{\mathcal{C}}(1,1) = 0$. Then there is an isomorphism of Hopf algebras $\mathcal{O}(G_\omega) \cong T(V)$.
\end{prop}
\begin{proof}
The proof proceeds in two steps:
First we construct a surjective morphism $\alpha : \spec T(V) \to G_\omega$, then we show it splits and must be an isomorphism.

\subsubsection{Step 1: Construction of $\alpha$}
We remember that $T(V)$ is the Hopf algebra of the pro-unipotent completion of a free group in $r$ generators.

Choose $\gamma_1,\ldots,\gamma_r \in G_\omega(\Q)$ minimal such that their image is a basis for
\[G_\omega(\Q)^{ab} = H_1(G_\omega) = H^1(G_\omega;k)^\vee = V^\vee.\]
Then we have a map from the free group generated by the $\gamma_i$ to $G_\omega(\Q)$.
From the universal property of the pro-unipotent completion of this free group we get a morphism $\alpha : \langle \gamma_1,\ldots,\gamma_r\rangle^{un} \to G_\omega$ (which depends on the choice of the $\gamma_i$).
It is also surjective on $H_1$ by construction, so from the previous lemma, it is surjective.


\subsubsection{Step 2: $\alpha$ splits and is an isomorphism}
From $0 = \Ext^2_{\mathcal{C}}(1,1) = H^2(G_\omega;k)$ classifying the extensions of $G_\omega$ by $\Ga$, we see that all these must be split extensions.

Since $\spec T(V)$ is pro-unipotent, we have an extension
\[U \inj \spec T(V) \surj G_\omega\]
with $U$ pro-unipotent as well. We look at the abelianization of $U$
\[U/[U,U]=\Ga^m \inj \spec T(V)/[U,U] \surj G_\omega\]
and this extension splits, giving a section $s : G_\omega \to \spec T(V)/[U,U]$.
Since $s \circ \alpha$ on $\{\gamma_1,\ldots,\gamma_r\} \subset \spec T(V)$ is the identity,
$s$ is an isomorphism, and in particular $U/[U,U] = 0$, so $U=0$ and we're finished.
\end{proof}








 
% Semi-direct product of a pro-unipotent group with Gm
% - assume C tensor generated by extensions of a fixed rank one object L, with morphisms respecting the corresponding weight structure
% - def canonical fiber functor
% - lemma fundamental group a semi-direct product of a pro-unipotent group with Gm
% - example MTH(k), L=Q(1)_H, w_n=W_2n, canonical fiber functor = deRham, then MTH(k)=Rep(Gm semi U_H)
% - proposition given some cohomology finiteness and vanishing assumptions, the Hopf algebra of the fundamental group is a tensor algebra over some direct sum of Ext-groups
% - proof
\section{Semi-direct product of a pro-unipotent group with Gm}

From now on, we assume that $\mathcal{C}$ is generated by extensions of tensor powers of a fixed rank one object $L$ (think: line bundle) and its dual $L^{-1} := L^\vee$.
Note that $G_\omega$ is no longer pro-unipotent, but we will show that it is still almost pro-unipotent.

 In other words: every object $S \in \mathcal{C}$ carries an increasing filtration $w_nS$, $n \in \Z$, whose $n$th adjoint quotient $gr^w_{n} = w_{n} S / w_{n-1} S$ is a direct sum of several copies of $L^{\otimes(-n)}$. We assume this filtration to be exact (in particular, $\gr^w_n$ is exact), respecting morphism in $\mathcal{C}$ and the tensor structure of $\mathcal{C}$, in particular $\Hom_{\mathcal{C}}(1,L^{\otimes n}) = 0$ for all $n \neq 0$ and

\begin{prop}
  $\Ext^1_{\mathcal{C}}(1,L^{\otimes (-n)}) = 0$ for $n \geq 0$ 
\end{prop}
\begin{proof}
Let $L^{\otimes (-n)} \inj S \surj 1$ be an extension,
then apply $\gr_0^w$ to get $\gr_0^w S \simeq 1$,
apply $\gr_n^w$ to get $\gr_n^w S \simeq L^{\otimes (-n)}$.
We have $1 = w_0 S \inj S$ and $L^{\otimes (-n)} \inj S$,
so we can form the direct sum and get an exact sequence
\[\ker \inj w_0 S \oplus L^{\otimes (-n)} \to S \surj \coker\]
where $\ker$ is pure of weight $n$ and $\coker$ is pure of weight $0$.
Applying $\gr_n$ and $\gr_0$ to the sequence show then that $\ker = 0$ and $\coker = 0$,
so the extension splits.
\end{proof}


\begin{defn}
 In this setting, one has a \emph{canonical fiber functor}
\[\omega : S \mapsto \bigoplus_{n \in \Z} \Hom_{\mathcal{C}}\left( L^{\otimes (-n)}, \gr_n^w S \right)\]
from $\mathcal{C}$ into $\Z$-graded $k$-vector spaces.
This defines a dual morphism of Tannakian fundamental groups $\Gm \to G_\omega$.
\end{defn}
\begin{proof} (that it is indeed a fiber functor)
The functor $\omega$ is $k$-linear, since $\gr_n^w$, covariant $Hom$ and $\bigoplus$ are $k$-linear.
It is a $\otimes$-functor, since the weight filtration respects the $\otimes$-structure.
It is also exact, since for $S' \inj S \surj S''$ we have $\Ext^1_{\mathcal{C}}(L^{\otimes (-n)},\gr_n^w S') = 0$
\end{proof}


\begin{lemma}
 The corresponding fundamental group has the form $G_\omega \simeq \Gm \rtimes U$ with $U$ a pro-unipotent group.
The category $\mathcal{C}$ is equivalent to the category of graded comodules over $\mathcal{O}(U)$.
\end{lemma}
\begin{proof}
Look at $\mathcal{C}' \subset \mathcal{C}$, the subcategory $\otimes$-generated by $L$. It has no non-trivial extensions,
and a natural grading, given by the tensor powers of $L$ that appear. This makes it a category of $\Gm$-representations,
which gives us a dual morphism of Tannakian fundamental groups $G_\omega \surj \Gm$,
with kernel $U$ a pro-unipotent group, since $L$ is the trivial $\Gm$-module, and $\mathcal{C}$ contains all iterated extensions of this trivial $\Gm$-module. The morphism induced by $\omega$ splits $G_\omega \surj \Gm$, since $\omega$ is a retraction of the full inclusion $\mathcal{C}' \inj \mathcal{C}$. This shows $G_\omega \simeq \Gm \rtimes U$.
\end{proof}

\begin{exam}
 Let $\mathcal{C} = MTH(k)$ and $L=\Q(1)_H$, with corresponding weight filtration $w_n = W_{2n}$. Then the canonical fiber functor coincides with the deRham realization functor $\omega = \omega_{dR}$, which amounts to
\[H_{dR}^n = \Hom_{MTH(k)}(\Q(-n),\gr^W_{2n} H),\]
which is a reformulation of the definition, that the $n$-th weight-graded part of $H$ is a pure Hodge structure $H_{dR}^n$ over $k$.

We have $G_{H} = \Gm \rtimes U_{H}$ and $MTH(k)$ is equivalent to the category of graded $\mathcal{O}(U_H)$-comodules.
\end{exam}

\begin{defn}
 We consider the grading on the tensor algebra $T(V)$ of the graded vector space $V = \bigoplus_{n > 0} V_n$ with $V_n = \Ext^1_{\mathcal{C}}(1,L^{\otimes n})$ to be such that $v \otimes w$ has degree $|v| + |w|$ (rather than $|v|+|w|+2$, which would also give a graded algebra).
\end{defn}


We would love to have something like this (which, by the way, doesn't make any sense):
\begin{lemma}
 The graded Hopf algebra $T(V)$ is the universal graded pro-unipotent group such that every morphism of graded groups from a free group in $r_n$ generators of degree $n$ into the $\Q$-points of a graded pro-unipotent group $G$ induces a morphism of graded pro-unipotent groups $T(V) \to G$.
\end{lemma}

Instead, we have to express what we need in terms of Lie algebras, to get a graded version:
\begin{lemma}
 Let $U$ be a graded pro-unipotent group with finitely many nonzero graded components. Then there is a surjective graded morphism $\spec T(V) \surj U$.

Let $U$ be a graded pro-unipotent group, then there is a surjective graded morphism $\spec T(V) \surj U$.
\end{lemma}
\begin{proof}
 The second statement follows from the first by a limit process, where we use $\spec T(V) = \lim \spec T(\bigoplus_{i=0}^n V_i)$ and $U = \lim U_n$ is the standard limit description.

The first statement comes from a graded re-statement of the last result of talk 3:

Let $F(V^\vee)$ be the free Lie algebra on $V^\vee$, where $V = H^1(U) = \mathfrak{g}/[\mathfrak{g},\mathfrak{g}]$.
We take a lift of $V^\vee$ to $\mathfrak{g}$, called $\tilde{V^\vee}$ (that is a choice, the same choice that the $\gamma_1,\ldots,\gamma_r$ were before).
So we get a morphism of graded Lie algebras $F(\tilde{V^\vee}) \to \mathfrak{g}$,
which induces a morphism of enveloping algebras $\mathcal{U}(F(\tilde{V^\vee})) \to \mathcal{U}(\mathfrak{g})$.
The latter one is a pro-unipotent Lie algebra, so the map factors through the pro-unipotent completion to a map
$\mathcal{U}(F(\tilde{V^\vee}))^{\wedge} \to \mathcal{U}(\mathfrak{g})$.
Dualizing gives us $T(V) \leftarrow \mathcal{O}(G)$ - as graded Hopf algebras, since everything respected the grading.
\end{proof}




\begin{prop}
 Assume for any $n$, the $k$-vector space $V_n$ has finite dimension $r_n$ and all $\Ext^2_ {\mathcal{C}}(1,L^{\otimes n})$ vanish. Then $r_n = 0$ for $n \leq 0$ and $V := \bigoplus_{n > 0} V_n$ is a graded vector space whose graded tensor algebra $T(V)$, with $V_n$ put in degree $n$, such that we have an isomorphism of graded Hopf algebras
\[\mathcal{O}(U) \cong T(V) = T\left(\bigoplus_{n>0} \Ext^1_{\mathcal{C}}(1,L^{\otimes n})\right).\]
\end{prop}
\begin{proof}
 Same argument as before, now with grading:

We get a morphism of graded pro-unipotent groups $T(\bigoplus_{n > 0} V_n) \surj U$.
It splits, as before, and we have an isomorphism of graded Hopf algebras.
\end{proof}















% Periods in the pro-unipotent case
% - def weigth filtration on objects induces weight filtration on periods
% - def filtered k-algebra $k[t^2] \otimes_k T(\bigoplus_{n>0} V_n)$
% - theorem as filtered algebra, $P^c$ is a subquotient of the latter.
% - proof
\section{Periods in the graded pro-unipotent case}

Now we assume, in addition to the canonical fiber functor $\omega$, to have a fiber functor $\eta$.

Suppose the involution $\tilde{c} : I(\omega,\eta) \to I(\omega,\eta)$ is given by the action of an order $2$ element $\epsilon \in G_\eta$ (with respect to the $G_\eta$-torsor structure on $I(\omega,\eta)$).

\begin{lemma}
 The element $\epsilon \in G_\eta$ is conjugate to $-1 \in \Gm \subset G_\eta$.
\end{lemma}
\begin{proof}
 Look at the commutative diagram
\[
\xymatrix{
G_{\omega} \times I(\omega, \eta) \ar[d]_{ad(\xi) \times \xi \cdot} \ar[rr]^{\mathrm{action}} && I(\omega, \eta) \ar[d]^{\xi \cdot} \\
G_{\omega} \times I(\omega, \eta) \ar[rr]_{\mathrm{action}} && I(\omega, \eta)
}
\]
for $\xi \in G_\omega$ any element. It commutes, since $\xi x \xi^{-1} \xi = \xi x$.
This shows that conjugation in $G_\eta$ corresponds to multiplication in $I(\omega,\eta)$.

In $G_\eta = \Gm \rtimes U$, the multiplication is
\[(\lambda,u) \cdot (\lambda',u') = (\lambda \lambda', u \lambda u' \lambda^{-1})\]
so if $\epsilon = (\lambda,u)$, we can multiply with $(-\lambda^{-1},\lambda^{-1} u^{-1} \lambda)$ to get
\[\epsilon \cdot (-\lambda^{-1},\lambda^{-1} u^{-1} \lambda) = (-1, 1). \qedhere\]
\end{proof}



\begin{defn}
 The filtration on $S$ defines a filtration on $\omega(S)$. Putting a trivial filtration on $\eta(S)^\vee$, this gives a filtration on $\omega(S) \otimes_k \eta(S)^\vee$ and thus a filtration on periods $P$. This filtration induces a filtration on the $c$-fixed points $P^c$.
\end{defn}

\begin{defn}
 Let $\mathcal{O}(I(\omega,\eta))_+ \subset \mathcal{O}(I(\omega,\eta))$ be the subspace generated (as $k$-algebra) by the image of all $\omega(S) \otimes_k \eta(S)^\vee$ under $\psi$, for $S$ of positive weight, i.e. $\gr_n^w S = 0$ for all $n < 0$.
\end{defn}


\begin{thm}\label{thm:realperiods}{\quad}
\begin{itemize}
 \item The involution $\tilde{c}$, resp. $\epsilon \in G_\eta$, respects the subspace $\mathcal{O}(I(\omega,\eta))_+$.
 \item The real periods $P^c$, as filtered $k$-vector space, are the image of $\mathcal{O}(I(\omega,\eta))^{\epsilon}_+$ under $p^\ast : \mathcal{O}(I(\omega,\eta)) \to K$.
 \item There is a graded Hopf algebra isomorphism $\mathcal{O}(I(\omega,\eta))_+^{\epsilon} \simeq k[t^2] \otimes_k \mathcal{O}(U)$.
 \item In particular, $P^c$ is a quotient of $k[t^2] \otimes_k T\left(\bigoplus_{n>0} \Ext^1_{\mathcal{C}}(1,L^{\otimes n})\right)$.
\end{itemize}
\end{thm}
\begin{proof}
The first is almost by definition, since any $\xi \in G_\eta$ respects the image of $\omega(S) \otimes \eta(S)^\vee$ in $\mathcal{O}(I(\omega,\eta))$, in particular those with only positive weights. In other words: translations respect the subalgebra of positive weights. The second statement also follows from this, by the definition of $P$.
From our knowledge about $\mathcal{O}(U)$, we only need to prove the third statement, to get the fourth one.
We will do this in four steps.

\subsubsection{Step 1: $\omega = \eta$}
 First, we know that all $G_\omega$-torsors are trivial (since $H^1(k,\Ga)=0=H^2(k,\Gm)$ and $G_\omega$ is an iterated extension of $\Ga$ and $\Gm$) so we know $\omega = \eta$.

\subsubsection{Step 2: the positive part}
Now we have the morphism
\[\Gm \times U \to G_\omega,\ (a,u) \mapsto a\cdot u\]
which defines an isomorphism of the corresponding graded $k$-algebras
\[\mathcal{O}(G_\omega) \cong k[t,t^{-1}] \otimes_k \mathcal{O}(U),\]
the grading on $\mathcal{O}(G_\omega)$ being induced by right translations of $\Gm \subset G_\omega$.

Since we consider only $S$ with $gr_W^n(S) = 0$ for $n < 0$, the action of $G_\omega$ on the image of $\omega(S) \otimes \eta(S)^\vee$ factors through the monoid $\spec( k[t] \otimes_k \mathcal{O}(U)) = \Ao_k \times_k U$.
Put differently, $\mathcal{O}(I(\omega,\eta))_+ \simeq k[t] \otimes_k \mathcal{O}(U)$, as $k$-algebra.

\subsubsection{Step 3: invariants under conjugation}
Since $\mathcal{O}(\Gm)^{-1}$, the invariants of $\mathcal{O}(\Gm)$ under $-1$, is isomorphic to $k[t^2,t^{-2}]$, we have
$\mathcal{O}({}_{-1}\backslash G_\omega) = k[t^2,t^{-2}] \otimes_k \mathcal{O}(U)$.

\subsubsection{Step 4: putting stepts 2 and 3 together}
We have $\mathcal{O}(I(\omega,\eta))_+ \simeq k[t] \otimes_k \mathcal{O}(U)$ and $\mathcal{O}(I(\omega,\eta))^\epsilon \simeq k[t^2,t^{-2}] \otimes_k \mathcal{O}(U)$ and altogether we get
\[\mathcal{O}(I(\omega,\eta))_+^{\epsilon} \simeq k[t^2] \otimes_k \mathcal{O}(U). \qedhere\]
\end{proof}
