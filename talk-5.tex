\chapter{Hodge structure: Proof of Chen's theorem by Sergey Rybakov}
%\addcontentsline{toc}{chapter}{*Hodge structure: Proof of Chen's theorem by Sergey Rybakov}

Sergey Rybakov on September 4th, 2012.

\medskip
\medskip

\noindent Recall that $M$ is a smooth connected manifold, $A_M^\bullet$ is the DG-algebra of complex valued differential forms, and $B(A_M^\bullet;a,b)$ is the bar complex of $A_M^\bullet$.

\section{The theorem}

Chen's theorem states that
\[
\mathrm{iter}^{\vee} : H^0(B(A_M^\bullet;a,b)) \to \C \otimes_{\Q} \cO(\pi_1(M;a,b)^{un}),
\]
is an isomorphism of algebras and respects coproducts. 
Note that both algebras are torsors under their corresponding Hopf algebras, thus it is enough to prove the Chen's theorem for $a=b$. Let us state it once more.

\begin{thm}[Chen's theorem]\label{thm:chentwo}
Let $H(M,a) = H^0(B(A_M^\bullet;a,a))$. Then
\[
\mathrm{iter}^{\vee} : H(M,a) \isom \C \otimes_{\Q} \cO(\pi_1(M,a)^{un})
\]
is an isomorphism of Hopf algebras.
\end{thm}

The morphism $\mathrm{iter}^{\vee}$ induces a functor
\[
\Phi : \begin{array}{rcl}
\mathrm{Comod~} H(M,a) & \longrightarrow & \mathrm{Rep}^{un}(\pi_1(M,a)) \\
V & \mapsto & \left( \gamma \mapsto \left( V \to V \otimes H(M,a) \stackrel{1 \otimes \int_{\gamma}}{\longrightarrow} V \right) \right).
\end{array}
\]

Clearly, $\Phi$ is faithful. 

\begin{lemma}[Key fact]
The theory of Tannakian categories shows that $\Phi$ is an equivalence of categories if and only if $\mathrm{iter}^{\vee}$ is an isomorphism.
\end{lemma}

By the Lemma the Chen's theorem is equivalent to the following result.

\begin{thm}\label{thm:comodsandreps}
The functor $\Phi$ is an equivalence of categories.
\end{thm}


\noindent To prove Theorem \ref{thm:comodsandreps}, we construct an inverse functor $\Psi$, which is a composition of Riemann-Hilbert correspondence functor and a functor $G$:
\[
\xymatrix{
\mathrm{Rep}^{un}(\pi_1(M,a)) \ar[d]^{RH} \ar[r]^{\Phi} & \mathrm{Comod}(H(M,a)) \\
 \mathrm{Conn}^{un}(M) \ar[ur]^{G}& \\
}
\]

\section{Riemann-Hilbert correspondence}

\begin{notation}
We denote by $\mathrm{Conn~}M$ the category of complex vector bundles with flat connection. There are adjoint functors
\[
\xymatrix{\mathrm{Conn~}M \ar@<0.2em>[r] & \mathrm{Rep~} \pi_1(M,a) \ar@<0.2em>[l]}.
\]
\end{notation}

A vector bundle $E$ with a connection $\nabla$ goes to the fiber $E_a$ with a monodromy representation.
Given a representation $V$ of $\pi_1(M,a)$, form $L = V \times \widetilde{M} / \pi_1(M,a)$, where $\widetilde{M}$ denotes the universal cover of $M$. Then
\[
V \mapsto L \otimes_{\C} \cA^0_M = \cE
\]
where $\cE$ is the sheaf of sections of $E$. The connection is defined as follows:
\[
\cE=L \otimes_{\C} \cA^0_M \stackrel{\nabla - \overline{d} \otimes_{\C} d}{\longrightarrow} L \otimes_{\C} \cA^1_M \cong L \otimes_{\C} \cA^0_M \otimes_{\cA^0_M} \cA^1_M \cong \cE \otimes_{\cA^0_M} \cA^1_M
\]

\begin{prop}
The Riemann-Hilbert correspondence takes unipotent representations to unipotent connections. In particular, it induces an equivalence:
\[
RH : \mathrm{Rep}^{un}(\pi_1(M,a)) \isom \mathrm{Conn}^{un}(M).
\]
\end{prop}

\section{Construction of the functor $G$ from unipotent connections to comodules}

Let $(E, \nabla) \in \mathrm{Conn}^{un}(M)$. Recall that an extension of trivial sheaves of $\cA^0_M$-modules is trivial. Since $E$ is unipotent, its sheaf of sections $\cE$ is trivial. 
Choose a trivialization  
\[
\phi : \cE \isom E_a \otimes_{\C} \cA^0_M
\]
such that $\phi$ induces $id_{E_a}$.

Put $N= \nabla- d$, where $N \in \End_{\C} E \otimes A^1_M$. Clearly, $\nabla$ is uniquly determined by $N$.
Since $\nabla$ is unipotent, $N$ is strict upper-triangular in a situable basis in $E_a$. Thus $N$ is nilpotent, i.e.
\[
N^{\rk E} = 0 \in (\End_{\C}) E \otimes (A^1_M)^{\otimes \rk E}.
\]

\begin{defn}
Let
\[
P_E = 1 + N + N^{\otimes 2} + \cdots \in (\End_{\C}E_a) \otimes_{\C} B(A_M,a)^0
\]
This is a finite sum.
\end{defn}

\begin{prop}
Given a vector bundle $E$ on $M$, $P_E$ is a cocycle.
\end{prop}
\begin{proof}

In fact, $P_E\in (\End_{\C}E_a) \otimes_{\C}(\oplus_{n\geq 0} (A_M^1)^{\otimes n})$. The differential on $\oplus_{n\geq 0} (A_M^1)^{\otimes n}$ is given by $d'+d''$, where
\begin{eqnarray*}
d'(\omega_1 \otimes \cdots \otimes \omega_n) & = & -\sum_{i=1}^n \omega_1 \otimes \cdots \otimes d \omega_i \otimes \cdots \otimes \omega_n \\
d''(\omega_1 \otimes \cdots \otimes \omega_n) & = & - \sum_{i=1}^{n-1} \omega_1 \otimes \cdots \otimes \omega_i \wedge \omega_{i+1} \otimes \cdots \otimes \omega_n
\end{eqnarray*}

By Proposition~\ref{prop:flatconn} given a vector bundle with connection $(E,\nabla)$, $\nabla$ is flat if and only if $dN = N \wedge N$. Altogether we have that $d' N^{\otimes n} = d'' N^{\otimes (n+1)}$. Thus $P_E$ is a cocycle.
\end{proof}

\begin{defn}
Cocycle $P_E$ defines an element $[P_E]\in (\End_{\C}E_a) \otimes_{\C} H(M,a)$. In turn, $[P_E]$ defines a $H(M,a)$-comodule structure on $E_a$.
\end{defn}

We are going to prove that $[P_E]$ does not depend on the choise of the trivialisation $\phi$, and that assignment $(E,\nabla)\to (E_a,[P_E])$ gives a functor $G:\mathrm{Conn}^{un}(M)\to\mathrm{Comod}(H(M,a))$.

\begin{prop}
Let $f : (E, \nabla_E) \to (F, \nabla_F)$ be a morphism in $\mathrm{Conn}^{un}(M)$.
The morphism
\[
P_F \cdot f_a \to f_a \cdot P_E
\]
is a coboundary in $\Hom_\C(E_a,F_a) \otimes_{\C} B(A_M, a)^0$.
\end{prop}

\begin{proof}
Fix trivialisations of $\cE$ and $\cF$ as before. We get the following isomorphism of complexes, independent of $\nabla$:
\[
\Hom_{\C}(E_a,F_a) \otimes_{\C} B(A_M,a) \isom \Hom_{\cA^0_M}(\cE,\cF) \otimes_{\cA^0_M} B(A_M,a)
\]
The differential of the right hand side is
\[
\delta : M \otimes \omega \mapsto dM \otimes \omega + M \otimes d\omega
\]
where $d\omega$ is a differential in $B(A_M,a)$, and tensor products are taken over $\cA^0_M$.
The term $dM \in \Hom_{\cA_M^0}(\cE,\cF) \otimes_{\cA^0_M} \cA^1_M$ comes from the trivial connection on trivialised sheaves. The proposition now follows from the key formula

\[
P_E f_a - f_a P_F = \delta \left( \sum_{n \geq 0} \sum_{i=0}^n N^{\otimes i}_F \otimes f \otimes N^{\otimes(n-i)}_E \right)
\]
where we consider the terms on the left hand side as an elements of $\Hom(\cE,\cF) \otimes_{\cA^0_M} B(A_M, a)$.

The formula can be proved by direct computation using the following observations:

\[
N_E f_a - f_a N_F = df,
\]

\[
\delta N_E=N_E\wedge N_E.
\]

\end{proof}

For an application, consider the case when $E = F$ and $\phi_1, \phi_2 : \cE \to E_a \otimes \cA_M^0$. Denote by $P_i$ the sum which come from $\phi_i$ for $i=1,2$. Since $(\phi_i)_a=1_{E_a}$, $P_1 - P_2$ is a coboundary in $\End(E_a) \otimes_{\C} B(A_m,a)^0$. Hence $[P_1] = [P_2]$. Functoriality can be proved in the same way.


\section{End of the proof}

\begin{prop}
The composition,
\[
\Phi \circ \Psi = 1
\]
is the identity on $\mathrm{Rep}^{un}(\pi_1(M;a))$.
\end{prop}
\begin{proof}
Let $V$ be a representation of $\pi_1(M;a)$. Then $\Psi(V) = G(RH(V))$, where $RH(V)$ is given by some nilpotent $N$. Put 
\[
P = \sum_{n=0}^{\infty} N^{\otimes n}.
\]
By Proposition \ref{prop:intismonodromy} the monodromy of $RH(V)$ is given by $\gamma \mapsto \int_{\gamma} P$.
Thus $\Phi(\Psi(V))\cong V$.
\end{proof}

\begin{prop}
The functor $\Psi$ is essentially surjective.
\end{prop}

\begin{proof}
The functor $\Psi$ send a unipotent connection described by a unipotent matrix to a comodule whose coproduct is described by a matrix. To show essential surjectivity, we begin with a matrix describing a coproduct and try to produce a connection that maps to the coproduct.

Decompose $A^0_M = V \oplus dA^0_M$. Then $A_M$ is quasi-isomorphic to the dg-algebra
\[
\widetilde{A}_M = (0 \to \C \stackrel{\sigma}{\to} V \stackrel{d}{\to} A^2_M \stackrel{d}{\to} \cdots )
\]

Given a vector space $V$, we define an associated bialgebra $T(V) := \bigoplus_{n \geq 0} V^{\otimes n}$ with product given by shuffle and coproduct given by deconcatenation.

We have an inclusion of coalgebras
\[
H(M,a)=H^0B(A^0_M,a)) \isom H^0(B(\widetilde{A}_M,a)) \isom H^0(\overline{B}(\widetilde{A}_M,a)) \subset T(V).
\]
Any comodule $C$ over $H(M,a)$ is also a comodule over $T(V)$. Any comodule over $T(V)$ comes from a comodule over $T(U)$ for $U \subset V$ with $\dim U < \infty$ because $\dim C < \infty$. So $U \subset A^1_M$ and $C$ is a comodule over $T(U)$.

By Talk 3 category of comodules over $H(M,a)$ is equivalent to the category of representations of the free group $\Gamma$ defined as follows. 
Let $e_1, \ldots, e_n$ be a basis of $U^{\vee}$, and $f_1, \ldots, f_n$ a basis of $U$.
Then $T(U)^{\vee} = \C\langle\langle e_1, \ldots, e_n \rangle\rangle$, and $\Gamma = \langle \gamma = \exp e_i \rangle$ is a free group.

Then $\gamma_i$ gives a unipotent matrix $M_i \in \End_{\C} C$. Put $N_i = \log M_i$, and
$$N = \sum_i N_i \otimes f_i \in \End C \otimes A^1_M.$$ The matrix $N$ defines a unipotent connection on a trivial vector bundle. We claim that $$P=1 + N + N^{\otimes 2} + \cdots$$ defines a comodule $C'$ which is isomorphic to $C$. Indeed, compute the action of $\gamma_i$ on $C'$.

\begin{eqnarray*}
\exp e_i \cdot \left( \sum_{j=1}^n N_j \otimes f_j \right)^{\otimes k} = \frac{e_i^{\otimes k}}{k!} N_i^k \times f_j^{\otimes k} = \frac{N_i^k}{k!} \\
\gamma_i(P) = \exp(e_i)(P) = \exp N_i = M_i
\end{eqnarray*}
\end{proof}

\begin{cor}
The composition,
\[
\Psi \circ \Psi = 1,
\]
is the identity on $\textrm{$H(M,a)$-Comod}$.
\end{cor}
\begin{proof}
It follows from the fact that $\Phi$ is faithful, $\Phi \circ \Psi = 1$ and $\Psi$ is essentially surjective.
\end{proof}
