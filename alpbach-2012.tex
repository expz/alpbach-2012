\documentclass[12pt]{report}
\usepackage{amsmath, amsthm, amssymb, amsfonts, latexsym, url, hyperref, tikz, enumerate, comment, colonequals, mathdots, stmaryrd}
\usepackage[utf8x]{inputenc}
\usepackage[OT2,OT1]{fontenc}
%\usepackage[russian,english]{babel}
\usepackage[all]{xypic}

\oddsidemargin		-0.25 in
\evensidemargin		-0.25 in
\textwidth		6.75  in
\headheight		0 in
\topmargin		0 in
\textheight		8.5 in

% -------- Environments ----------

\newtheorem{thm}{Theorem}[chapter]
\newtheorem{prop}[thm]{Proposition}
\newtheorem{cor}[thm]{Corollary}
\newtheorem{lemma}[thm]{Lemma}
\newtheorem{conj}[thm]{Conjecture}

\theoremstyle{definition}
\newtheorem{defn}[thm]{Definition}
\newtheorem{exam}[thm]{Example}
\newtheorem{xca}[thm]{Exercise}
\newtheorem{notation}[thm]{Notation}

\theoremstyle{remark}
\newtheorem{rem}[thm]{Remark}
\newtheorem{note}[thm]{Note}
\newtheorem{ques}[thm]{Question}
\newtheorem{todo}[thm]{TO DO}

\numberwithin{equation}{chapter}

% --------------- Other Alpbach Commands ----------------

\DeclareMathOperator{\Ab}{\bf{Ab}}
\DeclareMathOperator{\AdicDM}{\bf{AdicDM}}
\DeclareMathOperator{\Aff}{\bf{Aff}}
\DeclareMathOperator{\lAlg}{\bf{-Alg}}
\DeclareMathOperator{\Blpr}{\bf{Blpr}}
\DeclareMathOperator{\Ch}{\bf{Ch(R)}}
\DeclareMathOperator{\CHA}{\bf{CHA}}
\DeclareMathOperator{\DM}{\bf{DM}}
\DeclareMathOperator{\FSch}{\bf{FSch}}
\DeclareMathOperator{\Gps}{\bf{Gps}}
\DeclareMathOperator{\HA}{\bf{HA}}
\DeclareMathOperator{\LA}{\bf{LA}}
\DeclareMathOperator{\LRS}{\bf{LRS}}
\DeclareMathOperator{\MGps}{\bf{MGps}}
\DeclareMathOperator{\MLA}{\bf{MLA}}
\DeclareMathOperator{\Mod}{\bf{-Mod}}
\DeclareMathOperator{\Mon}{\bf{Mon}}
\DeclareMathOperator{\MS}{\bf{MS}}
\DeclareMathOperator{\PerfSH}{\bf{PerfSH}}
\DeclareMathOperator{\PerfDM}{\bf{PerfDM}}
\DeclareMathOperator{\RigDM}{\bf{RigDM}}
\DeclareMathOperator{\RigSH}{\bf{RigSH}}
\DeclareMathOperator{\Ring}{\bf{Ring}}
\DeclareMathOperator{\Psh}{\bf{Psh}}
\DeclareMathOperator{\PshAff}{\bf{Psh(Aff)}}
\DeclareMathOperator{\Set}{\bf{Set}}
\DeclareMathOperator{\lSet}{\bf{-Set}}
\DeclareMathOperator{\Sh}{\bf{Sh}}
\DeclareMathOperator{\Top}{\bf{Top}}
\DeclareMathOperator{\pUAG}{p\bf{UAG}}
\DeclareMathOperator{\UAG}{\bf{UAG}}
\DeclareMathOperator{\lVar}{\bf{-Var}}
\DeclareMathOperator{\Zar}{\bf{Zar}}

\newcommand \DMT  {\mathrm{DMT}(k)}
\newcommand \MHM {\mathrm{MHM}}
\newcommand \MTM {\mathrm{MTM}}
\newcommand \MTH  {\mathrm{MTH}(k)}
\newcommand \Ao   {\mathrm{Ao}}
\newcommand \MTMZ {\mathrm{MTM}(\mathbf{Z})}
\newcommand \DMk {\mathrm{DM}(k)}
\newcommand \SmCorr {\mathrm{SmCorr}(k)}
\newcommand \Smk  {\mathrm{Sm}/k}
\newcommand \MMk  {\mathrm{MM}(k)}
\newcommand \DbA  {\mathrm{D}^b(A)}
\newcommand \DbMMk {\mathrm{D}^b(\mathrm{MM}(k))}
\newcommand \DMeff {\mathrm{DM}^{\mathrm{eff}}(k)}
\newcommand \iterch {\mathrm{iter}^{\vee}}

% Martin's talk (talk-9.tex)

\providecommand{\DM}{\ensuremath{\mathbf{DM}}}
\providecommand{\mTm}{\ensuremath{\mathbf{MTM}}}
\providecommand{\Sm}{\ensuremath{\mathbf{Sm}}}
\providecommand{\MM}{\ensuremath{\mathbf{MM}}}
\providecommand{\ie}{\mbox{i.\,e.}\ }
\providecommand{\eg}{\mbox{e.\,g.}\ }
\providecommand{\SmCor}{\ensuremath{\mathbf{SmCor}}}
\providecommand{\D}{\ensuremath{\mathbf{D}}}
\providecommand{\Mod}{\ensuremath{\text{-}\mathrm{mod}}}
\providecommand \DTM {\ensuremath{\mathbf{DTM}}}
\providecommand \MTM {\ensuremath{\mathbf{MTM}}}

% -------- Russian ---------
\def\sha{{\text{\fontencoding{OT2}\selectfont sh}}}
\def\Sha{{\text{\fontencoding{OT2}\selectfont SH}}}

% -------- Roman ----------

\newcommand   \Ann    {\mathrm{Ann}}
\newcommand    \Aut    {\mathrm{Aut}}
\newcommand    \codim  {\mathrm{codim}}
\newcommand    \coker  {\mathrm{coker}}
\newcommand   \colim  {\mathrm{colim}}
\newcommand   \disc    {\mathrm{disc}}
\newcommand    \End   {\mathrm{End}}
\newcommand    \ext    {\mathrm{Ext}}
\newcommand   \Ext    {\mathrm{Ext}}
\newcommand  \gr	{\mathrm{gr}}
\newcommand    \Hom    {\mathrm{Hom}}
\newcommand	\HOM	{\mathrm{HOM}}
\newcommand    \id     {\mathrm{id}}
\newcommand    \im     {\mathrm{im}~}
\newcommand    \iter     {\mathrm{iter}}
\newcommand  \Ob   {\mathrm{Ob}}
\newcommand    \pic     {\mathrm{Pic}}
\newcommand	\proj	{\mathrm{Proj}~}
\newcommand   \Rep   {\mathrm{Rep}}
\newcommand    \rk       {\mathrm{rk}~}
\newcommand	\sch	{\mathrm{(Sch)}}
\newcommand	\schk	{\mathrm{(Sch/k)}}
\newcommand \schK   {\mathrm{(Sch/K)}}
\newcommand	\schS	{\mathrm{(Sch/S)}}
\newcommand	\schT	{\mathrm{(Sch/T)}}
\newcommand	\schU	{\mathrm{(Sch/U)}}
\newcommand    \spec   {\mathrm{Spec}~}
\newcommand   \supp   {\mathrm{supp}}
\newcommand     \tr       {\mathrm{tr}}
\newcommand   \Var     {\mathrm{Var}}
\newcommand   \Vect   {\mathrm{Vect}}

\newcommand{\catop}{\cat\op}
\newcommand{\cat}{\mathbf{C}}%{\mathscr{C}}
\newcommand{\catd}{\mathbf{D}}%{\mathscr{D}}
\newcommand{\catdg}{(\cat\downarrow G)}

% -------- Numbers --------

\def    \C     {\ensuremath{\mathbb{C}}}
\newcommand	\R	{\ensuremath{\mathbb{R}}}
\newcommand    \Q    {\ensuremath{\mathbb{Q}}}
\newcommand    \Z     {\ensuremath{\mathbb{Z}}}
\newcommand	\N	{\ensuremath{\mathbb{N}}}
\newcommand	\A	{\ensuremath{\mathbb{A}}}
\newcommand   \Aspace {\ensuremath{\mathbb{A}}}
\newcommand	\An	{\ensuremath{\mathbb{A}^n}}
\newcommand	\Pspace	{\ensuremath{\mathbb{P}}}
\newcommand    \Ga     {\ensuremath{\mathbb{G}_a}}
\newcommand    \Gm     {\ensuremath{\mathbb{G}_m}}
\newcommand    \Fp     {\ensuremath{\mathbb{F}_p}}
\newcommand    \Fq     {\ensuremath{\mathbb{F}_q}}

% -------- Groups ---------

\newcommand	\GLn	{\ensuremath{\boldsymbol{GL}_n}}
\newcommand	\SLn	{\ensuremath{\boldsymbol{SL}_n}}
\newcommand  \GL    {\ensuremath{\boldsymbol{GL}}}
\newcommand  \SL      {\ensuremath{\boldsymbol{SL}}}

% -------- Frakture ----------

\newcommand	\fa	{\ensuremath{\mathfrak{a}}}
\newcommand  \fg   {\ensuremath{\mathfrak{g}}}
\newcommand   \fH  {\ensuremath{\mathfrak{H}}}
\newcommand    \fm     {\ensuremath{\mathfrak{m}}}
\newcommand	\fp	{\ensuremath{\mathfrak{p}}}

% -------- Calligraphy -------

\newcommand    \cA     {\mathcal{A}}
\newcommand    \cB     {\mathcal{B}}
\newcommand    \cC     {\mathcal{C}}
\newcommand    \cE     {\mathcal{E}}
\newcommand    \cF     {\mathcal{F}}
\newcommand    \cG     {\mathcal{G}}
\newcommand    \cK     {\mathcal{K}}
\newcommand    \cM     {\mathcal{M}}
\newcommand    \cO     {\mathcal{O}}
\newcommand    \cP     {\mathcal{P}}
\newcommand    \cS     {\mathcal{S}}
\newcommand    \cT     {\mathcal{T}}
\newcommand    \cU     {\mathcal{U}}
\newcommand    \cV     {\mathcal{V}}
\newcommand    \cW     {\mathcal{W}}
\newcommand    \cX     {\mathcal{X}}
\newcommand    \cY     {\mathcal{Y}}
\newcommand    \cZ     {\mathcal{Z}}

% -------- Arrows --------

\newcommand    \inj    {\hookrightarrow}
\newcommand    \surj  {\twoheadrightarrow}
\newcommand    \onto   {\twoheadrightarrow}
\newcommand         {\rar}[1]       {\stackrel{#1}{\longrightarrow}}
\newcommand         {\isom}         {\rar{\sim}}

% -------- Miscellaneous --------

\newcommand    {\adj}[4]    {#1\negmedspace: #2\rightleftarrows #3:\negmedspace #4}

\newcommand  \TODO  {[Room for improvement]}

\begin{document}

\title{Summer School Alpbach: Multiple Zeta-Values\footnote{The workshop was organized by J. Ayoub, S. O. Gorchinsky and G. W{\"u}stholz (Chair) within the ETH Zurich ProDoc module in Arithmetic and Geometry. The speakers were Mario Huicochea, Roland Paulin, Fritz H\"ormann, Alberto Vezzani, Thomas Weissschuh, Sergey Rybakov, Javier Fres\'an, Rafael von K\"anel, Konrad V\"olkel, Martin Gallauer, Simon Pepin Lehalleur, Lars K\"uhne, Joseph Ayoub and Sergey Gorchinsky. The notes were recorded by Jonathan Skowera and in some cases dramatically altered by the speakers during editing.}}

\maketitle

{\small \setcounter{tocdepth}{1} \tableofcontents}


%%%%%%%%%%%%%%%%%%%%%%%%%%%%%%%%%%%

\setcounter{secnumdepth}{0}

\section{About these notes}
These are notes from the Workshop on Multiple Zeta-Values in Alpbach, Austria organized by the ProDoc Arithmetic and Geometry module of the ETH and the Universit\"at Z\"urich and meeting from September 2nd -- 7th, 2012. The workshop aims to study the motivic approach to multiple zeta values (MZV's) included recent advances due to Francis Brown. We especially rely on the presentation of P. Deligne in and \emph{Multizetas, d'apres Francis Brown} and refer to the paper \emph{Groupes fondamentaux motiviques de Tate mixte} by P. Deligne and A. Goncharov for many basic facts.

\medskip
\medskip

\noindent \emph{Please attribute all errors first to the scribe.}

\chapter*{Preliminary Talk: Unipotence, periods and motivations}
\addcontentsline{toc}{chapter}{**Preliminary Talk: Unipotence, periods and motivations by Brent Doran}

Brent Doran on August 27th, 2012.

Our introductory talks will concern themselves with three main questions.
\begin{enumerate}
\item What are periods?
\item How do they classically arise in Hodge theory?
\item How do they arise in our motivic context?
\end{enumerate}

\section{Toplogical and geometric motivations}

We start at the beginning, in order not to lose sight of the roots. Fix a field $k$, of characteristic 0 if necessary.

\begin{defn}[Algebraic group]
An algebraic group over $k$ is a $k$-variety with morphisms of varieties, $m : G \times G \to G$ and $i : G \to G$, satisfying the usual group axioms for multiplication, inversion and identity.
\end{defn}
\begin{exam}
Examples of algebraic groups are finite groups, the multiplicative group of $k$ ($\Gm$), the additive group of $k$ ($\Ga$), elliptic curves, etc.
\end{exam}
\begin{defn}[Group extension]
A group $G$ is an \emph{extension} of a group $Q$ by $N$ if there is a short exact sequence
\[
1 \to N \to G \to Q \to 1
\]
It is called a \emph{central extension} if $N$ lies in the center of $G$.
\end{defn}

\begin{prop}[Key Fact]
Every unipotent algebraic group over a field $k$ of characteristic 0 is an iterated central extension of $\Ga$.
\end{prop}
\noindent How general are these among algebraic groups?

\section{Classifying algebraic groups}

The classification proceeds according to the following plan.
\[
\xymatrix{
&& \textrm{algebraic groups} \ar[d] & & \ar@{-}[d]^{G/G^0 \textrm{~finite group}} \\
&& \textrm{connected} \ar[dl] \ar[dr] & & \ar@{-}[d]^{\textrm{Chevalley}} \\
& \textrm{affine} \ar@{=}[d] && \textrm{abelian variety} & \ar@{-}[d]^{\textrm{Linearize}} \\
& \textrm{linear} \ar[dl] \ar[dr] && & \ar@{-}[d]^{\textrm{Levi decomp.}} \\
\textrm{unipotent} \ar@{=}[d] && \textrm{reductive} \ar[d] & & \ar@{-}[d]^{\textrm{torus}} \\
\textrm{strictly upper triang.} && \textrm{semisimple} & &
}
\]
Classification of representations of reductive groups proceeds by the theory of weights in the co-character lattice of maximal torus. Classification of representations of unipotent groups is hard.

\subsection{Connected algebraic groups}
\begin{rem}
Let $G^0$ be the connected component of the identity. Then $G^0 \triangleleft G$, and $G^0$ is finite index. The quotient $G/G^0$ can be chosen to be an arbitrary finite group, so when we classify, we can simply consider connected $G$. The rest of the groups will be extensions of finite groups by these connected groups.
\end{rem}

\subsection{Reducing to affine and projective groups}

\begin{defn}[Affine algebraic group]
An algebraic group is \emph{affine} if its underlying $k$-variety is affine.
\end{defn}
\begin{defn}[Projective algebraic group]
Similarly, an algebraic group is \emph{projective} if its underlying $k$-variety is projective.
\end{defn}
For example, abelian varieties are projective algebraic groups.

\begin{rem}
Non-affine, non-projective algebraic groups arise naturally as Jacobians of singular curves and universal additive extensions of abelian varieties.
\end{rem}

\begin{rem}
But all algebraic groups are quasi-projective.
\end{rem}

\begin{thm}[Chevalley]
An algebraic group $G$ admits a unique normal affine subgroup $H$ such that
\[
1 \to H \to G \to A \to 1
\]
for an abelian variety $A$.
\end{thm}
\noindent The above theorem justifies studying affine and projective groups separately. We focus here on affine groups.

\begin{prop}[Linearize a $G$-variety]
Let $G$ be an affine algebraic group acting on an affine variety $X$. Then there exists a unique linearization, i.e., a $G$-equivariant closed immersion $X \inj V$ into a finite dimensional $G$-representation $V$.
\end{prop}

\begin{cor}\label{cor:affineislinear}
Affine algebraic groups are in fact linear algebraic groups, i.e., admit inclusions $G \inj GL(V)$.
\end{cor}

\section{Unipotent groups}

The definition of a unipotent group follows rests on the definition of unipotence for linear transformations.

\begin{defn}[Jordan decomposition]
Given $g \in GL_N$, there exists a unique decomposition $g = g_s \cdot g_u$ such that 
\begin{itemize}
\item $g_s$ is semisimple, i.e., diagonalizable over $\overline{k}$.
\item $g_u$ is unipotent, i.e., all eigenvalues are 1.
\item $g_s g_u = g_u g_s$.
\end{itemize}
\end{defn}

\begin{prop}
If $f : G \to H$ is a homomorphism of linear algebraic groups, then it preserves the Jordan decomposition
\[
f(g)_s = f(g_s) \qquad f(g)_u = f(g_u)
\]
\end{prop}

\noindent By the Corollary \ref{cor:affineislinear}, the notion of semisimplicity and unipotence is defined for elements of affine algebraic groups.

\begin{defn}[Unipotent group]
An affine algebraic group $G$ is \emph{unipotent} if all its elements are unipotent.
\end{defn}

\begin{rem}[Warning!]
The analogous is NOT true for the definition of semisimple groups.
\end{rem}

\begin{prop}
Any connected unipotent group is isomorphic to a subgroup of strictly upper triangular matrices.
\end{prop}

\begin{defn}[Derived group]
Let $G$ be an algebraic group. Then its \emph{derived groups} are given by $G' := \overline{[G, G]}$, $G'' := \overline{[G', G']}$, etc.
\end{defn}
\begin{defn}[Solvable group]
If this sequence terminates, then $G$ is said to be \emph{solvable}.
\end{defn}

\begin{thm}[Lie-Kolchin]
Every solvable algebraic group $G$ over an algebraically closed dield can be embedded into some $GL_N$ as upper triangular matrices.
\end{thm}

\begin{exam}
The group $\Gm \ltimes U$ is pro-unipotent.
\end{exam}

\begin{defn}[Unipotent radical]\label{def:uniradical}
Let $G$ be affine and connected. The \emph{unipotent radical} $r_u(G)$ of $G$ is the maximal connected unipotent normal subgroup of $G$.
\end{defn}

\begin{defn}[Reductive algebraic group]
An affine, connected algebraic group $G$ is \emph{reductive} if $r_u(G) = 0$. It is called redutive because its representations always decompose into a direct sum of irreducible representations. Hence to understand $Rep(G)$ for a reductive group $G$, one need only understand the irreducible representation, a feat accomplished by the theory of highest weights.
\end{defn}

\begin{defn}[Radical]
let $G$ be affine and connected. Then the \emph{radical} $r(G)$ of $G$ is the maximal connected solvable normal subgroup of $G$.
\end{defn}

Then $r_u(G) \subset r(G)$.

\begin{defn}[Semisimple algebraic group]
An affine, connected algebraic group $G$ is \emph{semisimple} if $r(G) = 0$.
\end{defn}
\begin{defn}[Torus]
An affine, connected algebraic group $G$ is a \emph{torus}, if there is a linearization $G \inj \GL_N$ such that all of its elements map to diagonal matrices.
\end{defn}

\begin{prop}
Every reductive group is an extension of a semisimple group by a torus, and torii are rigid (their morphisms cannot be deformed) and classified by a discrete invariant.
\end{prop}

\begin{thm}[Levi decomposition]
Let $G$ be an affine algebraic group. Then there exists a reductive subgroup $H$ such that
\[
G = r_u(G) \rtimes H
\]
The subgroup $H$ is unique up to conjugation and called the \emph{Levi factor}.
\end{thm}

\begin{exam}
If $G$ is commutative, the Levi decomposition has the form $G = U \times T = (\Ga)^k \times T$.
\end{exam}
\noindent So what can we say about representation theory? We study reductive and unipotent groups separately. There are only discrete choices with arbitrary dimensional moduli which are arbitrary badly behaved, a situation known as Murphy's Law.

The classification of unipotent groups themselves is very hard. What can be said:

\begin{itemize}
\item As a variety, every unipotent group is isomorphic to $\A^n$. 
\item The orbits of a unipotent group are all closed subvarieties.
\item Abelian unipotent groups in characteristic zero are of the form $\Ga^k$.
\item Unipotent groups in characteristic zero are iterated extensions by $\Ga$'s.
\end{itemize}

\begin{defn}[Lower central series]
Given a group $G$ (not necessarily algebraic), its \emph{lower central series} is
\[
G_1 = G, \qquad G_{i+1} = [G_i, G]
\]
\end{defn}

For example, consider the following unipotent group and its lower central series.
\[
G = G_1 = \left\{ \left(\begin{array}{ccc}
1 & a & c \\
0 & 1 & b \\
0 & 0 & 1
\end{array} \right) \left| a, b, c \in k \right. \right\} \qquad
G_2 = \left\{ \left(\begin{array}{ccc}
1 & 0 & t \\
0 & 1 & 0 \\
0 & 0 & 0
\end{array} \right) \left| t \in k \right. \right\}
\]
and $G_3 = 1$. Because $G_3 = 1$, we say that $G$ has nilpotency class 2, or that $G$ is a step 2 nilpotent group. More generally, the $n \times n$ strictly upper triangular matrices have nilpotency class $n-1$.

\begin{rem}
Affine algebraic groups $G$ are determined by $\mathrm{Rep}(G)$ in a Tannakian manner.
\end{rem}

\subsection{Discrete groups}
There exists a useful formalism for affine algebraic groups, but not for general discrete groups. Given a finitely generated discrete group $\Gamma$, e.g., $\GL(n, \Z)$, what can we say? If $\Gamma$ were an algebraic group, we might study it by looking for a variety $X$ and an embedding of $\Gamma \subset \Aut(X)$. In the discrete case, we find a manifold $M$ with $\Gamma = \pi_1(M)$. This leads to covering spaces! But a direct algebro-geometric interpretation of such $\Gamma$ is difficult, e.g., not every $\Gamma$ is realized as a profinite completion.

\begin{defn}
A \emph{$\Q$-local system} on a manifold $M$ is a locally constant sheaf\footnote{A locally constant sheaf $\cF$ is a sheaf such that $\cF(U) \cong V$ for all connected $U$ in the same connected component. For example $\underline{\Z}$ or $\underline{\Q}$.} of finite dimensional $\Q$-vector spaces.
\end{defn}
\noindent Local systems ``linearize'' the theory of covering spaces:
\begin{eqnarray*}
\mathrm{Rep}_{\Q}(\Gamma) & \longleftrightarrow & \textrm{$\Q$-local system on $M$} \\
\textrm{monodromy group is $\rho(\Gamma)$} & \longleftrightarrow & \textrm{monodromy group actions of $\pi_1(M)$}
\end{eqnarray*}
There are a number of issues with the scheme:
\begin{enumerate}
\item There is no reason these local systems underlie a variation of Hodge structure. Most do not. (See the next lecture for Hodge structures, Definition \ref{def:mixedhodge}.)
\item Representations of discrete $\Gamma$ can be arbitrarily bad. But they relate to representations of algebraic groups for which we have the Tannakian formalism for concrete understanding.
\end{enumerate}
We could try to overcome these issues by taking the closure of the discrete group $\Gamma$ in $\GL(V)$ for various $\Gamma$-representations $V$. That is sort of what we'll do. From a topologist's viewpoint, the homotopy theory of an arbitrary $M$ is hard. So we would like to ``linearize'' from homotopy theory to cohomology, and add additional structure to it, giving it a more geometric nature. We try ``rational homotopy theory'' (Sullivan, Deligne, Griffiths, Morgan, et al.) which is more manageable and is in a sense the ``rational cohomology'' of $\Gamma$ with homotopy tools (e.g., Postaikov towers). Then everything reduces to the study of a de Rham complex. Here arises a dg-algebra structure (Cf. Definition \ref{def:dgalgebra}).

There is no obvious way to encode all of $Rep_{\Q}(\Gamma)$. However, Sullivan, et al. realized quite early that it is quite powerful to encode all unipotent representations of $\Gamma$ via Malcev pro-unipotent completion of $\Gamma$.

\begin{rem} 
For a compactifiable K\"ahler manifold, this linearization of homotopy admits a mixed Hodge structure. Ultimately, Deligne-Gonehovov lift this structure to show it is a ``motive'', and hence the period formalism, etc. can be used.
\end{rem}
\begin{prop}[Morgan]
If $M$ is compact K\"ahler, then the Malcev algebra, i.e., $\mathrm{Lie} \pi_1(M)^{un}$ is generated by quadratic terms; K\"ahler groups.
\end{prop}
\noindent Many groups do not arise in this way. Even the free group on two generators does not!

By design, $\pi_1(M)^{un}$ encodes unipotent $\Q$-local systems on $M$. In other words, it encodes unipotent monodromy representations. Hence it encodes iterated extensions of trvial local systems.

\subsection{Why should unipotence be relevant to Hodge theory and periods?}
Maybe we never get unipotent $\Q$-local systems under variation of Hodge structure. Actually, it's almost the opposite: locally we always get, in effect, unipotent $\Q$-local systems. 

The basic picture of periods is found in the Lefschetz degeneration. For example, a torus degenerates to a pinched torus and a cylinder to a cone which produces nodes in both cases.

Let us first associate a local system $\mathcal{L}$. Then $\pi_1(\Delta^*)$, where $\Delta^*$ is the punctured disk, is generated by the obvious loop. Then,
\[
\mathcal{L}_{+} = H^1(E_t), t \neq 0 \qquad H^1(E_t, \Q) = Q^2, \left(\begin{array}{cc}
1 & 1 \\
0 & 1
\end{array}\right).
\]
There is a Picard-Lefschetz formula for the general case of local monodromy.

\subsection{Period picture}
$E$ has a unique 1-form $\omega$ and period vector $w(t) = S_{\gamma}$. The cocyles $\omega_i$ and $\gamma_i$ are basis vectors for $H^1(E, \Q)$. Then $\Gamma = \rho(\pi_1(\Delta^*))$ acts on $H^1(E_t, \Q)$. There is a map
\[
\Delta^* \to \textrm{$\Gamma$-orbits of periods}, \qquad t \mapsto \Gamma w(t)
\]
Then
\[
\Pspace(H^1(E_t, \Q)) \cong \Pspace^1 = \{[\omega_1 : \omega_2]\} = \{[1_{\epsilon} : \tau]\}
\]
The period domain is $\mathfrak{h} = H^1$ which is isomorphic to the Poincar\'e unit disk.

\begin{thm}[Monodromy theorem]
Let a local system $\mathcal{L}$ underlie a variation in Hodge structure. Then the monodromy operator $T$ is quasi-unipotent, i.e., its eigenvalues are roots of unity.
\end{thm}
\begin{thm}[Deligne's finiteness theorem]
Fix a compactifiable $B$ and an integer $N$. There exist at most finitely many conjugacy classes of rational maps of $\pi_1(B)$ of dimension $N$ giving local systems that occur as direct factor of a variation in mixed Hodge structure.
\end{thm}
\begin{rem} 
There are far too many unipotent representations, and very few variations of Hodge structure have unipotent monodromy.
\end{rem}
\noindent So there exists two examples of ``competing factors'' with unipotent local systems. On the one hand there are $\Q$-local systems with Deligne finiteness; on the other hand, there is global monodromy. Global monodromy is ``typically'' not unique, while local monodromy ``is'' unipotent.

Hence there is an incomplete ``motivic $\pi$'' theory. It was proved 40 years ago that $\pi_1^{un}$ admits a mixed Hodge structure. The proof that $\pi_1^{un}$ is a $\Q$-motive brought this into a modern period formalism.

\section{Nilpotent groups}
\subsection{What is a pro-nilpotent completion of $\pi_i$?}
It exists for any discrete group. It is convenient to reduce to a finitely generated group in characteristic 0. We'll be more concrete and conceptual than the formal way.

\begin{defn}[Nilpotent group]
A group $\Gamma$ (not necessarily algebraic) is \emph{nilpotent} if is lower central series, $F_{i+1} = [\Gamma_i, \Gamma]$, stabilizes in finitely many steps at the trivial group.
\end{defn}

\begin{rem}
Over a field of characteristic 0, unipotent groups are nilpotent.
\end{rem}

\begin{defn}[Pro-nilpotent completion]
The \emph{pro-nilpotent completion} of a group $\Gamma$ is a group $\Gamma^{nil}$ satisfying the universal property:
\[
\xymatrix{
\Gamma \ar[r] \ar[dr]_{\forall} & \Gamma^{nil} \ar[d]^{\exists !} & \\
& N & \textrm{$N$ nilpotent}
}
\]
It can be constructed as the limit of the projective system formed by morphisms $\Gamma \to N$ for nilpotent groups $N$. The construction can be simplified by restricting to the system of lower central series quotients, $\Gamma \to \Gamma / \Gamma_{n+1}$, which satisfy universal propreties for morphisms into step $n$ nilpotent groups. Then $\Gamma^{nil}$ can also be calculated as the natural projection of $\Gamma$ to the limit of
\[
\cdots \to \Gamma / \Gamma_3 \to \Gamma / \Gamma_2
\]
\end{defn}

\begin{lemma}
Let $N$ be a nilpotent group. The set of torsion elements $\mathrm{Tor~} N$ is a normal subgroup. If $N$ is finitely generated, then $\mathrm{Tor~} N$ is finite.
\end{lemma}

\noindent Torsion is no big deal, so we define torsion-free pro-nilpotent completion in a similar way.
\begin{defn}[Torsion-free pro-nilpotent completion]
\[
\xymatrix{
\Gamma \ar[r]^{j_0} \ar[dr] & \Gamma_0^{nil} \ar[d]^{\exists !} \\
& N_0 = N/\mathrm{Tor~} N
}
\]
It can similarly be calculated as a limit of $\cdots \to (\Gamma / \Gamma_3)_0 \to (\Gamma / \Gamma_2)_0$.
\end{defn}

\begin{exam}
A crucial example is a when $G$ is a finitely generated abelian group. Then the pro-nilpotent completion is $id : G \to G$. (This is obvious: $G$ is step 1 nilpotent and the identity satisfies the universal property.) The torsion-free pro-nilpotent completion is the quotient $G \to G / \mathrm{Tor~} G$.
\end{exam}

\begin{defn}[Pro-unipotent group]
A pro-unipotent group is an algebraic group over a field such that every representation has an increasing unipotent filtration.
\end{defn}

\begin{defn}[Pro-unipotent completion]\label{def:prounipotentcomp}
Let $\Gamma$ be a finitely generated group. Then its pro-unipotent completion $\Gamma^{un}$ (also known as the Malcev completion) is the pro-unipotent algebraic group $\Gamma^{un}$ over $\Q$ satisfying the universal property:
\[
\xymatrix{
\Gamma \ar[r] \ar[dr]_{\forall} & \Gamma^{un}(\Q) \ar[d]^{\exists !} & \\
& U(\Q) & \textrm{$U$ unipotent alg. group over $\Q$}
}
\]
Alternatively, the pro-unipotent completion can be calculated as
\[
\Gamma^{un} = \spec \varinjlim_n(\Q[\Gamma]/I^n)^{\vee}.
\]
where $I \subset \Q[\Gamma]$ is the augmentation ideal which will be defined later. This will be explained in the lecture on pro-unipotent completion \ref{ch:prounipotent}.
\end{defn}

\chapter*{Preliminary Talk: Cohomology, Hodge structures and Hopf algebras}
\addcontentsline{toc}{chapter}{*Preliminary Talk: Cohomology, Hodge structure and Hopf algebras by Sergey Gorchinskiy}

Sergey Gorchinskiy on September 30th, 2012.

\section{Fundamental groups and groupoids}

\begin{defn}[Fundamental groups and sets]\label{def:fundgroup}
Let $M$ be a path-connected topological space and $a, b \in M$. The \emph{fundamental group} $\pi_1(M;a)$ of $M$ is the group of homotopy classes of loops based at $a$ where multiplication denotes concatenation of loops. We use the convention that $\gamma_1 \circ \gamma_2$ denotes going along $\gamma_1$ and then $\gamma_2$. 

Let $\pi_1(M; a, b)$ denote the set of homotopy classes of paths from $a$ to $b$. It has no group structure, but is a left torsor under $\pi_1(M; a)$ and a right torsor under $\pi_1(M; b)$. 

These torsors combine to form the \emph{fundamental groupoid} with a multiplication of the form $\pi_1(M; a, b) \times \pi_1(M; b, c) \to \pi_1(M; a, c)$:
\[
\xymatrix{
& \coprod_{a,b} \pi_1(M; a, b) \ar[dl]_s \ar[dr]^t & \\
M & & M
}
\]
\end{defn}

\section{Filtered vector space}
Let $V$ be a $k$-vector space. $F^iV$ will denote a decreasing filtration of sub-$k$-vector spaces of $V$, e.g.,
\[
V \supset \cdots \supset F^{-1}V \supset F^0V \supset F^1V \supset \cdots,
\]
and $F_iV$, an increasing one. For example, a graded vector space $V = \bigoplus_{i \in I} V^i$ induces filtrations $F^pV = \bigoplus_{i \geq p} V^i$ and $F_p V = \bigoplus_{i \leq p} V^i$.
\begin{defn}[Associated Graded Vector Space]\label{def:assocgraded}
Given a decreasing filtration $F^iV$ on a vector space $V$, there is an associated graded vector space with a decreasing filtration,
\[
\gr_F V = \bigoplus_{i \in \Z} F^i V / F^{i+1}V, \qquad \gr_F^n = \bigoplus_{i \geq n} F^i V / F^{i+1}V.
\]
Given an increasing filtration $F_iV$, there is an analogous associated graded vector space $\gr^F V$ with graded parts $\gr_n^F V = \bigoplus_{i \leq n} F^{i+1} V / F^i V$.
\end{defn}
\begin{defn}[Exhaustive filtration]
A filtration $F$ on a vector space $V$ is \emph{exhaustive} if $\cup_{i \in \Z} F^i V = V$.
\end{defn}
\begin{defn}[Separated filtration]
A filtration $F$ on a vector space $V$ is \emph{separated} if $\cap_{i \in \Z} F^i V = \{0\}$.
\end{defn}
\begin{defn}[Completion of a filtered vector space]
The \emph{completion} of a filtered vector space is
\[
\widehat{V} = \varprojlim_i V / F^i V
\]
\end{defn}
\begin{defn}[Morphism of filtered vector spaces]
A \emph{morphism} $f : V \to U$ of filtered vector spaces satisfies $f(F^iV) = F^iU$. Its kernel is a filtered vector space with filtration,
\[
F^i\ker f := \ker f \cap F^iV.
\]
The filtration on $\coker f$ is similarly defined.
\end{defn}
\noindent Let $f$ be an endomorphism of the filtered vector space $V$. If the filtration $F^iV$ is exhaustive and separated, then $f$ is an isomorphism of filtered vector spaces (use $\gr^i f : \gr^i V \to \gr^i V$). Note that
\[
\ker (\coker) \neq \coker(\ker)
\]
\begin{prop}
Filtered vector spaces with strict morphisms form an abelian category.
\end{prop}

\section{Betti and de Rham cohomology}
\subsection{Betti cohomology}
\begin{defn}[Singular complex]
Let $M$ be a topological space. The \emph{singular complex} of $M$ is the complex of freely generated abelian groups,
\[
\cdots \to S_1(M) \to S_0(M) \to 0, \qquad S_i(M) := \langle f : \Delta^i \to M \rangle_{\Z},
\]
with alternating sums of faces for boundary maps. The singular cocomplex is the dual complex $S^i(M) := \Hom(S_i(M), \Z)$.
\end{defn}
\begin{defn}[Betti cohomology]\label{def:betti}
Given a topological space $M$, its Betti homology and cohomology is the 
\[
H^B_i(M) := H_i(S_{\bullet}(M)), \qquad H_B^i(M) := H_i(S^{\bullet}(M))
\]
The Betii homology with coefficients in a $\Z$-module $F$ is
\[
H^B_i(M, F) = H_i(S_{\bullet}(M)) \otimes_{\Z} F
\]
\end{defn}
\begin{prop}
Let $M$ be a (nice enough?) topological space and $a \in M$ a point. Then
\[
\pi_1(M;a)^{ab} := \pi_1(M;a) / [\pi_1(M;a), \pi_1(M;a)] \cong H_1(M, \Z)
\]
\end{prop}
\begin{defn}[Relative cohomology]
Let $M$ be a topological space, and let $N \inj M$ be a closed subspace. Let $F$ be a $\Z$-module. Form the relative complex and cocomplex
\[
S_i(M,N) := S_i(M) / S_i(N), \qquad S^i(M,N) = \ker (S^i(M) \to S^i(N))
\]
Then the cohomology of $M$ relative to $N$ is
\[
H^i_B(M,N;F) := H^i(S,(M,N;F))
\]
\end{defn}
There is an alternative formulation of relative cohomology using sheaves:
\begin{defn}[Relative cohomology by sheaves]
Let $M$ be a topological space, and let $N \inj M$ be a closed subspace. Let $F$ be a $\Z$-module. Form the relative cohomology
\[
H^i_B(M,N;F) := H^i(M,j_{!}F|_{M \setminus N})
\]
where $F$ is the constant sheaf of abelian groups on $M$ and $j_{!}$, where $j : N \inj M$, extends the sheaf by 0 over the closed subspace $N$.
\end{defn}

\subsection{de Rham Cohomology}
\begin{defn}[de Rham complex]\label{def:derham}
Let $M$ be a smooth manifold. Define the de Rham complex to be the complex $\C$-vector spaces,
\[
A^i_M := \{ \textrm{smooth $i$-forms on $M$} \},
\]
with the usual differential $d$.
\end{defn}
\begin{defn}[de Rham cohomology]
Let $M$ be a smooth manifold. Its de Rham cohomology is the complex of $\C$-vector spaces:
\[
H^i_{dR}(M, \C) := H^i(A^{\bullet}_M)
\]
\end{defn}

\begin{thm}[De Rham's theorem]
Let $M$ be a smooth manifold. Then there is an isomorphism,
\[
H^i_B(M, \C) \cong H^i_{dR}(M, \C),
\]
arising from the pairing
\[
\begin{array}{rcl}
S_i(M,\C) \otimes_{\C} A_M & \to & \C \\
(\sigma, f) \otimes \omega & \mapsto & \int_{\sigma} f^* \omega
\end{array}
\]
\end{thm}

\section{Mixed Hodge structures}
\begin{defn}\label{def:qpurehodge}
Let $H$ be a finite dimensional $\Q$-vector space. A $\Q$-pure Hodge structure of weight $n$ on $H$ is a decreasing filtration, $F^{\bullet}H_{\C}$, of $H_{\C} = H \otimes_{\Q} \C$ such that
\[
\bigoplus_{p \in \Z} H^{p,n-p}_{\C} \isom H_{\C},
\]
where
\[
H^{p,n-p}_{\C} := F^p H_{\C} \cap \overline{F^{n-p}H_{\C}}.
\]
\end{defn}

\begin{thm}[Hodge Theorem]\label{thm:hodge}
Let $X$ be a smooth complex variety and $X(\C)$ its corresponding topological space. Then $H_B(X(\C),\Q)$ admits a $\Q$-pure Hodge structure.
\end{thm}
\begin{proof}
Using the isomorphisms $H^n_B(X(\C),\Q)_{\C} = H^n_B(X(\C),\C) = H^n_{dR}(X(\C),\C) = H^n(A_{X(\C)})$ ($A^n_{X(\C)}$ is the $\C$-vector space of rank $n$ smooth, complex differential forms on $X(\C)$), it suffices to find a filtration on $H^n(A_{X(\C)})$. Define
\[
A^{p,q}_{X(\C)} = f(z) dz_{i_1} \wedge \cdots \wedge dz_{i_p} \wedge d\overline{z}_{j_1} \wedge \cdots \wedge d\overline{z}_{j_q}
\]
These form a double complex
\[
\xymatrix{
A^{1,0}_M \ar[r] & A^{1,1}_M \ar[r] & A^{1,2}_M \\
A^{0,0}_M \ar[u]^{\partial} \ar[r]^{\overline{\partial}} & A^{0,1}_M \ar[u]^{\partial} \ar[r]^{\overline{\partial}} & A^{0,2}_M \ar[u]
}
\]
where the differentials satsify $\partial \overline{\partial} = \overline{\partial} \partial$, $\partial^2 = \overline{\partial}^2 = 0$, and $d = \partial + \overline{\partial}$. Then
\[
A^n_{X(\C)} = \bigoplus_{p+q = n} A^{p,n-q}_{X(\C)}.
\]
Hence $A^{p,q}_{X(\C)}$ defines a Hodge structure.
\end{proof}

Compare this to the Hodge structure $H^{p,q} = H^q(X, \Omega^p)$, where $\Omega^p$ is the sheaf of holomorphic $p$-forms on $X$. There is an inclusion
\[
0 \to \Omega^p \to \cA^{p,0} \to \cA^{p,1} \to \cdots \to \cA^{p,d} \to 0
\]
where $\cA^{p,q}$ is the sheaf of smooth $(p,q)$-forms on the underlying manifold.

\subsection{Mixed Hodge Structure}
Let $X \subset \overline{X}$ be a subvariety of a smooth projective curve over $\C$, where $X = \overline{X} \setminus D$ for a non-empty divisor $D$.
\begin{notation}\label{not:qminusone}
We let $\Q(-1) = H^2(\Pspace^1)$.
\end{notation}
There is an exact sequence:
\[
0 \to H^1(\overline{X}, \Q) \to H^1(X, \Q) \to \bigoplus_{p \in D} \Q(-1) \to \Q(-1) \to 0
\]
Define a filtration on $H(\overline{X}, \Q)_{\C}$ by
\[
F^nH^1(\overline{X},\Q)_{\C} := \left\{ \begin{array}{ll}
H^2(\overline{X}, \Q)_{\C} & i < 1 \\
H^0(\overline{X}, \Omega_{\overline{X}}^1 \langle D \rangle) & i = 1 \\
0 & i > 1
\end{array} \right.
\]
Then
\[
\im(F^1H^1(X) \stackrel{r}{\longrightarrow} \bigoplus \Q_{\C} = H^0(D, \C) = \left\{ (a_x)_{x \in D} \left| \sum_{x \in D} a_x = 0 \right. \right\}
\]
Hence the Tate twist.

\begin{defn}[Mixed Hodge Structure]\label{def:mixedhodge}
Given a $\Q$-vector space $H$, a mixed Hodge structure on $H$ consists of two filtrations: 
\begin{itemize}
\item[] $W_{\bullet} H$: The weight filtration, an increasing filtration on $H$
\item[] $F^{\bullet}H_{\C}$: The Hodge filtration, a decreasing filtration on $H_{\C}$
\end{itemize}
These must satisfy the condition that the vector space associated to the weight filtration, $\gr^W H$ (Cf. Definition \ref{def:assocgraded}), is a pure $\Q$-Hodge structure of weight $n$ with respect to the strict subquotient filtration induced by $F^{\bullet}H$. In other words,
\[
\gr_n^W H \cap F^nH_{\C} = \bigoplus_{r \geq n} W_rH \cap F^nH_{\C} 
\]
is a pure $\Q$-Hodge structure of weight $n$. A morphism of mixed Hodge structures $H$ and $H'$ is a $\Q$-linear map $f : H \to H'$ which respects the filtrations, i.e., such that $f(W_nH) \subset W_nH'$ and $f_{\C}(F^nH_{\C}) \subset F^nH'_{\C}$.
\end{defn}

\begin{cor}
The category $\mathrm{MHS}$ of $Q$-vectors spaces with mixed Hodge structures is a tensor abelian category.
\end{cor}

\begin{thm}[Deligne]
Let $X$ be an arbitrary complex algebraic variety. Then $H^n(X(\C), \Q)$ has a canonical mixed Hodge structure, and given a morphism $\phi : X \to Y$ of complex varieties, the induces morphism on cohomology $\phi : H^n(Y(\C)) \to H^(X(\C))$ is a morphism of mixed Hodge structures.
\end{thm}

\begin{exam}
How many mixed Hodge structures are there on $H \cong \Q^2$ fitting into the exact sequence in $\mathrm{MHS}$:
\[
0 \to \Q(1) \stackrel{\iota}{\to} H \stackrel{p}{\to} \Q \to 0 ? 
\]
The extension are classified by 
\[
\Ext^1_{\mathrm{MHS}}(\Q, \Q(1)) = \C^* / \mathrm{Tor~}\C^* \stackrel{\exp(2\pi i t)}{\isom} \C / \Q
\]
The weight filtrations can be easily calculated,
\[
\begin{array}{lrcccccccl}
-3 \qquad & 0 & \to & 0 & \to & W_{-3}H & \to & 0 & \to 0 & \quad W_{-3}H = 0 \\
-2 \qquad & 0 & \to & \Q(1) & \to & W_{-2}H & \to & 0 & \to 0 & \quad W_{-2}H = \Q(1) \\
-1 \qquad & 0 & \to & \Q(1) & \to & W_{-1}H & \to & 0 & \to 0 & \quad W_{-1}H = \Q(1) \\
0 \qquad & 0 & \to & \Q(1) & \to & W_0H = H & \to & \Q & \to & \quad W_0H = H \\
1 \qquad & 0 & \to & \Q(1) & \to & W_1H & \to & \Q & \to 0 & \quad W_1H = H
\end{array}
\]
as well as the Hodge filtration (note the reversed direction!),
\[
\begin{array}{lrcccccccl}
1 \qquad & 0 & \to & 0 & \to & F^{1}H_{\C} & \to & 0 & \to 0 & \quad F^{1}H_{\C} = 0 \\
0 \qquad & 0 & \to & \C & \to & F^{0}H_{\C} & \to & 0 & \to 0 & \quad F^{0}H_{\C} = \C \\
-1 \qquad & 0 & \to & \C & \to & F^{-1}H_{\C} & \to & 0 & \to 0 & \quad F^{-1}H_{\C} = \C \\
-2 \qquad & 0 & \to & \C & \to & F^{-2}H_{\C} = H_{\C} & \to & \C & \to 0 & \quad F^{-2}H_{\C} = H_{\C} \\
\end{array}
\]
This shows that the mixed Hodge structure on $H$ is determined by a 1-dimensional subspace $F^H_{\C} \subset H_{\C}$ such that
\[
\iota(\Q(1)_{\C}) \cap F = 0.
\]
Let $e \in \Q(1)$ and $f \in H$ be defined by $\langle e\rangle_{\Q} = \Q(1)$ and $\langle p(f) \rangle_{\Q} = \Q$. Then $H = \langle e, f \rangle_{\Q}$ and $F = \langle ae + bf \rangle_{\Q} = \langle ae + f \rangle_{\C}$ for $b \neq 0$ and $a \in \C / \Q$.
\end{exam}

\begin{exam}
Let $X = \Gm \setminus \{a'\}$ for some $a' \in \C^*$. Then $H^1(X) \isom \Q(-1) \oplus \Q(-1)$, $a' = \exp(a)$.
\end{exam}
\begin{exam}
$a' \neq 1$ $(a + a' = 1) = \Gm / \{ a'=1 \} = X$. Hence
\[
0 \to \Q \to \Q \oplus \Q \to H^1(\Gm, \{1, a'\}) \to H^1(\Gm) \cong \Q(-1) \to 0
\]
The extension corresponding to $a \in \C / \Q \cong \C^* / \mathrm{Tor~}\C^*$.
\[
0 \to \Q(1) \to H^1(\Gm, \{1, a'\})(1) \to \Q \to 0
\]
$H(i) := H \otimes \Q(i)$.
\end{exam}

\section{Algebraic de Rham cohomology}
Let $X$ be a smooth projective algebraic variety over $C$. Let $\Omega_X$ be the sheaf of algebraic forms on $X$
\begin{prop}[GAGA]
\[
H^p(X, \Omega_X^q) \cong H^p(X(\C), \Omega^q_{X, an})
\]
\end{prop}
Let
\[
\Omega_X^0 \stackrel{d}{\to} \Omega_X^1 \stackrel{d}{\to} \cdots \stackrel{d}{\to} \Omega_X^d
\]
be a complex of Zariski sheaves on $X$.
\[
H^n_{dR}(X) := H^n(X, \Omega_X^*) \stackrel{\alpha}{\isom} H^n_{dR}(X(\C),\C)
\]
There is a filtration on $H^n_{dR}(X(\C),\C)$ induced by the filtration $F^p \Omega_X^* = \bigoplus_{r \geq p} \Omega_X^r$.

Let $k$ be a subfield of $\C$ and $X$ a $k$-variety. Then
\[
H^n_{dR}(X) \otimes_k \C \cong H^n_{dR}(X_{\C}) \cong H^n_{dR}(X(\C),\C)
\]

\begin{rem}
The same is true for any smooth algebraic variety $X$ over $k$, i.e., there exist filtrations $W_{\bullet}H^{\bullet}_{dR}(X)$ and $F^{\bullet}H^n_{dR}(X)_{\C}$.
\end{rem}

\begin{defn}[Period isomorphism]\label{def:periodisom}
Given a field $k \inj \C$ and a smooth $k$-variety $X$, the period isomorphism is
\[
H^n_B(X, \Q)_{\C} \isom H^n_{dR}(X) \otimes_k \C
\]
After fixing $\Q$-basis $\delta_1, \ldots, \delta_m$ of $H_n(X, \Q)$ and $k$-basis $\omega_1, \ldots, \omega_m$ of $H^n(X)_{\C}$, the isomorphism is specified by a choice of matrix $[M] \in GL_m(k) \\ GL_m(\C) / GL_m(\Q)$.
\end{defn}
\begin{exam}
\[
\Q(-1) = H^(\Pspace^1) \in \mathrm{Q}, \qquad \Pspace^1 / \Q \subset \C, \qquad \C^* / \Q^* \ni [2\pi i]
\]
\end{exam}
\begin{exam}
Let $X = \{y^2 = x^3 + ax + b\}$ be a (chart on a) torus. Then $H^1_{dR}(X) \supset F^1H^1_{dR}(X) = H^0(X, \Omega_X^1) = \langle \frac{dx}{y} \rangle_k$. Let $H^B_1(X, \Q) = \langle \sigma_1, \sigma_2 \rangle$ be a basis. Then the period matrix is
\[
\left( \begin{array}{cc}
\int_{\sigma_1} \frac{dx}{y} & \int_{\sigma_1} \frac{xdx}{y} \\
\int_{\sigma_2} \frac{dx}{y} & \int_{\sigma_2} \frac{xdx}{y}
\end{array}
\right)
\]
\end{exam}

\section{Hopf algebras}
In the section, let $G$ be a linear algebraic group and $A = \cO(G)$ its ring of regular functions. The multiplication $G \times G \to G$ induces a comultiplication $\Delta : A \to A \otimes_k A$. Similarly, $e \in G$ induces $\epsilon : A \to k$ and $g^{-1}$ induces $c : A \to A$.

\begin{defn}[Category of representations]
Given a group $G$, its category of \emph{finite dimensional} representations with $G$-equivariant linear maps between them will be denoted $\mathrm{Rep}(G)$.
\end{defn}
\begin{defn}[Category of comodules]
Given an commutative, associative, unital $k$-algebra $A$, its category comodules $\mathrm{Comod}(A)$ is the category of diagrams
\begin{eqnarray*}
\mathrm{Comod}(A) & = & \{ \xymatrix{
V \ar[r]^{\Delta} & V \otimes_k A \ar@<0.3em>[r]^{\Delta \otimes 1} \ar@<-0.3em>[r]_{1 \otimes \Delta} & V \otimes_k A \otimes_k A
} \} \\
& = & \{ \textrm{matrices with coefficients in $A$} \}
\end{eqnarray*}
\end{defn}
There is an equivalence of categories
\[
\mathrm{Rep}(G) \cong \mathrm{Comod}(A)
\]
\begin{exam}
If $G = \Gm$, then $G$ maps under the above equivalence to $A = k[t,t^{-1}]$ with $\Delta(t) = t \otimes t$.
\end{exam}
\begin{exam}
If $G = \Ga$, then $G$ maps under the above equivalence to $A = k[t]$ with $\Delta(t) = 1 \otimes t + t \otimes 1$.
\end{exam}
\begin{defn}[Lie algebra]
Given an algebraic group $G$, its Lie algebra is
\[
\mathrm{Lie}(G) = \fm / \fm^2, \quad \fm := \ker(\epsilon)
\]
where $\epsilon : \cO(G) \to k$, as above.
\end{defn}
\begin{rem}
There are such things as filtered and graded Hopf algebras. They are as you might guess.
\end{rem}

\section{dg-algebras}
\begin{defn}[dg-algebra]\label{def:dgalgebra}
Given a field $k$, a dg-algebra over $k$ is a complex $A$ of $k$-vector spaces with an associative product $\cdot : A \otimes_k \otimes \to A$ compatible with the differential of the complex by the Leibniz rule,
\[
d(a \cdot b) = da \cdot b + (-1)^{\deg(a)} a \cdot db.
\]
\end{defn}
\begin{defn}[Commutative dg-algebra]
A dg-algebra $(A, d)$ is commutative if
\[
a \cdot b = (-1)^{\deg a \deg b} b \cdot a
\]
\end{defn}

\begin{rem}
Given a dg-algebra $A$, its cohomology $H(A)$ is a graded $H^0(A)$-algebra.
\end{rem}

\begin{defn}[dg-Hopf algebra]
A dg-Hopf algebra is a commutative dg-algebra equipped with a coproduct $\Delta : A \to A \otimes_k A$, a counit $\epsilon : A \to k$ and coinverse $s : A \to A$ satisfying the usual commutative diagrams.
\end{defn}

\begin{rem}
Given a dg-Hopf algebra, its zeroth cohomology $H^0(A)$ is a Hopf algebra.
\end{rem}

%\chapter{Introduction}

September 2nd, 2012.

\noindent Fix an embedding $k \inj \C$.

There are adjoint pairs
\[
\xymatrix{
DA(k, \Lambda) \ar@<0.3em>[r]^{Bt_i^{\times}} & D(\Lambda) \ar@<0.3em>[l]^{Bt_{i, \times}} 
} \qquad
\xymatrix{
DA^{\mathrm{eff}}(k, \Lambda) \ar@<0.3em>[r]^{Bt_i^{\times}} & D(\Lambda) \ar@<0.3em>[l]^{Bt_{i, \times}} 
}
\]

Weak Tannaka duality for $Bt_i^{\times}$ and $Bt_{i, \times}$.

There is something called motivic homology $H_{mot}(k, \sigma, \Lambda)$ which forms a Hopf algebra. It also has an effective version $H^{\mathrm{eff}}_{mot}(k, \sigma, \Lambda)$. These are related by

\TODO

\chapter{Introduction to Multiple Zeta Values}
%\addcontentsline{toc}{chapter}{***Introduction to Multiple Zeta Values by Roland Paulin and Mario Huicochea}

Roland Paulin and Mario Huicochea on the September 2nd, 2012.

\medskip
\medskip

\noindent In this talk we define and give some basic properties of the multiple zeta values. In the first part we define give some motivation to  study  the multiple zeta values; In the second part we talk about the $\Q$-vector space generated by the multiple zeta values; In the third part,  we define the stuffle product; In the fourth part we define the shuffle product and in the last part we give some applications of the stuffle product, shuffle product and the regularized relations.

\section{Multiple Zeta Values}

Euler studied the Riemann Zeta function
\[
\zeta(s) = \sum_{n=1}^{\infty} \frac{1}{n^s}
\]
and he obtained a series of results:
\begin{itemize}
\item $\zeta(2) = \frac{\pi^2}{6}$
\item $\zeta(2n) = \frac{-(2\pi i)^{2n}}{4n} B_{2n}$ for all $n \in \mathbf{N}$ where the $B_k \in \mathbb{Q}$ are Bernoulli numbers defined by the equation $\frac{t}{e^t - 1} = 1 - \frac{t}{2} + \sum_{n=2}^{\infty} B_n \frac{t^n}{n!}$
\item $\mathbf{Q}[\zeta(2), \zeta(4), \ldots] = \mathbf{Q}[\pi^2]$.
\end{itemize}

However, much less is known about $\zeta(2n+1)$ for $n \in \mathbf{N}$. It is known, for example that:

\begin{itemize}
\item $\zeta(3)$ is irrational. (Ap\'ery 1978)
\item $\{ \zeta(2n+1) \}_{n \in \mathbf{N}}$ has infinitely many irrational values. (Rivoal 2000)
\end{itemize}

A famous conjecture in this direction is the following.

\begin{conj}
$\pi, \zeta(3), \zeta(5), \zeta(7), \ldots$ are algebraically independent over $\mathbf{Q}$.
\end{conj}


\begin{defn}[Zagier, Hoffman 1992]
Let $n_1, \ldots, n_k \in \mathbf{N}$ with $n_1 \geq 2$ and $\overline{n} = (n_1, \ldots, n_k)$. The real numbers
\[
\zeta(\overline{n}) = \sum_{i_1 > \cdots > i_k > 0} \frac{1}{i_1^{n_1} \cdots i_k^{n_k}}
\]
are known as the \emph{multiple zeta values}. The \emph{weight of $\zeta(\overline{n})$} is $|\overline{n}| = n_1 + \cdots + n_k$ and $k$ is the \emph{length}.
\end{defn}

\begin{defn}
A weight $\overline{n} = (n_1, \ldots, n_k)$ for $n_1, \ldots, n_k \in \mathbf{N}$ is \emph{admissible} if $n_1 \geq 2$. Let $\cW \subset \coprod_{r=1}^{\infty} \N^r$ denote the set of admissible weights.
\end{defn}

The multiple zeta values were discovered and studied by Euler for $k \leq 2$. Hoffman and Zagier defined the multiple zeta values for arbitrary $k \in \mathbf{N}$ independently in 1992.

\begin{defn}\label{def:mzvspaces}
Let
\begin{itemize}
\item $\cZ$ be the $\mathbf{Q}$-vector space generated by the MZV's.
\item $\cZ_n$ be the $\mathbf{Q}$-vector space generated by the MZV's of weight $n \geq 2$.
\item $\cF^n Z$ be the $\Q$-vector space generated by the MZV's of length $\leq k$.
\item $\cF^k Z_n$ be the $\Q$-vector space generated by the MZV's of weight $n$ and length $\leq k$ for $2 \leq k + 1 \leq n$.
\end{itemize}
\end{defn}

\begin{rem}
If $2 \leq k+1 \leq n$ then $\cF^k \cZ_n \subset \cF^k \cZ \cap \cZ_n$.
\end{rem}

\begin{conj}
$\cF^k \cZ_n = \cF^k \cZ \cap \cZ_n$.
\end{conj}

\begin{conj}
The weight defines a gradation $\cZ = \oplus_{n \geq 2} \cZ_n$. In particular $\cZ_2 \cap \cZ_3 = \{ 0 \}$ so $\zeta(3) / \pi^2 \notin \Q$.
\end{conj}

Let $d_0 = 1, d_1 = 0$ and $d_n = \dim_{\Q} \Z_n$ be the dimension of the nth graded part of $\cZ$ for all $n \in \mathbf{N}_{>2}$. 
\begin{itemize}
\item $d_2 = 1$ since $\cZ_2 = \Q \zeta(2)$.
\item $d_3 = 1$ since $\cZ_3 = \Q \zeta(2, 1) + \Q \zeta(3)$ and $\zeta(2, 1) = \zeta(3)$.
\item $d_4 = 1$ since $\cZ_4 = \Q \zeta(2, 1, 1) + \Q\zeta(2, 2) + \Q \zeta(3, 1) + \Q \zeta(4)$ and $\zeta(2, 1, 1) = \zeta(4), \zeta(2, 2) = \frac{3}{4} \zeta(4), \zeta(3, 1) = \zeta(4)/4$.
\end{itemize}

These are the unique known $d_n$. There are some upper bounds known for the rest. For example, $d_5$ is the dimension of a space generated by $\{ \zeta(3, 2), \zeta(2, 3) \}$ and $d_6$ by $\{ \zeta(2, 2, 2), \zeta(3, 3) \}$, so they are both at most 2.

A main result about the linear relations among multiple zeta values was proved by Goncharov and Terasoma.
\begin{thm}[Goncharov-Terasoma]
If $D_n \in \mathbf{N} \cup \{0\}$ for all $n \in \mathbf{N}$ such that $D_0 = D_2 = 1$, $D_1 = 0$ and $D_n = D_{n-2} + D_{n-3}$ for all $n \geq 3$ then $d_n \leq D_n$.
\end{thm}
The equality is still a conjecture.
\begin{conj}[Zagier 1992]
For all $n \geq 3, n \in \mathbf{N}$
\[
d_n = d_{n - 2} + d_{n -3}
\]
\end{conj}

The conjecture is proved for $n = 3,4$ and can be stated as
\[
\sum_{n=0}^{\infty} d_n x^n = \frac{1}{1 - x^2 - x^3}
\]
\begin{conj}[Hoffman 1997]
Let $n \in \mathbf{N}$ and $S_n = \{ (s_1, \ldots, s_n) \mid s_i \in \{2, 3\}, \sum_{j = 1}^k s_j = n \}$. The set $\{ \zeta(\overline{s})\}_{\overline{s} \in S_n}$ is a basis of $Z_n$.
\end{conj}

A main result about multiple zeta values is Brown's theorem:
\begin{thm}[Brown]
All multiple zeta values are linear combinations of $\zeta(\overline{n})$ with $n_i \in \{2,3\}$ for all $n_i$ in $\overline{n}$.
\end{thm}

\section{Words and Products of Words}

\begin{defn}[Words]\label{def:word}
Let $X$ be the set of words formed by the letters $\{X_0, X_1\}$. Let $1$ be the empty word $\varnothing \in X$, and let $Y_n := X_0^{n-1}X_1 \in X$ for all $n \in \N$. 
\end{defn}
\begin{defn}
Let $\fH = \Q\langle x_0, x_1 \rangle$ be the $\Q$-algebra of polynomials in the two non-commutative variables $\{ x_0, x_1 \}$ which is graded by the degree $(\deg x_0 = \deg x_1 = 1)$. Let $\fH_1$ be the $\Q$-vector space of polynomials generated by $\{1\} \cup \{Y_n\}_{n \in \N}$ and $\fH_0$ be the $\Q$-vector space of polynomials generated by $\{1\} \cup \{Y_n\}_{n \geq 2}$.
\end{defn}
\begin{defn}[Word associated to a vector]\label{def:assocword}
We associate to an admissible vector a word
\[
w : \begin{array}{rcl}
\cW & \to & X^k \\
(n_1, \ldots, n_k) = \overline{n} & \mapsto & Y_{\overline{n}} = (Y_{n_1}, \ldots, Y_{n_k})
\end{array}
\]
\end{defn}
\begin{defn}
We define a function from words associated to admissible vectors, $w(\cW)$, to multiple zeta values by
\[
\widehat{\zeta} : \begin{array}{rcl}
w(\cW) & \to & \cZ \\
Y_{\overline{n}} & \mapsto & \zeta(\overline{n})
\end{array}
\]
and let $|Y_{\overline{n}}| = |\overline{n}|$. The function $\widehat{\zeta}$ can be restricted to $\overline{n}$ with $n_i > 2$ for all $i$ and then extended by $\Q$-linearity to define a function
\[
\widehat{\zeta} : \fH_0 \to \cZ.
\]
\end{defn}
\begin{rem}
$\fH_0 \subset \fH_1 \subset \fH$
\end{rem}

\subsection{The stuffle product}

\begin{defn}[Stuffle product]
The stuffle product (also known as the harmonic product) $* : \fH \times \fH \to \fH$ on non-commutative polynomials is defined by extending by linearity from three rules:
\begin{enumerate}
\item $1 * w = w * 1 = w$ for all $w \in X$.
\item $Y_n w_1 * Y_m w_2 = Y_n(w_1 + Y_m w_2) + Y_m(Y_n w_1 + Y_2) + Y_{n+m}(w_1*w_2)$ for all $w_1, w_2 \in X$ and all $m, n \in \N$.
\item $X_0^n + W = W + x_0^n = W x_0^n$ for all $n \in \N, w \in X$.
\end{enumerate}
\end{defn}

\begin{rem}
The stuffle product is commutative and associative. Together with addition it forms a graded $\Q$-algebra $(\fH, +, *)$ with grading $\fH^n = \Q \langle \textrm{words of length $n$} \rangle$. $\fH_0$ and $\fH_1$ are subalgebras.
\end{rem}
\begin{exam}\label{ex:one}
\begin{eqnarray*}
Y_2 * Y_2 = Y_2 1 * Y_2 1 & = & Y_2(1 * Y_2 1) + Y_2(Y_2 1 * 1) + Y_4(1 * 1) \\
& = & Y_2 Y_2 + Y_2 Y_2 + Y_4 \\
& = & 2 Y_2 Y_2 + Y_4
\end{eqnarray*}
\end{exam}
\begin{exam}\label{ex:two}
\begin{eqnarray*}
Y_2 * Y_3 Y_2 = Y_2 1 * Y_3 Y_2 & = & Y_2(1 * Y_3 Y_2) + Y_3 (Y_2 * Y_2) + Y_5(1 * Y_2) \\
& = & Y_2 Y_3 Y_2 + Y_3 (2 Y_2 Y_2 + Y_4) + Y_5 Y_2 \\
& = & Y_2 Y_3 Y_2 + 2 Y_3 Y_2 Y_2 + Y_3 Y_4 + Y_5 Y_2.
\end{eqnarray*}
\end{exam}
\begin{prop}[Nielsen reflexion Formula]
$Y_n + Y_m = Y_n Y_m + Y_m Y_n + Y_{n + m}$ for all $m, n \geq 2$.
\end{prop}
\begin{thm}
The function $\widehat{\zeta} : (\fH_0, *) \to (\R, \cdot)$ is a homomorphism for the stuffle product $*$, i.e., for all $Z_1, Z_2 \in \fH_0$,
\[
\widehat{\zeta}(Z_1 * Z_2) = \widehat{\zeta}(Z_1) \widehat{\zeta}(Z_2).
\]
\end{thm}

So, for example, Example \ref{ex:one} implies 
\[
2 \zeta(2,2) + \zeta(4) = \widehat{\zeta}(Y_2*Y_2) = \zeta(2)^2
\]
and Example \ref{ex:two} implies 
\[
\zeta(2,3,2) + 2\zeta(3,2,2) + \zeta(3,4) + \zeta(5,2) = \widehat{\zeta}(Y_2 * Y_3 Y_2) = \zeta(2)\zeta(3,2).
\]

\subsection{The shuffle product}

\begin{defn}[Shuffle product]
Let $S_n$ denote the symmetric group of permutations on $n$ letters. Let subsets of $S_n$ be defined by
\[
S_{p,q} = \{ \sigma \in S_{p+q} \mid \sigma(1) < \cdots < \sigma(p) \textrm{~and~} \sigma(p+1) < \cdots < \sigma(p+q) \} \subset S_{p+q}.
\]
These are permutations which preserve the ordering of the first $p$ and the last $q$ letters while allowing shuffling between them. The shuffle product of $p+q$ words $u_1, \ldots, u_{p+q}$ is defined by
\[
\Sha : \begin{array}{rcl}
\fH \times \fH & \to & \fH \\
(u_1 \cdots u_p, u_{p+1} \cdots u_{p+q}) & \mapsto & \sum_{\sigma \in S_{p+q}} u_{\sigma^{-1}(1)} \cdots u_{\sigma^{-1}(p+q)}
\end{array}
\]
\end{defn}

\begin{exam}
For example, consider the shuffle product of two monomials,
\[
\frac{3}{2}Y_2 ~\Sha~ \frac{-1}{6}Y_2 = \frac{3}{2}X_0X_1 ~\Sha~ \frac{-1}{6}X_0X_1 = \frac{-1}{4}\left(4 X_0^2 X_1^2 + 2 X_0 X_1 X_0 X_1\right) = - Y_3 Y_1 - \frac{1}{2} Y_2 Y_2.
\]
\end{exam}

\begin{thm}
The function $\widehat{\zeta} : (\fH_0, \Sha) \to (\R, \cdot)$ is also a homomorphism for \Sha, i.e.,
\[
\widehat{\zeta}(Z_1 ~\Sha~ Z_2) = \widehat{\zeta}(Z_1)\widehat{\zeta}(Z_2)
\]
\end{thm}

\subsection{The regularized relation}

\begin{cor}[Regularized relation]
For all $w_1, w_2 \in \fH_0$,
\[
\widehat{\zeta}(w_1 ~\Sha~ w_2 - w_1 * w_2) = 0.
\]
\end{cor}

Let the shuffle algebra, $\mathfrak{H}_{\Sha}$, be the commutative $\mathbb{Q}$-algebra $(\mathfrak{H}, +, \Sha)$; then $\mathfrak{H}^0_{\Sha} \subset \mathfrak{H}^1_{\Sha} \subseteq \mathfrak{H}_{\Sha}$ are subalgebras. Similarly, let the stuffle algebra, $\mathfrak{H}_*$, be the commutative $\mathbb{Q}$-algebra $(\mathfrak{H}, +, *)$; then $\mathfrak{H}^0_{*} \subset \mathfrak{H}^1_{*} \subseteq \mathfrak{H}_{*}$ are subalgebras. The subalgebras are related by
\[
\mathfrak{H}^1_{\Sha} = \mathfrak{H}^0_{\Sha}[x_1] \qquad \mathfrak{H}_{\Sha} = \mathfrak{H}^1_{\Sha}[x_0],
\] 
and therefore $\mathfrak{H}_{\Sha} = \mathfrak{H}^0_{\Sha}[x_0,x_1]$. Similar relations hold for the stuffle algebras.
\[
\mathfrak{H}^1_* = \mathfrak{H}^0_* [x_1] \qquad \mathfrak{H}_* = \mathfrak{H}^1_* [x_0] = \mathfrak{H}^0_* [x_0,x_1].
\]

\begin{exam}
The regularized relation implies that $\widehat{\zeta}(Y_1 ~\Sha~ Y_2 - Y_1 * Y_2) = 0$, and calculation shows that
\[
Y_1 ~\Sha~ Y_2 = Y_1 Y_2 + 2Y_2Y_1 \qquad Y_1*Y_2 = Y_1Y_2 + Y_2Y_1 + Y_3.
\]
Since the difference $Y_2Y_1-Y_3 \in \fH_0$, this implies a relation for multiple zeta values, namely that $\zeta(2,1)=\zeta(3)$.
\end{exam}

\begin{conj}
The stuffle, shuffle and regularized relations generate all relations among the multiple zeta values. (A strong version restricts the regularized relations to those with $w_1 = X_1$.)
\end{conj}

\begin{thm}[Sum Theorem]
For all $p \geq 2$ and $1 \leq \ell \leq p-1$,
\[
\zeta(p) = \sum_{(n_1, \ldots, n_{\ell}) \in \cW} \zeta(n_1, \ldots, n_{\ell})
\]
\end{thm}

\begin{xca}
Derive from the regularized relation the following formula:
\[
\zeta(s) = \sum_{\begin{subarray}{c}
i+j=s \\
i\geq2
\end{subarray}}
\zeta(i,j)
\]
\end{xca}




\input{talk-2.tex}

\chapter*{The pro-unipotent completion by Alberto Vezzani}
%\addcontentsline{toc}{chapter}{***The pro-unipotent completion by Alberto Vezzani}

Alberto Vezzani on September 3rd, 2012.

\medskip
\medskip

The aim of these notes is to give an overview of Quillen's construction of the pro-unipotent completion of an abstract group (or a Lie algebra). Some consequences of the formulas and special cases are explained in more detail.

\section{Introduction}

We start by presenting the work of Quillen \cite[Appendix A]{quillen-r}, and translating it into the setting of algebraic groups, following the approach of \cite{em}. We also follow Cartier \cite{cartier-ha} for specific facts on Hopf algebras.
From now on, we work over the base field $\Q$. %We will try to underline the points in which this hypothesis is needed.

\begin{defn}
Given an abstract group $\Gamma$ [resp. a Lie algebra $\fg$], the pro-unipotent completion $\Gamma^{un}$ [resp. $\fg^{un}$] is the universal pro-unipotent algebraic group $G$ endowed with a map $\Gamma\to G(\Q)$ [resp. $\fg \to \mathrm{Lie~}G$]. 
\end{defn}

Let us focus on the case of groups. The meaning of the definition is that there is a map $u:\Gamma\to\Gamma^{un}(\Q)$ such that for any map $f:\Gamma\to G(\Q)$ to the $\Q$-points of a pro-unipotent algebraic group $G$, there exists a unique map $\phi:\Gamma^{un}\to G$ such that $f=\phi(\Q)\circ u$. In other words, we are looking for a left adjoint to the functor $G\mapsto G(\Q)$ defined from pro-unipotent algebraic groups to abstract groups. Sadly enough, we anticipate that we will need to restrict to a subcategory of abstract groups in order to find such a functor.

The category of pro-unipotent algebraic groups is a full subcategory of the category of pro-affine algebraic algebraic groups (the category of formal filtered limits of affine algebraic groups over quotients). This category is clearly equivalent to the opposite category of Hopf algebras (not necessarily finitely presented). What we need to do is therefore to associate to an abstract group a particular commutative Hopf algebra over $\Q$. There are some well-known examples of adjoint pairs which are close to reaching this aim.

\begin{prop}
There is an adjoint pair of functors
\[
\adj{\Q[\cdot]}{\Gps}{\Q\lAlg}{(\cdot)^\times}
\]
between the category of abstract groups and (not necessarily commutative) $\Q$-algebras.
\end{prop}

Any group algebra $\Q[\Gamma]$ can be endowed with the structure of a Hopf algebra with respect to the maps
\[
\Delta: g\mapsto g\otimes g\qquad
S: g\mapsto g^{-1}\qquad
\epsilon: g\mapsto1
\]
for all $g\in\Gamma$. Therefore the functor $\Q[\cdot]$ factors over the category of Hopf algebras. Also this new functor has an adjoint:
\begin{prop}
There is an adjoint pair of functors
\[
\adj{\Q[\cdot]}{\Gps}{\HA}{\cG}
\]
between the category of abstract groups and (not necessarily commutative) $\Q$-Hopf algebras where $\cG$ associates to a Hopf algebra $R$ the set of group-like elements:
\[
\cG R \colonequals\{x\in R^\times: \Delta x=x\otimes x\}
\]
endowed with the product inherited from $R$.
\end{prop}

\begin{proof}
This comes from the previous proposition. Indeed, given a map $\Q[\Gamma]\to R$ induced by $f:\Gamma\to R^\times$, the diagram 
$$\xymatrix{
\Q[\Gamma]\ar[r]\ar[d]&\Q[\Gamma]\otimes\Q[\Gamma]\ar[d]\\
R\ar[r]&R\otimes R
}$$
commutes if and only if all images of the elements of $G$ are group-like. Moreover, any group-like element $x$ satisfies $\epsilon(x)=1$, hence also the augmentation is preserved.
\end{proof}

Similarly for Lie algebras:
\begin{prop}\begin{enumerate}[(i)]
\item There is an adjoint pair of functors
\[
\adj{\cU}{\LA}{\Q\lAlg}{for}
\]
between the category of Lie algebras and (not necessarily commutative) $\Q$-algebras. The functor $for$ sends a $\Q$-algebra $R$ to the Lie algebra structure over $R$ induced by commutators.
\item The left adjoint factors over the category of Hopf algebras by endowing the universal enveloping algebra $\cU\fg$ of a Lie algebra $\fg$ with the structure of a Hopf algebra with respect to the maps
\[
\Delta: x\mapsto x\otimes1+1\otimes x\qquad
S: x\mapsto -x\qquad
\epsilon: x\mapsto0
\]
for all $x\in\fg$.
\item There is an adjoint pair of functors
\[
\adj{\cU}{\LA}{\HA}{\cP}
\]
between the category of Lie algebras and (not necessarily commutative) $\Q$-Hopf algebras where $\cP$ associates to a Hopf algebra $R$ the set of primitive elements:
\[
\cP R \colonequals\{x\in R: \Delta x=x\otimes 1+1\otimes x\}
\]
endowed with the Lie bracket induced by commutators.
\end{enumerate}
\end{prop}

Note that $\Q[\Gamma]$ and $\cU\fg$ are cocommutative, but not necessarily commutative (this happens iff $\Gamma$ or $\fg$ is abelian). Since our initial aim was to associate to a group (or to a Lie algebra) a commutative, but not necessarily cocommutative Hopf algebra, the natural idea is now to ``take duals''. Taking duals of vector spaces is a delicate operation whenever the dimension is not finite. Hence, we will need to restrict to a particular case where the situation is self-reflexive as in the finite-dimensional case.

\begin{defn}
A topological vector space $V$ is linearly compact if it is homeomorphic to $\varprojlim V/V_i$, where $V/V_i$ are discrete and finite dimensional, and the maps in the diagram are quotients.  
\end{defn}

We will denote by $(\cdot)^\vee$ the dual space and by $(\cdot)^*$ the topological dual.

\begin{exam}
If $V$ is a discrete vector space, we will always enodow its dual $V^\vee$ with the linearly compact topology $\varprojlim W_i^\vee$ by letting $W_i$ vary among the subvector spaces of $V$ which are finite dimensional.
\end{exam}

\begin{prop}
\begin{enumerate}
	\item If $V$ is discrete [resp. linearly compact], then $(V^\vee)^*\cong V$ [resp. $(V^*)^\vee\cong V$].
	\item If $V$ is discrete [resp. linearly compact], then $(V\otimes V)^\vee\cong V^\vee\hat{\otimes}V^\vee$ [resp. $(V\hat{\otimes}V)^*\cong V^*\otimes V^*$].
\end{enumerate}
\end{prop}

In particular, duality defines an equivalence of categories between commutative Hopf algebra and the category of linearly compact Hopf algebras.

Our attempt is now to use these dualities in order to obtain a commutative and cocommutative Hopf algebra out of $\Q[\Gamma]$ or $\cU\fg$. By what just stated, we need to get a complete topological Hopf algebra. Any Hopf algebra $R$ is augmented by the counit $\epsilon$. Let $I$ denote the augmentation ideal. We can endow $R$ with the $I$-adic topology, and complete it with respect to it.

\begin{defn}
A complete Hopf algebra is a complete topological augmented algebra $\epsilon: R\to R/I\cong\Q$, homeomorphic to $\varprojlim R/I^k$ and endowed with a map $\Delta: R\to R\hat{\otimes}R$ that fit in the usual diagrams of Hopf algebras.
% and which issuch that $R$ is homeomorphic to $\varprojlim R/I^k$, where $I$ is the augmentation ideal.
We denote the category of complete Hopf algebras by $\CHA$.
\end{defn}

We remark that our definition differs slightly from the one of \cite{quillen-r} since Quillen introduces also the choice of a filtration.

\begin{exam}
If $R$ is a Hopf algebra, then its $I$-adic completion $\hat{R}$ is a complete Hopf algebra. In particular, $\Gamma\mapsto \widehat{\Q[\Gamma]}$ and $\fg\mapsto\widehat{\cU\fg}$ define functors to the category $\CHA$.
\end{exam}

The following proposition is a formal consequence of the previous ones.

\begin{prop}
There are adjoint pairs of functors
\[
\adj{\hat{\Q}[\cdot]}{\Gps}{\CHA}{\cG}
\]
\[
\adj{\hat{\cU}}{\LA}{\CHA}{\cP}
\]
where $\cG$ and $\cP$ are defined like before.
\end{prop}

We can now isolate in $\CHA$ the full subcategory $\cat$ of those algebras $R$ which are also linearly compact. Since $R\cong\varprojlim R/I^k$, this condition is equivalent to asking that $R/I^k$ is finite dimensional for all $k$. Since this is obviously true for $k=1$, by induction we conclude that this is equivalent to the finite dimensionality of all $I^k/I^{k+1}$. Multiplication defines a surjection $(I/I^2)^{\otimes k}\to I^k/I^{k+1}$, and therefore this is equivalent to imposing $I/I^2$ finite dimensional.

\begin{exam}
\begin{enumerate}
	\item Let $\Gamma$ be an abstract group. Then
	\[
	I_{\hat{\Q}[\Gamma]}/I_{\hat{\Q}[\Gamma]}^2\cong I_{{\Q}[\Gamma]}/I_{{\Q}[\Gamma]}^2\cong \Gamma^{ab}\otimes_Z\Q
	\]
	where $\Gamma^{ab}$ is the abelianization of $\Gamma$, and where the last isomorphism is induced by $(g-e)\mapsto g$. In particular, if $\Gamma$ is such that $\Gamma^{ab}\otimes_{\Z} \Q$ has finite rank, then $\hat{\Q}[\Gamma]$ is linearly compact. We denote by $\widetilde{\Gps}$ the full subcategory of $\Gps$ of objects satisfying this property.
	\item 
	Let $\fg$ be a Lie algebra. Then
	\[
	I_{\hat{\cU}\fg}/I_{\hat{\cU}\fg}^2\cong I_{{\cU}\fg}/I_{{\cU}\fg}^2\cong \fg/[\fg,\fg]
	\]
	where the last isomorphism is induced by $x\mapsto x$. In particular, if $\fg$ is such that $\fg/[\fg,\fg]$ has finite rank, then $\hat{\cU}\fg$ is linearly compact. We denote by $\widetilde{\LA}$ the full subcategory of $\LA$ of objects satisfying this property.
\end{enumerate}
\end{exam}

%\begin{cor}
%There are adjoint pairs of functors
%\[
%\adj{\hat{\Q}[\cdot]}{\widetilde{\Gps}}{\cat}{\cG}
%\]
%\[
%\adj{\hat{\cU}}{\widetilde{\LA}}{\cat}{\cP}
%\]
%where $\cG$ and $\cP$ are defined like before.
%\end{cor}

%Consider now the category $\cat$ of linearly compact Hopf algebras which are also $I$-adically complete. 
%
%\begin{prop}
%There are adjoint pairs of functors
%\[
%\adj{\hat{\Q}[\cdot]}{\widetilde{\Gps}}{\cat}{\cG}
%\]
%\[
%\adj{\hat{\cU}}{\widetilde{\LA}}{\cat}{\cP}
%\]
%where $\cG$ and $\cP$ are defined like before.
%\end{prop}

By duality, the category  $\cat$ is equivalent to a full subcategory of $\HA^{op}$, and hence to a subcategory of pro-affine algebraic groups.

\begin{prop}
Let $G=\spec R$ be a pro-affine algebraic group. Then $R^\vee\in\cat$ if and only if $\cP R$ is finite dimensional and the ``conilpotency filtration''
\begin{equation}\label{cf}
0\subset C_0 \colonequals\Ann_R I\subset\ldots\subset C_k \colonequals\Ann_R I^{k+1}\subset\ldots
\end{equation}
is exhaustive, i.e. if $R=\bigcup C_i$, where $I$ is the augmentation ideal of $R^\vee$.
\end{prop}

\begin{proof}
We know that $R^\vee$ lies in $\cat$ if $I/I^2$ is finite dimensional, and if $R^\vee\cong\varprojlim R^\vee/I^k$. The dual space $(R^\vee/I^k)^*$ coincides with $C_k$. In particular, $C_0=\Q$ and $C_1=\Q^{op}lus\cP R$. Indeed, an element $x$ of $R$ lies in $\Ann I^2$ if and only if $\Delta x=y\otimes 1+1\otimes z$, and by using the axioms of Hopf algebra, this turns out to be equivalent to $\Delta x=x\otimes 1+1\otimes x$ if $\epsilon(x)=0$.

We then conclude that $I/I^2$ is finite dimensional if and only if $(I/I^2)^*=(\ker (R/I^2\to R/I))^*=C_1/C_0=\cP R$ is finite dimensional, and that $R^\vee\cong\varprojlim R^\vee/I^k$ if and only if $R=\varinjlim(R^\vee/I^k)^*=\varinjlim C_k$.
\end{proof}
%, which we denote with $\pUAG$ and whose objects are called pro-unipotent algebraic groups. 

We remark that the conilpotency filtration $\{C_i\}$ just defined coincides with the one of Cartier \cite[3.8 (A)]{cartier-ha}. This is part of the following proposition, whose proof comes by induction from the previous one.
\begin{prop}
Let $G=\spec R$ be a pro-affine algebraic group and let $\bar{R}$ be its augmentation ideal.
The elements of the filtration \eqref{cf} can be defined equivalently in the following ways:
\begin{enumerate}
	\item $C_i=\Q^{op}lus\ker\bar{\Delta}_n$, where $\bar{\Delta}: \bar{R}\to \bar{R}\otimes \bar{R}$ maps  $x$ to $\Delta x-x\otimes1-1\otimes x$ and $\bar{\Delta}_n: \bar{R}\to \bar{R}^{\otimes n}$ maps $x$ to $(\bar{\Delta}\otimes\id\otimes\ldots\otimes\id)(\bar{\Delta}_nx)$.
	\item $C_{i+1}/C_i$ is the trivial subrepresentation of $G$ inside $R/C_i$.
\end{enumerate}
\end{prop}

\begin{defn}
A pro-unipotent algebraic group is a pro-affine algebraic group $\spec R$ such that the conilpotency filtration \eqref{cf} is exhaustive.
%$R^\vee$ is $I$-adically complete, where $I$ is the augmentation ideal with respect to the complete topological Hopf algebra structure induced by duality. 
A unipotent algebraic group is a pro-unipotent algebraic group $\spec R$ such that $R$ is finitely presented. The category defined by [pro-]unipotent algebraic groups will be denoted with $\UAG$ [resp. $\pUAG$].
\end{defn} 

In particular, a unipotent algebraic group $G$ such that the Lie algebra $\cP\cO(G)$ is finite dimensional defines an object $\cO(G)^\vee$ of $\cat$.

Our definition is different from the ``standard'' one. We now prove the equivalence of the two notions. Recall that $\mathbf{UT}_n$ is the subgroup of $\GL_n$ defined by upper-triangular matrices, which have $1$'s on the main diagonal. 

\begin{prop}
Let $G$ be a pro-affine algebraic group. The following are equivalent:
\begin{enumerate}[(i)]
    \item The group $G$ is pro-unipotent.
    \item For every non-zero representation $V$ of $G$, there exists a non-zero vector $v\in V$ such that $G\cdot v=v$.
\end{enumerate}
In case $G$ is an algebraic group, the previous conditions are equivalent to:
\begin{enumerate}[(iii)]
    \item $G$ is isomorphic to a subgroup of $\mathbf{UT}_n$ for some $n$.
\end{enumerate}
\end{prop}

\begin{proof}
Let $G=\spec R$. Suppose $(i)$ is satisfied. Then any representation $\rho: V\to V\otimes R$ admits an exhaustive filtration $\{V_k\}$ where $V_k\colonequals\{v\in V\otimes C_k\}$. In particular, $V_0$ is a trivial subrepresentation since if $v\in V_0$, then $\rho v=v\otimes1$. We now prove $(ii)$ by showing that $V_k=0$ implies $V_{k+1}=0$.

It can be explicitly seen that $\Delta C_i\subset\sum_{a+b=i} C_a\otimes C_{b}$. Therefore if $x\in V_{k+1}$, then $(1\otimes\Delta)(\rho x)$ lies in $\sum_{a+b=k+1} V\otimes C_{a}\otimes C_{b}$. Since $a$ and $b$ can't be both bigger than $k$, if follows that $V_{k+1}$ is mapped to $0$ via the composite map
\[
V\to V\otimes R\stackrel{1\otimes\Delta}{\rightarrow}V\otimes R\otimes R\stackrel{\pi}{\rightarrow}V\otimes R/C_k\otimes R/C_k
\]
On the other hand, the previous map coincides (by the axioms of comodules) with
\[
V\to V\otimes R\stackrel{\rho\otimes1}{\rightarrow}V\otimes R\otimes R\stackrel{\pi}{\rightarrow}V\otimes R/C_k\otimes R/C_k
\]
which is an injection since $V_k=0$. Viceversa, if any representation $V$ has a non-zero trivial subrepresentation, by induction one can define an ascending filtration $\{V_i\}$ such that $V_{i+1}/V_{i}$ is the trivial subrepresentation of $V/V_i$. Since any element of $V$ generates a finite dimensional subrepresentation, it follows that this filtration is exhaustive. The conilpotency filtration corresponds to the filtration associated to the representation defined on $R$ itself. This proves $(i)\Leftrightarrow(ii)$.

If $V$ is finite dimensional, then (by induction on its dimension, since $V_0\neq0$) it is an extension of trivial representations. It follows in particular that, with respect to a suitable basis, $\rho: G\to\GL_V$ factors over $\mathbf{UT}_n$. If $G$ is an algebraic group, one can apply this fact to a faithful finite dimensional representation to prove $(iii)$.
\end{proof}

Being pro-unipotent is closed under quotients (using the condition $(ii)$ for example). Hence, if $\spec R$ is a pro-unipotent algebraic group, then any sub-Hopf algebra $R'$ of $R$ defines a pro-unipotent algebraic group. It follows that the category of $\pUAG$ coincides with the pro-objects of $\UAG$ and $\cat$ is a subcategory of it.%. Since any Hopf algebra is the union of finitely presented subalgebras, $\pUAG$ is also equivalent to pro-objects of $\cat$ since they both represent the category of pro-unipotent algebraic groups $G$ with a finite dimensional Hopf algebra $\cO(G)$.
%SECONDO ME SE G E AG, ALLORA PO(G) E' AUTOMATIC FIN DIM. 

We remark that our definition is slightly different from the one of Cartier \cite[end of p. 53]{cartier-ha}, since we do not impose that $R$ has countable dimension. 

\begin{exam}
\begin{enumerate}
	\item Let $\Gamma$ be an abstract group such that $\Gamma^{ab} \otimes_{\Z} \Q$ has finite rank. Then $\hat{\Q}[\Gamma]$ endowed with the $I$-adic topology is linearly compact, since it is homeomorphic to $\varprojlim \hat{\Q}[\Gamma]/I^k$ and all $I_k$ have finite codimension. In particular, $\spec(\hat{\Q}[\Gamma]^*)$ is pro-unipotent.
	\item Similarly, if $\fg$ is a Lie algebra such that $\fg/[\fg,\fg]$ has finite rank, then $\spec(\hat{\cU}\fg^*)$ is pro-unipotent.
	\item Consider $G=\mathbf{G}_a$. The Hopf algebra is $R=\Q[t]$, its dual vector space is ${\bf{pro}}d\Q\epsilon_k$ where $\epsilon_k(t^i)=\delta_{k,i}$. By duality, the augmentation ideal is $I=\ker((\Q\to R)^\vee)=\{\phi: R\to\Q: \phi(1)=0\}=\left\langle \epsilon_k\right\rangle_{k>0}$. The product is defined via duality from the coproduct of $R$ which sends $t$ to $t\otimes1 +1\otimes t$, so that
	\[
	(\epsilon_h\cdot\epsilon_k)(t^i)=(\epsilon_h\otimes\epsilon_k)(\Delta t^i)=(\epsilon_h\otimes\epsilon_k)(\sum_{\alpha+\beta=i} t^\alpha\otimes t^\beta)=\delta_{h+k,i}
	\]
	and hence $\epsilon_h\cdot\epsilon_k=\epsilon_{h+k}$.
	
	Therefore, $I$ is generated as an ideal by $\epsilon\colonequals\epsilon_1$. In particular, $R^\vee\cong\Q[[\epsilon]]$, which is $I$-adically complete. We conclude that $\mathbf{G}_a$ is unipotent.
	
	\item Consider $G=\mathbf{G}_m$. In this case, the coproduct on $R=\Q[t,t^{-1}]$ sends $t$ to $t\otimes t$. Therefore
	\[
	(\epsilon_h\cdot\epsilon_k)(t^i)=(\epsilon_h\otimes\epsilon_k)(\Delta t^i)=(\epsilon_h\otimes\epsilon_k)(t^i\otimes t^i)=\delta_{h,k,i}
	\]
	We conclude in particular that $I^2=I$, and hence $\mathbf{G}_m$ is not unipotent.
\end{enumerate}
\end{exam}

Let's now consider the functors we have obtained from $\pUAG$ to $\Gps$ and to $\LA$.
\begin{prop}
Let $G=\spec R$ be a pro-affine algebraic group. Then $\cG (R^\vee)\cong G(\Q)$ and $\cP(R^\vee)\cong\mathrm{Lie~} G$. 
\end{prop}

\begin{proof}
The unit of $R^\vee$ is the counit $\epsilon$. Also, for any $\phi\in R^\vee$ and any $x,y\in R$, $(\Delta\phi)(x\otimes y)=\phi(xy)$. Therefore
\[
\Delta\phi=\phi\otimes\phi \Leftrightarrow \phi(xy)=\phi(x)\phi(y)
\]
and
\[
\Delta\phi=\phi\otimes1+1\otimes\phi \Leftrightarrow \phi(xy)=\phi(x)\epsilon(y)+\epsilon(x)\phi(y)\Leftrightarrow \phi(I^2)=\phi(R/I)=0
\]
so that $\cG R^\vee\cong G(\Q)$ and $\cP R^\vee\cong (I/I^2)^\vee\cong\mathrm{Lie~} G$.
\end{proof}

%e postpone the proof of the following proposition to the next section.
%
\begin{prop}
If $G$ is a unipotent algebraic group, then $G(\Q)^{ab}\otimes_{\Z}\Q$ and $\mathrm{Lie~} G$ have finite rank.
\end{prop}

\begin{proof}
This is true for $\mathbf{UT}_n$, hence for any unipotent algebraic group.
\end{proof}

Recall that we have denoted by $\widetilde{\Gps}$ [resp. by $\widetilde{\LA}$] the subcategory of $\Gps$ [resp. of $\LA$] constituted by groups $\Gamma$ such that $\Gamma^{ab}\otimes_{\Z}\Q$ has finite rank [resp. by Lie algebras $\fg$ such that $\fg/[\fg,\fg]$ has finite rank]. Let $\bf{pro}\widetilde{\Gps}$ [resp. $\bf{pro}\widetilde{\LA}$] denote the associated category of pro-objects. It is equivalent to the category of topological groups $\Gamma$ which are homeomorphic to $\varprojlim \Gamma/\Gamma_i$, with $\Gamma/\Gamma_i$ discrete and lying in $\widetilde{\Gps}$ [resp. topological Lie algebras $\fg$ which are homeomorphic to $\varprojlim \fg/\fg_i$, with $\fg/\fg_i$ discrete and lying in $\widetilde{\LA}$]. 

By our construction, we have therefore obtained adjunction pairs
\begin{equation}\label{a}\begin{aligned}
\adj{\spec((\hat{\Q}[\cdot])^*)}{{\bf{pro}}\widetilde{\Gps}}{\pUAG}{(\Q)}
\\
\adj{\spec(\hat{\cU}(\cdot)^*)}{{\bf{pro}}\widetilde{\LA}}{\pUAG}{\mathrm{Lie~}}
\end{aligned}
\end{equation}
which are actually what we were looking for from the very beginning!

\begin{cor}[Quillen's formula]
\begin{enumerate}
	\item Let $\Gamma$ be an object of $\widetilde{\Gps}$ (e.g. if $\Gamma$ is finitely generated). Then $\spec((\hat{\Q}[\Gamma])^*)\cong\Gamma^{un}$. 
	\item Let $\fg$ be an object of $\widetilde{\LA}$ (e.g. if $\fg$ is finite dimensional). Then $\spec((\hat{\cU}\fg)^*)\cong\fg^{un}$. 
\end{enumerate}
\end{cor}

\begin{proof}
This follows formally from the previous adjunctions. We focus on the case of groups. Suppose that $G=\spec(\cO(G))$ is in $\pUAG$. Then $G$ is a filtered limit of unipotent algebraic groups $G_i$ with $\cP\cO(G_i)$ finite-dimensional. In particular, $\cO(G_i)^\vee$ as well as $\hat{\Q}[\Gamma]$ lie in $\cat$ and therefore:
\[
\begin{aligned}
\Hom(\Gamma,G(\Q))&=\varprojlim_i\Hom(\Gamma,G_i(\Q))=\varprojlim_i\Hom(\Gamma,\cG\cO(G_i)^\vee)=\varprojlim_i\Hom(\hat{\Q}[\Gamma],\cO(G_i)^\vee)=\\&=\varprojlim_i\Hom(\spec((\hat{\Q}[\Gamma])^*),G_i)=\Hom(\spec((\hat{\Q}[\Gamma])^*),G).
\end{aligned}\]
\end{proof}

We conclude our panorama on adjunctions by the following remark. There are well known adjunctions from $\Set$ to $\Gps$ and from $\Set$ to $\LA$. We wonder whether they are compatible with the rest of the diagram. In what follows we are crucially using the fact that we are working in characteristic $0$.

Let $R$ be in $\CHA$. Suppose that $x$ lies in the augmentation ideal. Then the series $\sum\frac{x^k}{k!}$ has a limit which we denote by $\exp x$. 

\begin{prop}\label{adj}
The adjunction diagram
$$\xymatrix{
&&\Set\ar@<0.5ex>[dll]\ar@<0.5ex>[drr]\\
\Gps\ar@<0.5ex>[urr]\ar@<0.5ex>[rr]&&\CHA\ar@<0.5ex>[rr]\ar@<0.5ex>[ll]&&\LA\ar@<0.5ex>[ull]\ar@<0.5ex>[ll]
}$$
commutes up to an equivalence of functors induced by the exponential map.
\end{prop}

\begin{proof}
It suffices to prove that the two right adjoints are equivalent. The claim then follows from the following lemma.
\end{proof}

\begin{lemma}
Let $R$ be an object of $\CHA$. Then $x\in\cP R\Leftrightarrow \exp x\in\cG R$.
\end{lemma}

\begin{proof}
This follows from the equalities
\[
\Delta x=x\otimes1+1\otimes x\Leftrightarrow \Delta\exp x=\exp(\Delta x)=\exp(x\otimes1+1\otimes x)=\exp(x)\otimes\exp(x)
\]
which come from the definition of the exponential.
\end{proof}

\section{Quillen's theorem and corollaries}

\begin{thm}[Quillen]Let $\MGps$ [resp. $\MLA$] be the subcategory of $\widetilde{\Gps}$ [resp. $\widetilde{\LA}$] constituted by nilpotent, uniquely divisible groups [resp. nilpotent algebras]. Then the adjunctions \eqref{a} induce equivalence of categories:
$$\xymatrix{{\bf{pro}}\MGps\ar@<0.5ex>[r]_\sim&\pUAG\ar@<0.5ex>[l]\ar@<0.5ex>[r]_\sim&{\bf{pro}}\MLA\ar@<0.5ex>[l]}$$
\end{thm}

We devote the rest of the section to sktching the proof of this theorem.

We begin with a useful fact from category theory. It is a generalization of well-known cases (Galois correspondences, closures of subsets, algebraic sets etc.) which usually involve ordered sets rather than general categories.

\begin{prop}\label{formal}
Let $\adj{F}{\cat}{\catd}{U}$ be an adjunction. The following are equivalent
\begin{enumerate}
	\item $FUF\to F$ is an isomorphism of functors.
	\item $U\to UFU$ is an isomorphism of functors.
\end{enumerate}
Moreover, if the previous conditions are satisfied, then the adjoint pair decomposes into three adjoint pairs
$$\xymatrix{\cat\ar@<0.5ex>[r]^{UF}&\cat^{UF}\ar@<0.5ex>[l]\ar@<0.5ex>[r]^{F}_\sim&\catd^{FU}\ar@<0.5ex>[l]^{U}\ar@<0.5ex>[r]&\catd\ar@<0.5ex>[l]^{FU}}$$
where $\cat^{UF}$ [resp. $\catd^{FU}$] is the full subcategory of $	\cat$ [resp. $\catd$] constituted by the objects $X$ such that $X\to UFX$ is an isomorphism [resp. $FUX\to X$ is an isomorphism], and where the pair in the center is an equivalence of categories.
\end{prop}

\begin{proof}
The first part is standard category theory (e.g. \cite[Lemma 4.3]{ls}), the second is a straightforward exercise.
\end{proof}
In particular, in order to prove the theorem we are left to prove the following facts:
\begin{enumerate}[(i)]
	\item $UF\cong\id$.
	\item $\Gamma\in{\bf{pro}}\MGps\Leftrightarrow\Gamma\in {\bf{pro}}\widetilde{\Gps}^{UF}$.
	\item $\fg\in{\bf{pro}}\MLA\Leftrightarrow\fg\in {\bf{pro}}\widetilde{\LA}^{UF}$.
\end{enumerate}
where $U$ is either $(\Q)$ or $\mathrm{Lie~}$ and $F$ is its respective left adjoint.

\begin{cor}
In order to prove the theorem, it suffices to prove
\begin{enumerate}[(I)]
	\item If $G\in\pUAG$, then $G(\Q)\in{\bf{pro}}\MGps$ and $\mathrm{Lie~} G\in{\bf{pro}}\MLA$.
	\item $\Gamma\in{\bf{pro}}\MGps\Rightarrow\Gamma\in {\bf{pro}}\widetilde{\Gps}^{UF}$.
	\item $\fg\in{\bf{pro}}\MLA\Rightarrow\fg\in {\bf{pro}}\widetilde{\LA}^{UF}$.
	\item $(\Q)$ and $\mathrm{Lie~}$ reflect isomorphisms.
\end{enumerate}
\end{cor}

\begin{proof}
The only non-trivial fact is the proof of condition $(i)$. Let $X$ be in $\pUAG$. Then by $(I)$ and $(II)$, $UX\to UFUX$ is an isomorphism. Because the compostion $UX\to UFUX\to UX$ is the identity, we also conclude that $UFUX\to UX$ is an isomorphism. By $(IV)$, we conclude $FUX\cong X$, as wanted.
\end{proof}

\begin{proof}[Sketch of the proof of Quillen's theorem]
Conditions $(II)$, $(III)$, $(IV)$ are proved by Quillen at the level of $\Gps\leftrightarrows\CHA\leftrightarrows\LA$ (see \cite[Theorem A.3.3]{quillen-r}). Condition $(I)$ comes from the fact that if $G$ is unipotent then $\mathrm{Lie~} G$ and $G(\Q)$ are nilpotent (it suffices to check this for $\mathbf{UT}_n$), $\mathrm{Lie~} G$ is finite dimensional and $G(\Q)^{ab}\otimes_{\Z}\Q$ has finite rank (already remarked), and $G(\Q)$ is uniquely divisible (it is isomorphic to $\exp\mathrm{Lie~} G$ by Proposition \ref{adj}).
\end{proof}

\begin{cor}\label{univ}
\begin{enumerate}
	\item Let $\Gamma$ be in $\widetilde{\Gps}$. Then $\Gamma^{un}$ is characterized by the fact that it is pro-unipotent and $\Gamma^{un}(\Q)$ is the universal pro-nilpotent uniquely divisible group associated to $\Gamma$.
	\item Let $\fg$ be in $\widetilde{\LA}$. Then $\fg^{un}$ is characterized by the fact that it is pro-unipotent and $\mathrm{Lie~}\fg^{un}$ is the universal pro-nilpotent Lie algebra associated to $\Gamma$.
\end{enumerate}
\end{cor}

\begin{proof}
This comes from Quillen's theorem and the lateral adjunctions of Proposition \ref{formal}.
\end{proof}

\begin{cor}
The category of unipotent algebraic group is equivalent to the category of finite dimensional nilpotent Lie algebras.
\end{cor}

\begin{proof}
The functor from $\UAG$ to nilpotent Lie algebras is fully faithful by Quillen's theorem. It is essentially surjective by \cite[Theorem 3.27]{milne-ag}.
\end{proof}

From the previous corollary and Proposition \ref{adj}, we can deduce a similar equivalence between unipotent algebraic groups and abstract groups which are exponentials of nilpotent, finite dimensional Lie algebras.

\section{The free case}

We now consider the free pro-unipotent group $G_S$ associated to a finite set $S=\{e_0,\ldots,e_n\}$. By the commutativity of the adjunction of Proposition \ref{adj}, it is isomorphic to the pro-unipotent completion of the free group over $S$, and of the free Lie algebra over $S$.

By what we already proved, $G_S\cong\spec(((\cU LS)^\wedge)^*)$, where $L$ is the free Lie algebra functor. The functor $\cU L$ from $\Set$ to $\Q\lAlg$ is left adjoint to the forgetful functor, and therefore $\cU LS\cong\Q\left\langle e_1,\ldots,e_n\right\rangle$, the algebra of non-commutative polynomials in $n$ variables. It is straightforward to check that $(\cU LS)^\wedge\cong\Q\left\langle \left\langle e_1,\ldots,e_n\right\rangle\right\rangle$, the formal non-commutative power series in $n$ variables. Its coproduct is defined via the relations $e_i\mapsto e_i\otimes1+1\otimes e_i$. If $I=(i_1,\ldots,i_k)$ is a multi-index, we indicate with $e_I$ the product $e_1\cdot\ldots\cdot e_k$. By induction, it follows $\Delta e_I=\sum_\sigma e_J\otimes e_K$, where $I,J$ vary among multi-indices and $\sigma$ varies among the permutations $\mathrm{Sym}(|J|,|K|)$ such that $\sigma(J,K)=I$. We clarify this with an example.

\begin{exam}
\[\Delta( e_1^2e_2 )= e_1^2e_2\otimes1+e_1^2\otimes e_2+2e_1 e_2\otimes e_1+2e_1\otimes e_1e_2+e_2\otimes e_1^2+1\otimes e_1^2e_2.\]
\end{exam}

This algebra is graded with respect to the degree, isomorphic to ${\bf{pro}}d V^{\otimes k}$, where $V$ is the free vector space generated by $S$.
It follows that 
\begin{equation}\label{eq:free}
((\cU LS)^\wedge)^*\cong\bigoplus (V^{\otimes k})^\vee\cong\bigoplus (V^\vee)^{\otimes k}\cong T(V^\vee).
\end{equation}
We now investigate its Hopf operations. We denote by $\epsilon_I$ the dual of $e_I$.

From the formulas
\[(\epsilon_I\cdot\epsilon_J)(e_K)=(\epsilon_I\otimes\epsilon_J)(\Delta e_K)=(\epsilon_I\otimes\epsilon_J)\left(\sum_\sigma e_M\otimes e_N\right)\]
we deduce that $(\epsilon_I\cdot\epsilon_J)(e_K)=1$ if there is a way to shuffle $I$ and $J$ to form $K$ and is $0$ otherwise. Hence, the product of $T(V^\vee)$ is the shuffle product $\sha$.

From the formula
\[(\Delta \epsilon_I)(e_J\otimes e_K)=\epsilon_I(e_{JK})\]
we deduce that $(\Delta\epsilon_I)(e_J\otimes e_K)=1$ if $JK=I$ and is $0$ otherwise. Therefore, $\Delta\epsilon_I=\sum_{JK=I}\epsilon_J\otimes\epsilon_K$, the so-called deconcatenation coproduct.

By Corollary \ref{univ}, we get that $\mathrm{Lie~} G_S$ is the universal pro-nilpotent algebra associated to $LS$, i.e. its completion by the lower central series. Also, by what we proved in the first part, $G_S(\Q)=\cG\Q\left\langle \left\langle e_1,\ldots,e_n\right\rangle\right\rangle$. %More generally, it can be computed $G_S(R)=\cG R\left\langle \left\langle e_1,\ldots,e_n\right\rangle\right\rangle$.

\section{Malcev original construction}

We now give an explicit description of another special case, originally studied by Malcev. We refer to \cite{suisse} for the group theory facts we need here. 
Suppose $\Gamma$ is nilpotent and finitely generated. In this case, the torsion elements constitute a subgroup $H$. Since any uniquely divisible group has no torsion, by Corollary \ref{adj} we conclude that $\Gamma^{un}\cong(\Gamma/H)^{un}$. We can therefore suppose that $\Gamma$ has no torsion.

Let
\[
\Gamma=\Gamma_1\geq\Gamma_2\geq\ldots\geq\Gamma_k=1
\]
be the lower central series ($\Gamma_i=[\Gamma_{i-1},\Gamma]$). Each factor $\Gamma_i/\Gamma_{i+1}$ is abelian and finitely generated since $[\Gamma_i,\Gamma_i]\leq[\Gamma_i,\Gamma]=\Gamma_{i+1}$. We can then refine the lower central series to obtain another central series with cyclic quotients. A group with such a central series is called polycyclic. Quotients and subgroup of polycyclic ones are again polycyclic (by studying the induced filtrations). In particular, the quotients of the upper central series ($Z_i/Z_{i+1}=Z(G/Z_{i+1})$)
\[
\Gamma=Z_1\geq Z_2\geq\ldots\geq Z_k=1
\]
are polycyclic. They are also without torsion by the next lemma.

\begin{lemma}
If $\Gamma$ is nilpotent and without torsion, then all quotients $\Gamma/Z_i$ are without torsion.
\end{lemma}

\begin{proof}
Since $\Gamma/Z_{i+1}\cong({G/Z_i})/({Z_{i+1}/Z_i})\cong({G/Z_i})/({Z(\Gamma/Z_i)})$, it suffices to prove that if $\Gamma$ is nilpotent and without torsion, then $\Gamma/Z(\Gamma)$ is without torsion. 

Suppose $x^m$ is central. We need to prove that also $x$ is. If $x^m$ is central, then for any $y$ we have $(y^{-1}xy)^m=x^m$. It suffices to prove uniqueness of roots in a torsion-free nilpotent group. We make induction on the nilpotency class, being the case of an abelian group trivial.

Let $a^m=b^m$ in a torsion-free nilpotent group $\Gamma$. In order to prove $a=b$, it suffices to prove $[a,b]=1$ since $\Gamma$ is torsion-free. Since $b^{-1}ab=a[a,b]$, both $b^{-1}ab$ and $a$ lie in the subgroup $H=\left\langle [\Gamma,\Gamma],a\right\rangle$, which has a stricly lower nilpotency class (see \cite[Proposition 2.5.5]{suisse}). By induction, from the equality $(b^{-1}ab)^m=b^{-1}a^mb=b^m=a^m$, we conclude $[a,b]=1$ as wanted.
\end{proof}

In conclusion, the upper central series can be enriched into a filtration 
\[
\Gamma=\Gamma^1\geq\Gamma^2\geq\ldots\Gamma^{s+1}=1
\]
such that $\Gamma^i/\Gamma^{i+1}\cong\left\langle e_i\right\rangle\cong\Z$. The set $\{e_1,\ldots,e_s\}$ is called a Malcev basis for $\Gamma$. We can associate to any element $g\in\Gamma$ a unique set of $s$ coordinates $t_i(g)\in\Z$ such that $g={\bf{pro}}d e_i^{t_i(g)}$, and $\Gamma^i$ coincides with the subset $\{g\in\Gamma: t_j(g)=0, j<i\}$. This defines a bijection from $\Gamma$ to $\Z^s$. We now recover also the product in terms of the coordinates.

\begin{prop}
Let $\Gamma$ be finitely generated, nilpotent and without torsion. Let $\{e_1,\ldots,e_s\}$ be a Malcev basis for $\Gamma$ and $t_i(g)$ the Malcev coordinates of an element $g$.
\begin{enumerate}
	\item The product is polynomial in the coordinates, i.e. there exist polynomials $P_{i}$ with rational coefficients such that
	\[t_i(gh)=t_i(g)+t_i(h)+P_{i}(t_j(g),t_j(h)).\]
	Moreover, the polynomial $P_{i}$ depends only on $t_j$'s with $j<i$.
	\item The inverse is polynomial in the coordinates, i.e. there exist polynomials $Q_{i,k}$ with rational coefficients such that
	\[t_i(g^k)=kt_i(g)+Q_{i,k}(t_j(g)).\]
	Moreover, the polynomial $Q_{i,k}$ depends only on $t_j$'s with $j<i$.
\end{enumerate}
\end{prop}

\begin{proof}
Make induction on the cardianlity of the Malcev basis. Details in \cite[Propri\'et\'e 3.1.5]{suisse}.
\end{proof}

Since the polynomials $P_{i}$ and $Q_{i,k}$ have rational coefficients, they define an algebraic group $G: R\mapsto (R^s,\cdot)$ where $\cdot$ is the product defined using the above formulas. 

\begin{prop}
The group $G$ is unipotent.
\end{prop}

\begin{proof}
Since $\cO(G)$ is a polynomial ring, $G(\Q)$ is dense in $G$. Therefore, if we prove that there is a faithful finite dimensional representation of $G$ such that $G(\Q)\to\mathbf{GL}_{V}(\Q)$ is made of unipotent morphisms (i.e. , for all $g\in G(\Q)$, $(g-\id)^n=0$ for $n\gg0$), we conclude that, with respect to a suitable basis, $G(\Q)$ factors through $\mathbf{UT}_n(\Q)$. By density, we can isomorphically embed $G$ in $\mathbf{UT}_n$, as wanted. Because the regular representation contains all representation, we can equivalently prove that all endomorphisms of $G(\Q)$ are unipotent with respect to it (i.e. $(g-\id)^n=0$ for $n\gg0$ when restricted to any subrepresentation of finite dimension).

This representation sends $g$ to the endomorphism $T_i\mapsto t_i(g)+T_i+P_i(t_j,T_j)$. By the formulas above, it follows that $(g-\id)$ it sends a monomial $T^I=T_1^{i_1}\cdot\ldots T_s^{i_s}$ to a linear combination of monomials which are stricly smaller with respect to the lexicographic order. In particular, for any multi-index $I$, $(g-\id)^n(T^I)=0$ for $n\gg0$. 
\end{proof}

We remark that the map $\Gamma\to G(\Q)$ is induced by the inclusion $\Z^s\to\Q^s$. It satisfies a universal property:

\begin{prop}
The abstract group $G(\Q)$ is the nilpotent, uniquely divisible closure of $\Gamma$.
\end{prop}

\begin{proof}
Using the formulas and induction on $i$, it is straightforward to see that $G(\Q)$ is uniquely divisible and if $x\in G(\Q)$, then $x^n\in\Gamma$ for $n\gg1$.
\end{proof}

\begin{cor}
$\Gamma^{un}\cong G$.
\end{cor}

\begin{proof}
This comes from the previous propositions and Corollary \ref{adj}.
\end{proof}

\section{Torsors}

Let $\Gamma$ be an abstract group. We remark that the functors we used to define $\Gamma^{un}$ make sense more generally for $\Gamma$-sets:
\[
\Gamma\lSet\stackrel{\Q[\cdot]}{\rightarrow}\Q[\Gamma]\Mod\stackrel{^\wedge}{\rightarrow}\hat{\Q}[\Gamma]\Mod\stackrel{\spec((\cdot)^*)}{\rightarrow}\Gamma^{un}\lVar
\]
and we denote again their composition with $S\mapsto S^{un}$.
Moreover, all these functors are tensorial with respect to the tensors defined in each category. The first and the last one are tensorial with respect to the cartesian product. It follows that if $S\in\Gamma\lSet$ is a torsor, i.e. if 
\[\Gamma\times S\to S\times S\qquad (g,s)\mapsto (g\cdot s,s)\]
is an isomorphism, then also
\[\Gamma^{un}\times S^{un}\cong(\Gamma\times S)^{un}\to (S\times S)^{un}\cong S^{un}\times S^{un}\]
is an isomorphism. Therefore, $S\mapsto S^{un}$ maps torsors to torsors.


\section{The Tannakian approach}

\begin{prop}
Let $\Gamma$ be an abstract group such that $\Gamma^{un}$ il well defined. Then $\Gamma^{un}$ is the pro-affine algebraic group associated to the Tannakian category of unipotent representations of $\Gamma$.
\end{prop}

\begin{proof}
The functor $\Gamma^{un}\Rep\to\Gamma \Rep$ sending $\Gamma^{un}\to\mathbf{GL}_V$ to $\Gamma\to\Gamma^{un}(\Q)\to\mathbf{GL}_V(\Q)$ factors over unipotent representations of $\Gamma$ since $\Gamma^{un}$ is pro-unipotent. Viceversa, if $\Gamma\to\mathbf{GL}_V(\Q)$ is unipotent then, with respect to a suitable basis, it factors over $\mathbf{UT}_n$. It follows that the subgroup $H$ of $\mathbf{GL}_V$ generated by $\Gamma$ is isomorphic to a subgroup of $\mathbf{UT}_n$, hence unipotent. By the universal property, the map $\Gamma\to H(\Q)$ then induces a map $\Gamma^{un}\to H\to\mathbf{GL}_V$, as wanted.
\end{proof}

Since we have proved the existence (and Quillen's construction) of $\Gamma^{un}$ only for groups $\Gamma$ with nice properties (i.e. $\Gamma^{ab}\otimes_{\Z}\Q$ has finite dimension), we wonder if this proposition gives a more general construction of $\Gamma^{un}$, i.e. if the Tannaka dual of unipotent representations of $\Gamma$ satisfies the universal property of the pro-unipotent completion.

\section{More on the conilpotency filtration}

We now present a last corollary of the first section and Quillen's paper. Suppose $G$ is unipotent, and let $\fg$ be its Lie algebra. We have $\fg\cong\cP\cO(G)^\vee$, and it inherits the $I$-adic filtration from $\cO(G)^\vee$:
\[\cP\cO(G)^\vee=\cP\cO(G)^\vee\cap I\supset \cP\cO(G)^\vee\cap I^2\supset\ldots\]

On the other hand, we can consider the graded Hopf algebra $\gr^\bullet\cO(G)^\vee$, obtained via the $I$-adic filtration. Its primitive elements constitute a graded subset $\cP\gr^\bullet\cO(G)^\vee$.

\begin{prop}\cite[Proposition A.2.14]{quillen-r}
The natural map $\gr^\bullet\cP\cO(G)^\vee\to\cP\gr^\bullet\cO(G)^\vee$ is an isomorphism.
\end{prop}

In particular, we deduce $\gr^1\fg\cong I/I^2\cong\fg/[\fg,\fg]$. This abelian Lie algebra has a natural map to $\gr^\bullet\cP\cO(G)^\vee\cong\cP\gr^\bullet\cO(G)^\vee$. It is a quotient of the free Lie algebra $LS$ generated by a chosen basis $S$ of $\fg/[\fg,\fg]$. By adjunction, we obtain a map $\hat{\cU}LS\to\gr^\bullet\cO(G)^\vee$, and by duality a map $(\gr^\bullet\cO(G)^\vee)^*\to\hat{(\cU}LS)^*$, and a map $\spec((\hat{\cU}LS)^*)\to\spec((\gr^\bullet\cO(G)^\vee)^*)$.

We recall that the conilpotency filtration on $\cO(G)$
\[
0\subset C_1\subset C_2\subset\ldots
\]
is dual to the $I$-adic filtration on $\cO(G)^\vee$
\[
\cO(G)^\vee\supset I\supset I^2\supset\ldots
\]
in the sense that $(\gr^i\cO(G)^\vee)^*\cong\gr_i\cO(G)$. Hence in particular, $\fg/[\fg,\fg]^*\cong(I/I^2)^*\cong(\gr_1\cO(G))$ and $(\gr^\bullet\cO(G)^\vee)^*\cong\gr_\bullet\cO(G)$.

We remark that $\spec((\hat{\cU}LS)^*)$ is a free pro-unipotent group generated by $S$. By Equation \ref{eq:free}, $\hat{(\cU}LS)^*\cong T(\fg/[\fg,\fg]^\vee)\cong T(\gr_1\cO(G))$.

\begin{prop}
The map $\gr_\bullet\cO(G)\to T(\gr_1\cO(G))$ is injective.
\end{prop}

\begin{proof}
We prove that the dual is surjective. It is the map $\hat{\cU}LS\to\gr^\bullet\cO(G)^\vee\cong\gr^\bullet \hat{\cU}\fg$, where the last equality follows from the unipotency of $G$.

The graded algebra $\gr^\bullet \hat{\cU}\fg$ is generated by its first graded piece $\gr^1 \hat{\cU}\fg=\cP\gr^1\hat{\cU}\cong\gr^1\cP\hat{\cU}\cong\fg/[\fg,\fg]$, hence $ \hat{\cU}LS\to\gr^\bullet \hat{\cU}\fg$ is surjective, as claimed.
\end{proof}


\input{talk-4.tex}

\chapter{Hodge structure: Proof of Chen's theorem by Sergey Rybakov}
%\addcontentsline{toc}{chapter}{*Hodge structure: Proof of Chen's theorem by Sergey Rybakov}

Sergey Rybakov on September 4th, 2012.

\medskip
\medskip

\noindent Recall that $M$ is a smooth connected manifold, $A_M^\bullet$ is the DG-algebra of complex valued differential forms, and $B(A_M^\bullet;a,b)$ is the bar complex of $A_M^\bullet$.

\section{The theorem}

Chen's theorem states that
\[
\mathrm{iter}^{\vee} : H^0(B(A_M^\bullet;a,b)) \to \C \otimes_{\Q} \cO(\pi_1(M;a,b)^{un}),
\]
is an isomorphism of algebras and respects coproducts. 
Note that both algebras are torsors under their corresponding Hopf algebras, thus it is enough to prove the Chen's theorem for $a=b$. Let us state it once more.

\begin{thm}[Chen's theorem]\label{thm:chentwo}
Let $H(M,a) = H^0(B(A_M^\bullet;a,a))$. Then
\[
\mathrm{iter}^{\vee} : H(M,a) \isom \C \otimes_{\Q} \cO(\pi_1(M,a)^{un})
\]
is an isomorphism of Hopf algebras.
\end{thm}

The morphism $\mathrm{iter}^{\vee}$ induces a functor
\[
\Phi : \begin{array}{rcl}
\mathrm{Comod~} H(M,a) & \longrightarrow & \mathrm{Rep}^{un}(\pi_1(M,a)) \\
V & \mapsto & \left( \gamma \mapsto \left( V \to V \otimes H(M,a) \stackrel{1 \otimes \int_{\gamma}}{\longrightarrow} V \right) \right).
\end{array}
\]

Clearly, $\Phi$ is faithful. 

\begin{lemma}[Key fact]
The theory of Tannakian categories shows that $\Phi$ is an equivalence of categories if and only if $\mathrm{iter}^{\vee}$ is an isomorphism.
\end{lemma}

By the Lemma the Chen's theorem is equivalent to the following result.

\begin{thm}\label{thm:comodsandreps}
The functor $\Phi$ is an equivalence of categories.
\end{thm}


\noindent To prove Theorem \ref{thm:comodsandreps}, we construct an inverse functor $\Psi$, which is a composition of Riemann-Hilbert correspondence functor and a functor $G$:
\[
\xymatrix{
\mathrm{Rep}^{un}(\pi_1(M,a)) \ar[d]^{RH} \ar[r]^{\Phi} & \mathrm{Comod}(H(M,a)) \\
 \mathrm{Conn}^{un}(M) \ar[ur]^{G}& \\
}
\]

\section{Riemann-Hilbert correspondence}

\begin{notation}
We denote by $\mathrm{Conn~}M$ the category of complex vector bundles with flat connection. There are adjoint functors
\[
\xymatrix{\mathrm{Conn~}M \ar@<0.2em>[r] & \mathrm{Rep~} \pi_1(M,a) \ar@<0.2em>[l]}.
\]
\end{notation}

A vector bundle $E$ with a connection $\nabla$ goes to the fiber $E_a$ with a monodromy representation.
Given a representation $V$ of $\pi_1(M,a)$, form $L = V \times \widetilde{M} / \pi_1(M,a)$, where $\widetilde{M}$ denotes the universal cover of $M$. Then
\[
V \mapsto L \otimes_{\C} \cA^0_M = \cE
\]
where $\cE$ is the sheaf of sections of $E$. The connection is defined as follows:
\[
\cE=L \otimes_{\C} \cA^0_M \stackrel{\nabla - \overline{d} \otimes_{\C} d}{\longrightarrow} L \otimes_{\C} \cA^1_M \cong L \otimes_{\C} \cA^0_M \otimes_{\cA^0_M} \cA^1_M \cong \cE \otimes_{\cA^0_M} \cA^1_M
\]

\begin{prop}
The Riemann-Hilbert correspondence takes unipotent representations to unipotent connections. In particular, it induces an equivalence:
\[
RH : \mathrm{Rep}^{un}(\pi_1(M,a)) \isom \mathrm{Conn}^{un}(M).
\]
\end{prop}

\section{Construction of the functor $G$ from unipotent connections to comodules}

Let $(E, \nabla) \in \mathrm{Conn}^{un}(M)$. Recall that an extension of trivial sheaves of $\cA^0_M$-modules is trivial. Since $E$ is unipotent, its sheaf of sections $\cE$ is trivial. 
Choose a trivialization  
\[
\phi : \cE \isom E_a \otimes_{\C} \cA^0_M
\]
such that $\phi$ induces $id_{E_a}$.

Put $N= \nabla- d$, where $N \in \End_{\C} E \otimes A^1_M$. Clearly, $\nabla$ is uniquly determined by $N$.
Since $\nabla$ is unipotent, $N$ is strict upper-triangular in a situable basis in $E_a$. Thus $N$ is nilpotent, i.e.
\[
N^{\rk E} = 0 \in (\End_{\C}) E \otimes (A^1_M)^{\otimes \rk E}.
\]

\begin{defn}
Let
\[
P_E = 1 + N + N^{\otimes 2} + \cdots \in (\End_{\C}E_a) \otimes_{\C} B(A_M,a)^0
\]
This is a finite sum.
\end{defn}

\begin{prop}
Given a vector bundle $E$ on $M$, $P_E$ is a cocycle.
\end{prop}
\begin{proof}

In fact, $P_E\in (\End_{\C}E_a) \otimes_{\C}(\oplus_{n\geq 0} (A_M^1)^{\otimes n})$. The differential on $\oplus_{n\geq 0} (A_M^1)^{\otimes n}$ is given by $d'+d''$, where
\begin{eqnarray*}
d'(\omega_1 \otimes \cdots \otimes \omega_n) & = & -\sum_{i=1}^n \omega_1 \otimes \cdots \otimes d \omega_i \otimes \cdots \otimes \omega_n \\
d''(\omega_1 \otimes \cdots \otimes \omega_n) & = & - \sum_{i=1}^{n-1} \omega_1 \otimes \cdots \otimes \omega_i \wedge \omega_{i+1} \otimes \cdots \otimes \omega_n
\end{eqnarray*}

By Proposition~\ref{prop:flatconn} given a vector bundle with connection $(E,\nabla)$, $\nabla$ is flat if and only if $dN = N \wedge N$. Altogether we have that $d' N^{\otimes n} = d'' N^{\otimes (n+1)}$. Thus $P_E$ is a cocycle.
\end{proof}

\begin{defn}
Cocycle $P_E$ defines an element $[P_E]\in (\End_{\C}E_a) \otimes_{\C} H(M,a)$. In turn, $[P_E]$ defines a $H(M,a)$-comodule structure on $E_a$.
\end{defn}

We are going to prove that $[P_E]$ does not depend on the choise of the trivialisation $\phi$, and that assignment $(E,\nabla)\to (E_a,[P_E])$ gives a functor $G:\mathrm{Conn}^{un}(M)\to\mathrm{Comod}(H(M,a))$.

\begin{prop}
Let $f : (E, \nabla_E) \to (F, \nabla_F)$ be a morphism in $\mathrm{Conn}^{un}(M)$.
The morphism
\[
P_F \cdot f_a \to f_a \cdot P_E
\]
is a coboundary in $\Hom_\C(E_a,F_a) \otimes_{\C} B(A_M, a)^0$.
\end{prop}

\begin{proof}
Fix trivialisations of $\cE$ and $\cF$ as before. We get the following isomorphism of complexes, independent of $\nabla$:
\[
\Hom_{\C}(E_a,F_a) \otimes_{\C} B(A_M,a) \isom \Hom_{\cA^0_M}(\cE,\cF) \otimes_{\cA^0_M} B(A_M,a)
\]
The differential of the right hand side is
\[
\delta : M \otimes \omega \mapsto dM \otimes \omega + M \otimes d\omega
\]
where $d\omega$ is a differential in $B(A_M,a)$, and tensor products are taken over $\cA^0_M$.
The term $dM \in \Hom_{\cA_M^0}(\cE,\cF) \otimes_{\cA^0_M} \cA^1_M$ comes from the trivial connection on trivialised sheaves. The proposition now follows from the key formula

\[
P_E f_a - f_a P_F = \delta \left( \sum_{n \geq 0} \sum_{i=0}^n N^{\otimes i}_F \otimes f \otimes N^{\otimes(n-i)}_E \right)
\]
where we consider the terms on the left hand side as an elements of $\Hom(\cE,\cF) \otimes_{\cA^0_M} B(A_M, a)$.

The formula can be proved by direct computation using the following observations:

\[
N_E f_a - f_a N_F = df,
\]

\[
\delta N_E=N_E\wedge N_E.
\]

\end{proof}

For an application, consider the case when $E = F$ and $\phi_1, \phi_2 : \cE \to E_a \otimes \cA_M^0$. Denote by $P_i$ the sum which come from $\phi_i$ for $i=1,2$. Since $(\phi_i)_a=1_{E_a}$, $P_1 - P_2$ is a coboundary in $\End(E_a) \otimes_{\C} B(A_m,a)^0$. Hence $[P_1] = [P_2]$. Functoriality can be proved in the same way.


\section{End of the proof}

\begin{prop}
The composition,
\[
\Phi \circ \Psi = 1
\]
is the identity on $\mathrm{Rep}^{un}(\pi_1(M;a))$.
\end{prop}
\begin{proof}
Let $V$ be a representation of $\pi_1(M;a)$. Then $\Psi(V) = G(RH(V))$, where $RH(V)$ is given by some nilpotent $N$. Put 
\[
P = \sum_{n=0}^{\infty} N^{\otimes n}.
\]
By Proposition \ref{prop:intismonodromy} the monodromy of $RH(V)$ is given by $\gamma \mapsto \int_{\gamma} P$.
Thus $\Phi(\Psi(V))\cong V$.
\end{proof}

\begin{prop}
The functor $\Psi$ is essentially surjective.
\end{prop}

\begin{proof}
The functor $\Psi$ send a unipotent connection described by a unipotent matrix to a comodule whose coproduct is described by a matrix. To show essential surjectivity, we begin with a matrix describing a coproduct and try to produce a connection that maps to the coproduct.

Decompose $A^0_M = V \oplus dA^0_M$. Then $A_M$ is quasi-isomorphic to the dg-algebra
\[
\widetilde{A}_M = (0 \to \C \stackrel{\sigma}{\to} V \stackrel{d}{\to} A^2_M \stackrel{d}{\to} \cdots )
\]

Given a vector space $V$, we define an associated bialgebra $T(V) := \bigoplus_{n \geq 0} V^{\otimes n}$ with product given by shuffle and coproduct given by deconcatenation.

We have an inclusion of coalgebras
\[
H(M,a)=H^0B(A^0_M,a)) \isom H^0(B(\widetilde{A}_M,a)) \isom H^0(\overline{B}(\widetilde{A}_M,a)) \subset T(V).
\]
Any comodule $C$ over $H(M,a)$ is also a comodule over $T(V)$. Any comodule over $T(V)$ comes from a comodule over $T(U)$ for $U \subset V$ with $\dim U < \infty$ because $\dim C < \infty$. So $U \subset A^1_M$ and $C$ is a comodule over $T(U)$.

By Talk 3 category of comodules over $H(M,a)$ is equivalent to the category of representations of the free group $\Gamma$ defined as follows. 
Let $e_1, \ldots, e_n$ be a basis of $U^{\vee}$, and $f_1, \ldots, f_n$ a basis of $U$.
Then $T(U)^{\vee} = \C\langle\langle e_1, \ldots, e_n \rangle\rangle$, and $\Gamma = \langle \gamma = \exp e_i \rangle$ is a free group.

Then $\gamma_i$ gives a unipotent matrix $M_i \in \End_{\C} C$. Put $N_i = \log M_i$, and
$$N = \sum_i N_i \otimes f_i \in \End C \otimes A^1_M.$$ The matrix $N$ defines a unipotent connection on a trivial vector bundle. We claim that $$P=1 + N + N^{\otimes 2} + \cdots$$ defines a comodule $C'$ which is isomorphic to $C$. Indeed, compute the action of $\gamma_i$ on $C'$.

\begin{eqnarray*}
\exp e_i \cdot \left( \sum_{j=1}^n N_j \otimes f_j \right)^{\otimes k} = \frac{e_i^{\otimes k}}{k!} N_i^k \times f_j^{\otimes k} = \frac{N_i^k}{k!} \\
\gamma_i(P) = \exp(e_i)(P) = \exp N_i = M_i
\end{eqnarray*}
\end{proof}

\begin{cor}
The composition,
\[
\Psi \circ \Psi = 1,
\]
is the identity on $\textrm{$H(M,a)$-Comod}$.
\end{cor}
\begin{proof}
It follows from the fact that $\Phi$ is faithful, $\Phi \circ \Psi = 1$ and $\Psi$ is essentially surjective.
\end{proof}


\chapter{Hodge structure: Mixed Hodge structure of the fundamental group by Javier Fres\'an}
%\addcontentsline{toc}{chapter}{*Hodge structure: Mixed Hodge structure of the fundamental group by Javier Fres\'an}

Javier Fres\'an on September 4th, 2012.

\medskip
\medskip

\noindent Our goal will be to understand all the relations among multiple zeta values $S(n_1, \ldots, n_k) \in \R$. For example,
\begin{itemize}
\item Stuffle: $S(2)^2 = 2 S(2,2) + S(4)$
\item Shuffle: $S(2)^2 = 2 S(2,2) + 4 S(1,3)$
\item Unknown: $28 S(3, 9) + 150 S(5,7) + 168 S(7,5) = \frac{5197}{691}S(12)$
\end{itemize}

\section{Formal multiple zeta values}

We must replace multiple zeta values by formal multiple zeta values (Cf. Definition \ref{def:mzvspaces}:
\[
\begin{array}{rcl}
Z & \longleftrightarrow & \fH_0 \subset \Q\langle x_0, x_1 \rangle \supset \fH_0 \\
S(n_1, \ldots, n_k) & \longleftrightarrow & x_0^{n_1-1} x_1 \ldots x_0^{n_k-1} x_1
\end{array}
\]
Some of the relations between MZV's are explained by the shuffle and stuffle relations between formal MZV's.

The next step will be a motivic interpretation of MZV's as periods of mixed Tate motives. Much of the theory is based on mixed Hodge theory, so we will first see how to interpret a MZV as a period of a mixed Hodge structure on $\pi_1^{un}$.

Recall from Definition \ref{def:fundgroup} that, for a smooth, connected complex manifold $M$ and $a, b \in M$, $\Q[\pi_1(M;a,b)]$ is the $\Q$-vector space generated by homotopy classes of paths from $a$ to $b$ and a free $\Q[\pi_1(M;a)]$-module of rank one.

Dualizing the Chen isomorphism from Proposition $\ref{prop:isompione}$ gives
\[
c_n^{\vee} : \Q_{a,b} \oplus H^n(M^n, Z^n_{a,b}) \isom (\Q[\pi_1(X; a, b)]/I^{n+1})^{\vee}
\]
These form a directed system. 

Let $\mathbf{H}$ denote the left-hand side. Then $\mathbf{H}$ has an algebra structure ($\mathbf{H} \otimes \mathbf{H} \to \mathbf{H}$) and a mixed Hodge structure. We aim to show that these two structures are compatible.

\TODO
 
\section{Mixed Hodge structure}

\begin{defn}
A \emph{$\Q$-mixed Hodge structure} (Cf. Definition \ref{def:qpurehodge}) is a triple $H = (H, W_{\bullet}, F^{\bullet})$ such that
\begin{enumerate}
\item $H$ is a $\Q$-vector space.
\item $W_{\bullet}$ is an ascending filtration on $H$ called the weight filtration.
\item $F^{\bullet}$ is a descending filtration on $H \otimes \C$ and called the Hodge filtration.
\end{enumerate}
\[
\gr^W_{\ell} H = W_{\ell} H / W_{\ell - 1} H
\]
And $F^{\bullet}_{ud}$ is a pure Hodge structure of weight $\ell$.
\end{defn}

Then $H$ is defined over $k \subset \C$. If there exists a $k$-vector space $H_{dR}$ and a filtration $T^{\bullet}$ on $H_{dR}$ and an isomorphism 
\[
\alpha : H_{dR} \otimes_k \C \isom H \otimes_{\Q} \C
\]
compatible with the filtration. This forms $(H, H_{dR}, W_{\bullet}, F^{\bullet}, \alpha)$
\[
\xymatrix{
MH(k) \ar[r]^{W_{dR}} \ar[dr]_{w_B} & H_{dR} \in \mathrm{Vect}(k) \\
& H \in Vect(\Q)
}
\]

\begin{defn}[Comparison isomorphism]
The comparison morphism is
\[
c_n : \C \otimes w_{dR} \isom \C \otimes w_B
\]
The comparison morphism encodes the periods.
\end{defn}
\begin{defn}[Period matrix]
Fix bases for the comparison isomorphism. Then the period matrix is the matrix of the comparison isomorphism in those bases, and it is generally chosen from a coset in $GL_n(k)  GL_n(\C) / GL_n(\Q)$.
\end{defn}

\begin{prop}
$\mathrm{MH}(k)$ is a ???, symmetric tensor category with tensor product
\[
W_p(H \otimes H) = \sum W_u H \otimes W_{p-u} H'
\]
The unit is $H^0(\spec k) = \Q(0)_H = \Q^{0,0}$ pure of weight $0$.
\end{prop}

\subsection{Examples}
\begin{enumerate}

\item $H^2(\Pspace^1_k) = \Q(-1)_H = (2\pi i) \Q = \Q(-1)_H$ pure of weight -2, $H_{dR} = k$. Period is $\frac{1}{2\pi i}$. $\Q(n)_H$ is pure of weight $-2n$. Recall that $\Q(-1) = H^2(\Pspace^1)$ (Cf. \ref{not:qminusone}).

\item (Deligne) Any algebraic variety over $k$ has a $k$-Mixed Hodge structure. Assume the variety $X$ is smooth and includes into a proper variety $j : X \inj \overline{X}$ such that $D = \overline{X} \setminus X$ is a simple normal crossing divisor. Then $\Omega^{\bullet}_{\overline{X}}(\log D) \to Rj_* \Omega_X$.

We can compute the cohomology of $X$ by computing the hypercohomology:
\[
\mathbf{H}^k(\overline{X}, \Omega^{\bullet}_{\overline{X}}(\log D))^{\otimes \C} \to H^k(X, \C)
\]
The period is $2\pi i$.
\begin{defn}[Weight filtration and Hodge filtration]
Given a smooth variety $X$ with (possibly singular) closure $\overline{X}$, its \emph{weight filtration} is defined over $\Q$ and given by
\[
W_m \Omega_{\overline{X}}^p(\log D) = \left\{ \begin{array}{ll}
0 & m \leq 0 \\
\Omega_{\overline{X}}^m(\log D) \wedge \Omega_{\overline{X}}^{p-m} & 0 \leq m \leq p \\
\Omega_X^p(\log D) & m \geq p
\end{array} \right.
\]
Its Hodge filtration is
\[
F^p \Omega_{\overline{X}}^{\bullet}(\log D) = \left( 0 \to \cdots \to 0 \to \Omega^p_{\overline{X}}(\log D) \to \cdots \right)
\]
\end{defn}
\[
W_k H^m(X, \C) = \mathrm{Im} H^m(X, W_{k-m} \Omega^{\bullet}(\log D))
\]
To show: defined over $\Q$.

\item Singular cohomology and relative cohomology. $Z^n_{a,b}$ and $H^n(X,Y)$ and mixed Hodge structure. $\Q_{a,b} \oplus \varinjlim H^n(M^n, Z^n_{a,b}) \in \mathrm{Ind~}\mathrm{MH}(k)$. (Recall that $\Q_{a,b}$ is defined to be $\Q$ if $a \neq b$ and $0$ otherwise.)

\begin{defn}[Ind-category]
Fix a field $k$. Given a tensor category $(\cC, \otimes)$ with a tensor functor $w : \cC \to \mathrm{Vect}$, it has an Ind-category $\mathrm{Ind~} \cC$.
\begin{itemize}
\item[objects]: Directed systems of $k$-varieties $(X_{\alpha})_{\alpha \in A}$.
\item[morphisms]: $\Hom((X_{\alpha}), (Y_{\beta})) = \varprojlim_{\alpha} \varinjlim_{\beta} \Hom(X_{\alpha}, Y_{\beta})$
\end{itemize}
It is a tensor category.
\end{defn}

\begin{defn}
A commutative algebra is ???
$M \in \mathrm{Ind~} \cC$ and $m : M \otimes M \to M$, $u : \mathbf{1} \to M$ and satisfies compatibility conditions.
This can be defined analogously for commutative Hopf algebras.
\end{defn}
\begin{defn}
A $\cC$-affine scheme $\spec(M) = \cO(X)$ is an object in the opposite category $(\mathrm{Ind~} \cC)^{op}$.
\[
\xymatrix{
MH(k) \ar[r] \ar[dr] & \mathrm{Vect}(\Q) \\
& \mathrm{Vect}(k)
}
\]
\end{defn}
\TODO
\end{enumerate}

\begin{prop}
If $A \to B$ is a quasi-isomorphism of dg-algebras, and $a$, $b$ abused notation for compatible augmentations
\[
\xymatrix{
A \ar[d] \ar[r]^a & k \\
B \ar[ur]^a & 
}
\]
then $B(A; a, b)$ and $B(B; a, b)$ are also quasi-isomorphic.
\end{prop}

\begin{enumerate}
\item $H^0(B(A_M;a,b))$ is a $\Q$-vector space.
\item Singular cochains $C^0_{X(\C)}$ with augmentations $\C^0_{X(\C)} \to \Q$. $H^0(B(C^0_{X(\C)};a,b))$ is a $\Q$-vector space. 
\item Let $\Omega_{\overline{X}}^{\bullet}(\log D)$ be the Godement resolution and $A_X^{\bullet} = R\Gamma(\overline{X}, \Omega^{\bullet}_{\overline{X}}(\log D))$ be the derived complex. $H^0(B(A_X; a, b))$ is a $k$-vector space.
\end{enumerate}

\begin{defn}[Mixed Hodge complex]
$(H_{\C}^{\bullet}, W_{\bullet}, F^{\bullet})$. 
\[
\begin{array}{ll}
(H^{\bullet}, W_{\bullet}) & \textrm{complex of $\Q$-vector spaces} \\
(H^{\bullet}_{dR}, W_{\bullet}, F^{\bullet}) & \textrm{complex of $k$-vector spaces} \\
\alpha^{\bullet} : H^{\bullet} \otimes \C \to H^{\bullet}_{\C}&  \textrm{quasi-isomorphisms} \\
\beta^{\bullet} : H_{dR}^{\bullet} \otimes \C \to H^{\bullet}_{\C}) & \textrm{quasi-isomorphisms}
\end{array}
\]
\end{defn}

\begin{prop}[Hain, Journal of $K$-theory 1987]
The bar construction applied to a mixed Hodge complex gives a mixed Hodge complex.
\end{prop}
Geometric origin:
\[
\begin{array}{rcl}
\mathbf{H} & \to & \mathbf{H} \otimes \mathbf{H} \\
\textrm{$[\omega_1 \mid \cdots \mid \omega_n]$} & \mapsto & \sum \textrm{$[\omega_1 \mid \cdots \mid \omega_i] \otimes [\omega_{i+1} \mid \cdots \mid \omega_k]$}
\end{array}
\]
To perform the bar construction, we take tensor powers of $A_M^{\bullet}$:
\[
(A_M^*)^{\otimes n} \to (A_M^*)^{\otimes (n-1)} \to \cdots
\]
\[
A^*_{M^n} \to A^*_{M^{n-1}} \to \cdots
\]
\[
\xymatrix{
M^n & M^{n-1} \ar@<0.5em>[l] \ar@<0.15em>[l] \ar@<-0.15em>[l] \ar@<-0.5em> & \ar[l]
}
\]

$\cO(\pi_1(X;a,b)_H)$ is an object in $\mathrm{Ind~} \mathrm{MH}(k)$.
\begin{eqnarray*}
\pi_1(X;a,b)_B = \spec H^0(B(C^*_{X(\C)}; a,b)) \qquad \textrm{(Betti)} \\
\pi_1(X;a,b)_{dR} = \spec H^0(B(A_X^*; a,b)) \qquad \textrm{(de Rham)}
\end{eqnarray*}

The comparison between realizations gives a mapping
\[
\xymatrix{
\cO(\pi_1(M;a,b)_{dR}) \otimes_k \C \isom \cO(\pi_1(M;a,b)_{p}) \otimes_{\Q} \C & H^0(\overline{B}(A_{X(\C)}; a,b)) \cong H^0(B(C_{X(\C)};a,b)) \ar[l]^{\mathrm{iter}^{\vee}}
}
\]

$w_1, \ldots, w_n$ closed differential $1$-forms such that $\omega_i \wedge \omega_{i+1} = 0$. $\omega_1 \otimes \cdots \otimes \omega_n \in \cO(\pi_1(X;a,b)_{dR}) \otimes \C$.
\[
\mathrm{comp}(\omega_1 \otimes \cdots \otimes \omega_n)(\gamma) = \int_{\gamma} \omega_1 \cdots \omega_n, \qquad \gamma \in \pi_1(X;a,b)(\Q)
\]
Hence we see that iterated integrals are periods of mixed Hodge structures.

Later we would like to apply to this to punctured $\Pspace^1$'s, but the punctures cause problems with the paths.

\section{Unipotent filtration}

\begin{defn}[Unipotent filtration]
Let $G$ be an algebraic group. The unipotent filtration is an ascending filtration on $\cO(G)$ given by
\begin{eqnarray*}
N_0 \cO(G) & = & \Q \\
\subset N_1 \cO(G) & = & \{ f \in \cO(G) \mid \Delta(f) = f \otimes 1 + 1 \otimes f + (\textrm{const}) 1 \otimes 1 \} \\
\subset N_2 \cO(G) & = & \{ f \in \cO(G) \mid \Delta^2(f) \in \cO(G) \otimes \cO(G) \otimes \cO(G) \}
\end{eqnarray*}
\end{defn}

\begin{defn}
A group $G$ is unipotent if and only if $\cO(G) = \cup N_{\ell}\cO(G)$
\end{defn}

We are concerned with the particular case $G = \cO(\pi_1(X;a)^{un})$.

There is a sub-mixed Hodge structure:
\[
\gr^N_m \cO(\pi_1(X;a)^{un}) \subset (\gr^N_1 \cO(\pi_1(X;a)^{un})^{\otimes m} = \Hom(\pi_1(X;a)^{ab}, \Q) = H^(X)
\]

For example, $H^1(X)$ is pure of weight 2 if and only if $H^1(X)$ is pure of type $(1,1)$ if and only if $H^1(\overline{X}, \cO) = 0$ if and only if $H^1_{dR}(\overline{X}) = 0$.

\[
0 \to H^1_{dR}(\overline{X}) \to H^1(X) \to \Q(-1)^{\oplus 2}
\]

\begin{rem}
In this case, the unipotent filtration is the weight filtration, i.e., $N_k = W_{2k}$.
\end{rem}

\begin{defn}[Mixed Tate type]
A mixed Hodge structure $H$ is of \emph{mixed Tate type} if all $\gr_{\ell}^W H$ are direct sums of $\Q(n)_H$.
\end{defn}

\begin{prop}
Under these conditions, $\cO(\pi_1(X;a,b)_H)$ is of mixed Tate type.
\end{prop}


\chapter{Hodge structure: The case of the punctured projective line by Rafael von K\"anel}
%\addcontentsline{toc}{chapter}{*Hodge structure on fundamental groups: The case of the punctured projective line}

Rafael von K\"anel on September 4th, 2012.

\medskip
\medskip

\noindent In the first section, we shall apply the theory of the previous chapters to the variety $X=\overline{X}-D$, where $\overline{X}$ is the projective line over $\C$ and $D$ are finitely many complex points of $\overline{X}$. 
In this fundamental case, we shall be able to describe explicitly the Betti and de Rham realizations of the affine $\mathrm{MH}(k)$-scheme $\pi_1(X;a,b)_H$ associated to $X$ and $a,b$, where $k\subseteq \mathbb \C$ is a field over which all points of $D$ are defined  and  $a,b$ are $k$-rational points of $X$. Furthermore, we shall show that $\pi_1(X;a,b)_H$ is of mixed Tate type and has non-negative weights.


In the second section, we shall use Deligne's theory of fundamental groups with tangential base points to define affine $\MTH$-schemes $\pi_1(X;u,v)_H$ with non-negative weights, where $u,v$ are non-zero tangential vectors to the curve $\overline{X}$ at points in $D$. We also work out explicitly the example $D=\{0,\infty\}$, which will play an important role in the following chapters.

In the last section, we shall apply the theory of tangential base points in the crucial case $X=\overline{X}-\{0,1,\infty\}$. This shall lead to the main result of this chapter which asserts that multi zeta values are real periods of an affine $\mathrm{MTH}(\Q)$-scheme, with non-negative weights.


\section{Hodge structure on the fundamental group of the punctured projective line}\label{sec:hodgepuncteredpline}

Let $\overline{X}$ be the projective line over $\C$ and write $D=\{a_1,\dotsc,a_r,\infty\}$ with $a_1,\dotsc,a_r$ distinct points in a field $k\subseteq \C$, where $r\in \Z_{\geq 1}$. We take $k$-rational points $a,b$ of $X=\overline{X}-D$. The structure sheaf $\mathcal O_{\overline{X}}$ of the smooth compactification $\overline{X}$ of $X$ satisfies  
\begin{equation}\label{eq:h1=0}
H^1(\overline{X},\cO_{\overline{X}})=0.
\end{equation}
We remark that (\ref{eq:h1=0}) holds more generally for any rational variety over $k$, and thus several results in this chapter may be generalized to rational varieties over $k$.

We now consider the de Rham realization $\pi_1(X;a,b)_{dR}$  of the affine $\textnormal{MH}(k)$-scheme $\pi_1(X;a,b)_H$ associated to $X$ and $a,b$. See the previous chapter for a definition of $\pi_1(X;a,b)_H$. 
Let $$\Omega= H^0(\overline{X}, \Omega_{\overline{X}}^1(\log D))$$ be the $k$-vector space of algebraic differential 1-forms on $\overline{X}$, with at most first order poles along $D$. We remark that the differentials $\frac{dz}{z-a_i}, 1\leq i\leq r$, form a basis of the $k$-vector space $\Omega$. Let $T(\Omega)$ be the tensor algebra of $\Omega$. 

The next lemma gives an explicit description of the coordinate ring  $\cO(\pi_1(X;a,b)_{dR})$ of the pro-unipotent group $\pi_1(X;a,b)_{dR}$ over $k$.

\begin{lemma}\label{lem:explicitderham}
There is an isomorphism of Hopf algebras between $\cO(\pi_1(X;a,b)_{dR})$ and $T(\Omega).$
\end{lemma}
\begin{proof}
Let $\Omega_{\overline{X}}^{\bullet}(\log D)$ be the Godement resolution and let $A_X^{\bullet} = R\Gamma(\overline{X}, \Omega^{\bullet}_{\overline{X}}(\log D))$ be the derived complex. 
Since $X$ is an affine curve with (\ref{eq:h1=0}), we obtain a quasi-isomorphism
 $$A_X^{\bullet}\cong k\oplus \Omega[-1].$$ 
On the other hand, we observe that $T(\Omega)$ is isomorphic to the $H^0$ of the bar complex associated to $k\oplus \Omega[-1]$ and the natural morphisms induced by $a,b$, 
and we recall that $$\pi_1(X;a,b)_{dR}\cong\spec H^0(B(A_X^{\bullet};a,b))$$ for $B(A_X^{\bullet};a,b)$ the bar complex associated to $A_X^{\bullet}$ and $a,b$, see Chapter \ref{chapxavier}. 
Then the statement follows from the existence of a quasi-isomorphism between $B(A_X^{\bullet};a,b)$ and the bar complex associated to $k\oplus \Omega[-1]$ and $a,b$, which is induced by the quasi-isomorphism $A_X^{\bullet}\cong k\oplus \Omega[-1]$.
\end{proof}
The above lemma, which relies on computations with the bar complex, shows in particular that $\pi_1(X;a,b)_{dR}$ is independent of $a,b$. 
We remark that the independence of $a,b$ can also be seen without computing with the bar complex. 
Indeed, we get $$\pi_1(X;a,b)_{dR}\cong\underline{\textnormal{Aut}}^{\otimes}_k(\omega)$$ with $\omega$ a fiber functor, independent of $a$ and $b$, of the Tannakian category given by vector bundles on $X$ with unipotent flat connection. 
Moreover, the Tannakian approach allows to show  that $\pi_1(X;a,b)_{dR}$ does not depend on $a,b$ for any smooth variety $X$ over $k$, with smooth compactification  $\overline{X}$ such that $H^1(\overline{X},\cO_{\overline{X}})=0$ and such that $\overline{X}-X$ is a normal crossing divisor. Here one uses the assumption $H^1(\overline{X},\cO_{\overline{X}})=0$ to construct a suitable fiber functor $\omega$ which is independent of $a$ and $b$, see Deligne's \cite[Section 12]{deligne:galoisgroups}.
We mention that further interesting properties of Tannakian categories shall be discussed in Chapter \ref{chapterkonrad}. 



Next, we consider the Betti realization $\pi_1(X;a,b)_{B}$ of $\pi_1(X;a,b)_{H}$. 
Let $\pi_1(X;a)$ be the topological fundamental group of $X$ with base point $a$. 
It is a free group of rank $r$. 
A result of Chapter \ref{chap:alberto} gives that the Lie algebra of the pro-unipotent completion $\pi_1(X;a)^{un}$ of $\pi_1(X;a)$ is the completion of the free graded Lie algebra over $\Q$,
on generators $e_1,\dotsc,e_r$ in degree one, by the lower central series.  We define $$V=(e_1\cdot\Q\oplus\dotsc\oplus e_r\cdot\Q)^{\vee}.$$ 
The generators of $\pi_1(X;a)$ are given by the homotopy classes of simple loops around points in $D$ based at $a$. 
We remark that they are related to the $e_i$. Let $T(V)$ be the Tensor algebra of $V$.

The following lemma provides an explicit description of the coordinate ring  $\cO(\pi_1(X;a,b)_{B})$ of the pro-unipotent group $\pi_1(X;a,b)_{B}$ over $\Q$.

\begin{lemma}\label{lem:explicitbetti}
There is an isomorphism of Hopf algebras between $\cO(\pi_1(X;a,b)_{B})$ and $T(V).$
\end{lemma}
\begin{proof}
The set $\pi_1(X;a,b)$ is a torsor under $\pi_1(X;a)$. Hence the functioriality observations in Chapter \ref{chapteralberto} give a torsor $\pi_1(X;a,b)^{un}$ under  $\pi_1(X;a)^{un}$. 
Let $\cO(\pi_1(X;a,b)^{un})$  be the coordinate ring of the pro-unipotent group $\pi_1(X;a,b)^{un}$ over $\Q$.
The bar complex construction of $\pi_1(X;a,b)_{B}$ in Chapter \ref{chapxavier} implies  $$\cO(\pi_1(X;a,b)_{B})\cong\cO(\pi_1(X;a,b)^{un}).$$  
On the other hand, the computations in Chapter \ref{chapalberto} are applicable to our finitely generated free group $\pi_1(X;a)$ and they show that $\cO(\pi_1(X;a,b)^{un})\cong T(V)$. Therefore we conclude the statement.
\end{proof}

To consider the comparison isomorphism between 
$\pi_1(X;a,b)_{B}\times_{\Q} \C$ and  $\pi_1(X;a,b)_{dR}\times_k \C$, we take $n\in \Z_{\geq 1},$ $\omega_1,\dotsc,\omega_n\in \Omega$  and a homotopy class of paths $[\gamma]\in \pi_1(X;a,b)$. 
It follows that the iterated integral $\int_{\gamma}\omega_1\cdots\omega_n$ depends only on the homotopy class $[\gamma]$. We obtain a morphism
$$\iterch: T(\Omega)\otimes_k \C\to T(V)\otimes_{\Q} \C$$
induced by $\omega_1\otimes\dotsc\otimes\omega_n\mapsto \{[\gamma]\mapsto \int_{\gamma}\omega_1\cdots\omega_n \}$. Now we can state

\begin{prop}\label{prop:hodgeactualbasepoints}
The affine $\textnormal{MH}(k)$-scheme $\pi_1(X;a,b)_H$ has the following properties. 
\begin{itemize}
\item[(i)] The coordinate ring $\cO(\pi_1(X;a,b)_{dR})$ of its de Rham realization $\pi_1(X;a,b)_{dR}$ satisfies
$$\cO(\pi_1(X;a,b)_{dR})\cong T(\Omega),$$
and the weight and Hodge filtration are given by $W_{2m}\cO(\pi_1(X;a,b)_{dR})=\bigoplus_{i\leq m}\Omega^{\otimes i}$  and $F^p\cO(\pi_1(X;a,b)_{dR})=\bigoplus_{i\geq p}\Omega^{\otimes i}$ respectively. 
\item[(ii)] The coordinate ring $\cO(\pi_1(X;a,b)_{B})$ of its Betti realization $\pi_1(X;a,b)_{B}$ satisfies
$$\cO(\pi_1(X;a,b)_{B})\cong T(V),$$
and the weight filtration, which is twice the unipotent filtration induced by the action of $\pi_1(X;a)^{un}$ on $\pi_1(X;a,b)_B\cong \pi_1(X;a,b)^{un}$, has non-negative weights. 
\item[(iii)] The comparison isomorphism is induced by $\iterch$, and $\pi_1(X;a,b)_H$ is of mixed Tate type.
\end{itemize}
\end{prop}
\begin{proof}
Chen's thereom gives that $\iterch$ is an isomorphism and then Lemma \ref{lem:explicitderham} and \ref{lem:explicitbetti} imply all assertions of the proposition, except the second part of (ii) and (iii). 
To show these remaining claims we recall that (\ref{eq:h1=0}) gives $H^1(\overline{X},\cO_{\overline{X}})=0$. Further, we observe that $H^1(X)$ is a finite direct sum of $\Q(-1)$, since $D$ may be identified with a normal-crossing divisor whose irreducible components are all defined over $k$.
Hence all assumptions of the example of the previous chapter are satisfied. 
This example therefore shows that the                                                                                                                                                                                                                                                                                                                                                                                                                                                                                                                                                                                                                                                                                                                                                                                                                                                                                                                                                                                         
                                                                                                                                                                                                                                                                                                                                                                                                                                                                                                                                                                                                                                                                                                                                                                                                                                                                                                                                                                                                                              
                                                                                                                                                                                                                                                                                                                                                                                                                                                                                                                                                                                                                                                                                                                                                                                                                                                                                                                                                                                                                              
                                                                                                                                                                                                                                                                                                                                                                                                                                                                                                                                                                                                                                                                                                                                                                                                                                                                                                                                                                                                                              
                                                                                                                                                                                                                                                                                                                                                                                                                                                                                                                                                                                                                                                                                                                                                                                                                                                                                                                                                                                                                              
                                                                                                                                                                                                                                                                                                                                                                                                                                                                                                                                                                                                                                                                                                                                                                                                                                                                                                                                                                                                                              
                                                                                                                                                                                                                                                                                                                                                                                                                                                                                                                                                                                                                                                                                                                                                                                                                                                                                                                                                                                                                              
                                                                                                                                                                                                                                                                                                                                                                                                                                                                                                                                                                                                                                                                                                                                                                                                                                                                                                                                                                                                                              
                                                                                                                                                                                                                                                                                                                                                                                                                                                                                                                                                                                                                                                                                                                                                                                                                                                                                                                                                                                                                              
                                                                                                                                                                                                                                                                                                                                                                                                                                                                                                                                                                                                                                                                                                                                                                                                                                                                                                                                                                                                                              
                                                          filtration of $\pi_1(X;a,b)_H$ is twice the unipotent filtration which is induced by the action of $\pi_1(X;a)_H$, and that $\pi_1(X;a,b)_H$ is of mixed Tate type. Then the definition of the unipotent filtration implies that all weights of $\pi_1(X;a,b)_H$ are non-negative. Thus we verified all claims and this completes the proof of the proposition.
\end{proof}

\section{Tangential base points}\label{sec:tangentialpoints}

In this section we review Deligne's concept of fundamental groups with tangential base points. 
Here we shall restrict again to the case $X=\overline{X}-D$, where $\overline{X}$ is the projective line over $\C$ and $D$ are finitely many complex points of $\overline{X}$ which are all defined over a field $k\subseteq \C$. 
We mention that Deligne \cite{deligne:galoisgroups} developed his theory for arbitrary smooth projective curves over $k$ and Deligne-Goncharov \cite{dego:motivicfundamentalgroups} studied the case of unirational varieties over $k$.

We take $x,y \in D$. Let $u$ and $v$ be non-zero tangent vectors of $\overline{X}$ at $x$ and $y$ respectively, and let $M$ and $\overline{M}$ be the complex points of $X$ and $\overline{X}$ respectively. We define the set of paths $P_{u,v}$ from $u$ to $v$ by
\begin{align*}P_{u,v} = \{ &\gamma : I \to \overline{M} \mid \textrm{$\gamma$ piecewise smooth}, \gamma(]0,1[) \subset M, \\
&\gamma(0) = x, \gamma(1) = y, \gamma'(x) = u, \gamma'(y) = -v \}
\end{align*}
for $\gamma'(x)$ and $\gamma'(y)$ the tangent vectors of $\gamma$ at $x$ and $y$ respectively; here we choose a coordinate function on the real interval $I=[0,1]$. 
We say that $\gamma_1,\gamma_2\in P_{u,v}$ are homotopic if there is a homotopy $F:I\times I\to \overline{M}$ relating $\gamma_1,\gamma_2$ as paths in $\overline{M}$, with $F\rvert_{\{t\}\times I}\in P_{u,v}$ for any $t\in [0,1]$. 
This gives an equivalence relation $\sim$ on $P_{u,v}$ and then we define
$$\pi_1(M; u,v)=P_{u,v}/\sim.$$ 
Let $w$ be a non-zero tangent vector to $\overline{X}$ at $z\in D$. 
The condition on the tangent vectors allows to define a composition $\gamma=\gamma_1\circ_{\epsilon}\gamma_2$ of paths $\gamma_1\in P_{u,w}$ and $\gamma_2\in P_{w,v}$  such that  $$\circ:\pi_1(M;u,w)\times \pi_1(M;w,v)\to \pi_1(M;u,v),$$ given by $([\gamma_1],[\gamma_2])\mapsto [\gamma_1\circ_{\epsilon}\gamma_2]$,   does not depend on any choices.
Then $(\pi_1(M;u),\circ)$ is a group. 
For example, there is a canonical small counter-clockwise loop $$[\gamma_x]\in \pi_1(M; u)$$ ``almost around" $x$. %Its class is non-trivial, which is a crucial difference to the case of base points in $X$. 
We say that $\pi_1(M;u)$ is the fundamental group of $M$ with tangential base point $u$. 

Let $E$ be a unipotent vector bundle on $\overline{X}$ with a connection $\nabla$, which has at most first order poles along $D$.  From (\ref{eq:h1=0}) we get $$\textnormal{Ext}^1(\cO_{\overline{X}},\cO_{\overline{X}})=H^1(\overline{X},\cO_{\overline{X}})=0.$$ Then on doing induction on the rank of $E$ we see that $E$ is trivial as a vector bundle, since it is an iterated extension of trivial vector bundles. We fix a trivialization of $E$ and we write $$\nabla=d+N, \ \ \ N\in \Omega\otimes_k \textnormal{End}(E),$$ 
where $\Omega=H^0(\overline{X},\Omega^1_{\overline{X}}(\log D))$ is as in the previous section. 
We view $N$ as a matrix with values in $\Omega$. %This matrix is nilpotent, since $\nabla$ is unipotent. 
Let $\epsilon>0$ be a real number and let $U$ be a nilpotent matrix. We write $\mathrm{res}_x(N)$ for the matrix of residues of $N$ at $x$ and we define 
$$\textnormal{res}_x(\nabla)= \textnormal{res}_x(N), \ \ \ \epsilon^U = \exp(\log(\epsilon)U).$$
For any $[\gamma] \in \pi_1(M; u,v)$, we let $\gamma_{\epsilon}:[\epsilon,1-\epsilon]\to M$ be the restriction of $\gamma$ to $[\epsilon,1-\epsilon]$. Next, we define the monodromy along $\gamma$ by the regularization
$$
\int_{\gamma} \nabla= \lim_{\epsilon \to 0} \epsilon^{\mathrm{res}_y(\nabla)} \circ \int_{\gamma_\epsilon}\nabla \circ \epsilon^{-\mathrm{res}_x(\nabla)}.
$$
A local calculation shows that this limit exists. Further, one can prove that $\int_{\gamma} \nabla$ depends only on the homotopy class of $\gamma$. 
For example, we get  $$\int_{\gamma_x} \nabla=\exp(2\pi i\cdot \textnormal{res}_x(\nabla)),$$
where $\gamma_x$ is the canonical counter-clockwise loop ``almost around" $x$ introduced above.

\section{Mixed Hodge structure on the fundamental group with tangential base points}\label{sec:hodgetangential}

We continue the notation introduced above. The purpose of this section is to define an affine $\MTH$-scheme $\pi_1(X;u,v)_H$ with non-negative weights. To construct an analogue of the map $\iterch$ we take $n\in\Z_{\geq 1}, \omega_1, \ldots, \omega_n \in \Omega$
and
$$
N = \left( \begin{array}{cccc}
0 & \omega_1 & \cdots & 0 \\
 & \ddots & \ddots &  \vdots \\
 & &  \ddots & \omega_n \\
 & &     & 0
\end{array}
\right).
$$
We define $\int_{\gamma} \omega_1 \cdots \omega_n$ by the monodromy interpretation from Chapter \ref{chap:fritz} applied to the connection $\nabla=d+N$. This gives the explicit formula,
$$
\int_{\gamma} \omega_1 \cdots \omega_n = \lim_{\epsilon \to 0} \sum_{0 \leq i \leq j \leq n} \frac{(-1)^i}{i!(n-j)!} \prod_{l=1}^i \mathrm{res}_y(\omega_l) \cdot \int_{\gamma_{\epsilon}} \omega_{i+1} \cdots \omega_{i+j} \cdot \prod_{l=j+1}^n \mathrm{res}_x(\omega_l) \log(\epsilon)^{i+n-j}.
$$
Let $\Q[\pi_1(M;u,v)]$ be the $\Q$-vectorspace, freely generated by the set $\pi_1(M;u,v)$, and let $T(\Omega)^\vee$ be the dual of the Tensor algebra of $\Omega$. 
We get a map
$$\iter: \Q[\pi_1(M;u,v)]\to T(\Omega)^\vee\otimes_k\C$$
defined by $\gamma\mapsto \{\omega_1\otimes\dotsc\otimes\omega_n\mapsto\int_\gamma \omega_1\cdots\omega_n\}$. 
The definition of $\int_\gamma \omega_1\cdots\omega_n$  by the monodromy interpretation implies the composition of path formula for iterated integrals, and the explicit formula of $\int_\gamma \omega_1\cdots\omega_n$ shows the product of integrals formula. 
Then, in exactly the same way as in the case of actual base points in $M$, we see that the map $\iter$ induces a morphism
$$\iter^\vee:  T(\Omega)\otimes_k\C\to \cO(\pi_1(M;u,v)^{un})\otimes_{\Q}\C.$$
Here $\cO(\pi_1(M;u,v)^{un})$ is the coordinate ring of the pro-unipotent group $\pi_1(M;u,v)^{un}$ over $\Q$, which is a torsor under the pro-unipotent completion $\pi_1(M;u)^{un}$ of $\pi_1(M;u)$, see Chapter \ref{chap:alberto}. 

\begin{lemma}
The morphism $\iterch$ is an isomorphism.
\end{lemma}
\begin{proof}
We recall that $a$ is a $k$-rational point of $X$. Then we observe that all statements above have a version when we replace $u$ or $v$ by the actual base point $a$.

First, we consider the case when $u$ is replaced by the actual base point $a$.
Then Proposition \ref{prop:hodgeactualbasepoints} gives an isomorphism  of coordinate rings of pro-unipotent groups over $\C$
\begin{equation*}
\iterch_a: T(\Omega)\otimes_k\C\cong \cO(\pi_1(M;a)^{un})\otimes_{\Q}\C.
\end{equation*}
Chapter \ref{chap:alberto} delivers a  $\pi_1(M;a)^{un}$-torsor $\pi_1(M;a,v)^{un}$  coming from the $\pi_1(M;a)$-torsor $\pi_1(M;a,v)$,
and we see that the isomorphism $\iterch_a$ is compatible with the morphism of torsors
\begin{equation*}
\iterch_{a,v}:  T(\Omega)\otimes_k\C\to \cO(\pi_1(M;a,v)^{un})\otimes_{\Q}\C.
\end{equation*}
Then basic torsor theory shows that $\iterch_{a,v}$ is  an isomorphism as well. 

To consider the case where both $u$ and $v$ are tangential base points of $M$, we observe that $\pi_1(M;u,a)$ is a $\pi_1(M;a)$-torsor. Hence, on  replacing $v$ by $u$ in the case above, we obtain
\begin{equation*}
\iterch_{u,a}: T(\Omega)\otimes_k\C\cong \cO(\pi_1(M;u,a)^{un})\otimes_{\Q}\C.
\end{equation*}
We form the product of right and left $\pi_1(M;a)^{un}$-torsors, given by  $\pi_1(M;u,a)^{un}$ and $\pi_1(M;a,v)^{un}$ respectively, and then we quotient by the relation $(g_1g,g_2) \sim (g_1,gg_2)$ with $g\in \pi_1(M;a)^{un}$, $g_1\in \pi_1(M;u,a)^{un}$ and $g_2\in\pi_1(M;a,v)^{un}$. 
It follows that the resulting quotient
$$
\pi_1(M;u,a)^{un} \times \pi_1(M;a,v)^{un}/ \sim
$$
is isomorphic to $\pi_1(M;u,v)^{un}$.
Then, on using the isomorphisms $\iterch_{u,a}$ and  $\iterch_{a,v}$ of pro-unipotent groups over $\C$, we deduce that the morphism
\begin{equation*}
\iter^\vee:  T(\Omega)\otimes_k\C\to \cO(\pi_1(M;u,v)^{un})\otimes_{\Q}\C
\end{equation*}
is an isomorphism as well. This completes the proof of the lemma.
\end{proof}

As in the case of actual base points, we see that 
$\pi_1(M;u)$ is a finitely generated free group of rank $r$, and that the Lie algebra of $\pi_1(M;u)^{un}$ is the completion of the free graded Lie algebra over $\Q$,
on generators $e_1,\dotsc,e_r$ in degree one, by the lower central series. Let $V$ be the dual to the $\Q$-vector space spanned by the $e_i$. We obtain

\begin{prop}\label{prop:hodgetangbasepoints}
There is an affine $\MTH$-scheme $\pi_1(X;u,v)_H$ with the following properties. 
\begin{itemize}
\item[(i)] The coordinate ring $\cO(\pi_1(X;u,v)_{dR}$ of its de Rham realization $\pi_1(X;u,v)_{dR}$ is defined by
$$\cO(\pi_1(X;u,v)_{dR})=T(\Omega),$$
and the weight and Hodge filtration are given by $W_{2m}\cO(\pi_1(X;u,v)_{dR})=\bigoplus_{i\leq m}\Omega^{\otimes i}$ and $F^p\cO(\pi_1(X;u,v)_{dR})=\bigoplus_{i\geq p}\Omega^{\otimes i}$ respectively. 
\item[(ii)] The coordinate ring $\cO(\pi_1(X;u,v)_{B})$ of its Betti realization $\pi_1(X;u,v)_{B}$ satisfies
$$\cO(\pi_1(X;u,v)_{B})=\cO(\pi_1(M;u,v)^{un})\cong T(V),$$
and the weight filtration, which is twice the unipotent filtration induced by the action of $\pi_1(M;u)^{un}$ on $\pi_1(X;u,v)_{B}$, has non-negative weights. 
\item[(iii)] The comparison isomorphism is induced by $\iterch$.
\end{itemize}
\end{prop}
\begin{proof}
All claims follow on combining the above lemma with exactly the same arguments as used in the proof of Proposition \ref{prop:hodgeactualbasepoints}.
\end{proof}

\section{The case of $\Gm$}

We continue the notation of the previous sections. In addition, we assume that $X=\overline{X}-\{0,\infty\}$. 
Then we get that $k=\Q$, that $X$ is the multiplicative group $\Gm$ and that $\pi_1(M;a)$ is generated by the homotopy class $[\gamma]$ of a simple loop $\gamma$, around 0, based at a $\Q$-rational point $a$ of $X$. Here $M$ denotes the complex points of $X$. 
In particular, it follows that the group $\pi_1(M;a)$ is commutative and does not depend on the choice of $a$. Further, we obtain
\begin{prop}
There is an isomorphism in $\textnormal{Ind MTH}(\Q)$ $$\cO(\pi_1(X;a,b)_H)\cong \bigoplus_{n \geq 0}\Q(-n)_H\cong T(\Q(-1)_H).$$ 
\end{prop}
\begin{proof}
To prove this proposition we may and do assume that $a=b$. We now explicitly compute the comparison isomorphism of 
$\pi_1(X;a)_H$. In our case of $D=\{0,\infty\}$, we get that the $\Q$-vector spaces $V$ and $\Omega$ of Section \ref{sec:hodgepuncteredpline} satisfy $V=e_1^{\vee}\cdot \Q$ and  $\Omega=\frac{dz}{z}\cdot \Q$. 
To simplify notation we write $\omega=\frac{dz}{z}$ and $e=e_1^{\vee}$. 
Let $n\geq 1$ be an integer. We obtain $$n!\int_\gamma \omega^{\otimes n}=\left(\int_\gamma \omega\right)^n=(2\pi i)^n,$$
where the first equality follows from the shuffle product formula. Further, we recall that the isomorphism $\cO(\pi_1(M;a)^{un})\cong T(V)$ is induced by $\gamma\mapsto \exp(e_1)$ and that $\iterch$ is defined by $\omega^{\otimes n}\mapsto \{\gamma\mapsto \int_\gamma \omega^{\otimes n}\}$. Then we deduce  $$\iterch(\omega^{\otimes n})=(2\pi i)^ne^{\otimes n}.$$ This implies the statement, since Proposition \ref{prop:hodgeactualbasepoints} shows that $\iterch$ induces the comparison isomorphism of $\pi_1(X;a)_H$. 
\end{proof}

We remark that the proof of the above proposition shows in addition that $\pi_1(X;a)_{dR}=\Ga$ and that the comparison isomorphism identifies $\pi_1(\Gm;a)_B(\Q)$ with $2\pi i\Q$ in $\Ga(\C)$. 

By abuse of notation, we denote by $\Q(1)_H$ the affine $\textnormal{MTH}(\Q)$-scheme that corresponds to $T(\Q(-1)_H)$. More generally, we now return to the setup of Section \ref{sec:hodgetangential} and we assume that $X=\overline{X}-D$ is the projective line $\overline{X}$ over $\C$ without finitely many $k$-rational points $D$ of $\overline{X}$, where $k\subseteq \C$ is a field. 
Let $\pi_1(X;u)_H$ be the affine $\MTH$-scheme associated to $X$ and $u$, where $u$ is a tangential base point of $X$. We get

\begin{prop}
There is a morphism $\Q(1)_H \to \pi_1(X;u)_H$ of affine $\MTH$-schemes.
\end{prop}
\begin{proof}
We start with the Betti realization, and we recall that $u$ is a non-zero tangent vector to $\overline{X}$ at $x\in D$. 
Let $\gamma_x$ be a small counter-clockwise simple loop around $x$. 
The morphism $\Z\to \pi_1(M;u)$, defined by $1 \mapsto \gamma_x$, induces a map between the coordinate rings of the corresponding pro-unipotent completions $\cO(\Z^{un})\to \cO(\pi_1(M;u)^{un})$, where $M$ is  the set of complex points of $X=\overline{X}-D$. 
This gives a morphism
$$\cO(\Q(1)_B)\to \cO(\pi_1(X;u)_B),$$
since $\cO(\Q(1)_B)\cong \cO(\Z^{un})$ and  $\cO(\pi_1(M;u)^{un})=\cO(\pi_1(X;u)_B)$. Next, we consider the de Rham realization. 
Taking residues of differential forms at $x$ gives a morphism $\mathrm{res}_x : \Omega \to k$, where we recall that $\Omega=H^0(\overline{X},\Omega^1_{\overline{X}}(\log D))$. This induces a morphism
$$
\cO(\pi(X;u)_{dR})\to \cO(\Q(1)_{dR})\otimes_{\Q} k,
$$
since $\cO(\pi(X;u)_{dR})=T(\Omega)$ and $T(k)\cong \cO(\Q(1)_{dR})\otimes_{\Q}k$.
We now use that iterated integrals are defined via the monodromy interpretation and that $\iterch$ induces the comparison isomorphisms, see Proposition \ref{prop:hodgeactualbasepoints} and \ref{prop:hodgetangbasepoints}. 
Then we see that the above displayed morphisms between the Betti and de Rham realizations are compatible with the comparison isomorphisms. Hence, they define a morphism $\Q(1)_H \to \pi_1(X;u)_H$ of affine $\MTH$-schemes as desired.
\end{proof}


\section{MZV as periods of an affine $\textnormal{MTH}(\Q)$-scheme with non-negative weights}

In this section, we apply the theory of tangential base points in the fundamental case $X=\overline{X}-D$, where
$D= \{0,1,\infty\}$ and  $\overline{X}$ is the projective line over $\C$. Let $u$ and $v$ be the tangential vectors $(0,1)$ and $(1,0)$ of $\overline{X}$ at 0 and 1 respectively. An application of Proposition \ref{prop:hodgetangbasepoints}, with $X$ and $u,v$, gives an affine $\textnormal{MTH}(\Q)$-scheme $\pi_1(X;u,v)_H$, with non-negative weights. We write $\pi_1(X;0,1)_H=\pi_1(X;u,v)_H$ and we obtain

\begin{thm}
Multi zeta values are real periods of $\pi_1(X;0,1)_H$.
\end{thm} 
\begin{proof}
In the case $D=\{0,1,\infty\}$, we get $k=\Q$ and $\Omega = \omega_0\cdot\Q \oplus \omega_1\cdot\Q$ with $\omega_0=\frac{dz}{z}$  and $\omega_1=\frac{dz}{1-z}$. 
For any $m,n_1,\dotsc,n_{m-1}\in \Z_{\geq 1}$ and $n_m\in \Z_{\geq 2}$, we define $\overline{n} = (n_1, \ldots, n_m)$. 
As in Definition \ref{def:word}, we see that $\overline{n}$ corresponds to a word $w(\overline{n})$ in $0$ and $1$ starting in $0$ and ending in $1$. 
Let $\overline{w} = w(\overline{n})$ be the corresponding sequence in $w_0$ and $w_1$. 
We denote by $\mathrm{dch}$ the real interval $[0,1]$ viewed as a path in $\overline{M}$ from 0 to 1, where we recall that $M$ and $\overline{M}$ are the complex points of $X$ and $\overline{X}$ respectively. It follows  
$$[\mathrm{dch}] \in \pi_1(M;u,v).$$ 
For any real $\epsilon>0$, let $dch_{\epsilon}:[\epsilon,1-\epsilon]\to M$ be the restriction of dch to the closed interval $[\epsilon,1-\epsilon]$. Then  a computation in Chapter \ref{chap:fritz} gives
$
\zeta(\overline{n}) = \lim_{\epsilon \to 0} \int_{dch_{\epsilon}} \overline{w},
$
and the definitions show that $\lim_{\epsilon \to 0}\int_{dch_{\epsilon}}\overline{w}=\int_{dch} \overline{w}
= \iterch(\overline{w})(\mathrm{dch})$. On combining these equalities we deduce
$$\zeta(\overline{n})=\iterch(\overline{w})(\mathrm{dch}).$$
Proposition \ref{prop:hodgetangbasepoints} gives that $\iterch$ induces the comparison isomorphism. Therefore the above displayed representation of $\zeta(\overline{n})$ shows that multi zeta values are real periods of $\pi_1(X;0,1)_H$. This completes the proof of the theorem.
\end{proof}



\chapter{Motivic structure: Tannakian categories by Konrad V\"olkel}
%\addcontentsline{toc}{chapter}{**Motivic structure on the fundamental group: Tannakian categories by Konrad V\"olker}

Konrad V\"olkel on September 6th, 2012.

% Was \semidirect supposed to be \rtimes or \ltimes? I tried \rtimes.
\medskip
\medskip

\indent After reminding you of the basic properties of Tannakian categories, along some examples, we will work out a strategy to prove upper bounds on periods of mixed Tate motives, which will be used by the following talk, and which will finally lead to the theorem of Goncharov-Terasoma. The strategy consists of defining a weigth filtration on the real periods, obtained as fixed points of all periods under complex conjugation, and then to show that it's a quotient of a certain explicit graded $\Q$-subalgebra $\mathcal{O}(I(\omega,\eta))^\epsilon_+ \subseteq \mathcal{O}(I(\omega,\eta))$, whose Poincar\'e series can be computed. In the proof, we exhibit the structure of graded k-algebra
\[\mathcal{O}(I(\omega,\eta))^\epsilon_+ \simeq \mathcal{O}(\Gm \rtimes U) = k[t,t^{-1}] \otimes_k T\left(\bigoplus_{n>0}Ext^1_{MTM(\Z)}\left(\Q(0),\Q(n)\right)\right).\]
Since we don't have mixed Tate motives over $\Z$ at hand at this time, we will do this in a very abstract setting, using mixed Tate Hodge structures as main example. We will have to make several assumptions for the proofs, all of which will be proven in the case of mixed Tate motives over $\Z$ in the next talk. Mixed Tate Hodge structures don't suffice for the proof of Goncharov-Terasoma, since the Ext-groups are too large.

% Basic notions:
% - def neutral tannakian cat over a field
% - def fiber functor
% - example G-rep for G linear pro-algebraic
% - def fundamental group
% - def isomorphism scheme
% - example tensor generated by one object

\section{Basic notions}

\begin{defn}
 A \emph{tensor category} over $k$ is an abelian category $\mathcal{C}$ enriched over $k$-vector spaces, equiped with a ``tensor'' product $k$-bifunctor $\otimes : \mathcal{C} \times \mathcal{C} \to \mathcal{C}$ and an ``identity'' object $1 \in \mathcal{C}$ together with natural isomorphisms $(\cdot) \otimes 1 \cong \mathrm{id}$ and $1 \otimes (\cdot) \cong \mathrm{id}$ that make the identity object worth its name, and natural isomorphisms $\alpha_{A,B,C} : (A \otimes B) \otimes C \cong A \otimes (B \otimes C)$, that are called \emph{associators}, which have to obey the pentagon and the triangle identites. From MacLane's coherence theorem, this implies all further combinations of associators and identities that can be isomorphic, are isomorphic. It is called \emph{symmetric} if there are symmetry isomorphisms $A \otimes B \cong B \otimes A$. These conditions are often abbreviated to ACU (for associativity, commutativity, unit). We require furthermore that a tensor category is \emph{rigid}, i.e. internal Homs exist, and 
$End(1)\simeq k$.

 A neutral Tannakian category over $k$ is a rigid abelian tensor category over $k$, together with a $k$-tensor functor to $k$-vector spaces, that is exact and faithful. A Tannakian category is a category which admits a functor to $k$-vector spaces that makes it neutral Tannakian (in other words: we don't fix the fiber functor).
\end{defn}

One can do the same not only for $k$-vector spaces but for quasicoherent sheaves over a $k$-scheme $S$. Deligne discusses this in detail, but we won't need the more general theory right now.

\begin{exam}
 Let $G$ be a linear pro-algebraic group over $k$, e.g. $\SL_{\infty} = \lim \SL_n$ over $\Q$, then the category of all finite-dimensional $k$-rational representations is a rigid abelian tensor category over $k$,
 and the forgetful functor to $k$-vector spaces is exact and faithful, making $\mathrm{Rep}(G)$ a neutral Tannakian category.
\end{exam}

\begin{prop}
 Let $\omega : \mathrm{Rep}(G) \to \Vect(k)$ be the forgetful functor. Then the tensor-automorphisms $G_\omega := \mathcal{A}ut^{\otimes}_k(\omega)$ form a linear pro-algebraic group, called \emph{fundamental group}. There is a canonical isomorphism $G \to G_\omega$.
\end{prop}
\begin{proof}
 Let $R$ be a $k$-algebra. The $R$-points of $G_\omega$ are the automorphisms $\mathcal{A}ut^{\otimes}_k(\omega)(R)$, i.e.,
\[
G_\omega(R) = \left\{ (\lambda_X)_{X \in \mathrm{Rep}(G)} 
\left| \begin{array}{c}
 \lambda_X \in \Aut(X \otimes R) \qquad \lambda_X\ R\textrm{-linear} \qquad \lambda_{X\otimes Y} = \lambda_X \otimes \lambda_Y \\ 
 \lambda_1 = \id_R \textrm{ and for all } G\textrm{-equivariant } \alpha : X \to Y \\ 
 \lambda_Y \circ (\alpha \otimes 1) = (\alpha \otimes 1) \circ \lambda_X : X \otimes R \to Y \otimes R
\end{array} \right.
 \right\}
 \]
so we can map $G(R) \to G_\omega(R)$, since every $g \in G(R)$ acts on every $G$-representation $X$ tensored with $R$.
This gives $G \to G_\omega$ and is in fact an isomorphism of functors of $k$-algebras:

We restrict to the full subcategory $\mathcal{C}_X$ of subquotients of any sum of tensor powers of $X$ and $X^\vee$ for a fixed object $X \in \mathcal{C}$. Then $\Aut^{\otimes}(\omega|_{\mathcal{C}_X})(R)$ can be considered as a subgroup of $\GL(X \otimes R)$ by $\lambda \mapsto \lambda_X$. Let $G_X$ be the image of $G$ in $\GL_X$, which is a closed algebraic subgroup, then we have
\[G_X(R) \subseteq \Aut^{\otimes}(\omega|_{\mathcal{C}_X})(R) \subseteq \GL_X(R) = \GL(X \otimes R).\]
If $V \in \mathcal{C}_X$ and $t \in V^G$, then the 1-parameter group $\alpha : k \to V,\ a \mapsto at$ is $G$-equivariant, and so $\lambda_V(t \otimes 1) = t \otimes 1$.
Thus $\Aut^{\otimes}(\omega|_{\mathcal{C}_X})$ is the subgroup of $\GL_X$ fixing all tensors in representations of $G_X$ fixed by $G_X$, which implies that $G_X = \Aut^{\otimes}(\omega|_{\mathcal{C}_X})$.

Now this works for all $X$, and we can take a limit construction to get the general result.
\end{proof}



\begin{thm}[Tannakian Reconstruction]
 Let $\mathcal{C}$ be a neutral Tannakian category with fiber functor $\omega$. Then the representation category of the fundamental group $G_\omega := \mathcal{A}ut^{\otimes}_k(\omega)$, as a neutral Tannakian category $(\mathrm{Rep}(G_\omega),\omega_\textrm{forget})$, is canonically equivalent to $(\mathcal{C},\omega)$.
\end{thm}

 Vague proof idea: We can write down a functor $(\mathcal{C},\omega) \to (\mathrm{Rep}(G_\omega),\omega_\textrm{forget})$ by mapping any $S \in \mathcal{C}$ to $\omega(S)$ with the $G_\omega$-action given by
\[
\mathrm{Aut}_k^{\otimes}(\omega) \times \omega(S) \to \omega(S),\qquad (\alpha,v) \mapsto \alpha_{S}(v).
\]


\begin{defn}
 Let $I(\omega,\eta) := \mathrm{Iso}^{\otimes}_k(\omega,\eta)$ be the isomorphism scheme from one fiber functor $\omega : \mathcal{C} \to \Vect(k)$ to another $\eta$. The $S$-points of this scheme (for $u : S \to k$ a $k$-scheme) consists of the set of isomorphisms of the fiber functor $u^\ast \omega$ with $u^\ast \eta$.

The scheme $I(\omega,\eta)$ is a right torsor under $G_\omega$ and a left torsor under $G_\eta$.
If $\mathcal{C}$ is tensor generated by a single object $S$, then $I(\omega,\eta)$ is a closed subscheme of $\mathrm{Iso}_k(\omega(S),\eta(S))$, the relations corresponding to the coherence constraints on the tensor product.
\end{defn}

















% Examples
% - ex graded vector spaces, Gm
% - ex loc sys and fibers
% - ex unipot completion of abstract group
% - ex MTHS/Q with deRham and Betti fiber functors and comparison

\section{Examples}

\begin{exam}
 The category of $\Z$-graded vector spaces over a field $k$ is neutral Tannakian with fiber functor the forgetful functor to ungraded $k$-vector spaces. A $\Z$-grading can be thought of as the weight grading of a $\Gm$-representation, where $\lambda \in \Gm(k)$ acts as multiplication with $\lambda^{-n}$ on the $n$th graded part. This shows that the category of $\Z$-graded vector spaces over a field $k$ is equivalent to the category of $k$-rational $\Gm$-representations, with forgetful functors on both sides corresponding to each other.
\end{exam}


\begin{exam}
 Take an abstract group $G$ and the subcategory of all finite-dimensional representations in $k$-vector spaces where $G$ acts by unipotent matrices. This is a Tannakian subcategory (sums and tensor products of unipotent representations are still unipotent). Its fundamental group wrt. the forgetful functor is then the solution to the universal problem of a group over which unipotent representations factor, and it is called the unipotent completion of $G$. It coincides with the Malcev completion because it satisfies the same universal property. In short:
\[\mathrm{Rep}^{un}(G) \simeq \mathrm{Rep}(G^{un}).\]
\end{exam}

\begin{exam}
 The category $MTH(\Q)$ carries at least two interesting fiber functors: deRham realization
\[\omega_{dR} : MTH(\Q) \to \Vect(\Q),\qquad (H,H_{dR},W_\bullet,F^\bullet,\alpha) \mapsto H_{dR})\]
 and Betti realization
\[\omega_B : MTH(\Q) \to \Vect(\Q),\qquad (H,H_{dR},W_\bullet,F^\bullet,\alpha) \mapsto H.\]
There is a comparison isomorphism over $\C$, i.e. a $\C$-point of $I(\omega_{dR},\omega_B)$,
given by $\alpha : H_{dR} \otimes \C \cong H \otimes \C$.
\end{exam}


% \begin{exam} %TODO not sure if explicit enough. Maybe should elaborate the topological case.
%  It is a classical result that for $(X,x)$ a topological space with basepoint $x \in X$, the category of local systems is equivalent to the category of $\pi_1(X;x)$-representations.
%  The same can be said for $X$ a scheme with $x \in X(k)$ a rational point, then local systems $E$ can be reconstructed from their monodromy representations, i.e. $E_x$ as $\pi_1^{et}(X,x)$-set.
% 
% %  Let $X$ be a smooth $k$-scheme and $Loc(X)$ the category of local systems (that is, locally constant \'etale sheaves). From any local system $E \in Loc(X)$ and any element $\gamma \in \pi_1^{et}(X,x)$ we get a chain of isomorphisms of fibers
% % \[E_{\gamma(0)} \cong E_{\gamma(t_1)} \cong \cdots \cong E_{\gamma(t_{n-1})} \cong E_{\gamma(0)}\]
% % for some segmentation $[0,1] = [t_0,t_1] \sqcup [t_1,t_2] \sqcup \cdots \sqcup [t_{n-1},t_n]$, where $t_0=0$, $t_1 = 1$ and $\gamma(t_i)$ are points in ... ahrg
% % 
% %  The category of local systems on a topological space $X$, name it $Loc(X)$, has a forgetful functor to $k$-vector spaces which gives just the fiber at a fixed basepoint $x \in X$. The $\otimes$-automorphisms are precisely the monodromy of all local systems, i.e. $\pi_1(X;x)$.
% \end{exam}










% Upper bounds on periods
% - def periods wrt field extension K/k and comparison iso and object S in Tannakian cat C
% - proposition periods as subquotient of certain Hopf algebra
% - proof
% - def involution action on periods and Hopf algebra


\section{Upper bounds on periods}

\begin{defn}
 Let $K/k$ be a field extension, $S \in \mathcal{C}$ an ind-object in a Tannakian category $\mathcal{C}$ with two fiber functors $\omega, \eta : \mathcal{C} \to \Vect(k)$ and a point $p \in I(\omega,\eta)(K)$. Then a \emph{period} of this data is an element of the $k$-vector space $P \subset K$ of \emph{periods} generated by the numbers $\langle \alpha,p^\vee \beta\rangle$ for all $\alpha \in \omega(S)$ and all $\beta \in \eta(S)^\vee$.
\end{defn}

We can use this to define various maps, in particular the main actor of this talk:
\begin{defn}
 Let $S$ be an object in $\mathcal{C}$.
 We define a $k$-linear map
\[\psi: \omega(S) \otimes_k \eta(S)^\vee \to \mathcal{O}(I(\omega,\eta))\]
which assigns to every $\gamma \otimes \sigma \in \omega(S) \otimes_k \eta(S)^\vee$ the function on $I(\omega,\eta)$ that sends a point $p \in I(\omega,\eta)(K)$ to the value of $\gamma \otimes p^\vee(\sigma) \in \omega(S)_K \otimes_K \omega(S)^\vee_K$ under the canonical pairing (i.e. evaluation map).
\end{defn}

\begin{prop}
 Periods $P \subset K$ for fixed $(k,K,\mathcal{C},S,\omega,\eta,p)$ are a quotient of the subset of the Hopf algebra $\mathcal{O}(I(\omega,\eta))$ which is generated by the image of $\omega(S) \otimes_k \eta(S)^\vee$ under $\psi$.
\end{prop}
\begin{proof}
 With the point $p \in I(\omega,\eta)(K)$ we can define an evaluation map $p^\ast : \mathcal{O}(I(\omega,\eta)) \to K$, whose concatenation with $\psi$ gives a generating set for the period $k$-vector space $P \subset K$ of $(k,K,\mathcal{C},S,\omega,\eta,p)$. The evaluation map is $k$-linear and the map $\mathrm{Span}_k\psi\left(\omega(S) \otimes_k \eta(S)^\vee\right) \to P$ is surjective by definition of $P$.
\end{proof}

\begin{lemma}
 Let $c : K \to K$ be a field involution over $k$, assume $\mathrm{char~} k \neq 2$. Suppose $c$ extends (not necessarily uniquely) to an involution $\tilde{c}$ of $I(\omega,\eta)$ over $k$ that commutes with $p : \spec(K) \to I(\omega,\eta)$, i.e. $c \circ p^\ast = p^\ast \circ \tilde{c}$.
Then we have not only the $c$-fixed periods $P^c$, but also the $\tilde{c}$-fixed space $\mathcal{O}(I(\omega,\eta))^{\tilde{c}}$,
and $\mathcal{O}(I(\omega,\eta))^{\tilde{c}} \to P^c$ is still surjective.

In particular,
The $c$-fixed periods $P^c$ are a subquotient of $\mathcal{O}(I(\omega,\eta))^{\tilde{c}}$.
\end{lemma}
\begin{proof}
 Let $x \in P^c$, then there is a preimage $y \in \mathcal{O}(I(\omega,\eta))$, and $(y+\tilde{c}(y))/2 \in \mathcal{O}(I(\omega,\eta))^{\tilde{c}}$ is a preimage of $x$ (which fails for characteristics $2$).
\end{proof}


 In the application, the involution $c$ of $I(\omega,\eta)$ will be complex conjugation on the $\C$-points of varieties, which is a natural choice.


























% =========================================================================================================


% Pro-unipotent group
% - now char k = 0
% - lemma fundamental group pro-unipotent iff all objects filtered with unit object subquotients
% - proposition given some cohomology finiteness and vanishing assumptions, the Hopf algebra of the fundamental group is a tensor algebra over some Ext-group
% - proof

\section{Pro-unipotent groups}

Now we work only in characteristics $0$ (since unipotent groups in positive characteristics behave worse than in characteristics $0$).
Furthermore, we assume in this section the fundamental group $G_\omega$ of our Tannakian category $(\mathcal{C},\omega)$ to be pro-unipotent.

\begin{lemma}
 The fundamental group $G_\omega$ of a neutral Tannakian category $(\mathcal{C},\omega)$ is a pro-unipotent algebraic group if and only if every object $S \in \mathcal{C}$ has a filtration such that the subquotients are isomorphic to the unit object $1 \in \mathcal{C}$.
\end{lemma}
\begin{proof}
If every object has such a filtration, then in particular $\mathcal{O}(G_\omega)$ has one, and therefore is pro-unipotent.

If, on the other hand, $G_\omega$ is pro-unipotent, then every $S$ has a unipotent filtration, and this can be refined to one with the properties of the lemma.
\end{proof}


\begin{lemma}\label{lem:prounisurj}
 Whenever a morphism of pro-unipotent groups $f : G' \to G$ is surjective on $H_1(-,k)$, it is already surjective.
\end{lemma}
\begin{proof}
Since $H_1(G,k)^\vee \simeq H^1(G,k)$, the fact that $H_1 f$ is surjective implies that $H^1 f$ is injective.
Also, surjectivity of $f$ is equivalent to injectivity of $f^\natural : \mathcal{O}(G) \to \mathcal{O}(G')$.

We know from a previous talk that $V := H^1(G,k) \simeq \gr_1^N \mathcal{O}(G)$, since $G$ is pro-unipotent.
There is a canonical morphism $\gr_\bullet^N \mathcal{O}(G) \inj T(\gr_\bullet^1 \mathcal{O}(G)) = T(V)$,
which commutes with the morphisms $\gr_\bullet f$ and $T(\gr_\bullet^1 f^\natural)$, so the former is injective:
\[
\xymatrix{
\gr_{\bullet}^N \cO(G) \ar[d]^{\mathrm{can}} \ar[r]^{\gr_{\bullet} f^\natural} & \gr_{\bullet}^N \cO(G') \ar[d]^{\mathrm{can}} \\
T(V) \ar[r]_{T(\gr_1 f^\natural)} & T(V')
}
\]
The result follows from the fact that a morphism which is injective on $\gr_\bullet$ is already injective,
since the kernel has $\gr_\bullet = 0$, so it must vanish as well.
\end{proof}



\begin{prop}
 Suppose $V := \Ext^1_{\mathcal{C}}(1,1)$ has $k$-dimension $r < \infty$ and $\Ext^2_{\mathcal{C}}(1,1) = 0$. Then there is an isomorphism of Hopf algebras $\mathcal{O}(G_\omega) \cong T(V)$.
\end{prop}
\begin{proof}
The proof proceeds in two steps:
First we construct a surjective morphism $\alpha : \spec T(V) \to G_\omega$, then we show it splits and must be an isomorphism.

\subsubsection{Step 1: Construction of $\alpha$}
We remember that $T(V)$ is the Hopf algebra of the pro-unipotent completion of a free group in $r$ generators.

Choose $\gamma_1,\ldots,\gamma_r \in G_\omega(\Q)$ minimal such that their image is a basis for
\[G_\omega(\Q)^{ab} = H_1(G_\omega) = H^1(G_\omega;k)^\vee = V^\vee.\]
Then we have a map from the free group generated by the $\gamma_i$ to $G_\omega(\Q)$.
From the universal property of the pro-unipotent completion of this free group we get a morphism $\alpha : \langle \gamma_1,\ldots,\gamma_r\rangle^{un} \to G_\omega$ (which depends on the choice of the $\gamma_i$).
It is also surjective on $H_1$ by construction, so from the previous lemma, it is surjective.


\subsubsection{Step 2: $\alpha$ splits and is an isomorphism}
From $0 = \Ext^2_{\mathcal{C}}(1,1) = H^2(G_\omega;k)$ classifying the extensions of $G_\omega$ by $\Ga$, we see that all these must be split extensions.

Since $\spec T(V)$ is pro-unipotent, we have an extension
\[U \inj \spec T(V) \surj G_\omega\]
with $U$ pro-unipotent as well. We look at the abelianization of $U$
\[U/[U,U]=\Ga^m \inj \spec T(V)/[U,U] \surj G_\omega\]
and this extension splits, giving a section $s : G_\omega \to \spec T(V)/[U,U]$.
Since $s \circ \alpha$ on $\{\gamma_1,\ldots,\gamma_r\} \subset \spec T(V)$ is the identity,
$s$ is an isomorphism, and in particular $U/[U,U] = 0$, so $U=0$ and we're finished.
\end{proof}








 
% Semi-direct product of a pro-unipotent group with Gm
% - assume C tensor generated by extensions of a fixed rank one object L, with morphisms respecting the corresponding weight structure
% - def canonical fiber functor
% - lemma fundamental group a semi-direct product of a pro-unipotent group with Gm
% - example MTH(k), L=Q(1)_H, w_n=W_2n, canonical fiber functor = deRham, then MTH(k)=Rep(Gm semi U_H)
% - proposition given some cohomology finiteness and vanishing assumptions, the Hopf algebra of the fundamental group is a tensor algebra over some direct sum of Ext-groups
% - proof
\section{Semi-direct product of a pro-unipotent group with Gm}

From now on, we assume that $\mathcal{C}$ is generated by extensions of tensor powers of a fixed rank one object $L$ (think: line bundle) and its dual $L^{-1} := L^\vee$.
Note that $G_\omega$ is no longer pro-unipotent, but we will show that it is still almost pro-unipotent.

 In other words: every object $S \in \mathcal{C}$ carries an increasing filtration $w_nS$, $n \in \Z$, whose $n$th adjoint quotient $gr^w_{n} = w_{n} S / w_{n-1} S$ is a direct sum of several copies of $L^{\otimes(-n)}$. We assume this filtration to be exact (in particular, $\gr^w_n$ is exact), respecting morphism in $\mathcal{C}$ and the tensor structure of $\mathcal{C}$, in particular $\Hom_{\mathcal{C}}(1,L^{\otimes n}) = 0$ for all $n \neq 0$ and

\begin{prop}
  $\Ext^1_{\mathcal{C}}(1,L^{\otimes (-n)}) = 0$ for $n \geq 0$ 
\end{prop}
\begin{proof}
Let $L^{\otimes (-n)} \inj S \surj 1$ be an extension,
then apply $\gr_0^w$ to get $\gr_0^w S \simeq 1$,
apply $\gr_n^w$ to get $\gr_n^w S \simeq L^{\otimes (-n)}$.
We have $1 = w_0 S \inj S$ and $L^{\otimes (-n)} \inj S$,
so we can form the direct sum and get an exact sequence
\[\ker \inj w_0 S \oplus L^{\otimes (-n)} \to S \surj \coker\]
where $\ker$ is pure of weight $n$ and $\coker$ is pure of weight $0$.
Applying $\gr_n$ and $\gr_0$ to the sequence show then that $\ker = 0$ and $\coker = 0$,
so the extension splits.
\end{proof}


\begin{defn}
 In this setting, one has a \emph{canonical fiber functor}
\[\omega : S \mapsto \bigoplus_{n \in \Z} \Hom_{\mathcal{C}}\left( L^{\otimes (-n)}, \gr_n^w S \right)\]
from $\mathcal{C}$ into $\Z$-graded $k$-vector spaces.
This defines a dual morphism of Tannakian fundamental groups $\Gm \to G_\omega$.
\end{defn}
\begin{proof} (that it is indeed a fiber functor)
The functor $\omega$ is $k$-linear, since $\gr_n^w$, covariant $Hom$ and $\bigoplus$ are $k$-linear.
It is a $\otimes$-functor, since the weight filtration respects the $\otimes$-structure.
It is also exact, since for $S' \inj S \surj S''$ we have $\Ext^1_{\mathcal{C}}(L^{\otimes (-n)},\gr_n^w S') = 0$
\end{proof}


\begin{lemma}
 The corresponding fundamental group has the form $G_\omega \simeq \Gm \rtimes U$ with $U$ a pro-unipotent group.
The category $\mathcal{C}$ is equivalent to the category of graded comodules over $\mathcal{O}(U)$.
\end{lemma}
\begin{proof}
Look at $\mathcal{C}' \subset \mathcal{C}$, the subcategory $\otimes$-generated by $L$. It has no non-trivial extensions,
and a natural grading, given by the tensor powers of $L$ that appear. This makes it a category of $\Gm$-representations,
which gives us a dual morphism of Tannakian fundamental groups $G_\omega \surj \Gm$,
with kernel $U$ a pro-unipotent group, since $L$ is the trivial $\Gm$-module, and $\mathcal{C}$ contains all iterated extensions of this trivial $\Gm$-module. The morphism induced by $\omega$ splits $G_\omega \surj \Gm$, since $\omega$ is a retraction of the full inclusion $\mathcal{C}' \inj \mathcal{C}$. This shows $G_\omega \simeq \Gm \rtimes U$.
\end{proof}

\begin{exam}
 Let $\mathcal{C} = MTH(k)$ and $L=\Q(1)_H$, with corresponding weight filtration $w_n = W_{2n}$. Then the canonical fiber functor coincides with the deRham realization functor $\omega = \omega_{dR}$, which amounts to
\[H_{dR}^n = \Hom_{MTH(k)}(\Q(-n),\gr^W_{2n} H),\]
which is a reformulation of the definition, that the $n$-th weight-graded part of $H$ is a pure Hodge structure $H_{dR}^n$ over $k$.

We have $G_{H} = \Gm \rtimes U_{H}$ and $MTH(k)$ is equivalent to the category of graded $\mathcal{O}(U_H)$-comodules.
\end{exam}

\begin{defn}
 We consider the grading on the tensor algebra $T(V)$ of the graded vector space $V = \bigoplus_{n > 0} V_n$ with $V_n = \Ext^1_{\mathcal{C}}(1,L^{\otimes n})$ to be such that $v \otimes w$ has degree $|v| + |w|$ (rather than $|v|+|w|+2$, which would also give a graded algebra).
\end{defn}


We would love to have something like this (which, by the way, doesn't make any sense):
\begin{lemma}
 The graded Hopf algebra $T(V)$ is the universal graded pro-unipotent group such that every morphism of graded groups from a free group in $r_n$ generators of degree $n$ into the $\Q$-points of a graded pro-unipotent group $G$ induces a morphism of graded pro-unipotent groups $T(V) \to G$.
\end{lemma}

Instead, we have to express what we need in terms of Lie algebras, to get a graded version:
\begin{lemma}
 Let $U$ be a graded pro-unipotent group with finitely many nonzero graded components. Then there is a surjective graded morphism $\spec T(V) \surj U$.

Let $U$ be a graded pro-unipotent group, then there is a surjective graded morphism $\spec T(V) \surj U$.
\end{lemma}
\begin{proof}
 The second statement follows from the first by a limit process, where we use $\spec T(V) = \lim \spec T(\bigoplus_{i=0}^n V_i)$ and $U = \lim U_n$ is the standard limit description.

The first statement comes from a graded re-statement of the last result of talk 3:

Let $F(V^\vee)$ be the free Lie algebra on $V^\vee$, where $V = H^1(U) = \mathfrak{g}/[\mathfrak{g},\mathfrak{g}]$.
We take a lift of $V^\vee$ to $\mathfrak{g}$, called $\tilde{V^\vee}$ (that is a choice, the same choice that the $\gamma_1,\ldots,\gamma_r$ were before).
So we get a morphism of graded Lie algebras $F(\tilde{V^\vee}) \to \mathfrak{g}$,
which induces a morphism of enveloping algebras $\mathcal{U}(F(\tilde{V^\vee})) \to \mathcal{U}(\mathfrak{g})$.
The latter one is a pro-unipotent Lie algebra, so the map factors through the pro-unipotent completion to a map
$\mathcal{U}(F(\tilde{V^\vee}))^{\wedge} \to \mathcal{U}(\mathfrak{g})$.
Dualizing gives us $T(V) \leftarrow \mathcal{O}(G)$ - as graded Hopf algebras, since everything respected the grading.
\end{proof}




\begin{prop}
 Assume for any $n$, the $k$-vector space $V_n$ has finite dimension $r_n$ and all $\Ext^2_ {\mathcal{C}}(1,L^{\otimes n})$ vanish. Then $r_n = 0$ for $n \leq 0$ and $V := \bigoplus_{n > 0} V_n$ is a graded vector space whose graded tensor algebra $T(V)$, with $V_n$ put in degree $n$, such that we have an isomorphism of graded Hopf algebras
\[\mathcal{O}(U) \cong T(V) = T\left(\bigoplus_{n>0} \Ext^1_{\mathcal{C}}(1,L^{\otimes n})\right).\]
\end{prop}
\begin{proof}
 Same argument as before, now with grading:

We get a morphism of graded pro-unipotent groups $T(\bigoplus_{n > 0} V_n) \surj U$.
It splits, as before, and we have an isomorphism of graded Hopf algebras.
\end{proof}















% Periods in the pro-unipotent case
% - def weigth filtration on objects induces weight filtration on periods
% - def filtered k-algebra $k[t^2] \otimes_k T(\bigoplus_{n>0} V_n)$
% - theorem as filtered algebra, $P^c$ is a subquotient of the latter.
% - proof
\section{Periods in the graded pro-unipotent case}

Now we assume, in addition to the canonical fiber functor $\omega$, to have a fiber functor $\eta$.

Suppose the involution $\tilde{c} : I(\omega,\eta) \to I(\omega,\eta)$ is given by the action of an order $2$ element $\epsilon \in G_\eta$ (with respect to the $G_\eta$-torsor structure on $I(\omega,\eta)$).

\begin{lemma}
 The element $\epsilon \in G_\eta$ is conjugate to $-1 \in \Gm \subset G_\eta$.
\end{lemma}
\begin{proof}
 Look at the commutative diagram
\[
\xymatrix{
G_{\omega} \times I(\omega, \eta) \ar[d]_{ad(\xi) \times \xi \cdot} \ar[rr]^{\mathrm{action}} && I(\omega, \eta) \ar[d]^{\xi \cdot} \\
G_{\omega} \times I(\omega, \eta) \ar[rr]_{\mathrm{action}} && I(\omega, \eta)
}
\]
for $\xi \in G_\omega$ any element. It commutes, since $\xi x \xi^{-1} \xi = \xi x$.
This shows that conjugation in $G_\eta$ corresponds to multiplication in $I(\omega,\eta)$.

In $G_\eta = \Gm \rtimes U$, the multiplication is
\[(\lambda,u) \cdot (\lambda',u') = (\lambda \lambda', u \lambda u' \lambda^{-1})\]
so if $\epsilon = (\lambda,u)$, we can multiply with $(-\lambda^{-1},\lambda^{-1} u^{-1} \lambda)$ to get
\[\epsilon \cdot (-\lambda^{-1},\lambda^{-1} u^{-1} \lambda) = (-1, 1). \qedhere\]
\end{proof}



\begin{defn}
 The filtration on $S$ defines a filtration on $\omega(S)$. Putting a trivial filtration on $\eta(S)^\vee$, this gives a filtration on $\omega(S) \otimes_k \eta(S)^\vee$ and thus a filtration on periods $P$. This filtration induces a filtration on the $c$-fixed points $P^c$.
\end{defn}

\begin{defn}
 Let $\mathcal{O}(I(\omega,\eta))_+ \subset \mathcal{O}(I(\omega,\eta))$ be the subspace generated (as $k$-algebra) by the image of all $\omega(S) \otimes_k \eta(S)^\vee$ under $\psi$, for $S$ of positive weight, i.e. $\gr_n^w S = 0$ for all $n < 0$.
\end{defn}


\begin{thm}\label{thm:realperiods}{\quad}
\begin{itemize}
 \item The involution $\tilde{c}$, resp. $\epsilon \in G_\eta$, respects the subspace $\mathcal{O}(I(\omega,\eta))_+$.
 \item The real periods $P^c$, as filtered $k$-vector space, are the image of $\mathcal{O}(I(\omega,\eta))^{\epsilon}_+$ under $p^\ast : \mathcal{O}(I(\omega,\eta)) \to K$.
 \item There is a graded Hopf algebra isomorphism $\mathcal{O}(I(\omega,\eta))_+^{\epsilon} \simeq k[t^2] \otimes_k \mathcal{O}(U)$.
 \item In particular, $P^c$ is a quotient of $k[t^2] \otimes_k T\left(\bigoplus_{n>0} \Ext^1_{\mathcal{C}}(1,L^{\otimes n})\right)$.
\end{itemize}
\end{thm}
\begin{proof}
The first is almost by definition, since any $\xi \in G_\eta$ respects the image of $\omega(S) \otimes \eta(S)^\vee$ in $\mathcal{O}(I(\omega,\eta))$, in particular those with only positive weights. In other words: translations respect the subalgebra of positive weights. The second statement also follows from this, by the definition of $P$.
From our knowledge about $\mathcal{O}(U)$, we only need to prove the third statement, to get the fourth one.
We will do this in four steps.

\subsubsection{Step 1: $\omega = \eta$}
 First, we know that all $G_\omega$-torsors are trivial (since $H^1(k,\Ga)=0=H^2(k,\Gm)$ and $G_\omega$ is an iterated extension of $\Ga$ and $\Gm$) so we know $\omega = \eta$.

\subsubsection{Step 2: the positive part}
Now we have the morphism
\[\Gm \times U \to G_\omega,\ (a,u) \mapsto a\cdot u\]
which defines an isomorphism of the corresponding graded $k$-algebras
\[\mathcal{O}(G_\omega) \cong k[t,t^{-1}] \otimes_k \mathcal{O}(U),\]
the grading on $\mathcal{O}(G_\omega)$ being induced by right translations of $\Gm \subset G_\omega$.

Since we consider only $S$ with $gr_W^n(S) = 0$ for $n < 0$, the action of $G_\omega$ on the image of $\omega(S) \otimes \eta(S)^\vee$ factors through the monoid $\spec( k[t] \otimes_k \mathcal{O}(U)) = \Ao_k \times_k U$.
Put differently, $\mathcal{O}(I(\omega,\eta))_+ \simeq k[t] \otimes_k \mathcal{O}(U)$, as $k$-algebra.

\subsubsection{Step 3: invariants under conjugation}
Since $\mathcal{O}(\Gm)^{-1}$, the invariants of $\mathcal{O}(\Gm)$ under $-1$, is isomorphic to $k[t^2,t^{-2}]$, we have
$\mathcal{O}({}_{-1}\backslash G_\omega) = k[t^2,t^{-2}] \otimes_k \mathcal{O}(U)$.

\subsubsection{Step 4: putting stepts 2 and 3 together}
We have $\mathcal{O}(I(\omega,\eta))_+ \simeq k[t] \otimes_k \mathcal{O}(U)$ and $\mathcal{O}(I(\omega,\eta))^\epsilon \simeq k[t^2,t^{-2}] \otimes_k \mathcal{O}(U)$ and altogether we get
\[\mathcal{O}(I(\omega,\eta))_+^{\epsilon} \simeq k[t^2] \otimes_k \mathcal{O}(U). \qedhere\]
\end{proof}


\chapter{Mixed Tate motives over $\Z$ by Martin Gallauer}

Martin Gallauer on September 6th, 2012.

\medskip
\medskip

\noindent Possible references for this chapter are \cite{voevodsky00-mm}, \cite{andre04-motifs} or \cite{MVW-motcoh}.

\section{Introduction}
\label{sec:9-intro}

The two principal tasks of this chapter are to construct the category
of mixed Tate motives over $\Z$ and to apply results from the last
chapter to it. We won't construct this category in a ``bottom-up''
approach but instead extract it as a full subcategory of the larger
triangulated category of mixed motives over $\Q$. Thus more precisely
our tasks are, in order:
\begin{enumerate}
\item Construct $\DM(\Q)$, the triangulated category of mixed motives
  over $\Q$, or more generally, $\DM(k)$ for any field $k$.
\item Extract $\mTm(\Z)$ as a full subcategory of $\DM(\Q)$.
\item Apply results from last chapter.
\end{enumerate}

The construction of $\DM(k)$ proceeds in several steps and for someone
unacquainted with motives it might be difficult to see \emph{why}
these steps ought to be taken and what \emph{results} from them. For
these readers we would like to make life easier by, firstly, describing
the general philosophy of mixed motives in the rest of the
introduction, and secondly, stating some of the most important
properties of the resulting category $\DM(k)$ after the construction
has been done (see~\ref{sec:9-prop}).

To describe the idea of mixed motives (very roughly, of course; we
follow \cite[chapter~14]{andre04-motifs}), fix a field $k$ and
consider the category $\Sm/k$ of smooth varieties over $k$, \ie{}
separated smooth finite type schemes over $k$ (actually, one often
considers \emph{all} varieties over $k$). A \emph{mixed Weil homology
  theory} is a functor $H_{*}:\Sm/k\to {\cal A}$ to an abelian
$\otimes$-category endowed with an action of finite correspondences
which satisfies at least the following axioms: homotopy invariance,
Künneth formula and long exact sequence of Mayer-Vietoris
type. (Sometimes more axioms are postulated; and usually contravariant
functors are considered, giving rise to the notion of a mixed Weil
\emph{co}homology theory.) Typical examples are Betti, algebraic de
Rham or $\ell$-adic homology.

\emph{Mixed motives} should be considered as a universal Weil homology
theory. More explicitly, there should exist an abelian
$\otimes$-category $\MM(k)$ with a mixed Weil homology theory
$\Sm/k\to\MM(k)$ such that for any mixed Weil homology theory $H$ the
following diagram can be completed commutatively in a unique way with
a dotted arrow, called a \emph{realization functor}:
\begin{equation*}
  \xymatrix{\Sm/k\ar[d]_{H}\ar[r]&\MM(k)\ar@{.>}[ld]^{\mathrm{real}_{H}}\\
    {\cal A}}
\end{equation*}
Alas, this category $\MM(k)$ is not known to exist although people
have certainly tried to construct it for the last decades. Deligne was
it who, in the eighties of the last century, devised another approach
to mixed motives: Instead of abelian categories ${\cal A}$ as targets
of homology theories the focus is now shifted to functors into
$\otimes$-\emph{triangulated} categories ${\cal D}$ from which a mixed
Weil homology theory would be obtained via a functor ${\cal D}\to{\cal
  A}$. The axioms for the mixed Weil homology theory are thus replaced
by their triangulated translations (in particular, the Mayer-Vietoris
long exact sequence is replaced by a distinguished triangle) and,
again, the \emph{triangulated category of mixed motives} $\DM(k)$
would be the universal such theory. The diagram above is thus replaced
by its triangulated version:
\begin{equation*}
  \xymatrix{\Sm/k\ar[d]_{H}\ar[r]&\DM(k)\ar@{.>}[ld]^{\mathrm{real}_{H}}\\
    {\cal D}}  
\end{equation*}
In contrast to the abelian case there are indeed candidates for this
universal category one of which we will describe in more detail in the
next section.

In addition, one hopes to recover $\MM(k)$ from $\DM(k)$ by a
construction which is known as a $t$-structure. One way to think about
this is to imagine $\DM(k)$ to be the (bounded) derived category of
$\MM(k)$ and to use the shift functor on $\DM(k)$ to recover $\MM(k)$
as the ``homological 0-part''. A $t$-structure is a natural
generalization of this idea. (We will come back to this idea in
section~\ref{sec:9-tate}.)

We hope that keeping the universal property of $\DM(k)$ as well as its
intended relation with $\MM(k)$ in mind, it will be easier to follow
the construction in the next section.

\section{Mixed motives: Construction}
\label{sec:9-construction}
Let $k$ be a field, and let $\Sm/k$ be the category of smooth
separated finite type schemes over $k$. If not mentioned otherwise,
all schemes considered from now on will be such smooth varieties.

We will construct the triangulated category of mixed motives over $k$,
$\DM(k)$, following Voevodsky (see \cite[2]{voevodsky00-mm}; there this category is
denoted $\DM_{gm}(k)$ (``geometric motives''))

\subsubsection{Step 1: Correspondences and linearization}

\begin{defn}Let $X, Y \in \Sm/k$.
  \begin{enumerate}
  \item A closed integral subscheme $\alpha \subset X \times Y$ is
    called an \emph{elementary correspondence from $X$ to $Y$} if it
    is finite and surjective over a connected component of $X$.
  \item Set $c(X,Y)$ to be the $\Q$-vector space generated by
    elementary correspondences from $X$ to $Y$. Elements of this
    vector space are called \emph{finite correspondences}.
\end{enumerate}
\end{defn}

\begin{exam}
  The graph $\Gamma_f$ of a morphism $f : X \to Y$ of schemes defines
  a finite correspondence, namely its associated reduced subscheme. It
  is an elementary correspondence if and only if $X$ is connected.
\end{exam}

\begin{rem}
  One can consider elementary correspondences as ``multi-valued
  functions'' as follows (\cite[15.1.2]{andre04-motifs}): Let $\alpha$
  be an elementary correspondence from $X$ to $Y$ and let $x$ be a
  closed point of $X$. The pre-image of $x$ under the finite morphism
  $\alpha\to X\times Y\to X$ is a 0-cycle $\sum_{i}n_{i}a_{i}$,
  $n_{i}\in\Z, a_{i}$ closed points of $\alpha$. This ``multi-valued''
  morphism
  \begin{align*}
    X&\to \alpha\\
    x&\mapsto \sum_{i}n_{i}a_{i}
  \end{align*}
  is called \emph{transfer}, and $\alpha$ can be considered as the
  composition of this transfer with the projection $\mathrm{p}_{2}:\alpha\to X\times
  Y\to Y$, thus $x\mapsto\sum_{i}n_{i}p_{2}(a_{i})$.
\end{rem}

The definition of our correspondences is so chosen that they can
easily been composed. (For the claims made in the sequel
see~\cite[chapter~1]{MVW-motcoh}.) Namely, let $\alpha\subset X\times
Y$ and $\beta\subset Y\times Z$ be elementary correspondences and
consider their pullback to the triple product $X\times Y\times
Z$. They intersect properly and we may consider their intersection
product $(\alpha\times Z)\cdot(X\times \beta)$ which is finite over
$X\times Z$. Denoting by $p$ the projection from the triple product to
$X\times Z$ we set:
\begin{defn} The composition of $\alpha$ and $\beta$, denoted by
  $\beta\circ \alpha$, is defined to be the pushforward
  $p_{*}((\alpha\times Z)\cdot(X\times \beta))$. The composition is
  extended $\Q$-linearly to all finite correspondences.
\end{defn}
$\beta\circ\alpha$ is again a finite correspondence and the
composition is associative. Moreover, $\Gamma_{\mathrm{Id}_{X}}$ acts
as identity with respect to this composition.
\begin{defn}
  We denote by $\SmCor(k)$ the \emph{category of finite
    correspondences}. Its objects are the objects of $\Sm/k$ and for
  $X, Y \in \Sm/k$, the morphisms from $X$ to $Y$ are the finite
  correspondences $c(X,Y)$.
\end{defn}

\begin{rem}
\begin{itemize}
\item $\SmCor(k)$ is a $\Q$-linear $\otimes$-category, with coproducts
  given by disjoint union and the monoidal product by the direct
  product (over $k$). More explicitly, this means that the category
  has a $\Q$-linear as well as a symmetric monoidal unitary structure
  and the monoidal product is $\Q$-linear in both variables. TODO:
                               Refer to definition of this in the
                                last chapter
                              \item The functor $\Sm/k \to \SmCor(k)$
                                is a $\otimes$-functor, in particular
                                $\Gamma_{g}\circ
                                \Gamma_{f}=\Gamma_{g\circ f}$.
\end{itemize}
\end{rem}

\subsubsection{Step 2: Triangulation}
We proceed in the usual way to pass from an additive to a triangulated
category.
\begin{defn}
  Let $K^{b}(\SmCor(k))$ be the category of bounded complexes
  in $\SmCor(k)$ up to chain homotopy. In other words, objects are
  bounded chain complexes in $\SmCor(k)$ and morphisms are equivalence
  classes of chain complex morphisms with respect to chain
  homotopies. (The differential in the complexes has degree -1.)
\end{defn}

\begin{rem}
  The category $K^b(\SmCor(k))$ is a $\Q$-linear triangulated
  $\otimes$-category. More explicitly, it is a $\Q$-linear
  $\otimes$-category with a triangulated structure such that $\otimes$
  is triangulated in both variables. (See\cite[8A]{MVW-motcoh} for the
  precise definition.)

  Moreover, the functor $\SmCor(k)\to K^b(\SmCor(k))$ respects the
  monoidal and the $\Q$-linear structure.
\end{rem}

\subsubsection{Step 3: Impose relations}
Let $X$ be a smooth variety and consider the following complex
$[\Aspace^{1}_{X}]\to [X]$ concentrated in degrees 1 and 0 (say). To ensure
homotopy invariance we would like this complex to be 0. Similarly,
suppose there is an open covering $X=U\cup V$ of $X$ and consider the
following complex in degrees 2, 1 and 0:
\begin{equation*}
  \xymatrix{
[U \cap V] \ar[r]^{\iota} & [U] \oplus [V] \ar[r]^{\quad(+, -)} & [X] 
}
\end{equation*}
To ensure we get a distinguished triangle of Mayer-Vietoris type we
would also like this complex to be 0. This is what we will now impose.

For this, set ${\cal T}$ to be the thick subcategory of
$K^b(\SmCor(k))$ generated by the two types of complexes above, \ie{}
the smallest full triangulated subcategory containing these objects
and closed under direct factors.
\begin{defn}
  \begin{enumerate}
  \item $K^b_{\Aspace^{1},\mathrm{MV}}(\SmCor(k))$ is the quotient category
    of $K^b(\SmCor(k))$ with respect to ${\cal T}$.
  \end{enumerate}
\end{defn}

More explicitly, the objects are the same but the morphisms between
two complexes $A$ and $B$ are $\varinjlim \Hom_{K^b(\SmCor(k))}(A',B)$
where $A'\to A$ runs over all morphisms in $K^b(\SmCor(k))$ whose cone
lies in ${\cal T}$. There is a canonical quotient functor
$K^b(\SmCor(k))\to K^b_{\Aspace^{1},\mathrm{MV}}(\SmCor(k))$.
\begin{rem}
There is (in general) a canonical triangulated structure on the
quotient which makes the quotient functor triangulated. It is easy to
check that (in our case) there is also a canonical $\otimes$-structure
on the quotient which makes the quotient functor a
$\otimes$-functor (this boils down to check that ${\cal T}$ is closed
under tensoring with an arbitrary complex). Moreover these structures
are compatible with each other and also with the $\Q$-linear structure.
\end{rem}

\subsubsection{Step 4: Pseudo-abelianization}
\begin{defn}
  An additive category $(\cC, \otimes)$ is called
  \emph{pseudo-abelian} if for all objects $X \in \cC$ and for all
  idempotents $p : X \to X \in \cC$ (i.e., $p^2 = p$), $X = \ker(p)
  \oplus \ker(\mathrm{Id}_{X} - p)$.
\end{defn}
\begin{rem}
  The bounded derived category of an abelian category is
  pseudo-abelian (\cite[2.8]{Balmer-Schlichting}). Thus keeping with
  the intention described in the introduction we will want our
  category $\DM(k)$ to be pseudo-abelian. There is a universal way to
  pass from an additive to a pseudo-abelian category which is
  described as taking the \emph{pseudo-abelian hull}.
\end{rem}
\begin{defn}
  We set $\DM^{\mathrm{eff}}(k)$ to be the pseudo abelian hull of
  $K^b_{\Aspace^1,\mathrm{MV}}(\SmCor(k))$. It is called the
  \emph{triangulated category of effective mixed motives over
    $k$}. Accordingly, objects of this category are called
  \emph{effective motives}.
\end{defn}
Explicitly, the objects are pairs $(A,p)$ where $A$ is an object in
$K^b_{\Aspace^1,\mathrm{MV}}(\SmCor(k))$ and $p$ is an idempotent on $A$. A
morphism $(A,p)\to (A',p')$ is a morphism in
$K^b_{\Aspace^1,\mathrm{MV}}(\SmCor(k))$ of the form
\begin{equation*}
  A\stackrel{p}{\to}A\to A'\stackrel{p'}{\to}A',
\end{equation*}
and composition is induced by the original one. There is a canonical
functor $K^b_{\Aspace^1,\mathrm{MV}}(\SmCor(k))\to \DM^{\mathrm{eff}}(k)$
which sends $A$ to $(A,\mathrm{Id}_{A})$.
\begin{rem}
  $\DM^{\mathrm{eff}}(k)$ is a $\Q$-linear triangulated tensor
  category and the functor just defined is a fully faithful embedding
  preserving all the structure. The only non-trivial part of this
  statement is about the triangulation. For this see
  \cite[1.5]{Balmer-Schlichting}.
\end{rem}
\begin{defn}
  We write $M$ for the $\otimes$-functor
  $\Sm/k\to\DM^{\mathrm{eff}}(k)$. For any smooth variety $X$, $M(X)$
  is called the \emph{motive of} $X$.
\end{defn}
\begin{exam}
  \begin{enumerate}\item 
    The $\otimes$-unit in $\DM^{\mathrm{eff}}(k)$ is the motive of the
    $\otimes$-unit in $\Sm/k$:
    \begin{equation*}
      \mathbf{1_{\otimes}} = M(\spec k) =: \Q(0)
    \end{equation*}
  \item One important motive is $M(\Pspace^1)$. In any homology theory $H$
    we have $H(\Pspace^{1})=H_{0}(\Pspace^{1})\oplus
    H_{2}(\Pspace^{1})$ thus the same decomposition should be true of
    the motive $M(\Pspace^{1})$. Our next goal is to define the ``Tate
    object'' $\Q(1)$ so that
    \begin{equation}
      \label{eq:9-p1-decomposition}
      M(\Pspace^{1})=\Q(0)\oplus\Q(1)[2].
    \end{equation}
  \end{enumerate}
\end{exam}

Starting with any smooth variety $X$ consider the following complex in 
$K^b(\SmCor(k))$
\begin{equation*}
  [X]\to [\spec k]
\end{equation*}
concentrated in degrees $0$ and $-1$.
\begin{defn}
  The image of this complex in $\DM^{\mathrm{eff}}(k)$ is denoted by
  $\tilde{M}(X)$ and is called the \emph{reduced motive of} $X$.
\end{defn}

\begin{rem}
  There is a canonical distinguished triangle
  \begin{equation*}
    \tilde{M}(X)\to M(X)\to\Q(0)\to^{+}
  \end{equation*}
  in $\DM^{\mathrm{eff}}(k)$ associated to any smooth variety $X$.
\end{rem}

\begin{exam}
  Let $x: \spec k \to \Pspace^1$ be a rational point. Then the
  composition
\begin{equation*}
  p: [\Pspace^1] \to [\spec k] \stackrel{x}{\to} [\Pspace^1]
\end{equation*}
is an idempotent in $K^b(\SmCor(k))$, i.e., $p \circ p = p$. Thus we
have a decomposition in $\DM^{\mathrm{eff}}(k)$:
$M(\Pspace^1)=\im(p)\oplus\ker(\mathrm{Id}-p)$. It is easy to see
that $\im(p)=\Q(0)$ and the canonical projection
$M(\Pspace^1)\to\Q(0)$ is compatible with the decomposition. Hence we
get a distinguished triangle
\begin{equation*}
  \ker(\mathrm{Id}-p)\to M(X)\to\Q(0)\to^{+}
\end{equation*}
and it follows from the previous remark that
$\ker(\mathrm{Id}-p)=\tilde{M}(\Pspace^{1})$ and we have a decomposition
\begin{equation*}
  M(\Pspace^{1})=\Q(0)\oplus\tilde{M}(\Pspace^{1})
\end{equation*}  
\end{exam}
\begin{defn}
  The \emph{Tate motive} $\Q(1)$ is defined as
  $\tilde{M}(\Pspace^{1})[-2]$. For any effective motive $N$ and any
  $n\in\N$ we define
  \begin{equation*}
    N(n):=N\otimes\Q(n):=N\otimes\Q(1)^{\otimes n}
  \end{equation*}
\end{defn}
It is easy to show, using homotopy invariance, that the decomposition
of the motive of $\Pspace^{1}$ does not depend on the choice of the
rational point (see \cite[16.3.1.3]{andre04-motifs}) hence neither
does~(\ref{eq:9-p1-decomposition}).
\begin{rem}
  The argument in the previous example shows more generally that for
  any smooth variety $X$ with a rational point there is a
  decomposition $M(X)=\Q(0)\oplus \tilde{M}(X)$. In contrast to the
  example above however, the decomposition depends in general on the
  choice of the rational point.
\end{rem}
\subsubsection{Step 5: Towards rigidity}

Tensoring with $\Q(1)$ defines an endofunctor of
$\DM^{\mathrm{eff}}(k)$ which we invert to get a ``dual'' to $\Q(1)$.
\begin{defn}
Set $\DM(k):= \DM^{\mathrm{eff}}(k)[\Q(1)^{-1}]$. This category is
called the \emph{triangulated category of mixed motives over} $k$. Its
objects are accordingly called \emph{motives}.
\end{defn}
Explicitly, objects are pairs $(N,n)$ where $N$ is an effective
motive and $n\in\Z$ while the set of morphisms $(N,n)\to (N',n')$ is
\begin{equation*}
  \varinjlim_{k\geq -n,-n'}\Hom_{\DM^{\mathrm{eff}}(k)}(N(k+n),N'(k+n')).
\end{equation*}
There is a canonical functor $\DM^{\mathrm{eff}}(k)\to\DM(k)$ sending
$N$ to $(N,0)$ and we denote the composition $\Sm/k\to \DM(k)$ also by
$M$.
\begin{rem}
The category $\DM(k)$ is canonically a $\Q$-linear triangulated
$\otimes$-category and the functor just defined preserves all the
structure. However, for the symmetric monoidal structure there is
something to prove (see \cite[17.1.2]{andre04-motifs}).

It is not difficult to see that $\DM(k)$ is still pseudo-abelian.
\end{rem}

\begin{defn}
  Let $\Q(-1)$ be an object in $\DM(k)$ such that
  $\Q(-1)\otimes\Q(1)=\Q(0)$. We then define for any motive $N$ and
  any $n\in\Z$ the $n$\emph{th twist of} $N$ to be
  \begin{equation*}
    N(n):=N\otimes \Q(n):=N\otimes \Q\left(\frac{n}{|n|}\right)^{\otimes |n|}.
  \end{equation*}
\end{defn}

We will see in the next section that inverting the Tate object already
suffices to render the whole category rigid. $\Q(-1)$ is the dual of
the Tate object.

\section{Mixed motives: Properties}
\label{sec:9-prop}
From now on we assume $\mathrm{char~}(k) = 0$ although this is not always
necessary.

In this section we will state some of the most important properties of
the category $\DM(k)$ defined above (although not everything will be
needed in the rest of the chapter). The proof of these properties lies
at the center of the theory developed by Voevodsky et al. The method
used is to embed $\DM(k)$ into a larger category in which one can
``apply all the standard machinery of sheaves and their cohomology.''
\cite[p.~7]{voevodsky00-mm}

See \cite[2.2]{voevodsky00-mm}, \cite[14]{MVW-motcoh},
\cite[18]{andre04-motifs} for the statements.


\subsection{``Homological'' properties}

\begin{thm}
  The functor $M:\Sm/k\to\DM(k)$ extends to all varieties over $k$ and
  satisfies:
  \begin{itemize}
  \item the Künneth formula: $M(X\times Y)= M(X)\otimes M(Y)$;
  \item homotopy invariance: $M(\Aspace^{1}_{X})=M(X)$;
  \item Mayer-Vietoris distinguished triangle: $M(U\cap V)\to
    M(U)\oplus M(V)\to M(X)\to^{+}$ (for any open covering $X=U\cup
    V$);
  \item blow-up distinguished triangle: $M(p^{-1}(Z))\to
    M(X_{Z})\oplus M(Z)\to M(X)\to^{+}$ (for $p:X_{Z}\to X$ a blow-up
    with center $Z$);
  \item projective bundle formula:
    $M(\Pspace(E))=\oplus_{i=0}^{n}M(X)(i)[2i]$ (for $E$ a rank $n+1$
    vector bundle over $X$).
  \end{itemize}
  In addition, there is also a functor $M^{c}$ of motives with compact
  support, a Gysin distinguished triangle\ldots{}
\end{thm}

In summary, the behaviour of this functor $M$ is as expected from a
homological theory of algebraic varieties. 

\begin{exam}\label{exam:9-gm}
  We want to use the previous theorem to compute $M(\Gm^n)$.
  \begin{enumerate}
  \item If $n=1$, then we use the Mayer-Vietoris decomposition of
    $\Pspace^{1} =
    (\Pspace^{1}\backslash\{0\})\cup(\Pspace^{1}\backslash\{\infty\})$
    to get a distinguished triangle
    \begin{equation*}
      M(\Gm) \to M(\Aspace^1) \oplus M(\Aspace^1) \to M(\Pspace^1) \to^{+}.
    \end{equation*}
    By choosing a common base point this yields another distinguished
    triangle
    \begin{equation*}
      \tilde{M}(\Gm) \to \tilde{M}(\Aspace^1) \oplus \tilde{M}(\Aspace^1) \to \tilde{M}(\Pspace^1) \to^{+}.
    \end{equation*}
    By homotopy invariance, the two summands in the middle are 0 and
    we conclude $\tilde{M}(\Gm)=\tilde{M}(\Pspace^{1})[1]=\Q(1)[1]$.
  \item If $n \geq 1$, then
    \begin{align*}
      M(\Gm^n) &= \otimes_{i=1}^{n} M(\Gm)&&\text{Künneth formula}\\
      &= \otimes_{i=1}^{n}(\Q(0) \oplus \Q(1)[1]) &&\text{previous part}\\
      &= \bigoplus_{i=0}^n \binom{n}{i} \Q(i)[i]
    \end{align*}
  \end{enumerate}
\end{exam}

\subsection{Cancellation}

\begin{thm}
The endofunctor 
\begin{equation*}
- \otimes \Q(1) : \DM(k) \to \DM(k)
\end{equation*}
is fullly faithful.
\end{thm}
It follows that the canonical functor $\DM^{\mathrm{eff}}(k)\to\DM(k)$
is a fully faithful embedding.

\subsection{Rigidity}

\begin{thm}
  There exists an autoduality
  \begin{equation*}
    ^{\vee}:\DM(k)^{\mathrm{op}}\to\DM(k)
  \end{equation*}
  making $\DM(k)$ a rigid $\Q$-linear category.

  Moreover, if $X$ is smooth and projective of dimension $d$, then
  $M(X)^{\vee}=M(X)(-d)[-2d]$.
\end{thm}

\begin{exam}
  For any (smooth) variety $X$ over $k$, the motive $M(X)^{\vee}$ is
  canonically an algebra object in $\DM(k)$. Indeed, since $\Sm/k$ is
  cartesian monoidal, $X$ carries a unique coalgebra structure given
  by the diagonal embedding and the canonical projection to
  $\spec(k)$. Thus it is an algebra object in $(\Sm/k)^{\mathrm{op}}$
  and is mapped to an algebra object in $\DM(k)$ under the
  $\otimes$-functor $^{\vee}\circ M$.

  Explicitly, the ``multiplication'' is \eg{} given by
  \begin{align*}
    M(X)^{\vee} \otimes M(X)^{\vee} &= (M(X) \otimes
    M(X))^{\vee}&&\text{autoduality}\\
    &= M(X \times X)^{\vee} &&\text{Künneth formula}\\
    &\to M(X)^{\vee}&&\text{diagonal embedding } X\to X\times X
  \end{align*}
\end{exam}

\subsection{Realization functors}
There are general sufficient conditions under which a functor
$\Sm/k\to {\cal D}$ induces a ``realization'' functor $\DM(k)\to{\cal
  D}$ (see~\cite{huber00-realization}) however we are interested only
in the following examples:
\begin{exam}
  \begin{enumerate}
  \item There is a \emph{de Rham realization} functor
    \begin{equation*}
    \omega_{\mathrm{dR}} : \DM(k) \to \D^b(k\Mod)
  \end{equation*}
  which comes with an increasing weight and a decreasing Hodge
  filtration.

  ``Realization'' here simply means that the following diagram
  commutes: 
  \begin{equation*}
    \xymatrix{
      \Sm/k \ar[r]^{M} \ar[d]_{H_{*}^{\textrm{dR}}} & \DM(k) \ar[dl]^{\omega_{dR}} \\
      \D^b(k\Mod)
    }
  \end{equation*}
  where $H_{*}^{\mathrm{dR}}$ is algebraic de Rham homology. Also the
  diagram is compatible with the filtrations.
\item Let $k$ be a subfield of $\C$ with embedding $\sigma : k \inj
  \C$. Then there is an associated \emph{Betti realization} functor
  \begin{equation*}
    \omega_{\mathrm{B},\sigma} : \DM(k) \to \D^b(\Q\Mod)
  \end{equation*}
\item Let $\ell$ be a prime. There is an \emph{$\ell$-adic
    realization} functor
  \begin{equation*}
    \omega_{\ell} : \DM(k) \to \D^b(\Q_{\ell}\Mod)
  \end{equation*}
  This is a \emph{geometric} realization thus endowed with an action
  of the Galois group.
\end{enumerate}
\end{exam}

\begin{exam}\label{exam:9-hodge-realization}
  For any embedding $\sigma:k\inj \C$ there is an isomorphism of the
  Betti and de Rham functors after scalar extension to $\C$ (the
  ``period'' or ``comparison'' isomorphism)
  \begin{equation*}
    \omega_{\mathrm{dR}} \otimes \C \cong \omega_{\mathrm{B},\sigma} \otimes \C
  \end{equation*}
  From this one deduces a \emph{Hodge realization} functor
  \begin{equation*}
    \omega_{\mathrm{H}} : \DM(k) \to \D^b(\mathrm{MHC}(k)) \text{ TODO: Target category}
  \end{equation*}

  It is not hard to see that $\omega_{\mathrm{H}}(\Q(1)) =
  \Q(1)_{\mathrm{H}}$ (see TODO: Reference to example in one of the
  previous chapters).
\end{exam}

\subsection{Motivic cohomology}

\begin{defn}
  Let $X$ be a smooth variety over $k$. Then the \emph{motivic
    cohomology of $X$ with $\Q$-coefficients} is
  \begin{equation*}
    H^{i,n}(X, \Q) := \Hom_{\DM(k)}(M(X), \Q(n)[i]), \qquad i,n\in\Z
  \end{equation*}
\end{defn}

\begin{thm}
  Let $X$ be a smooth variety over $k$. Then
  \begin{equation*}
    H^{i,n}(X, \Q) = K_{2n-i}(X)^{(n)}_{\Q}
  \end{equation*}
  where the right hand side denotes the subspace of weight $n$ (with
  respect to the Adams operator) of algebraic $K$-theory tensored with
  $\Q$.
\end{thm}

\begin{exam}\label{exam:9-hom-base}
  It follows that $\Hom_{\DM(k)}(\Q(m),\Q(n))=0$ whenever $n<m$, and
  $=\Q$ if $m=n$.
\end{exam}

\begin{exam}
  Let $X = \spec(k)$. Then the claim that 
  \begin{equation*}
    \text{(B-S)}_{k}:\quad H^{i,n}(\spec(k), \Q) = K_{2n-i}(k)^{(n)}_{\Q}=0,\qquad \forall n \geq 0, \forall i < 0
  \end{equation*}
  is known as the \emph{Beilinson-Soulé vanishing conjecture for
    $k$}. It plays a strikingly important role in the theory of mixed
  (Tate) motives as we will see a little bit in the next section. It
  is not known to hold except in some special cases, among which is
  the only one we're interested in:
\end{exam}

\begin{thm}[Borel]\label{thm:9-borel}
  Let $k = \Q$. Then
  \begin{align*}
    K_{2n-1}(\Q)_{\Q}^{(n)} &= K_{2n-1}(k)_{\Q} = \left\{ \begin{array}{ll}
        \Q^{\times} \otimes \Q &:n=1 \\
        \Q &:\textrm{$n\geq 3$ odd} \\
        0 &:\textrm{otherwise}
      \end{array} \right.\\
    K_{2n}(\Q)_{\Q} &= 0 \qquad \forall n>0
  \end{align*}
  Hence (B-S)$_{\Q}$ is true.
\end{thm}

The term $\Q^{\times}\otimes\Q$ is easily computed. Indeed, the maps
\begin{eqnarray*}
  \Q^{\times}\otimes \Q&\leftrightarrow&\oplus_{p\text{ prime}}\Q\\
  x\otimes y&\mapsto&y\cdot(v_{p}(x))_{p}\\
  q\otimes z&\mapsfrom&z\in \Q\stackrel{q}{\inj}\oplus_{p}\Q
\end{eqnarray*}
define a $Q$-linear bijection ($v_{p}$ is the $p$-valuation).
\begin{rem}
  In fact, Borel proved a similar formula for all number fields $k$;
  in particular, (B-S)$_{k}$ is true.
\end{rem}
\section{Mixed Tate motives}
\label{sec:9-tate}
\subsection{Mixed Tate motives over $\Q$}

\begin{defn}
  A \emph{mixed Tate motive over $k$} is an iterated extension in
  $\DM(k)$ of motives of the form $\Q(n), n \in \Z$.
\end{defn}
This means that a mixed Tate motive $N$ fits into a distinguished triangle
\begin{equation*}
  N'\to N\to \Q(n)\to^{+},
\end{equation*}
where $N'$ is a mixed Tate motive and $n\in\Z$.
\begin{defn}
  \begin{enumerate}
  \item The \emph{category of mixed Tate motives over $k$}, $\MTM(k)$,
    is the full subcategory of $\DM(k)$ spanned by mixed Tate motives.
  \item The \emph{triangulated category of mixed Tate motives over
      $k$}, $\DTM(k)$, is the full triangulated subcategory of
    $\DM(k)$ generated by mixed Tate motives.
  \end{enumerate}
\end{defn}

\begin{exam}
  Hence $\MTM(k) \subset \DTM(k)$ as full subcategories. If
  (B-S)$_{k}$ is satisfied, it is easy to see---using~\ref{exam:9-gm}
  and the next result~\ref{thm:9-mtmk}---
  that $\Gm\in \DTM(k)$ but $\notin\MTM(k)$.%Write ses N\to M(G_m)\to
                                %Q(n) in MTM. last arrow is 0 unless
                                %n=0 by B-S. Hence n=0 and last arrow
                                %is projection onto first factor. Thus
                                %we get N=Q(1)[1] but this is not MTM
                                %by the same argument.
\end{exam}

The following theorem is a formal consequence of how the sets
$\Hom_{\DM(k)}(\Q(m)[i],\Q(n)[j])$ look like for different
$m,n,i,j\in\Z$---assuming the Beilinson-Soulé vanishing conjecture
(see~\cite{levine92-tatemotives}):
\begin{thm}\label{thm:9-mtmk}
  If (B-S)$_k$, then
  \begin{enumerate}
  \item\label{thm:9-abelian} There exists a $t$-structure on $\DTM(k)$ whose heart is
    $\MTM(k)$. $\MTM(k)$ is abelian and closed under extensions.
  \item\label{thm:9-tensor} The $\otimes$-structure restricts to $\MTM(k)$ and makes it
    rigid. More generally, the $t$-structure is compatible with the
    tensor structure.
  \item There exists a finite, functorial, increasing $2\Z$-filtration
    $W$ on $\DTM(k)$ which induces an exact filtration on
    $\MTM(k)$. Moreover,
    \begin{equation*}
      \gr_{2n}^W M = \oplus_{i=1}^N \Q(-n), \qquad \gr_{2n}^W \otimes \gr_{2m}^W = \gr_{2(n+m)}^W
    \end{equation*}
  \item\label{thm:9-fiber} \begin{align*}
      \omega_W:\MTM(k)&\to\Q\Mod\\
      N &\mapsto \oplus_{n \in \Z}\Hom_{\MTM(k)}(\Q(-n), \gr_{2n}^WN)
      \end{align*}
      is an exact, faithful $\otimes$-functor.
  \item\label{thm:9-ext} The $\Ext$-groups satisfy
    \begin{eqnarray*}
      \Ext^1_{\MTM(k)}(N,N') & = & \Hom_{\DTM(k)}(N,N'[1]) \\
      \Ext^2_{\MTM(k)}(N,N') & \inj & \Hom_{\DTM(k)}(N,N'[2])
    \end{eqnarray*}
  \end{enumerate}
\end{thm}

\begin{cor}\label{cor:9-mtm-tannaka}
  $\MTM(\Q)$ is a neutral Tannakian category with fiber functor
  $\omega_{W}$, and
  \begin{align*}
    \Ext^1_{\MTM(\Q)}(\Q(0), \Q(n)) &= \left\{ \begin{array}{ll}
        \Q^{\times} \otimes \Q &:n = 1 \\
        \Q &:n\geq 3 \textrm{ odd} \\
        0 &:\textrm{otherwise}
      \end{array} \right.\\
    \Ext^2_{\MTM(\Q)}(\Q(0), \Q(n)) &= 0
  \end{align*}
\end{cor}
\begin{proof} By Borel's theorem~\ref{thm:9-borel}, $\Q$ satisfies the
  Beilinson-Soulé vanishing conjecture hence the theorem applies.  By
  parts~\ref{thm:9-abelian} and \ref{thm:9-tensor}, $\MTM(\Q)$ is a
  rigid abelian $\otimes$-category. By~\ref{thm:9-fiber}, $\omega_{W}$
  is a fiber functor, and by Example~\ref{exam:9-hom-base},
  $\Hom_{\MTM(k)}(\Q(0),\Q(0))=\Q$.

  The computation of the $\Ext$-groups follows from
  part~\ref{thm:9-ext} of the theorem together with~\ref{thm:9-borel}.
\end{proof}

\begin{rem}
  The theorem exemplifies the idea we mentioned in the introduction,
  namely that one should be able to recover $\MM(k)$ from $\DM(k)$ via
  the construction of a $t$-structure. The theorem shows that this is
  possible in the case of mixed \emph{Tate} motives.

  There is a canonical functor $\D^{b}(\MTM(k))\to \DTM(k)$ which in
  general is not expected to be an equivalence. It is one however in
  the case of number fields. TODO: reference
\end{rem}
\subsection{Mixed Tate motives over $\Z$}

At this stage we could apply the Tannakian theory developed in the
last chapter and get a fundamental group associated to
$(\MTM(\Q),\omega_{W})$. However, this fundamental group would be
``too big'' for our purposes; in particular, if we want to apply the
results from the last chapter (as we do), 
we should ensure that the groups $\Ext^1(\Q(0), \Q(n))$ are
finite-dimensional, in particular that $\Ext^1(\Q(0), \Q(1))$ is. The
restriction to motives ``defined over $\Z$'' is intended to achieve
exactly this.

Let us first describe explicitly the elements of the group
$\Ext^1_{\MTM(k)}(\Q(0), \Q(1))$.
\begin{defn}
  Let $f : Y \to X$ be a morphism of smooth varieties. This defines a
  complex 
  \begin{equation*}
    [Y]\to[X]
  \end{equation*}
  concentrated in degrees 1 and 0 in $K^{b}(\SmCor(k))$. Its image in
  $\DM(k)$ is denoted by $M(X,Y)$ and is called the \emph{relative
    motive associated to $f$}.
\end{defn}

\begin{rem}
  Every such $f$ induces a distinguished triangle
  \begin{equation*}
    M(X)\to M(X,Y)\to M(Y)[1]\to^{+}%the degree 1 map actually comes
                                %with a minus sign
  \end{equation*}
  in $\DM(k)$.
\end{rem}
\begin{exam}
  Let $t \in k^* / \{ 1 \}$ and consider the embedding
  \begin{equation*}
    \{ 1,t \} \inj \Pspace^1 \setminus \{0, \infty\}=\Gm.
  \end{equation*}
  The associated motive gives rise to the distinguished triangle
  \begin{equation*}
    M(\Gm)\to M(\Gm,\{1,t\})\to (\Q(0)\oplus\Q(0))[1]\to^{+}
  \end{equation*}
  It's not difficult to deduce from this another distinguished triangle
  \begin{equation*}
    \tilde{M}(\Gm)\to M(\Gm,\{1,t\})\to \Q(0)[1]\to^{+}
  \end{equation*}%use base point 1 for the two outer objects and
                 %apply 9-lemma for triangulated categories
  and by applying the shift functor:
  \begin{equation*}
    \Q(1)\to M(\Gm,\{1,t\})[-1]\to \Q(0)\to^{+}.
  \end{equation*}
  By the theorem, this induces a short exact sequence
  \begin{equation*}
    0\to\Q(1)\to M(\Gm,\{1,t\})[-1]\to \Q(0)\to 0,
  \end{equation*}
  hence we obtain an element $K(t)\in\Ext^1_{\MTM(k)}(\Q(0), \Q(1))$,
  called the \emph{Kummer extension associated with
    $t$}. $M(\Gm,\{1,t\})[-1]\in\MTM(k)$ is called the \emph{Kummer
    motive associated with $t$}.

  The association
  \begin{align*}
    k^{*}\otimes \Q&\to \Ext^{1}_{\MTM(k)}(\Q(0),\Q(1))\\
    t\otimes 1&\mapsto K(t)\\
  \end{align*}
  induces a $Q$-vector space isomorphism.

  Finally, let us consider the realizations of $K(t)$. The Hodge
  realization of this extension was described in chapter (TODO:
  Pre-Alpbach). For the étale realization fix a prime $\ell$. Then
  $\omega_{\ell}(K(t))$ arises from an extension
  \begin{equation}
    0\to \Z_{\ell}(1)\to K_{t,\ell}\to \Z_{\ell}\to 0\label{eq:9-kummer-ladic}
  \end{equation}
  by tensoring with $\Q_{\ell}$. (\ref{eq:9-kummer-ladic}) in turn
  arises from a projective system of extensions
  \begin{equation*}
    0\to \mu_{\ell^{n}}(\overline{k})\to K_{t,\ell^{n}}\stackrel{f_{n}}{\to} \Z/\ell^{n}\Z\to 0,
  \end{equation*}
  where $\mu_{\ell^{n}}(\overline{k})$ is the group of $\ell^{n}$-th
  roots of 1 in the algebraic closure of $k$. Such an extension is
  given by a $\mu_{\ell^{n}}(\overline{k})$-torsor (corresponding to
  $f_{n}^{-1}(1)$); in this case it is the set of $\ell^{n}$-th roots
  of $t$ in $\overline{k}$.

  Now, let $v$ be a finite place with residue characteristic distinct
  from $\ell$. Then $\omega_{\ell}(K(t))$ is unramified at $v$ if and
  only if $t\in {\cal O}_{(v)}^{\times}$.
\end{exam}

Having described the elements of the infinite dimensional $\Q$-vector
space $\Ext^{1}_{\MTM(\Q)}(\Q(0),\Q(1))$ as $\Q$-linear combinations
of Kummer extensions we will simply remove all Kummer motives from
$\MTM(\Q)$ to obtain $\MTM(\Z)$:
\begin{defn}
  \begin{enumerate}
  \item We say a motive $N\in\MTM(\Q)$ is \emph{defined over $\Z$} if
    for all primes $p$ there exists a prime $\ell\neq p$
    such that $\omega_{\ell}(N)$ is unramified at $p$.
  \item The \emph{category of mixed Tate motives over $\Z$},
    $\MTM(\Z)$ is the full subcategory of $\MTM(\Q)$ spanned by
    motives defined over $\Z$.
  \end{enumerate}
\end{defn}

\begin{rem}\label{rem:9-mtmZ}
\begin{itemize}
\item The category $\MTM(\Z)$ is a Tannakian subcategory of $\MTM(\Q)$.
\item The $\Ext$-groups now look as follows:
  \begin{align*}
    \Ext^1_{\MTM(\Z)}(\Q(0), \Q(n)) &= \left\{ \begin{array}{ll}
        \Q &:n\geq 3 \textrm{ odd} \\
        0 &:\textrm{otherwise}
      \end{array} \right.\\
    \Ext^2_{\MTM(\Z)}(\Q(0), \Q(n)) &= 0
  \end{align*}
\end{itemize}
\end{rem}

\begin{exam}
  Let $X$ be a smooth scheme over $k$ such that $M(X) \in
  \MTM(\Q)$, and assume that $X$ has good reduction at all
  primes. Then $M(X) \in \MTM(\Z)$.
\end{exam}

\section{Fundamental group and periods}
Now we are ready to apply the results from the last chapter to mixed
Tate motives over $\Z$.

\subsection{Tannakian structure}
We saw in~\ref{cor:9-mtm-tannaka} that $\MTM(\Q)$ is a neutral Tannaka
category with fiber functor $\omega_{W}$; by~\ref{rem:9-mtmZ},
$\MTM(\Z)$ is also a neutral Tannaka category with the restriction of
$\omega_{W}$ as fiber functor. Hence we may apply (TODO: Tannaka main
theorem) from the last chapter to obtain an algebraic group $G_M$ over
$\Q$ such that
\begin{equation}\label{eq:9-mtmZ-gm}
  \mathrm{Rep}(G_M) = (\MTM(\Z), \omega_{W}).
\end{equation}
By Theorem~\ref{thm:9-mtmk} and (TODO: last chapter, semidirect
product), $G_{M}$ is of the form
\begin{equation*}
  G_{M}=\Gm\ltimes U_{M}
\end{equation*}
with $U_{M}$ a pro-unipotent group over $\Q$. Moreover,
by~\ref{rem:9-mtmZ} and (TODO: last chapter, ring of functions), the
ring of functions of $U_{M}$ is a graded Hopf algebra of the form
\begin{equation}\label{eq:9-hopfalgebra}
  T(\oplus_{n>0}\Q\cdot e_{2n+1}),
\end{equation}
$\Q\cdot e_{2n+1}$ being a one-dimensional $\Q$-vector space in degree
$2n+1$, corresponding to $\Ext^{1}_{\MTM(\Z)}(\Q(0),\Q(2n+1))$.

The Hodge realization functor $\omega_{H}$
(\ref{exam:9-hodge-realization}) induces a $\otimes$-functor which
makes the following diagram commutative:
\begin{equation}
\xymatrix{
\MTM(\Q) \ar[rr]^{\omega_H} \ar[dr]_{\omega_{W}} && \MTH(\Q) \ar[dl]^{\omega_{dR}} \\
&\Q\Mod}\label{eq:9-group-morphism}
\end{equation}
Here, $\MTH(\Q)$ denotes the category of mixed Hodge structures over
$\Q$ of Tate type defined in (TODO: chapter 6). Commutativity follows
from the fact that $\omega_{W}$ on $\MTM(\Q)$ is the same as
$\omega_{\mathrm{dR}}$.

(\ref{eq:9-group-morphism}) induces a morphism of fundamental groups
by Tannaka duality:
\begin{equation*}
  \omega^{H}:G_{H}\to G_{M},
\end{equation*}
where $G_{H}$ is the fundamental group associated to
$(\MTH(\Q),\omega_{\mathrm{dR}})$ (see (TODO: last chapter, example)).

\begin{prop}
  $\omega^{H}$ is an epimorphism.
\end{prop}
\begin{proof}
  It suffices to prove surjectivity on the pro-unipotent factors
  $U_{H}\to U_{M}$ and by Lemma (TODO: last chapter) it suffices to
  prove this on $H_{1}$ hence on the abelianization of the associated
  graded pro-Lie algebras ${\cal U}_{H}^{\mathrm{ab}}\to{\cal
    U}_{M}^{\mathrm{ab}}$. But in degree $n$ this is dual to the
  canonical map
  \begin{equation*}
    \Ext^{1}_{\MTM(\Z)}(\Q(0),\Q(n))\to \Ext^{1}_{\MTH(\Q)}(\Q(0)_{H},\Q(n)_{H}),
  \end{equation*}
  see~\cite[A.15]{deligne-goncharov05}. Injectivity of this map
  follows from injectivity of the ``regulator map'' in
  $K$-theory. (TODO: Give reference)%TODO: Check this argument.
\end{proof}

\begin{cor}\label{cor:9-mixed-hodge-tate-realization}
  $\omega_{H}$ is fully faithful and its image is closed under taking
  subquotients.
\end{cor}
\begin{proof}
  By the Proposition, we can write
  \begin{equation*}
    \mathrm{Rep}(G_M) = \left\{ \phi \in \mathrm{Rep}(G_H) \mid \textrm{$\phi$ factors through $G_M$} \right\}    
  \end{equation*}
  The claim is now clear by~(\ref{eq:9-mtmZ-gm}) and the corresponding
  equivalence for $G_{H}$.
\end{proof}

\subsection{Upper bounds on periods}

Let $X$ be a variety over $\Q$. Complex conjugation $\mathrm{c}$
defines an involution on the $\C$-points of $X$ and hence on the
singular homology of $X(\C)$. This induces an involutive automorphism
of the $\otimes$-functor 
\begin{equation*}
\omega_{\mathrm{B}}:\MTM(\Z) \to \D^b(\Q\Mod)
\end{equation*}
(this requires an understanding of the construction of
$\omega_{\textrm{B}}$ TODO: The details are rather long; should I
write them down? Also: Isn't there an easier way to see
this?)%we want to prove that omega c = c omega and actually
% we are going to prove this on whole DM. By the
% universality of the steps in the construction of DM
% it suffices to prove functoriality on SmCor. Here
% one has to see how singular homology is extended to
% SmCor by Huber in 2.1.6 (proof on page 779). So fix
% elementary correspondence alpha from X to Y, both
% smooth varieties. (we will in the following discuss
% cohomology; homology is obtained by dualization
% on DM) Then the image of alpha will be the composition tilde{R}'(Y)
% to tilde{R}'(alpha) to tilde{R}'(X) and we have to show
% functoriality with respect to both arrows. This means that two
% squares have to commute, where the horizontal arrows are these ones
% and the vertical ones are the ones induced by c on tilde{R}' (this
% is a colimit, thus the arrow is induced by a morphism of directed
% systems). This is clear for the second arrow since this "is" simply
% functoriality of betti cohomology with respect to c. For the first
% one has to see the definition of "covariant functoriality" in
% 2.1.10; but it also  boils down to the same functoriality.
, in other words we get an element $\varepsilon$ of order 2 in
$G_{\mathrm{B}}$, the fundamental group of
$(\MTM(\Z),\omega_{\mathrm{B}})$ (that $\omega_{\mathrm{B}}$ is indeed a fiber
functor follows from the factorization
$\omega_{\mathrm{B}}=\omega_{\mathrm{B}}\circ\omega_{\mathrm{H}}$
together with Corollary~\ref{cor:9-mixed-hodge-tate-realization} and
Example (TODO: reference to example in last chapter, where it is
stated/shown that Betti realization on MTH(Q) is fiber
functor)). %Indeed, both functors in the factorization are tensor,
           %faithful and exact
$\varepsilon$ induces an involutive action $\tilde{c}$ on
$\underline{\mathrm{Isom}}_{\Q}^{\otimes}(\omega_{\mathrm{dR}},
\omega_{\mathrm{B}})$ and we claim that it makes the following square
commute:
\begin{equation*}
\xymatrix{
\underline{\mathrm{Isom}}_{\Q}^{\otimes}(\omega_{\mathrm{dR}}, \omega_{\mathrm{B}}) \ar[r]^{\tilde{c}} & \underline{\mathrm{Isom}}_{\Q}^{\otimes}(\omega_{\mathrm{dR}}, \omega_{\mathrm{B}}) \\
\spec \C \ar[r]^c \ar[u]^{\mathrm{comp}} & \spec \C \ar[u]^{\mathrm{comp}}}
\end{equation*}
In other words, we claim that the following square commutes:
\begin{equation*}
\xymatrix{\omega_{\mathrm{dR}}\otimes_{\Q}\C\ar[r]^{\mathrm{comp}}&\omega_{\mathrm{B}}\otimes_{\Q}\C\ar[d]^{\varepsilon}\\
&\omega_{\mathrm{B}}\otimes_{\Q}\C\\
\ar[uu]^{\mathrm{can}}\omega_{\mathrm{dR}}\otimes_{\Q}\C\otimes_{\mathrm{Id},\C,c}\C\ar[r]_{\mathrm{comp}\otimes\C}&\omega_{\mathrm{B}}\otimes_{\Q}\C\otimes_{\mathrm{Id},\C,c}\C\ar[u]_{\mathrm{can}}}
\end{equation*}
But this can be checked on varieties %Homology
% is cohomology \circ dual, so it suffices to prove it for cohomology.
% And every tensor triangulated Q-linear realization functor on DM is
% uniquely determined on its objects by the corresponding morphism on
% varieties.
and, taking into account the definition of $\mathrm{comp}$, one
gets for $X$ a smooth variety over $\Q$:
\begin{equation*}
\xymatrix{H^{*}_{\mathrm{dR}}(X)\otimes_{\Q}\C\ar[r]^{\int}&H^{*}_{B}(X_{\C}^{\mathrm{an}},\C)\ar[d]^{\varepsilon}&H^{*}_{B}(X_{\C}^{\mathrm{an}},\Q)\otimes_{\Q}\C\ar[d]^{\varepsilon}\ar[l]_{\mathrm{can}}\\
&H^{*}_{\mathrm{B}}(X_{\C}^{\mathrm{an}},\C)&\ar[l]_{\mathrm{can}}H^{*}_{B}(X_{\C}^{\mathrm{an}},\Q)\otimes\C\\
\ar[uu]^{\mathrm{can}}H^{*}_{\mathrm{dR}}(X)\otimes_{\Q}\C\otimes_{\mathrm{Id},\C,c}\C\ar[r]_{\int\otimes\C}
&H^{*}_{\mathrm{B}}(X_{\C}^{\mathrm{an}},\C)\otimes_{\mathrm{Id},\C,c}\C\ar[u]^{\psi}&\ar[l]_{\mathrm{can}}H^{*}_{\mathrm{B}}(X_{\C}^{\mathrm{an}},\Q)\otimes\C\otimes_{\mathrm{Id},\C,c}\C\ar[u]_{\mathrm{can}}
}  
\end{equation*}
Here, $H^{*}_{dR}$ is algebraic deRham cohomology over $\Q$,
$H^{*}_{B}$ is singular cohomology and $\psi$ is the $\C$-linear map
which takes a function $f:C_{p}(X_{\C}^{\mathrm{an}})\to\C$ on the
singular group to $c\circ f$. This makes the lower square on the right
commute. Clearly also the upper square on the right is commutative. It
remains to check commutativity of the left part of the diagram, which
is the following equality
\begin{equation*}
  \int_{\gamma}c^{*}\theta=\int_{c\circ\gamma}\theta=\overline{\int_{\gamma}\theta}=\int_{\gamma}\overline{\theta}
\end{equation*}
for algebraic $p$-forms $\theta$ on $X$ over $\Q$, and $p$-cycles
$\gamma$ on $X_{\C}^{\mathrm{an}}$. This follows from the fact that
$\theta$ is defined over $\Q$.

Now, we are in the setting where we can apply the (TODO: reference to
last chapter) and we deduce that the real periods,
$P^c_{\mathrm{comp},+}$, are a graded quotient of
\begin{equation}\label{eq:9-graded-algebra-upper-bound}
  \Q[t^2] \otimes
  T\left(\bigoplus_{n > 0} \Q\cdot e_{2n+1}\right),
\end{equation}
where $t$ has degree 1 (cf~(\ref{eq:9-hopfalgebra})). To get a bound
on the graded components of $P^c_{\mathrm{comp},+}$, we
compute the Poincaré series
of~(\ref{eq:9-graded-algebra-upper-bound}) over $\Q$; it is given by:
\begin{eqnarray*}
\varphi(x) & = & \frac{1}{1-x^2} \cdot\sum_{i \geq 0}
\left(
\frac{x^3}{1-x^2}\right)^{i} \\
& = & \frac{1}{1-x^2} \cdot \frac{1}{1-\frac{x^3}{1-x^2}} \\
& = & \frac{1}{1-x^2} \cdot \frac{1-x^2}{1-x^2-x^3} \\
& = & \frac{1}{1-x^2-x^3}.
\end{eqnarray*}
Setting 
\begin{equation*}
D_n = D_{n-2} + D_{n-3}, \qquad D_0 = D_2 = 1, D_1 = 0,
\end{equation*}
we thus conclude
\begin{equation*}
d_n := \dim_{\Q} \gr_n P^c_{\mathrm{comp},+}\leq D_n,\quad\text{ for all }n.
\end{equation*}



\chapter{Motivic structure: Motivic fundamental group by Simon Pepin Lehalleur}
%\addcontentsline{toc}{chapter}{Motivic structure: Motivic fundamental group by Simon Pepin Lehalleur}

Simon Pepin Lehalleur on September 6th, 2012.

\medskip
\medskip

Our starting point is the following basic situation and facts from talk 7 :

\begin{itemize}
\item Let $X = \Pspace^1_k \setminus D$ for $k \inj \C$, $D$ the support of a divisor defined over $k$ and containing $\infty$, and $a, b \in X(k)$. When needed, we denote by $t$ a coordinate function on $X$.
\item We have defined $\pi_1(X;a,b)_H$, a $\mathrm{MH}(k)$-affine scheme, and the ``composition of paths'' groupoid structure when varying $a,b$.
\item Let $x, y \in D(k), u \in T_x \Pspace^1 \setminus \{0\}, v \in T_y \Pspace^1 \setminus \{0\}$. We have $\pi_1(X;a,u)_H$ and $\pi_1(X;u,v)_h$ $\mathrm{MH}(\Q)$-affine schemes, again with composition of paths.
\item Multiple zeta values are real periods of  $\pi_1(\Pspace^1 \setminus \{ 0,1,\infty\},\overrightarrow{01},\overrightarrow{10})$ :
\[
\xymatrix{
\stackrel{0}{\mathrm{x}} \ar@{-<}[r] & \stackrel{1}{\mathrm{x}} \ar@{-<}[l] & \stackrel{\infty}{\mathrm{x}}
}
\]
\end{itemize}

The goal of this talk is to prove similar facts in the category $\MTM(k)$ (even $\MTM(\Z)$ for the last item), and to combine this with the bounds on periods that was derived from the structure of $\MTM(\Z)$ in talks 8 and 9 to prove the Goncharov-Terasoma bounds on the dimension of spaces of multizeta values.

We start in the more general setting of $X\in \Smk$ for any field $k$ of characteristic $0$, and add increasingly strong hypotheses (which are all satisfied in the case of $\Pspace^1_\Q\setminus\{0,1,\infty\}$) to perform the construction. 

\section{The case of interior base points}

\subsection{The motivic bar complex}

The idea is to perform the bar construction on $M(X)^{\vee}$ in order to produce a motivic bar complex $B_M^*(X;a,b) \in \mathrm{Ind}(\DMk)$. The idea meets the following technical problems :

\begin{enumerate}
\item $\DMk$ is too small, it does not contain the analogues of unbounded complexes used in the bar construction. Therefore we define the analogue of the truncated bar complexes and get an Ind-object.
\item The triangulated structure of $\DMk$ alone does not provide a functor $Tot : K^b(\DMk) \to \DMk$. So we need to do the construction at the level of $K^b(\SmCorr)$.
\item On the other hand, duality is only defined on $\DMk$. So we have to take duals at the end.
\end{enumerate}

We define :
\[
\begin{array}{rccccccl}
\mathrm{CPX}^{a,b}_* = [ & \spec k & \stackrel{d_0}{\longrightarrow} & X & \stackrel{d_1}{\longrightarrow} & X^2 & \stackrel{d_2}{\longrightarrow} & \cdots ] \in K^b(\SmCorr) \\
& 0 &  & -1 & & -2 & &
\end{array}
\]
where
\begin{eqnarray*}
d_i = \sum_{j=0}^{i} (-1)^j d_i^j, \qquad d_i^j(x_1, \ldots, x_i) = \left\{ \begin{array}{ll}
(a,x_1, \ldots, x_i), & j=0 \\
(x_1, \ldots, x_j, x_j, \ldots, x_i), & 0 < j < i \\
(x_1, \ldots, x_i, b), & j=i
\end{array} \right.
\end{eqnarray*}

We do a naive truncation of this complex :
\[
\sigma_{\geq -n} \mathrm{CPX}_*^{a,b} = [ \spec k \to X \to \cdots \to X^n] \in K^b(\SmCorr)
\]
\[
B_M^{\geq -n}(X;a,b) = M([\sigma_{\geq -n} \mathrm{CPX}_*^{a,b}])^{\vee} \in \DMk
\]
And finally :
\[
B_M(X;a,b) = \varinjlim_n B^{\geq -n}_M(X;a,b) \textrm{``$=$''}[ \cdots \to \stackrel{-2}{M(X^2)^{\vee}} \to \stackrel{-1}{M(X)^{\vee}} \to \stackrel{0}{\spec k}]
\]
This is the motivic bar complex.

\subsection{Operations}
By analogy with the bar construction of Talk 4., we want to define on the motivic bar complex : a product $\nabla$ which encodes the shuffle, a coproduct $\Delta$ which encodes the composition of paths and an antipode $S$ which encodes the reversal of paths. Because of the duality step in the construction, we need operations the other way at the geometric level.
\subsubsection{Shuffle product}
We need to define shuffle coproducts:
\[
\nabla^\vee:\sigma_{\geq -n} \mathrm{CPX}_*^{a,b} \to \sigma_{\geq -n} \mathrm{CPX}_*^{a,b} \otimes \sigma_{-n} \mathrm{CPX}_*^{a,b}
\]
The shuffle coproduct before truncations:
\[
\begin{array}{rcl}
\nabla^\vee\mathrm{CPX}_*^{a,b} & \to & \mathrm{CPX}_*^{a,b} \otimes \mathrm{CPX}_*^{a,b} \\
(x_1, \ldots, x_n) & \mapsto & (\sum_{\sigma \in \Sigma_{p,q}} \mathrm{sgn}(\sigma)(x_{\sigma(1)}, \ldots, x_{\sigma(p)}) \otimes (x_{\sigma(p+1)}, \ldots, x_{\sigma(n)}))_{p+q=n}
\end{array}
\]
The sum is indexed by $(p,q)$-shuffles, as in Talk 4.

The counit map before truncations is simply the projection $u^\vee:\mathrm{CPX}_*^{a,b}\rightarrow [Spec(k)]$.

\begin{lemma}
This coproduct is :
\begin{itemize}
\item coassociative
\item graded cocommutative
\item counital with respect to $u^{\vee}$
\end{itemize}
\end{lemma}

\subsubsection{Composition of paths products}
This time we need to define composition of paths products for $a,b,c\in X(k)$ :
\[
\begin{array}{rcl}
\mathrm{CPX}_*^{a,b} \otimes \mathrm{CPX}_*^{b,c} & \to & \mathrm{CPX}_*^{a,c} \\
(x_1, \ldots, x_k) \otimes (y_1, \ldots, y_l) & \mapsto & (x_1,b,\ldots,b,y_1,\ldots,y_l)+(x_1,x_2,b,\ldots b,y_1,\ldots,y_l)+\ldots\\
& & +(x_1,\ldots,x_k,y_1,\ldots y_l)+(x_1,\ldots,x_k,b,y_2,\ldots y_l)+(x_1,\ldots,x_k,b,\ldots,b,y_l)
\end{array}
\]

When $a=b$, the differential $d_0$ of $\mathrm{CPX}_*^{a,a}$ is zero, which allows to define a unit map $\epsilon^\vee:[Spec(k)]\rightarrow \mathrm{CPX}_*^{a,a}$.

\begin{lemma}
The composition of paths is :
\begin{itemize}
\item associative
\item unital with respect to $\epsilon^\vee$
\end{itemize}
\end{lemma}

\begin{lemma}
The shuffle coproduct and the composition of paths products on $\mathrm{CPX}_*^{a,b}$ are compatible.
\end{lemma}

\subsubsection{Reversal of paths antipode}
\[
\begin{array}{cccc}
S^{\vee}: & \mathrm{CPX}_*^{a,b}&\to & \mathrm{CPX}_*^{b,a}\\
& (x_1,\ldots,x_n) & \mapsto & (x_n,\ldots,x_1)
\end{array}
\]

\begin{lemma}
The map $S^{\vee}$ makes the previous bialgebra structure on $(\mathrm{CPX}_*^{a,a})$ into an Hopf algebra.
\end{lemma}

\subsubsection{Truncations}
It remains to check that these operations on the non-truncated complex $\mathrm{CPX}_*^{a,b}$ induce similar operations on the truncations, which we can then dualize, and take the direct limit (checking that the operations are compatible with the direct system). This is somewhat complicated by the fact that the tensor product of complexes mixes various degrees. A clearer picture of what goes on will be sketched below using the cosimplicial point of view. We simply admit it for the time being.

To sum up the four previous sections :

\begin{prop}
There is a natural structure of Hopf algebroid structure on the system $(B_M(X;a,b))_{a,b\in X(k)}$.
\end{prop}

Like in Talk 4., one can define a normalized version $\tilde{B}_M(X;a,b)$ of the motivic bar complex, and this is the choice made in the original paper \cite{deligne-goncharov05}.

\subsection{Passing to $\MTMZ$}

We want to use the more precise results we have for mixed Tate motives. We have :

\begin{prop}
 Assume that $M(X)\in \DMT$. Then $B_M(X;a,b) \in \mathrm{Ind}(\DMT)$.
\end{prop}
\begin{proof}
We have to check that the truncated complexes $B_M^{\geq -n}(X;a,b)$ are in $\DMT$. Since duality preserves $\DMT$, we have to show that $M(\sigma_{\geq -n}\mathrm{CPX}^{a,b}_*)$ is in $\DMT$. We proceed by induction on $n$. This is clear for $M(\sigma_{\geq 0}\mathrm{CPX}^{a,b}_*)=M(Spec(k))=\Q(0)$. We have a short exact sequence in $K^b(\SmCorr)$ :
\[
0\rightarrow X^{n+1}[-n-1]\rightarrow \sigma_{\geq -(n+1)}\mathrm{CPX}^{a,b}_*\rightarrow \sigma_{\geq n}\mathrm{CPX}^{a,b}_*\rightarrow 0
\]
This provides an exact triangle in $\DMT$ :
\[
M(X^{n+1}[-n-1])\rightarrow M(\sigma_{\geq -(n+1)}\mathrm{CPX}^{a,b}_*)\rightarrow M(\sigma_{\geq n}\mathrm{CPX},^{a,b}_*)\rightarrow +
\]
 But $M(X^{n+1}[-n-1])=M(X)^{\otimes n+1}[-n-1]\in \DMT$, and we conclude using the induction hypothesis.
\end{proof}

Recall the following proposition from Talk 9 :

\begin{prop}
Assume $(B-S)_k$. Then there is a $t$-structure on $\DMT$, with heart $\MTM(k)$.
\end{prop}

We denote the cohomological functor to the heart by :
\[
H^0 = H_0 : \DMT \to \MTM(k)
\]
(Here $H^n = H_{-n}$.)

For the rest of the section, we assume that $M(X)\in \DMT$ and that $(B-S)_k$ holds. This is satisfied in the basic situation. We then define :
\[
H^0(B_M(X;a,b)):=\varinjlim_{n}H^0(B_M^{\geq -n}(X;a,b))\in\mathrm{Ind}(\MTM(k))
\]
A more elegant approach would be to extend the triangulated structure and the $t$-structure to $\DMT$, but we will not need this.

Since the tensor product is $t$-exact, we have for any objects $M,M'\in \DMT$ :
\[
H^0(M\otimes M')\simeq \sum_{n\in\Z}H^n(M)\otimes H^{-n}(M')
\]
and we get lax (resp. oplax) monoidal structures on the functor $H^0$, namely the inclusion of the $H^0\otimes H^0$ factor (resp. the projection to it). (It is not so obvious that the projection is oplax)

This allows to transfer our algebraic structures on $B_M(X;a,b)$ to $H^0(B_M(X;a,b))\in \mathrm{Ind}(\MTM(k))$ (one has to go back to finite level and use the truncated operations...). So we get a shuffle product, and ``composition of paths'' coproduct. The end result is :

\begin{defn}[Motivic fundamental group]
Let $X$ be a smooth variety over $k$, $a,b\in X(k)$. Then its motivic fundamental path torsor (motivic fundamental group if $a=b$) $\pi_1(X;a,b)$ is the $\MTM(k)$-affine scheme $\spec (H^0(B_M(X;a,b)))$. For varying $a,b$, it forms a pro-algebraic groupoid.
\end{defn}

Now we make the following extra assumption on $X$ : $M(X)\in \DMT$ has non-positive weights, i.e. $M(X)=W_0M(X)\in \DMT$, or equivalently that $W_{-1}(M(X)^{\vee})=0$. This does not seem to follow from $M(X)\in \DMT$ and $(B-S)_k$, because this would imply the Beilinson-Soul\'e conjecture for any such variety. But it is satisfied in the basic situation, because $M(\Pspace^1)=\Q\oplus \Q(1)[2]$ has negative weights and the localization exact sequence only adds $\Q(1)[1]$ factors.

Note that because of the functoriality of the weight filtration on $\MTM(k)$, it extends to a functorial filtration on $\mathrm{Ind}(\MTM(k))$ defined by :
\[
W_k(\varinjlim_{i\in I}M_i):=\varinjlim_{i\in I} W_kM_i
\]

\begin{prop}
The $\MTM(k)$-scheme $\pi_1(X;a,b)$ has non-negative weights, i.e. $W_{-1}\mathcal{O}(\pi_1(X;a,b))=0$.
\end{prop}
\begin{proof}
We have :
\begin{eqnarray*}
W_{-1}(H^0(B_M(X;a,b)) & = & W_{-1}(\varinjlim_{n\in\N}H^0(B^{\geq -n}_M(X;a,b)))\\
& = & \varinjlim_{n\in\N} H^0(W_{-1}B^{\geq -n}_M(X;a,b))\\
\end{eqnarray*}
Now, by duality :
\[
W_{-1}B^{\leq n}_M(X;a,b) = 0\Leftrightarrow W_0\sigma_{\geq -n}\mathrm{CPX}^{a,b}_*=\sigma_{\geq -n}\mathrm{CPX}^{a,b}_*
\]
We prove this by induction on $n$. It is clear for $n=0$. In general, we apply the weight truncation functor $W_0$ on a previous exact triangle to get the following :
\[
W_0(M(X)^{n+1})[n+1]\rightarrow W_0(M(\sigma_{\geq -(n+1)}\mathrm{CPX}^{a,b}_*))\rightarrow W_0(M(\sigma_{\geq n}\mathrm{CPX}^{a,b}_*))\rightarrow +
\]
Now $M(X)$ has non-positive weights, so $M(X^{n+1})$ has non-positive weights, and we conclude by the induction hypothesis.
\end{proof}

\subsubsection{The cosimplicial point of view on the operations}
\label{cosimp}
This section is not strictly necessary to understand the end of this talk, but it may help to understand where the formulas come from and how to replace some computations by functoriality arguments. This requires to know the language of cosimplicial objects. For an introduction see for instance \cite{brbr}, Chapters 2 and 7.

Let $\Delta^\bullet$ be the standard cosimplicial simplicial set (i.e., a cosimplicial set in the category of simplicial sets), and $\Delta^1$ the constant cosimplicial simplicial set built out of an interval. For any $X\in \SmCorr$, there is a cosimplicial ``free path space'' : $\mathrm{PX}^\bullet:=X^{Hom(\Delta^\bullet,\Delta^1)}\in \Delta \SmCorr$, with a map  $\partial :\mathrm{PX}^\bullet\rightarrow X\times X=X^{Hom(\Delta^\bullet,\partial \Delta^1)}$ (which should be thought of as (starting point, ending point)), and for $a,b\in X(k)$, we define $\mathrm{PX}^\bullet_{a,b}$ as the fiber product $\mathrm{PX}^{\bullet} _{\partial}\times_{X\times X,(a,b)}Spec(k)$.

More concretely, $\mathrm{PX}^{n}_{a,b}\simeq X^n$, with cofaces $d^i:\mathrm{PX}^{n}_{a,b}\rightarrow\mathrm{PX}^{n+1}_{a,b}$, $i=0\ldots n+1$ given by :
\[
d^i(x_1,\ldots x_n)=\left\{\begin{array}{c}(a,x_1,\ldots x_n), i=0\\(x_1,\ldots,x_i,x_i,\ldots,x_n), i\neq 0,n+1\\(x_1,\ldots,x_n,b), i=n+1\end{array} \right.
\]
and codegeneracies $s^i:\mathrm{PX}^{n}_{a,b}\rightarrow\mathrm{PX}^{n-1}_{a,b}$, $i=0,\ldots n-1$ :
\[
s^i(x_1,\ldots,x_n)=(x_1,\ldots, \widehat{x_{i+1}},\ldots, x_n)
\]
These path objects come with two natural algebraic structures coming from the geometry of $X$. First, a diagonal embedding coproduct :
\[
\begin{array}{cccc} \nabla^{\vee}: &\mathrm{PX}^\bullet_{a,b} & \rightarrow &\mathrm{PX}^\bullet_{a,b}\times\mathrm{PX}^\bullet_{a,b}\\ & (x_1,\ldots,x_n) & \mapsto & ((x_1,\ldots x_n), (x_1,\ldots, x_n))\end{array}
\] 
with a counit map $u^\vee$ being simply the structure map as $k$-schemes, and composition of paths products :
\[
\begin{array}{cccc} \Delta^{\vee}: &\mathrm{PX}^\bullet_{a,b}\times\mathrm{PX}^\bullet_{b,c} & \rightarrow &\mathrm{PX}^\bullet_{a,c}\\ & ((x_1,\ldots,x_n),(y_1,\ldots y_n)) & \mapsto & \sum_{i=0}^{n}(x_1,\ldots x_i,y_{i+1},\ldots,y_n) 
\end{array}
\]
The formula should be interpreted as a finite correspondance sum of actual morphisms of schemes, hence as a morphism in $\SmCorr$. When $a=b$, there is a unit map $\epsilon^\vee$ given by $Spec(k)\rightarrow\mathrm{PX}^\bullet_{a,a}, *\mapsto (a,\ldots,a)$.

There is also an reversal of path antipode :
\[
\begin{array}{cccc} S^{\vee}: &\mathrm{PX}^\bullet_{a,b} & \rightarrow &\mathrm{PX}^\bullet_{b,a}\\ & (x_1,\ldots,x_n) & \mapsto & (x_n,\ldots,x_1)\end{array}
\]

The structures are compatible with each other, i.e., the composition of paths products are morphisms of counital coalgebra objects, and the Hopf algebra diagrams commute. This is easy to check because the coproduct is a diagonal embedding.

Because $\SmCorr$ is additive, there is the ``associated chain complex'' functor :
\[
C:\Delta\SmCorr\rightarrow C_{\leq 0}(\SmCorr)
\]
We have $CPX^{a,b}_*=C(PX^\bullet_{a,b})$, justifying the notation.

Now, the functor $C$ is not strongly monoidal with respect to the product of simplicial objects and the tensor product of complexes, so transferring algebraic structures is not straightforward, but there is a well-known fix to this, namely the Eilenberg-Zilber (or shuffle) map and the Alexander-Whitney map. For any $X^\bullet,Y^\bullet\in \Delta\SmCorr$, those are maps (functorial in $X^\bullet$, $Y^\bullet$) :
\[
\begin{array}{cccc}
EZ :& C(X^\bullet\times Y^\bullet) &\rightarrow & C(X^\bullet)\otimes C(Y^\bullet)\\
& x\otimes y & \mapsto & (\sum_{\sigma\in \Sigma_{p,q}}sgn(\sigma)s^{\sigma(1)-1}\ldots s^{\sigma(p)-1}x\otimes s^{\sigma(p+1)-1}\ldots s^{\sigma(n)-1}y)_{p,q}
\end{array}
\]
where again we index on shuffles, and :
\[
\begin{array}{cccc}
AW :& C(X^\bullet)\otimes C(Y^\bullet)&\rightarrow & C(X^\bullet\times Y^\bullet)\\
& x\otimes y & \mapsto & (d^nd^{n-1}\ldots d^{p+1},d^0\ldots d^0y)
\end{array}
\]
The map $AW$ gives $C$ the structure of a lax monoidal functor, and the map $EZ$ gives $C$ the structure of an oplax monoidal functor. This allows to transfer algebra object (using $AW$) and coalgebra objects (using $EZ$). 

When applied to $\mathrm{PX}^{\bullet}_{a,b}$ we recover the operations $\nabla^{\vee},\ldots$ considered previously. 

For $n\in\N$, there are truncation functors both on the cosimplicial and chain complex level :
\[
sk_n:\Delta\SmCorr\rightarrow \Delta\SmCorr
\]
\[
\sigma_{\geq -n}:C_{\geq 0}(\SmCorr)\rightarrow C_{\geq 0}(\SmCorr)
\]
The functor $sk_n$ commutes with products of simplicial sets because it is a left adjoint.

We do not have a natural isomorphism of functors $C\circ sk_n\simeq \sigma_{\geq -n}\circ C$. However, we do have natural maps in both direction : an inclusion of $\sigma_{\geq -n}\circ C\rightarrow C\circ sk_n$ and a projection (modding out degenerate simplices) $C\circ sk_n
\rightarrow \sigma_{\geq -n}\circ C$. One could also replace the functor $C$ by its normalized subcomplex $N$ as in \cite{deligne-goncharov05}, in which case there is a natural equivalence of functors $N\circ sk_n\simeq \sigma_{\geq -n}\circ N$.

 We can now define the operations on the truncated bar complex by functoriality. For instance, the shuffle coproduct  $\nabla^\vee_n:\sigma_{\geq -n}\mathrm{CPX}^{a,b}_*\rightarrow \sigma_{\geq -n}\mathrm{CPX}^{a,b}_*\otimes \sigma_{\geq -n}\mathrm{CPX}^{a,b}_*$ is defined as the composition :
\begin{eqnarray*}
\sigma_{\geq -n}C(\mathrm{PX}^\bullet_{a,b}) & \rightarrow  & C(sk_n\mathrm{PX}^\bullet_{a,b})\\
& \stackrel{\sigma_{\leq n}\nabla^\vee}{\rightarrow} & C(sk_n(\mathrm{PX}^\bullet_{a,b}\times \mathrm{PX}^\bullet_{a,b}))\\
& \simeq & C(sk_n(\mathrm{PX}^\bullet_{a,b})\times (sk_n\mathrm{PX}^\bullet_{a,b}))\\
& \stackrel{EZ}{\rightarrow} & C(sk_n(\mathrm{PX}^\bullet_{a,b}))\otimes C(sk_n(\mathrm{PX}^\bullet_{a,b}))\\
& \rightarrow &  \sigma_{\geq -n}\mathrm{CPX}^{a,b}_*\otimes \sigma_{\geq -n}\mathrm{CPX}^{a,b}_*
\end{eqnarray*}

The functoriality makes it then relatively easy to check the compatibility of these operations with the maps relating the various trunctations.

\section{The motivic fundamental group of $\Gm$}
In this section, we work out the special case of $\Gm=\Pspace^1_k\setminus \{0,\infty\}$. Compare with the computations in the Hodge setting in Talk. 7.

\begin{prop}
For all $x\in\Gm(k)$, we have a canonical isomorphism :
\[
\pi_1(\Gm;x)_M\simeq Spec(T(\Q(-1)))
\]
with the usual Hopf algebra structure on a tensor algebra. In particular, the fundamental group $\pi_1(\Gm;x)_M$ does not depend on the choice of base-point
\end{prop}
\begin{proof}
Let $n\geq 0$. We write :
\[
C^n:=\sigma_{\geq -n}(CP\Gm)_*^{x,x}=[\spec(k)\rightarrow 
\Gm\rightarrow\Gm^2\rightarrow\ldots\rightarrow\Gm^n\rightarrow 0]\in 
K^b(\SmCorr)
\]
To simplify the proof, we use the normalized complexes using the underlying cosimplicial structure. For this, we need to pass to the pseudo-abelian completion $K^b(\SmCorr)_{psa}:=Psa(K^b(\SmCorr))$. This is possible because $DMT(k)$ is pseudo-abelian by construction so that the functor $M:K^b(\SmCorr)\rightarrow DMT(k)$ extends to $K^b(\SmCorr)_{psa}$. This leads to the following :
\[
\bar{C}^n: [Spec(k)\rightarrow \Gm\rightarrow (\Gm^{\wedge 2})^{Alt}\rightarrow\ldots\rightarrow (\Gm^{\wedge n})^{Alt}]\in Psa(\SmCorr(k))
\]
with $\Gm^{\wedge n}=Im(p_n)$, $p_n=(id-[x])\times (id-[x])\times\ldots \times (id-[x])$ projector on $\Gm^n$ with $[x]:\Gm\rightarrow \{x\}\rightarrow \Gm$.

Now, arguments from the proof of the Dold-Kan correspondance show that $\varinjlim \bar{C}^n$ is quasi-isomorphic to $\varinjlim C^n$, although this is not true at finite levels. Moreover, the operations on the complex can also be defined at the level of the normalized version, as mentioned in the previous section. 

 We have short exact sequences of complexes :
\[
0\rightarrow \bar{C}^{n}\rightarrow \bar{C}^{n+1}\rightarrow (\Gm^{\wedge (n+1)})[-n-1]\rightarrow 0
\]
which leads to a distinguished triangle :
\[
M((\Gm^{\wedge (n+1)}))^\vee[n+1]\rightarrow M(\bar{C}^{n+1})^{\vee}\rightarrow M(\bar{C}^n)^{\vee}\rightarrow +
\]
We have $M(\Gm^{\wedge (n+1)})\simeq \Q(n+1)[n+1]$.

Now, consider the actions on the terms $C^n_i=\Gm^i$ by the symmetric group $\Sigma_i$ by permutations of the factors. This action induces one on $\bar{C}^n_i=\Gm^{\wedge i}$. Now put $(\bar{C}^n_{Alt})_i=(\bar{C}^n_i)^{\sigma^*=\epsilon}$ the sub-object where the action of the symmetric group is via the signature character (This is a subobject because we are in a $\Q$-linear category). Recall the definition of the differential on $C^n$ :

\begin{eqnarray*}
d_i = \sum_{j=0}^{i} (-1)^j d_i^j, \qquad d_i^j(x_1, \ldots, x_i) = \left\{ \begin{array}{ll}
(a,x_1, \ldots, x_i), & j=0 \\
(x_1, \ldots, x_j, x_j, \ldots, x_i), & 0 < j < i \\
(x_1, \ldots, x_i, b), & j=i
\end{array} \right.
\end{eqnarray*}

From it, we see that :
\begin{itemize}
\item $\bar{C}^n_{Alt}$ forms a subcomplex of $\bar{C}_n$
\item The differentials of $\bar{C}^n_{Alt}$ are zero (the differentials $d_i^0$ and $d_i^i$ are killed by passing to $\bar{C}^n$, and the differentials $d_i^j$ for $0<j<i$ feature a repeated term so they are killed in $\bar{C}^n_{Alt}$)
\end{itemize}
We get :
\[
\bar{C}^n_{Alt}=\bigoplus_{i=0}^n(\Gm^{\wedge i})^{Alt}[-i]
\]

Now, the special feature of $\Gm$ compared to more general curves is that the action of $\Sigma_n$ on $M(\Gm^{\wedge n})\simeq \Q(n+1)[n+1]$ is via the signature character, so that $M((\Gm^{\wedge n})^{Alt})\simeq M(\Gm^{\wedge n})$.  Using the previous exact triangle and induction on $n$, we conclude that $M(\bar{C}^n_{Alt})^\vee\rightarrow M(\bar{C}^n)^\vee$ is an isomorphism. This implies :
\[
M(\bar{C}^n)^\vee\simeq \bigoplus_{i=0}^n \Q(-n)
\]
and passing to the limit :
\[
\mathcal{O}(\pi_1(\Gm;x))\simeq T(\Q(-1))
\]
We skip the verification that the algebraic operations are the same.
\end{proof}

\section{Realizations}

We want to compare the motivic construction to the Hodge construction, and also to use the ramification criterion via $l$-adic cohomology on mixed Tate motives. For this we need a little more information on the construction of the Hodge realization and the $l$-adic realization.

The general principle is the following : since $\DMk$ is built out of $C^b(\SmCorr)$ by modding out chain homotopy, localization, pseudo-abelianization and inversion of $\Q(1)$, the essential step is to define the realization on $C^b(\SmCorr)$, and then checking that the other steps go through (that the would-be realization is invariant by chain homotopy, satisfies Mayer-Vietoris, $\A^1$-invariance, etc.). We will only give the construction on $C^b(\SmCorr)$. Moreover, all the finite correspondances that enter the construction of $\pi_1(X;a,b)_M$ are actually morphisms of schemes or sums of morphisms of schemes. So we will construct the realization only on such complexes. To see details about how to handle finite correspondances in this context, see \cite[Paragraph 1.5]{deligne-goncharov05}, \cite{huber00-realization}, \cite{ab}. Let $X^*\in C^b(\SmCorr)$.

\underline{l-adic realization :} Fix an algebraic closure $\bar{k}$ of $k$. For any $n\in\N$ and $S$ a $k$-variety, denote by $R\Gamma(S_{\bar{k}},\Z/l^n\Z)$ the global sections of an functorial injective resolution of the \'etale sheaf $(\Z/l^n\Z)_S$ (for instance obtained via the Godement resolution). Applying this functorially to $X^*$ defines then a double complex $(R\Gamma(X^*_{\bar{k}},\Z/l^n\Z))$ of $\Z/l^n$-modules with an action of $Gal_k$. We define the $l$-adic realization $\omega_l(M(X^*\bullet))\in D^b(\Q_l[Gal_K]-Mod)$ by :
\[
\omega_l(M(X^{\bullet})):=(\varprojlim_{n\in \N}(Tot((R\Gamma(X^*,\Z/l^n)(X^*)))\otimes \Q_l)^\vee
\]
The final duality is used because we want the homological (covariant) resolution.

\underline{Hodge realization :} For a smooth variety $S/k$ with a smooth compactification $\bar{S}$ by a divisor with normal crossings $T$, we denote by $R\Gamma(\Omega^*_{\bar{S}}(log T))$ the global sections of the total complex of a functorial injective resolution of the complex of coherent sheaves of logarithmic differential form on $(\bar{S},T)$. We equip it with its two natural filtrations : the Hodge filtration (= naive decresing filtration) and the weight filtration (by order of poles). We can then add in the Betti realization and the comparison isomorphism to form the Hodge complex $R\Gamma_{MHC}(X)$, as in Talk 6.

 Using Hironaka's theorem on resolution of singularities and its corollary for resolving ambiguities of rational maps, one can find smooth compactifications $\bar{X}^n$ of each $X^n$ such that the boundary is a divisor with normal crossings $D^n$ and the differentials of $X^*$ extend to $\bar{X}^*$ (giving a complex by continuity). In our basic situation, we have a natural choice of such compactifications : $\bar{X}^n=(\Pspace^1_k)^n$ and $D^n$ is built out of $D$.

We then form the complex of Mixed Hodge complex $R\Gamma(\Omega^*_{\bar{X}^*}(log D^*))$, form the total complex (The weight filtration has to be translated in an appropriate way) and take its dual. This is the Hodge realization.

We now get to the applications :

\begin{prop}
The Hodge realization of $\pi_1(X;a,b)_M$ is $\pi_1(X;a,b)_H$.
\end{prop}

This results from the parallel constructions of the Hodge and motivic bar complexes. By anology, we will also use the notation :
\[
\pi_1(X;a,b)_l=\omega_l(\pi_1(X;a,b)_M)\in \mathrm{Ind}\Q_l[Gal_k]-Mod
\]

\begin{prop}
\label{pi_1_a_unr}
Let $k$ be a number field. Let $\mathfrak{p}$ be a prime ideal of $\mathcal{O}_k$ such that $D_{k_{\mathfrak{p}}}\subset \Pspace^1_{\mathfrak{p}}$ extends to a divisor $\mathcal{D}$ in $\Pspace^1_{\mathcal{O}_\mathfrak{p}}$ smooth over $Spec(\mathcal{O}_{\mathfrak{p}})$, giving us a smooth model $\mathcal{X}=\Pspace^1_{\mathcal{O}_\mathfrak{p}}\setminus \mathcal{D}$ of $X_{k_\mathfrak{p}}$ over $Spec(\mathcal{O}_{\mathfrak{p}})$, and such that the points $a$ and $b$ extend to $a,b\in \mathcal{X}(\Z_p)$. Let $l$ prime such that $\mathfrak{p}\nmid l$. 

Then $\omega_l(\pi_1(X;a,b)_M)\in \mathrm{Ind}-Gal_k-Mod$ is unramified at $\mathfrak{p}$.
\end{prop}
\begin{proof}
The existence of the model $\mathcal{X}$ implies that the complex $R\Gamma(S,\Z/l^n\Z)$ of $\Z/l^n\Z[Gal_k]$-modules can be represented by a complex $R\Gamma(S,\Z/l^nZ)'$ of Galois representations unramified at $\mathfrak{p}$. The differentials in $CPX^{a,b}_*$ are defined in terms of $a$ and $b$, and their extension to $\mathcal{X}$ allows to choose $R\Gamma(S,\Z/l^n\Z)'$ such that the differentials extend. Finally, we see that we can compute the $l$-adic realization of the (truncated) motivic bar construction using complexes of representations unramified at $\mathfrak{p}$, so that the $l$-adic realization of $B_M(X;a,b)$ is unramified at $\mathfrak{p}$. Since the functor $\omega_l$ on $\DMT$ interchanges the motivic $t$-structure and the standard $t$-structure on $D^b(\Q_l-Gal_k-Mod)$, this implies that $\omega_l(\pi_1(X;a,b)_M)$ is unramified at $\mathfrak{p}$. 
\end{proof}

\section{Tangential base points}

The previous constructions do not cover the case of tangential base points. We will sketch the approach used by \cite{deligne-goncharov05} : it is indirect and works by reduction to the case of interior base points.

We put ourselves, from here on, in the basic situation of the start of the talk. Let $x,y\in D(k)$ be points of the boundary, $u\in T_x\Pspace^1_k\setminus \{0\}$ and $v\in T_y\Pspace^1_k\setminus\{0\}$. Talk 7. has defined $\mathrm{MH}(k)$-schemes $\pi_1(X;a,v)_H$, $\pi_1(X;u,v)_H$, etc. together with composition of paths operations giving them structures of groups or torsors. The space $\pi_1(X;u,v)_H$ of paths between two tangential base points is built out of the case of $\pi_1(X;a,v)_H$, i.e. of paths between an interior point and a tangential point. We will also follow this procedure.

In this section, we make the hypothesis that $k$ is a number field. Recall from Talk 9. that by Borel's theorem, this implies not only $(B-S)_k$, but also the stronger result that $\omega_H:\MTM(k)\rightarrow \mathrm{MH}(k)$ is fully faithful and its image is stable by subquotients. We say that an object in the image is \emph{motivic}, and it will admit an unique $\MTM(k)$-structure up to isomorphism. Moreover, any $\mathrm{MH}(k)$-algebraic structure on a motivic object will lift to an $\MTM(k)$-algebraic structure.

From this also follows a recognition principle for $\MTM(k)$-affine schemes :

\begin{prop}
Let $Y$ be an $\MTM$-affine scheme, and $Z_H\subset \omega_H(Y)$ be a closed $\mathrm{MH}(k)$-subscheme. Then there exists a unique $\MTM$-affine subscheme $Z$ such that $Z_H=\omega_H(Z)$,i.e. $Z_H$ is motivic.
\end{prop}
\begin{proof}
The closed subscheme of the affine scheme $\omega_H(Y)$ is defined by an $\mathrm{MH}(k)$-ideal $I$ in $\cO(\omega_H(Y))=\omega_H(\cO(Y))$. Now $\cO(Y)$ is an inductive limit of sub-$\MHM$-algebras $A_i$, and we look at the intersection $I_i$ of $I$ with $A_i$. Because the image of $\omega_H$ is stable by subobjects, $I_i$ is in the image of $\omega_H$, and because $\omega_H$ is fully faithful, there is a unique lift to $I^M_i\in \MHM$. By uniqueness, these fit together in the direct limit to make a lift $I^M$ which is an $\MHM$-ideal of $\cO(Y)$. We then put $Z=Spec(\cO(Y)/I^M)$. 
\end{proof}

\subsection{The case of $\pi_1(\Gm,x,v)$}
In this section, $X=\Gm$ with coordinate $t$, $a=x\in \Gm(k)$, $v\in T_0\Pspace^1_k\setminus\{0\}$ such that $dt(v)=z$. We have the following :
\begin{prop}
\label{pi_1_Gm}
There exists isomorphisms of $\mathrm{MH}(k)_H$-affine schemes
\[
\pi_1(\Gm;x,v)_H\simeq \pi_1(\Gm;x,z)_H\simeq \pi_1(\Gm;1,z/x)_H
\]
which are isomorphisms of left $\pi_1(\Gm)_H$-torsors.
\end{prop}
\begin{proof}
The second isomorphism simply comes from the unique automorphism $t\mapsto t/x$ of $\Gm(k)$ sending $x$ to $1$.

We define the first isomorphism $\theta:\pi_1(\Gm;x,z)_H\simeq \pi_1(\Gm;x,v)_H$ as follows. Let $\gamma_{z,v}\in\pi_1(\Gm(\C);z,v)$ be the straight path from $z$ to $0$. We put $\theta_B:\gamma\in \pi_1(\Gm(\C);x,z)\mapsto \gamma\gamma_{z,v}\in\pi_1(\Gm(\C);x,v)$ and $\theta_{dR}=id_{T(\Omega^1)}:\pi_1(\Gm;x,v)_{dR}\simeq T(\Omega^1)\rightarrow \pi_1(\Gm;x,z)_{dR}\simeq T(\Omega^1)$. It remains to show that these two maps define an morphism of Hodge structure (automatically an isomorphism since $\theta_{dR}$ is), i.e. to check the compatibility with the comparison map given by iterated integrals. Let $\gamma\in \pi_1(\Gm(\C);x,z)$ and $\omega_1,\ldots,\omega_n\in \Omega^1$. Since on $\Gm$, all closed 1-forms are proportional, we can assume $\omega_1=\ldots=\omega_n=\omega=\frac{dt}{t}$. We compute, using the definition of the tangential iterated integrals :
\[
\int_{\gamma\gamma_{z,v}}\omega^{\otimes n} = \lim_{\epsilon\rightarrow 0}\sum_{i=0}^n\frac{(-1)^{n-i}}{(n-i)!}\int_{(\gamma\gamma_{z,v})_\epsilon}\omega^{\otimes i} \log(\epsilon)^{n-i}
\]
Let us show that this last expression is equal to $\int_{\gamma}\omega^{\otimes n}$, even before taking the limit. We use the formula $\int_\gamma\omega^\otimes k=\frac{1}{k!}(\int_\gamma\omega)^k$.
\[
\sum_{i=0}^n\frac{(-1)^{n-i}}{(n-i)!}\int_{(\gamma(\gamma_{z,v})_\epsilon}\omega^{\otimes i} log(\epsilon)^{n-i} = \sum_{i=0}^n\frac{(-1)^{n-i}}{i!(n-i)!}\left(\int_{\gamma}\omega +\int_{(\gamma_{z,v})_\epsilon}\omega\right)^i \log(\epsilon)^{n-i}
\]
We have $\int_{\gamma_{z,v},\epsilon}\frac{dt}{t}=log(dt(v)\epsilon)-log(z)=log(z\epsilon)-log(z)=log(\epsilon)$ (Because $\gamma_{z,v}$ has derivative $v$ at $1$ ; this is where we use that $dt(z)=v$). So 
\begin{eqnarray*}
\sum_{i=0}^n\frac{(-1)^{n-i}}{(n-i)!}(\int_{\gamma}\omega +\int_{(\gamma_{z,v})_\epsilon}\omega)^i \log(\epsilon)^{n-i} & = & \frac{1}{n!}\sum_{i=0}^n(-1)^{n-i}\binom{n}{i}\left(\int_{\gamma}\omega+\log(\epsilon)\right)^i\log(\epsilon)^{n-i}\\
& = & \frac{1}{n!}(\int_{\gamma}\omega)^n\\
& = & \int_{\gamma}\omega^{\otimes n}
\end{eqnarray*}
This concludes the proof.
\end{proof}

These isomorphisms are consistent with the idea that $\pi_1(\Gm)$ is commutative and that the choice of a base point, even tangential, do not matter. 

\begin{cor}
The $\mathrm{MH}(k)$-group scheme $\pi_1(\Gm;x,v)$ is motivic with positive weights, and the isomorphisms of Proposition \ref{pi_1_Gm} hold at the motivic level.
\end{cor}


\subsection{The case of $\pi_1(X;a,v)$}

Our aim is the following :

\begin{prop}
\label{pi_1_av}
Let $X'=\Pspace^1\setminus \{y,\infty\}$ ($X'\simeq \Gm$ as $k$-variety).

There exists a closed embedding of $\mathrm{MH}(k)$-affine schemes :
\[
\pi_1(X;a,v)_H\hookrightarrow \pi_1(X';a,v)_H\times Lie(\pi_1(X;a)_H)(-1)
\]
where we identify $Lie(\pi_1(X;a)_H)\in \mathrm{MH}(k)$ with the corresponding $\mathrm{MH}(k)$-vector scheme.
\end{prop}
\begin{proof}
We will define the map and refer to \cite[4.4-4.10]{deligne-goncharov05} for the proof that it is a closed embedding.

First, the map $\pi_1(X;a,v)_H\rightarrow \pi_1(X';a,v)_H$ is induced by the inclusion $X'\subset X$.

Then, the map $\pi_1(X;a,v)_H\rightarrow Lie(\pi_1(X;a)_H)(-1)$ is defined in the following way. We have a morphism of $MH(k)$-affine schemes :
\[
\pi_1(X;a,v)_H\times \Q(1)_H\rightarrow \pi_1(X;a)_H
\]
which one can see as the following composition of morphisms already defined in Talk 7 (reversal of paths, local monodromy, composition of paths) :
\[
\pi_1(X;a,v)_H\times \Q(1)_H\rightarrow \pi_1(X;a,v)_H\times \Q(1)_H\times\pi_1(X;v,a)_H
\]
\[
\rightarrow \pi_1(X;a,v)_H\times \pi_1(X;v)_H\times \pi_1(X;v,a)_H\rightarrow \pi_1(X;a)_H
\]
Using internal Homs in $MH(k)$, we can write this as a morphism :
\[
\pi_1(X;a,v)_H\rightarrow \underline{Hom}_{MH(k)-\mbox{grp sch.}}(\Q(1)_H,\pi_1(X;a)_H)
\]
But since those are unipotent pro-algebraic groups in characteristic 0, the homomorphisms between them are determined by the induced morphism on Lie algebras, and we have (with the abuse of notation $Lie(\Q(1)_H)\simeq \Q(1)_H$ since $\Q(1)_H$ is a vector group scheme) :
\[
\pi_1(X;a,v)_H\Rightarrow \underline{Hom}_{MH(k)-\mbox{Lie alg.}}(\Q(1)_H,Lie(\pi_1(X;a)_H)
\]
Finally, since $\Q(1)_H$ is an invertible object, this corresponds to
\[
\pi_1(X;a,v)\longrightarrow Lie(\pi_1(X;a)_H)(-1)
\]
as required.
\end{proof}

\begin{lemma}
\label{pi_1_X'}
There is an isomophism of $\MTM(k)$-affine schemes :
\[
\pi_1(X',a,v)_M\simeq \pi_1(\Gm,1,\frac{dt(v)}{a-y})_M
\]
\end{lemma}
\begin{proof}
We use the translation isomorphism $x\mapsto x-y$ to identify $X'$ and $\Gm(k)$ and then apply the result of the previous section.
\end{proof}

\begin{cor}
The $\mathrm{MH}(k)$-group scheme $\pi_1(X;a,v)_H$ is motivic, with positive weights.
\end{cor}
\begin{proof}
  The fact that it is motivic follows from the proposition and the recognition principle. We have the positivity of the weights for $\pi_1(X',a,v)_H\simeq \pi_1(\Gm,1,\frac{dt(v)}{a-y})_H$ (from previous lemma) and $\pi_1(X,a)_H$, and it is preserved by products, passing to the Lie algebra, and twisting by $\Q(-1)_H$.
\end{proof}

Now we can also lift the operations. Namely, $\pi_1(X;a,v)_M$ is a left $\pi_1(X;a)_M$-torsor. If $b\in X(k)$ is another point, there is an isomorphism of torsors :
\[
\pi_1(X;a,v)_M\simeq \pi_1(X;a,b)_M\times_{\pi_1(X;b)_M}\pi_1(X;b,v)_M
\]

All the results of the section hold for the case of $\pi_1(X;u,a)$ as well.

\subsection{The case of $\pi_1(X;u,v)$}

\begin{defn}
Let $a\in X(k)$. We can now define :
\[
\pi_1(X;u,v)_M=\pi_1(X;u,a)_M\times_{\pi_1(X,a)_M}\pi_1(X;a,v)_M
\]
\end{defn}
This definition is actually independent of the choice of $a$ by the last isomorphism of the previous section. Of course, since we defined $\pi_1(X;u,v)_H$ in the same fashion, we have $\omega_H(\pi_1(X;u,v)_M)\simeq \pi_1(X;u,v)_H$.

\section{The theorem of Goncharov-Terasoma}

To sum up, we now know that the multizeta values are periods of the $\MTM(k)$-scheme $\pi_1(\Pspace^1\setminus\{0,1,\infty),\overrightarrow{01},\overrightarrow{10})_M$, with positive weights. 

\begin{prop}
The $\MTM(k)$-scheme $pi_1(\Pspace^1\setminus\{0,1,\infty),\overrightarrow{01},\overrightarrow{10})_M$ is an $\MTMZ$-scheme.
\end{prop}

\begin{proof}
Write $X=\Pspace^1_k\setminus\{0,1,\infty\}$, $\mathcal{X}=\Pspace^1_\Z\setminus\{0,1,\infty\}$, $u=\overrightarrow{01}$, $v=\overrightarrow{10}$. Let $p$ be a prime number. From Talk 9, we now that it suffices to show that for some $l\neq p$ prime, the $l$-adic realization $\pi_1(X, u,v)_l$ is unramified at $p$. 

We choose an auxiliary point $a\in X(\Q)\subset \Q$, arbitrary for the time being. We will gather sufficient conditions on $a$ and show at the end that they can be met. Then by definition, it suffices to check that $\pi_1(X;u,a)_l$, $\pi_1(X;a)_l$ and $\pi_1(X;a,v)_l$ are unramified at $p$. 

To ensure that $\pi_1(X;a)_l$ is unramified at $p$, by Proposition \ref{pi_1_a_unr} it suffices to choose $a\in \Q\subset \Q_p$ such that it extends to a $\Z_p$-point of $\mathcal{X}$, i.e., that $a\in \Q\cap \Z_p^\times=\Z_{(p)}$ and $a-1\in \Z_{(p)}^\times$.

Let $X'=\Gm(k)\subset X$ and $X''=\Pspace^1_k\setminus \{1,\infty\}$. By \ref{pi_1_av}, it suffices to show that $\pi_1(X';u,a)_l$, $\pi_1(X'';a,v)$ and $Lie(\pi_1(X;a)_l)(-1)$ are unramified at $p$. Passing to the Lie algebra is a quotient at the level of functions and $\Q(-1)\in \MTMZ$, so the last one is unramified if $\pi_1(X;a)_l$ is. By \ref{pi_1_X'}, we have isomorphisms :
\[
\pi_1(X';u,a)_l\simeq \pi_1(\Gm(k);1,a)_l
\]
\[
\pi_1(X'';a,v)_l\simeq \pi_1(\Gm(k);1,\frac{1}{a-1})_l
\]
By Talk 9, we know that those Kummer torsors are unramified at $p$ if and only if $a,\frac{1}{a-1}\in \Z_{(p)}^{\times}$.

To sum up, if we can find an $a\in \Q\setminus \{0,1\}$ such that $a\in \Z_{(p)}^\times,a-1\in\Z_{(p)}^\times$, then we are done. This always exist... except when $p=2$, because $\mathbb{F}_2$ has 2 elements. In this case, we extend the situation to the quadratic extension $K=\Q(\sqrt{5})$ where the prime $2$ is unramified and inert. Then $\omega_l(\pi_1(X;a,b))\simeq \omega_l(\pi_l(X_K,a,b))$ as $\Q_l[Gal_{\Q}]$-modules. Since $2$ is unramified in $K$, the inclusion $Gal(\bar{K}_2/K_2)\rightarrow Gal(\bar{\Q}_2/\Q_2)$ (well defined up to conjugation) is an isomorphism on inertia, so we can test for ramification at $2$ after extension. Now we do have a $K$-point $a$ such that $a,1-a\in\mathcal{O}_{(2)}^{\times}$ (take any antecedent of a residue class in $\mathbb{F}_4$ different from $0$ or $1$), and the argument goes through.
\end{proof}

Applying the final result from Talk 9, we get :

\begin{thm}[Goncharov-Terasoma]
The dimension $d_n$ of the $\Q$-vector space of multizeta values is bounded by $D_n$ with $D_n = D_{n-2} + D_{n-3}, \qquad D_0 = D_2 = 1, D_1 = 0$.
\end{thm}





\chapter{Brown's proof: Strategy of the proof by Sergey Gorchinsky}
%\addcontentsline{toc}{chapter}{Brown's proof: Strategy of the proof by Sergey Gorchinsky}

Sergey Gorchinsky on September 7th, 2012.

\medskip
\medskip

\section{Statements}

Write $\zeta(2//3)$ to mean $\{\zeta(a_1, \ldots, a_n) \mid n \in \N, a_i = \textrm{2 or 3} \}$.
\begin{conj}[Hoffman]
The set $\zeta(2//3)$ forms a $\Q$-basis for $\cZ = \langle \zeta(\overline{n}) \rangle_{\Q} \subset \R$. where $\overline{n} = (n_1, \ldots, n_r), n_r \geq 2$.
\end{conj}
\begin{conj}[Brown's theorem]
Let $\zeta(2//3)$ $\Q$-linearly generate $\cZ$.
\end{conj}
\begin{thm}\label{thm:mtmgenerates}
The mixed Tate motive $cO(\pi_1(X;0,1)_M)$ generates $\MTMZ$ under the tensor product.
\end{thm}
\begin{thm}\label{thm:strictquot}
The morphism $\cO(I(dR,B))_{+} \surj \cZ$ is a strict quotient of a filtered algebra (not just a subquotient).
\end{thm}

\begin{ques}
Does Theorem \ref{thm:strictquot} imply stronger upper bounds on $d_n$? For example
\[
d_n \leq d_{n-2} + d_{n-3}
\]
\end{ques}

\section{Motivic lift of Hodge class}
\subsection{Set-up}
\begin{itemize}
\item $X = \Pspace^1 \setminus \{0,1,\infty\}$
\item $\pi_1(X;0,1)$-motivic scheme of paths 
\item $\pi_1(X;0,1)_{dR} = T(\Omega)_1$.
\item $\Omega = \langle \frac{dz}{z} = \omega^0, \frac{dz}{1-z} = \omega^1 \rangle_{\Q}$
\item $\mathrm{dch} \in \pi_1(X;0,1)_B(\Q)$\footnote{Recall from Definition \ref{def:dch} that $\mathrm{dch}$ denotes the unit path $[0,1]$.}
\item \[
\xymatrix{
1 \ar[r] & U_M \ar[r] & G_M \ar@<-0.2em>[r] & \Gm \ar@<-0.2em>[l] \ar[r] & 1
}
\]
\item $I := I(\omega_{dR}, \omega_B)$ and $\cO(01)_{dR} := \cO(\pi_1(X;0,1)_{dR}) \cong T(\Omega)$ are in $\mathrm{Ind}(\mathrm{Rep}(G_M))$
\item $\mathrm{comp} \in I(\C)$.
\end{itemize}

\begin{defn}
The path $\mathrm{dch}$ leads to the composition:
\[
\xymatrix{
\cO(01)_{dR} \ar[r] \ar@/_/@{->}[rr]_g & \cO(01)_{dR} \otimes \cO(01)^{\vee}_B \ar[r] & \cO(I) \ar[d]^{\mathrm{comp}^*} \\
& & \C
}
\]
\end{defn}
\begin{defn}
Let
\[
\cZ_M := \im g
\]
\end{defn}
$\zeta_M(\overline{n}) \to \zeta(\overline{n}) := \im(\tau_0(\tau_1))$

\begin{rem}
All maps above are mixed motives of $G_M$ representations. In particular, $\cZ_M$ is a $G_M$-representation. Furthermore, all of the above vector spaces are canonically graded with respect to $\Gm \inj G_M$.
\begin{itemize}
\item $\mathrm{comp}^*(\cZ_M) = \cZ$
\item $\cZ_M \subset \cO(I)^{\epsilon}_+$\footnote{The + denotes taking the positive degree functions.}
\item $\epsilon \in G_{\omega_B}(\Q)$ is given by complex conjugation and correspondence to non-negative weights using the fact that $\mathrm{dch} \subset X(\R)$.
\end{itemize}
\end{rem}

\begin{rem}
Let $LI$ denote the proposition that $\zeta_M(2//3)$ is linearly independent in $\cO(I)$.
Then HC implies LI. Via upper bounds on $\cO(I)_+$, this implies in turn Brown's theorem, as well as Theorems \ref{thm:mtmgenerates} and \ref{thm:strictquot}. Use that $\Q(-1)_M$ is a subquotient of $\cO(01)$ by weight applied to $\cO(0,1)_{dR}$.

Since $\cZ_M$ is graded, it is enough to prove $LI_n$ for each weight $n \geq 0$.
\end{rem}

\subsection{The guiding principle}
\[
\xymatrix{
\cZ_M \ar@{->>}[r]^{\mathrm{comp}^*} & \cZ \\
\cO(I) \ar[r]^{\mathrm{comp}^*} & \C
}
\]
The left side is algebro-geometric and has functions. The right side is analytic and has numbers.

The Kontsevich-Zagier conjecture implies that $\textrm{comp}^*$ is injective.

Examples of relations of multiple zeta values.
\begin{center}
\begin{tabular}{cl}
Weight & Relations \\
0 & $1$ \\
1 & $0$ \\
2 & $\zeta_M(2)$ \\
3 & $\zeta_M(3)$ \\
4 & $\zeta_M(2,2)$ \\
5 & $\zeta_M(3,2), \zeta_M(2,3)$
\end{tabular}
\end{center}

Kontsevich-Zagier implies that
\[\label{eq:KZ1}
\left( \begin{array}{c}
\zeta_M(2,3) \\
\zeta_M(3,2)
\end{array} \right)
=
\left( \begin{array}{cc}
\frac{3}{2} & -2 \\
-\frac{11}{2} & 3
\end{array} \right)
\left( \begin{array}{c}
\zeta_M(5) \\
\zeta_M(2,3,2)
\end{array} \right)
\]

The Lie algebra $\mathfrak{u}_M := \mathrm{Lie~} U_M$ acts by derivations on regular functions in order to reduce to $\zeta_M(2), 1$. It is pro-nilpotent.
\[\label{eq:KZ2}
\begin{array}{rcl}
\mathfrak{u}_M \to \mathfrak{u}_M^{ab} = \mathfrak{u}_M |_{[\mathfrak{u}_M, \mathfrak{u}_M]} & \cong & \coprod_{r \geq 1} \Ext_{\MTMZ}(\Q(0), \Q(2r+1))^{\vee} \\
\partial_{2r+1} & \mapsto & \Q[\partial_{2r+1}]
\end{array}
\]
\[
\mathrm{comp}^*(\zeta_M(2, \ldots, 2)) = \zeta(\underbrace{2, \ldots, 2}_m) \sim \pi^{2m}
\]
We need to derviate $\zeta_M(2, \ldots, 2)$. Apply Kontsevich-Zagier formula again to infer that $\zeta_M(\underbrace{2,\ldots,2}_n)$ is a function of the period of $\Q(2n)$.

As $\Q(1)_{dR}$ is a $G_M$ representation factors through $G_M \to \Gm$. Hence $\partial_{2r+1}(\zeta(2,\ldots,2)) = 0$.

\subsection{Aside on algebraic groups}

\begin{defn}[Linear function]
Let $G$ be an algebraic group acting algebraically on a scheme $X$ over $k$. Then a function $f \in \cO(X)$ is \emph{linear} with respect to $G$ if for all $\partial \in \mathrm{Lie~}(G)$, $\partial(f)$ is constant on $X$.
\end{defn}

For example, if $G = \Ga$ acts on $X = \Ga$, then linear functions $g$ are those such that $dg \leq 1$.

\begin{rem}
Any linear $f \in \cO(X)$ defines a morphism $X_f : g \to k$ out of $\chi_f$. We get
\[
0 \to 1 \to V_f \to 1 \to 0 \qquad V_f = \left( \begin{array}{cc}
0 & \chi_f \\
0 & 0
\end{array} \right)
\]
where $V_f$ are the representations of $g$.
\end{rem}

\begin{lemma}
Let $\xymatrix{ \cC \ar@<0.2em>[r]^{\omega} \ar@<-0.2em>[r]^{\eta} & \mathrm{Vect}(k) }$.
\[
0 \to L_1 \to S \to L_2 \to 0
\]
$\rk L_i = 1$, $S \in \cC$, $f \in \cO(I(\omega, \eta))$ any period of $S$ of kind $\left( \begin{array}{cc}
* & f \\
0 & *
\end{array} \right)$. Then $f$ is linear with respect to $U = r_U(G_M)$, where $r_U$ denotes the unipotent radical (Cf. Definition \ref{def:uniradical}). In addition, $V_f \cong \omega(S)$ as $U$-representations.
\end{lemma}

Borel's result on $\zeta$-values as regulators and the Borel-Beilinson comparison theorem together imply that the periods of $M_{2r+1} = \left( \begin{array}{cc}
(2\pi i)^{2r+1} & \zeta(2r+1) \\
0 & 1
\end{array} \right)$

Kontsevich-Zagier and the lemma together imply that $\zeta_M(2r+1)$ is linear with respect to $U_M$ and
\[\label{eq:KZ3}
\partial_{2r+1}(\zeta_M(2s+1)) \sim \delta_{r,s}
\]
Why is this proportional? Take the linear function $\zeta$ it gives you an extension which gives you a character of the Lie algebra, and mapping over to the Ext group, we see that it vanishes.

So we've deduced three formulas from Kontsevich-Zagier: Equations \ref{eq:KZ1}, \ref{eq:KZ2} and \ref{eq:KZ3}.



\subsection{Application}
\[
(\partial_{\epsilon} + \partial_s)\left( \begin{array}{cc} \zeta_M(2,3) \\ \zeta_M(3,2) \end{array} \right) = A \left( \begin{array}{c} 1 \\ \zeta_M(2) \end{array} \right)
\]
where $A := \left( \begin{array}{cc} 3/2 & -2 \\ -11/2 & 3 \end{array} \right)$ is the matrix of \ref{eq:KZ1}. The matrix $A$ is invertible. Indeed, we may multiple the first column by 2. Then it is an integer matrix. Examine it modulo 2 and it becomes lower triangle with ones on the diagonal, and hence it is invertible. This will be a general tactic for see that a matrix is invertible.

``So'', $\zeta_M(2,3)$ and $\zeta_M(3,2)$ are linearly independent.

After applying these derivations we are still in the subspace generated by $\zeta_M(3)$. By the way if you apply the derivations to $(1 \zeta_M(2))^T$, you get zero.

\section{Strategy of the proof}

\begin{defn}
We define
\[
3_{\ell} \cZ_M := \langle \zeta_M(2 //3) \mid \textrm{at most $\ell$ threes appear} \rangle
\]
\end{defn}

\begin{rem} 
The filtration $3_{\ell}$ commutes with $W_n$.
\end{rem}

\begin{itemize}
\item $3_{\ell} \cZ_M$ are stable under $\mathfrak{u}_M \ni \partial = \sum_{k=1}^{\infty} \partial_{2k+1}$.
\item $\partial : \gr^{3_{\ell}}_e \cZ_M \to \gr^{3_{\ell}}_{e-1} \cZ_M$. But we need to prove the dependency in each (weight) degree. So consider the $n$-part. They all have negative weights, so in degree $n$ we have
\[
(\gr^{3_{\ell}}_e \cZ_M)_n \stackrel{\partial}{\to} \bigoplus_{r \geq 1} \left(\gr_{e-1}^{3_{\ell}} \cZ_M\right)_{n-(2r+1)}
\]
\end{itemize}

Explicit bases: 
\[
\begin{array}{rcl}
\{ \zeta_M(2 //3) \mid \textrm{with $m$ 2's and $\ell$ 3's where $2m+3\ell = n$} \} & \to & \{ \zeta_M(2//3) \mid \textrm{the same for various weights} \} \\
\zeta_M(2//3, \ldots, 2//3)(3,2, \ldots,2)) & \mapsto & \zeta_M((2//3, \ldots, 2//3))
\end{array}
\]

By the main theorem, there exists an \emph{invertible} matrix $A$ such that
\[
\partial(\textrm{LHS basis}) = A \cdot \partial(\textrm{RHS basis})
\]

\begin{prop}
M.th. implies LI.
\end{prop}
\begin{proof}
Induction on $\ell$. The induction step is M.th. The base step is $\ell = $. In the base case, $\zeta_M(2, \ldots, 2) \neq 0$. $\R \ni \zeta(2,\ldots, 2) \geq 0$.
\end{proof}



\chapter{Brown's proof: Part 2 by Sergey Gorchinsky}
%\addcontentsline{toc}{chapter}{Brown's proof: Part 2 by Sergey Gorchinsky}

Sergey Gorchinsky on September 7th, 2012.

\medskip
\medskip

\subsection{Plan of the proof}
\begin{enumerate}
\item Prove the property of $\partial$ and $3_{\ell} \cZ_M$. ``Then'' $\partial_{2r+1}(\zeta_M(2, \ldots, 2)) = 0$.
\item $\partial_{2r+1}(\zeta_M(2s+1)) \sim \delta_{rs}$.
\item Case $\ell = 1$: Motivic lift of Zeta function.
\item Case $\ell \geq 2$: Recurrency process.
\end{enumerate}
Step 1 is algebro-geometric. Steps 2, 3 and 4 require step 1.5, the use of Goncharov's formula which gives restrictions on these derivations $\partial_{2r+1}(\zeta_M(2s+1))$. This combinatorial results will be applied to show Steps 3 and 4.

\section{Step 1}
\subsection{Motivic origin of $3_{\ell} \cZ_M$}

We would like to show that $3_{\ell}\cZ_M \subset \cZ_M$ is a $G_M$-subrepresentation and moreover, that
\[
3_{\ell} \cO(0,1)_{dR} = 3_{\ell} T(\Omega) := \langle \textrm{ $\overline{\omega}$ that contain only $\omega^1 \omega^0 \omega^1 \cdots \omega^1_{\leq \ell} \omega^0 \cdots \omega^0$} \rangle_{\Q}
\]
\begin{prop}
\[
3_{ell}\cO(0,1)_{dR} \subset \cO(0,1)_{dR}
\]
\end{prop}
The action of $G_M$ on that path space is a difficult object. But we have a geometric interpretation of all this. By the Tannakian theory, we see that subrepresentations of $G_M$ correspond to mixed motives.
\[
\MTMZ \isom \Rep(G_M)
\]
Equivalently, we want that $3_{\ell} \cO(0,1)_{dR}$ is motivic, i.e., that there exists a subobject in $\cO(0,1)_{dR}$ whose $\omega_{dR}(-)$ is $3_{ell} \cO(0,1)_{dR}$. To do this, we prove two lemmas.

We will first show that the subspace of admissible $\overline{\omega}$ are admissible, and then we will show a lemma to extend the result to all $\overline{\omega}$ in $3_{\ell} \cO(0,1)_{dR}$.

\begin{lemma}\label{lem:sergeylem1}
The span
\[
\langle \overline{omega} \mid \overline{\omega} = \omega \otimes \cdots \omega^0 \textrm{~is admissible} \rangle_{\Q} \subset \cO(0,1)_{dR}
\]
is motivic.
\end{lemma}
\begin{proof}
Consider
\[
\cO(\Q(1) \setminus \pi_1(X;0,1)_M / \Q(1)) \subset \cO(\pi_1(X;0,1)_M)
\]
The left $\Q(1)$ corresponds to a loop around 0 exiting and reentering the point along the tangent vector pointing toward 1. The right $\Q(1)$ similarly corresponds to a loop around 1 exiting and reentering the point along the tangent vector pointing toward 0. The paths of the $\pi_1$ exit the point 0 along the vector pointing toward 1 and enter the point 1 along the vector pointing from 0. This geometric picture shows that
\[
\omega_{dR}(\cO(\Q(1) \setminus \pi_1(X;0,1)_M / \Q(1))) = a concave diamond
\]
\begin{eqnarray*}
\overline{\omega} & \mapsto & \sum \overline{\omega}_{i_1} \otimes \overline{\omega}_{i_2} \otimes \overline{\omega}_{i_3} \\
& \mapsto & \sum \mathrm{res}_0(\overline{\omega}_{i_1}) \otimes \overline{\omega}_{i_2} \otimes \mathrm{res}_1(\overline{\omega}_{i_3}) \\
& \stackrel{?}{=} \overline{\omega}
\end{eqnarray*}
\end{proof}
The difference is $\sum \mathrm{res}_0(\overline{\omega} \otimes \overline{\omega}_{i_2} \otimes \mathrm{res}_1(\overline{\omega}_{i_3})$. The first factor of the tensor product has positive coefficients. The third factor does as well. Hence $\omega_1 = \omega^1$ and $\omega_n = \omega^0$.
\begin{lemma}\label{lem:sergeylem2}
Fix p.m. Then
\[
\heartsuit = \left\langle \overline{\omega} \mid \textrm{$\overline{\omega}$ does not contain $p$ blocks of $\omega^0$'s of total length $m$} \right\rangle_{\Q} \subset \cO(0,1)_{dR} = T(\Omega)
\]
is motivic.
\end{lemma}
\begin{cor}
$3_{\ell}\cO(0,1)$ is motivic.
\end{cor}
The proof of the corollary is left as an exercise, but we consider the example of canceling those series of 0's of length greater than 4. Lemma \ref{lem:sergeylem1} implies
\[
\underbrace{10\cdots 0}_{n_1} \underbrace{10\cdots 0}_{n_2} 10 \cdots 01 \cdots \underbrace{10 \cdots 0}{n_n}
\]
Apply Lemma \ref{lem:sergeylem2} with $p=1$ and $m=3$. For a general $3_{\ell}$, $p=\ell$ and $m=2\ell$ apply Lemma \ref{lem:sergeylem2} with reg'd by 1 $p=1$, $m=2$.

\begin{proof}[Proof of Lemma \ref{lem:sergeylem2}]
Define $S := \pi_1(X;0)_M^{\times p} \times \pi_1(X;0,1)_M$. Consider
\[
f: \Q(1) \times S \to \pi_1(X;0,1)_M
\]
The Betti realization is $(n, \gamma_1, \ldots, \gamma_0, \gamma) \mapsto (\gamma_1 \gamma_2 \ldots \gamma_p \gamma)$.

Then $\heartsuit$ is $\omega_{dR}$.
\[
(f^*)^{-1}((\Q(0) \oplus \Q(-1) \oplus \cdots \oplus \cO(-n)) \otimes \cO(s)) \subset \cO(0,1)
\]
Blocks correspond to $\partial^n$'s. The degree of polynomials is exactly the length.
\end{proof}

\begin{prop}[Step 1]
$\mathfrak{u}_M$ acts trivially on the graded pieces $\gr_{\ell}^3 \cZ_M$.
\end{prop}
\begin{proof}
\[
\partial_{2r+1}(\zeta_M(\ldots, 3, 2, \ldots, 2, 3, \ldots) \mapsto \sum \zeta_M(2, \ldots, 2, 3, \ldots, 3, 2, \ldots, 2)
\]
...missing...
\end{proof}
This immediately implies Step 1.

\section{Step 2}
Consider the action of $\partial_{\geq r+1}$ on $\cO(0,1)_{dR} = T(\Omega)$. We identify tensors $(\omega^1 \omega^0 \cdots \omega^0)$ with words $(10 \cdots 0)$.
\begin{thm}[Goncharov's Formula]\label{thm:goncharov}
Let $\partial \in \mathfrak{u}_M$ be a derivation. Then the following relation between words holds in $\cO(01)_{dR}$.
\[
\partial(w) = \sum_{\begin{subarray}{c}
v \neq \varnothing \\
(0v1) \subset (0w1)
\end{subarray}}
c t(\partial(v)) \cdot (w \setminus v) + \sum_{\begin{subarray}{c}
v \neq \varnothing \\
(1v0) \subset (0w1)
\end{subarray}} (-1)^{|v|} c t(\partial(v^*)) \cdot (w \setminus w)
\]
where $v^{*}$ denotes the inverse word to $v$.
\end{thm}
There is nothing motivic behind this formula. There is some Lie algebra canonically acting on $T(\Omega)$, and a chain of reasoning proves it, but we omit it.

\subsection{$\partial_{2r+1}(\zeta_M(2s+1))$}

\begin{lemma}
The images of $\underbrace{0 \ldots 0}_m = (\omega^0)^{\otimes m}$ under $\cO(0,1)_{dR} \surj \cZ_M$ vanish. The same is true when 0 is replaced by 1, i.e., $(\omega^1)^{\otimes m} \mapsto 0$.
\end{lemma}
\begin{proof}
Consider $\gr_{0,1}^w \cO(0,1)_M$. Looking at the de Rham realization, we see that
\begin{eqnarray*}
\gr_0^w \cO(0,1)_M & = & \Q(0) \\
\gr_1^w \cO(0,1)_M & = & \Q(-1) \oplus \Q(-1)
\end{eqnarray*}

$\Ext^{1,0}_{\MTMZ}(\Q(0),\Q(1)) = 0$. Hence there is a canonical splitting, so that $\omega_1\cO(0,1) \cong \Q(0) \otimes (\Q(1) \otimes \Q(-1))$ by a canonical isomorphism. Also, $\omega_{dR} = \Q \otimes \Omega$.

So the images of $\omega^{0/1}$ in $\cZ_M \subset \cO(I)$ are $G_M$-eigenfunctions of $I$. Evaluating at $\mathrm{comp} \in I(\C)$ gives
\[
\langle \mathrm{comp}(\overline{\omega})(\gamma) = \int_{\gamma} \overline{\omega} \rangle
\]
\[
\int_{\mathrm{dch}} \omega^{0/1} = 0
\]
Since $G_M$ acts transitively on $I$, we see that 
\[
\begin{array}{rcl}
\cO(0,1)_{dR} & \to & \cO(I) \\
\omega^{0/1} & \mapsto & 0 \in \cO(I)
\end{array}
\]
Since $\int_{\gamma}(\omega^0)^{\otimes n} = \frac{1}{m!}\int_{\gamma} \omega^0$, $(\omega^0)^{\otimes m} \mapsto 0^m = 0$.
\end{proof}

\begin{prop}
\[
\partial_{2r+1}(\zeta_M(2s+1)) \sim \delta_{r,s}
\]
\end{prop}
\begin{proof}
\subsubsection{Proof Part 1}
We show that $\zeta_M(2s+1) \in \cO(I)$ is a linear function with respect to $U_M$. The proof proceeds by cases.
\begin{itemize}
\item[] In the case $r > s$, $\partial_{2r+1}(\zeta_M(2s+1)) = 0$. 
\item[] In the case $r=s$, $\partial_{2r+1}(1 \underbrace{0 \ldots 0}_{2r}) = ct(\partial_{2r+1}(10\cdots0)$. $(0v1) \subset (0|1\underbrace{0 \ldots 0}_{2r}|1)$. $|r| = 2r+1$. $(1v0) \subset (0|10\ldots0)|1) \to 0$.
\item[] In the case $r < s$, $\partial_{2r+1}(1\underbrace{0\ldots0}_{2s} = ct(\partial_{2r+1}(\underbrace{0\ldots0}_{2r+1}))(1\underbrace{0 \ldots 0}_{2s-2r-1}) - ct()$. Because the map $\cO(0,1)_{dR} \to \cO(I)$ is graded, the term $\partial_{2r+1}(0 \cdots 0)$ vanishes.
\begin{eqnarray*}
(0v1) \subset (0|1\underbrace{0\ldots0}_{2s}|1) & |v|=2r+1 \\
(1v0) \subset (0|1\underbrace{0\ldots0}_{2s}|1) & |v|=2r+1
\end{eqnarray*}
\end{itemize}

\subsubsection{Proof Part 2}
To show: $\partial_{2r+1}(\zeta_M(2r+1)) \neq 0$. Suppose the converse. Then $\zeta_M(2r+1) \in \cO(I)^{U_M}$ as computed by Konrad (Cf. missing).
\[\label{eq:explicitdesc}
\cO(I)^{\epsilon}_+ \cong \Q[t^2] \otimes_{\Q} T(\bigoplus_{r \geq 1} \Q e_{2r+1})
\]
is graded and $\deg t = 1$. So $\cO(I)^{\epsilon}_+)^{U_M} \cong \Q[t^2]$ has only even components. Hence $\zeta_M(2r+1)=0$. Contradiction! We already know that $\zeta_M(2r+1) > 0$.
\end{proof}

This completes Step 2.

\section{Step 3}
\begin{rem}
Now we fix a normalization $\partial_{2r+1}$ so that $\partial_{2r+1}(\zeta_M(2r+1)) = 1$.
\end{rem}

\begin{prop}
The embedding $\cZ \inj \cO(I)^{\epsilon}_+$ of $U_M$-representations induces equalities
\begin{eqnarray*}
N_0 \cZ_M & = & N_0 \cO(I)^{\epsilon}_+ \\
\{\textrm{$V_M$-linear}\}\cZ_M & = & \{ \textrm{$U_M$-linear} \} \cO(I)^{\epsilon}_+ \\
N_1 \cZ_M & = & N_1 \cO(I)^{\epsilon}_+
\end{eqnarray*}
\end{prop}
\begin{proof}
The proof is by calculation using the explicit description of $\cO(I)^{\epsilon}_+$ given in Equation \ref{eq:explicitdesc}. Then
\[
N_0 \cZ_M \subset N_0 \cO(I)_+^{\epsilon}.
\]
The left hand side equals $\langle 0 \neq \zeta_M(\underbrace{2, \ldots, 2}_m) \rangle_{\Q}$ and the right hand side, $\Q[t^2]$. But these are equal, hence $N_0 \cZ_M$ is the improper subset.

\begin{rem}
It follows that $\zeta_M(\underbrace{2, \ldots, 2}_m) \sim \zeta_M(2)^m$. Hence $\zeta(\underbrace{2, \ldots, 2}_m) \sim \zeta(2)^m \sim \mathrm{comp}^*(t)^{2m}$. Then $t$ is the period of $\Q(-1)$, so $\mathrm{comp}^*(t) \sim 2\pi i$. Hence $\zeta(\underbrace{2, \ldots, 2}_m) \sim \pi^{2m}$.
\end{rem}
Note that
\[
\begin{array}{ccc}
\{ \textrm{linear in $\cZ_M$} \} & \subset & \{ \textrm{linear in $\cO(I)^{\epsilon}_+$} \} \\
\parallel & & \parallel \\
\langle \zeta_M(2r+1) \neq 0, \zeta_M(\underbrace{2,\ldots,2}) \rangle_{\Q} & \subset & \langle t^{2m}, e_{2r+1} \rangle_{\Q}
\end{array}
\]
Since $\zeta_M(2r+1) \sim e_{2r+1}$, the bottom subset relation is actually an equality, and hence the top one is too.
\[
\begin{array}{ccc}
N_1 \cZ_M & \subset & N_1 \cO(I) \\
\cup & & \parallel \\
\langle \zeta_M(\underbrace{2, \ldots, 2}_m), \zeta_M(2r+1) \rangle & \subset & \langle t^{2m} \cdot e_{2r+1} \rangle_{\Q}
\end{array}
\]
\end{proof}

\begin{cor}
There exists a Zagier type formula:
\[
\zeta_M(2, \ldots, 2, 3, 2, \ldots, 2) = \sum_{r \geq 1} c \zeta_M(2, \ldots, 2) \cdot \zeta_M(2r+1)
\]
where $c \in \Q$ is a constant.
\end{cor}
\begin{proof}
The Lie algebra $\mathfrak{u}_M$ acts trivially on $\gr^3_{\ell} \cZ_M$. Hence
\[
3_1\cZ_M \subset N_1 \cZ_M.
\]
\end{proof}

\subsection{The Goncharov formula and combinatorial part}
\subsubsection{The Ihara group: An explanation of the Gondcharov formula}
\begin{rem}
The action of $G_M$ preserves the following structure.
\begin{itemize}
\item The groupoid structure of $\pi_1(X;0)_{dR}$, $\pi_1(X;0,1)_{dR}$, and $\pi_1(X;1)_{dR}$.
\item The morphisms $\Q(1)_{dR} \to \pi_1(X;0)_{dR}$ and $\Q(1)_{dR} \to \pi_1(X;1)_{dR}$.
\end{itemize}
The action of $U_M$ leaves the images of
\[\label{eq:invariantim}
\Q(1)_{dR} \to \pi_1(X;0)_{dR} \qquad \Q(1)_{dR} \to \pi_1(X;1)_{dR}
\]
\end{rem}

\begin{defn}[Ihara group]
Given a smooth scheme $X$ over $k$, its Ihara group is
\[
IH := \{ (\phi_1, \phi_2, \phi_3) \in \Aut(\pi_1(X;0)_{dR}, \pi_1(X;0,1)_{dR}, \pi_1(X;1)_{dR}) \mid \textrm{$\phi_1$, $\phi_2$ and $\phi_3$ preserve the groupoid structure} \}
\]
\end{defn}
The action of $U_M$ on $\pi(X;0,1)_{dR}$ factors through $U_M \to IH$. The Goncharov formula holds for $\partial \in \mathrm{Lie~}(IH)$.

\begin{prop}
\[
\begin{array}{rcl}
IH & \isom & \pi_1(X;0,1)_{dR} \\
g & \mapsto & g(dc_{dR})
\end{array}, \qquad
dc_{dR} \in \pi_1(X;0,1)_{dR}(\Q) \textrm{~corresponding to $ct$}
\]
\end{prop}
The explicit description of $IH$ begins with the free group on two generators, $\Gamma := \langle \gamma_1, \gamma_2 \rangle$. We associate to $\Gamma$ three its pro-unipotent completion $\Gamma^{un}$ and two sets, $\Gamma^{un}_l$ and $\Gamma^{un}_r$ which are left and right torsors under $\Gamma^{un}$.
\[
IH \cong \{ \phi \in \Aut(\Gamma^{un}_l, \Gamma^{un}, \Gamma^{un}_r) \mid \textrm{$\phi$ preserves $\gamma_1 \in \Gamma^{un}_l$, $\gamma_2 \in \Gamma^{un}_r$ and the product groupoid structure} \}
\]
\begin{prop}
\[
\begin{array}{rcl}
IH & \isom & \Gamma^{un} \\
g & \mapsto & g(1)
\end{array}
\]
\end{prop}
To prove the proposition, one does this for $(\Gamma_l, \Gamma, \Gamma_r)$.

Moreover, one obtains a new group structure $\ast$ on $\Gamma$ and $\Gamma^{un}$ defined as
\[
\gamma \ast \gamma' = \gamma(\gamma') \cdot \gamma
\]
where $\gamma' \mapsto \gamma(\gamma')$ is a group automorphism of $\Gamma$ such that
\[
\gamma_1 \mapsto \gamma_1, \quad \gamma_2 \mapsto \gamma \gamma_2 \gamma^{-1}
\]
From this one gets a new
\[
\Delta^{\ast} : \cO(\Gamma^{un}) \to \cO(\Gamma^{un}) \otimes_{\Q} \cO(\Gamma^{un})
\]
But $\cO(\Gamma^{un}) \cong T(\Omega)$. In ``our Goncharov formula''

\begin{exam}
We would like to find the coefficients
\[
\zeta_M(2,3) = c_1 \zeta_M(5) + c_2 \zeta_M(2) \cdots \zeta_M(3)
\]
Then $\partial_3 \zeta_M(2,3) = \underbrace{ct(\partial_3(100)) \cdot (10)}_{\zeta_M(2)} - \underline{ct(\partial_3(010)) \cdot (10)}_{-2\zeta_M(2)} = 3\zeta_M(2)$.
$(0|10100|)$, $|v| = 3$.
The trick is
\[
\begin{array}{rcl}
\cO(01) & \to & \cZ_M \\
(0) = \omega^0 & \mapsto & 0
\end{array}
\qquad
\begin{array}{rcccc}
(0) \cdot (10) & = &(010) & + & 2(100) \\
\downarrow & & \downarrow & & \downarrow \\
0 & = & (010) & + & 2\zeta_M(3)
\end{array}
\]

\[
\partial_5 \zeta_M(2,3) = ct(\partial_5(10100)) = ct(\partial_5 \zeta_M(2,3))
\]
$(0|10100|1) \supset (0v1)$ for $|v|=5$ implies $v=w$.
Hence
\[
\zeta_M(2,3) - 3\zeta_M(2) \cdot \zeta_M(3) = c \cdot \zeta_M(5)
\]
for some $c \in \Q$. Apply the morphism $\mathrm{comp}^*$ and Zagier's formula to see that $c = -11/2$.
\end{exam}

\begin{prop}
Zagier's formula holds for $\zeta_M(2//3)$.
\end{prop}
\begin{proof}
Apply $\partial_{2r+1}$ to $\zeta_M(2 \ldots 232 \ldots 2)$ such that $2r+1 < n$, where $n$ is the weight of $\zeta_M(2 \ldots 232 \ldots 2)$. One obtains words $v$ that are either $(10)^{?}(100)(10)^{?}$, in which case $n = 2r+1$ or $0(10)^{?}$, in which case the weight $n > 2r+1$. Examining the weights gives $ct(\partial_{2r+1}(v))$. Then
\[
\zeta_M(2 \ldots 232 \ldots 2) = c \cdot \zeta(n) + \epsilon
\]
Apply $\mathrm{comp}$ to get $c = c_{\cZ}$.
\end{proof}

This completes Step 3.

\section{Step 4}

Consider the case $\ell \geq 2$. We are interested in describing $\partial = \sum_{r \geq 1} \partial_{2r+1}$ from $\gr^3_{\ell} \cZ_{\ell} \to \gr^3_{\ell-1} \cZ_M$. That is, we want to describe $\partial_{2r+1} \zeta_M(\underbrace{2\ldots2332\ldots3\ldots3}_{\ell 3's})$ modulo $\zeta_M(2//3)$ with at most $\ell-2$ 3's. Apply Goncharov's formula. We get many $v$'s and those which matter are in $3_1 \cZ_M$. Apply motivic Zagier and do an explicit calculation. By the explicit Zagier formula, one knows $v_2$.

If you look more precisely at the Zagier formula. We need to show that the matrix is invertible. The point is that the coefficients in Goncharov's formula (\ref{thm:goncharov}), the only non-integrality comes out of $ct(\partial(v))$. But if you look in the explicit Zagier formula, there are only powers of two in the denominators.

\begin{prop}
For all $\ell \geq 2$. Let $A$ be the matrix for $\partial$. There exists a way to multiply columns by a power of 2 such that $A(mod 2)$ is lower triangular with ones along the diagonal. Hence it is invertible.
\end{prop}
We use induction on $\ell$ where the base case, $\ell=1$ follows from Zagier's formula.

This completes Step 4 and the proof.


%\bibliographystyle{amsalphaabbrv}

\bibliographystyle{amsalpha}
\bibliography{alpbach-2012}

%\begin{thebibliography}{DM69}
%\end{thebibliography}

\end{document}