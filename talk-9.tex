\chapter{Mixed Tate motives over $\Z$ by Martin Gallauer}

Martin Gallauer on September 6th, 2012.

\medskip
\medskip

\noindent Possible references for this chapter are \cite{voevodsky00-mm}, \cite{andre04-motifs} or \cite{MVW-motcoh}.

\section{Introduction}
\label{sec:9-intro}

The two principal tasks of this chapter are to construct the category
of mixed Tate motives over $\Z$ and to apply results from the last
chapter to it. We won't construct this category in a ``bottom-up''
approach but instead extract it as a full subcategory of the larger
triangulated category of mixed motives over $\Q$. Thus more precisely
our tasks are, in order:
\begin{enumerate}
\item Construct $\DM(\Q)$, the triangulated category of mixed motives
  over $\Q$, or more generally, $\DM(k)$ for any field $k$.
\item Extract $\mTm(\Z)$ as a full subcategory of $\DM(\Q)$.
\item Apply results from last chapter.
\end{enumerate}

The construction of $\DM(k)$ proceeds in several steps and for someone
unacquainted with motives it might be difficult to see \emph{why}
these steps ought to be taken and what \emph{results} from them. For
these readers we would like to make life easier by, firstly, describing
the general philosophy of mixed motives in the rest of the
introduction, and secondly, stating some of the most important
properties of the resulting category $\DM(k)$ after the construction
has been done (see~\ref{sec:9-prop}).

To describe the idea of mixed motives (very roughly, of course; we
follow \cite[chapter~14]{andre04-motifs}), fix a field $k$ and
consider the category $\Sm/k$ of smooth varieties over $k$, \ie{}
separated smooth finite type schemes over $k$ (actually, one often
considers \emph{all} varieties over $k$). A \emph{mixed Weil homology
  theory} is a functor $H_{*}:\Sm/k\to {\cal A}$ to an abelian
$\otimes$-category endowed with an action of finite correspondences
which satisfies at least the following axioms: homotopy invariance,
Künneth formula and long exact sequence of Mayer-Vietoris
type. (Sometimes more axioms are postulated; and usually contravariant
functors are considered, giving rise to the notion of a mixed Weil
\emph{co}homology theory.) Typical examples are Betti, algebraic de
Rham or $\ell$-adic homology.

\emph{Mixed motives} should be considered as a universal Weil homology
theory. More explicitly, there should exist an abelian
$\otimes$-category $\MM(k)$ with a mixed Weil homology theory
$\Sm/k\to\MM(k)$ such that for any mixed Weil homology theory $H$ the
following diagram can be completed commutatively in a unique way with
a dotted arrow, called a \emph{realization functor}:
\begin{equation*}
  \xymatrix{\Sm/k\ar[d]_{H}\ar[r]&\MM(k)\ar@{.>}[ld]^{\mathrm{real}_{H}}\\
    {\cal A}}
\end{equation*}
Alas, this category $\MM(k)$ is not known to exist although people
have certainly tried to construct it for the last decades. Deligne was
it who, in the eighties of the last century, devised another approach
to mixed motives: Instead of abelian categories ${\cal A}$ as targets
of homology theories the focus is now shifted to functors into
$\otimes$-\emph{triangulated} categories ${\cal D}$ from which a mixed
Weil homology theory would be obtained via a functor ${\cal D}\to{\cal
  A}$. The axioms for the mixed Weil homology theory are thus replaced
by their triangulated translations (in particular, the Mayer-Vietoris
long exact sequence is replaced by a distinguished triangle) and,
again, the \emph{triangulated category of mixed motives} $\DM(k)$
would be the universal such theory. The diagram above is thus replaced
by its triangulated version:
\begin{equation*}
  \xymatrix{\Sm/k\ar[d]_{H}\ar[r]&\DM(k)\ar@{.>}[ld]^{\mathrm{real}_{H}}\\
    {\cal D}}  
\end{equation*}
In contrast to the abelian case there are indeed candidates for this
universal category one of which we will describe in more detail in the
next section.

In addition, one hopes to recover $\MM(k)$ from $\DM(k)$ by a
construction which is known as a $t$-structure. One way to think about
this is to imagine $\DM(k)$ to be the (bounded) derived category of
$\MM(k)$ and to use the shift functor on $\DM(k)$ to recover $\MM(k)$
as the ``homological 0-part''. A $t$-structure is a natural
generalization of this idea. (We will come back to this idea in
section~\ref{sec:9-tate}.)

We hope that keeping the universal property of $\DM(k)$ as well as its
intended relation with $\MM(k)$ in mind, it will be easier to follow
the construction in the next section.

\section{Mixed motives: Construction}
\label{sec:9-construction}
Let $k$ be a field, and let $\Sm/k$ be the category of smooth
separated finite type schemes over $k$. If not mentioned otherwise,
all schemes considered from now on will be such smooth varieties.

We will construct the triangulated category of mixed motives over $k$,
$\DM(k)$, following Voevodsky (see \cite[2]{voevodsky00-mm}; there this category is
denoted $\DM_{gm}(k)$ (``geometric motives''))

\subsubsection{Step 1: Correspondences and linearization}

\begin{defn}Let $X, Y \in \Sm/k$.
  \begin{enumerate}
  \item A closed integral subscheme $\alpha \subset X \times Y$ is
    called an \emph{elementary correspondence from $X$ to $Y$} if it
    is finite and surjective over a connected component of $X$.
  \item Set $c(X,Y)$ to be the $\Q$-vector space generated by
    elementary correspondences from $X$ to $Y$. Elements of this
    vector space are called \emph{finite correspondences}.
\end{enumerate}
\end{defn}

\begin{exam}
  The graph $\Gamma_f$ of a morphism $f : X \to Y$ of schemes defines
  a finite correspondence, namely its associated reduced subscheme. It
  is an elementary correspondence if and only if $X$ is connected.
\end{exam}

\begin{rem}
  One can consider elementary correspondences as ``multi-valued
  functions'' as follows (\cite[15.1.2]{andre04-motifs}): Let $\alpha$
  be an elementary correspondence from $X$ to $Y$ and let $x$ be a
  closed point of $X$. The pre-image of $x$ under the finite morphism
  $\alpha\to X\times Y\to X$ is a 0-cycle $\sum_{i}n_{i}a_{i}$,
  $n_{i}\in\Z, a_{i}$ closed points of $\alpha$. This ``multi-valued''
  morphism
  \begin{align*}
    X&\to \alpha\\
    x&\mapsto \sum_{i}n_{i}a_{i}
  \end{align*}
  is called \emph{transfer}, and $\alpha$ can be considered as the
  composition of this transfer with the projection $\mathrm{p}_{2}:\alpha\to X\times
  Y\to Y$, thus $x\mapsto\sum_{i}n_{i}p_{2}(a_{i})$.
\end{rem}

The definition of our correspondences is so chosen that they can
easily been composed. (For the claims made in the sequel
see~\cite[chapter~1]{MVW-motcoh}.) Namely, let $\alpha\subset X\times
Y$ and $\beta\subset Y\times Z$ be elementary correspondences and
consider their pullback to the triple product $X\times Y\times
Z$. They intersect properly and we may consider their intersection
product $(\alpha\times Z)\cdot(X\times \beta)$ which is finite over
$X\times Z$. Denoting by $p$ the projection from the triple product to
$X\times Z$ we set:
\begin{defn} The composition of $\alpha$ and $\beta$, denoted by
  $\beta\circ \alpha$, is defined to be the pushforward
  $p_{*}((\alpha\times Z)\cdot(X\times \beta))$. The composition is
  extended $\Q$-linearly to all finite correspondences.
\end{defn}
$\beta\circ\alpha$ is again a finite correspondence and the
composition is associative. Moreover, $\Gamma_{\mathrm{Id}_{X}}$ acts
as identity with respect to this composition.
\begin{defn}
  We denote by $\SmCor(k)$ the \emph{category of finite
    correspondences}. Its objects are the objects of $\Sm/k$ and for
  $X, Y \in \Sm/k$, the morphisms from $X$ to $Y$ are the finite
  correspondences $c(X,Y)$.
\end{defn}

\begin{rem}
\begin{itemize}
\item $\SmCor(k)$ is a $\Q$-linear $\otimes$-category, with coproducts
  given by disjoint union and the monoidal product by the direct
  product (over $k$). More explicitly, this means that the category
  has a $\Q$-linear as well as a symmetric monoidal unitary structure
  and the monoidal product is $\Q$-linear in both variables. TODO:
                               Refer to definition of this in the
                                last chapter
                              \item The functor $\Sm/k \to \SmCor(k)$
                                is a $\otimes$-functor, in particular
                                $\Gamma_{g}\circ
                                \Gamma_{f}=\Gamma_{g\circ f}$.
\end{itemize}
\end{rem}

\subsubsection{Step 2: Triangulation}
We proceed in the usual way to pass from an additive to a triangulated
category.
\begin{defn}
  Let $K^{b}(\SmCor(k))$ be the category of bounded complexes
  in $\SmCor(k)$ up to chain homotopy. In other words, objects are
  bounded chain complexes in $\SmCor(k)$ and morphisms are equivalence
  classes of chain complex morphisms with respect to chain
  homotopies. (The differential in the complexes has degree -1.)
\end{defn}

\begin{rem}
  The category $K^b(\SmCor(k))$ is a $\Q$-linear triangulated
  $\otimes$-category. More explicitly, it is a $\Q$-linear
  $\otimes$-category with a triangulated structure such that $\otimes$
  is triangulated in both variables. (See\cite[8A]{MVW-motcoh} for the
  precise definition.)

  Moreover, the functor $\SmCor(k)\to K^b(\SmCor(k))$ respects the
  monoidal and the $\Q$-linear structure.
\end{rem}

\subsubsection{Step 3: Impose relations}
Let $X$ be a smooth variety and consider the following complex
$[\Aspace^{1}_{X}]\to [X]$ concentrated in degrees 1 and 0 (say). To ensure
homotopy invariance we would like this complex to be 0. Similarly,
suppose there is an open covering $X=U\cup V$ of $X$ and consider the
following complex in degrees 2, 1 and 0:
\begin{equation*}
  \xymatrix{
[U \cap V] \ar[r]^{\iota} & [U] \oplus [V] \ar[r]^{\quad(+, -)} & [X] 
}
\end{equation*}
To ensure we get a distinguished triangle of Mayer-Vietoris type we
would also like this complex to be 0. This is what we will now impose.

For this, set ${\cal T}$ to be the thick subcategory of
$K^b(\SmCor(k))$ generated by the two types of complexes above, \ie{}
the smallest full triangulated subcategory containing these objects
and closed under direct factors.
\begin{defn}
  \begin{enumerate}
  \item $K^b_{\Aspace^{1},\mathrm{MV}}(\SmCor(k))$ is the quotient category
    of $K^b(\SmCor(k))$ with respect to ${\cal T}$.
  \end{enumerate}
\end{defn}

More explicitly, the objects are the same but the morphisms between
two complexes $A$ and $B$ are $\varinjlim \Hom_{K^b(\SmCor(k))}(A',B)$
where $A'\to A$ runs over all morphisms in $K^b(\SmCor(k))$ whose cone
lies in ${\cal T}$. There is a canonical quotient functor
$K^b(\SmCor(k))\to K^b_{\Aspace^{1},\mathrm{MV}}(\SmCor(k))$.
\begin{rem}
There is (in general) a canonical triangulated structure on the
quotient which makes the quotient functor triangulated. It is easy to
check that (in our case) there is also a canonical $\otimes$-structure
on the quotient which makes the quotient functor a
$\otimes$-functor (this boils down to check that ${\cal T}$ is closed
under tensoring with an arbitrary complex). Moreover these structures
are compatible with each other and also with the $\Q$-linear structure.
\end{rem}

\subsubsection{Step 4: Pseudo-abelianization}
\begin{defn}
  An additive category $(\cC, \otimes)$ is called
  \emph{pseudo-abelian} if for all objects $X \in \cC$ and for all
  idempotents $p : X \to X \in \cC$ (i.e., $p^2 = p$), $X = \ker(p)
  \oplus \ker(\mathrm{Id}_{X} - p)$.
\end{defn}
\begin{rem}
  The bounded derived category of an abelian category is
  pseudo-abelian (\cite[2.8]{Balmer-Schlichting}). Thus keeping with
  the intention described in the introduction we will want our
  category $\DM(k)$ to be pseudo-abelian. There is a universal way to
  pass from an additive to a pseudo-abelian category which is
  described as taking the \emph{pseudo-abelian hull}.
\end{rem}
\begin{defn}
  We set $\DM^{\mathrm{eff}}(k)$ to be the pseudo abelian hull of
  $K^b_{\Aspace^1,\mathrm{MV}}(\SmCor(k))$. It is called the
  \emph{triangulated category of effective mixed motives over
    $k$}. Accordingly, objects of this category are called
  \emph{effective motives}.
\end{defn}
Explicitly, the objects are pairs $(A,p)$ where $A$ is an object in
$K^b_{\Aspace^1,\mathrm{MV}}(\SmCor(k))$ and $p$ is an idempotent on $A$. A
morphism $(A,p)\to (A',p')$ is a morphism in
$K^b_{\Aspace^1,\mathrm{MV}}(\SmCor(k))$ of the form
\begin{equation*}
  A\stackrel{p}{\to}A\to A'\stackrel{p'}{\to}A',
\end{equation*}
and composition is induced by the original one. There is a canonical
functor $K^b_{\Aspace^1,\mathrm{MV}}(\SmCor(k))\to \DM^{\mathrm{eff}}(k)$
which sends $A$ to $(A,\mathrm{Id}_{A})$.
\begin{rem}
  $\DM^{\mathrm{eff}}(k)$ is a $\Q$-linear triangulated tensor
  category and the functor just defined is a fully faithful embedding
  preserving all the structure. The only non-trivial part of this
  statement is about the triangulation. For this see
  \cite[1.5]{Balmer-Schlichting}.
\end{rem}
\begin{defn}
  We write $M$ for the $\otimes$-functor
  $\Sm/k\to\DM^{\mathrm{eff}}(k)$. For any smooth variety $X$, $M(X)$
  is called the \emph{motive of} $X$.
\end{defn}
\begin{exam}
  \begin{enumerate}\item 
    The $\otimes$-unit in $\DM^{\mathrm{eff}}(k)$ is the motive of the
    $\otimes$-unit in $\Sm/k$:
    \begin{equation*}
      \mathbf{1_{\otimes}} = M(\spec k) =: \Q(0)
    \end{equation*}
  \item One important motive is $M(\Pspace^1)$. In any homology theory $H$
    we have $H(\Pspace^{1})=H_{0}(\Pspace^{1})\oplus
    H_{2}(\Pspace^{1})$ thus the same decomposition should be true of
    the motive $M(\Pspace^{1})$. Our next goal is to define the ``Tate
    object'' $\Q(1)$ so that
    \begin{equation}
      \label{eq:9-p1-decomposition}
      M(\Pspace^{1})=\Q(0)\oplus\Q(1)[2].
    \end{equation}
  \end{enumerate}
\end{exam}

Starting with any smooth variety $X$ consider the following complex in 
$K^b(\SmCor(k))$
\begin{equation*}
  [X]\to [\spec k]
\end{equation*}
concentrated in degrees $0$ and $-1$.
\begin{defn}
  The image of this complex in $\DM^{\mathrm{eff}}(k)$ is denoted by
  $\tilde{M}(X)$ and is called the \emph{reduced motive of} $X$.
\end{defn}

\begin{rem}
  There is a canonical distinguished triangle
  \begin{equation*}
    \tilde{M}(X)\to M(X)\to\Q(0)\to^{+}
  \end{equation*}
  in $\DM^{\mathrm{eff}}(k)$ associated to any smooth variety $X$.
\end{rem}

\begin{exam}
  Let $x: \spec k \to \Pspace^1$ be a rational point. Then the
  composition
\begin{equation*}
  p: [\Pspace^1] \to [\spec k] \stackrel{x}{\to} [\Pspace^1]
\end{equation*}
is an idempotent in $K^b(\SmCor(k))$, i.e., $p \circ p = p$. Thus we
have a decomposition in $\DM^{\mathrm{eff}}(k)$:
$M(\Pspace^1)=\im(p)\oplus\ker(\mathrm{Id}-p)$. It is easy to see
that $\im(p)=\Q(0)$ and the canonical projection
$M(\Pspace^1)\to\Q(0)$ is compatible with the decomposition. Hence we
get a distinguished triangle
\begin{equation*}
  \ker(\mathrm{Id}-p)\to M(X)\to\Q(0)\to^{+}
\end{equation*}
and it follows from the previous remark that
$\ker(\mathrm{Id}-p)=\tilde{M}(\Pspace^{1})$ and we have a decomposition
\begin{equation*}
  M(\Pspace^{1})=\Q(0)\oplus\tilde{M}(\Pspace^{1})
\end{equation*}  
\end{exam}
\begin{defn}
  The \emph{Tate motive} $\Q(1)$ is defined as
  $\tilde{M}(\Pspace^{1})[-2]$. For any effective motive $N$ and any
  $n\in\N$ we define
  \begin{equation*}
    N(n):=N\otimes\Q(n):=N\otimes\Q(1)^{\otimes n}
  \end{equation*}
\end{defn}
It is easy to show, using homotopy invariance, that the decomposition
of the motive of $\Pspace^{1}$ does not depend on the choice of the
rational point (see \cite[16.3.1.3]{andre04-motifs}) hence neither
does~(\ref{eq:9-p1-decomposition}).
\begin{rem}
  The argument in the previous example shows more generally that for
  any smooth variety $X$ with a rational point there is a
  decomposition $M(X)=\Q(0)\oplus \tilde{M}(X)$. In contrast to the
  example above however, the decomposition depends in general on the
  choice of the rational point.
\end{rem}
\subsubsection{Step 5: Towards rigidity}

Tensoring with $\Q(1)$ defines an endofunctor of
$\DM^{\mathrm{eff}}(k)$ which we invert to get a ``dual'' to $\Q(1)$.
\begin{defn}
Set $\DM(k):= \DM^{\mathrm{eff}}(k)[\Q(1)^{-1}]$. This category is
called the \emph{triangulated category of mixed motives over} $k$. Its
objects are accordingly called \emph{motives}.
\end{defn}
Explicitly, objects are pairs $(N,n)$ where $N$ is an effective
motive and $n\in\Z$ while the set of morphisms $(N,n)\to (N',n')$ is
\begin{equation*}
  \varinjlim_{k\geq -n,-n'}\Hom_{\DM^{\mathrm{eff}}(k)}(N(k+n),N'(k+n')).
\end{equation*}
There is a canonical functor $\DM^{\mathrm{eff}}(k)\to\DM(k)$ sending
$N$ to $(N,0)$ and we denote the composition $\Sm/k\to \DM(k)$ also by
$M$.
\begin{rem}
The category $\DM(k)$ is canonically a $\Q$-linear triangulated
$\otimes$-category and the functor just defined preserves all the
structure. However, for the symmetric monoidal structure there is
something to prove (see \cite[17.1.2]{andre04-motifs}).

It is not difficult to see that $\DM(k)$ is still pseudo-abelian.
\end{rem}

\begin{defn}
  Let $\Q(-1)$ be an object in $\DM(k)$ such that
  $\Q(-1)\otimes\Q(1)=\Q(0)$. We then define for any motive $N$ and
  any $n\in\Z$ the $n$\emph{th twist of} $N$ to be
  \begin{equation*}
    N(n):=N\otimes \Q(n):=N\otimes \Q\left(\frac{n}{|n|}\right)^{\otimes |n|}.
  \end{equation*}
\end{defn}

We will see in the next section that inverting the Tate object already
suffices to render the whole category rigid. $\Q(-1)$ is the dual of
the Tate object.

\section{Mixed motives: Properties}
\label{sec:9-prop}
From now on we assume $\mathrm{char~}(k) = 0$ although this is not always
necessary.

In this section we will state some of the most important properties of
the category $\DM(k)$ defined above (although not everything will be
needed in the rest of the chapter). The proof of these properties lies
at the center of the theory developed by Voevodsky et al. The method
used is to embed $\DM(k)$ into a larger category in which one can
``apply all the standard machinery of sheaves and their cohomology.''
\cite[p.~7]{voevodsky00-mm}

See \cite[2.2]{voevodsky00-mm}, \cite[14]{MVW-motcoh},
\cite[18]{andre04-motifs} for the statements.


\subsection{``Homological'' properties}

\begin{thm}
  The functor $M:\Sm/k\to\DM(k)$ extends to all varieties over $k$ and
  satisfies:
  \begin{itemize}
  \item the Künneth formula: $M(X\times Y)= M(X)\otimes M(Y)$;
  \item homotopy invariance: $M(\Aspace^{1}_{X})=M(X)$;
  \item Mayer-Vietoris distinguished triangle: $M(U\cap V)\to
    M(U)\oplus M(V)\to M(X)\to^{+}$ (for any open covering $X=U\cup
    V$);
  \item blow-up distinguished triangle: $M(p^{-1}(Z))\to
    M(X_{Z})\oplus M(Z)\to M(X)\to^{+}$ (for $p:X_{Z}\to X$ a blow-up
    with center $Z$);
  \item projective bundle formula:
    $M(\Pspace(E))=\oplus_{i=0}^{n}M(X)(i)[2i]$ (for $E$ a rank $n+1$
    vector bundle over $X$).
  \end{itemize}
  In addition, there is also a functor $M^{c}$ of motives with compact
  support, a Gysin distinguished triangle\ldots{}
\end{thm}

In summary, the behaviour of this functor $M$ is as expected from a
homological theory of algebraic varieties. 

\begin{exam}\label{exam:9-gm}
  We want to use the previous theorem to compute $M(\Gm^n)$.
  \begin{enumerate}
  \item If $n=1$, then we use the Mayer-Vietoris decomposition of
    $\Pspace^{1} =
    (\Pspace^{1}\backslash\{0\})\cup(\Pspace^{1}\backslash\{\infty\})$
    to get a distinguished triangle
    \begin{equation*}
      M(\Gm) \to M(\Aspace^1) \oplus M(\Aspace^1) \to M(\Pspace^1) \to^{+}.
    \end{equation*}
    By choosing a common base point this yields another distinguished
    triangle
    \begin{equation*}
      \tilde{M}(\Gm) \to \tilde{M}(\Aspace^1) \oplus \tilde{M}(\Aspace^1) \to \tilde{M}(\Pspace^1) \to^{+}.
    \end{equation*}
    By homotopy invariance, the two summands in the middle are 0 and
    we conclude $\tilde{M}(\Gm)=\tilde{M}(\Pspace^{1})[1]=\Q(1)[1]$.
  \item If $n \geq 1$, then
    \begin{align*}
      M(\Gm^n) &= \otimes_{i=1}^{n} M(\Gm)&&\text{Künneth formula}\\
      &= \otimes_{i=1}^{n}(\Q(0) \oplus \Q(1)[1]) &&\text{previous part}\\
      &= \bigoplus_{i=0}^n \binom{n}{i} \Q(i)[i]
    \end{align*}
  \end{enumerate}
\end{exam}

\subsection{Cancellation}

\begin{thm}
The endofunctor 
\begin{equation*}
- \otimes \Q(1) : \DM(k) \to \DM(k)
\end{equation*}
is fullly faithful.
\end{thm}
It follows that the canonical functor $\DM^{\mathrm{eff}}(k)\to\DM(k)$
is a fully faithful embedding.

\subsection{Rigidity}

\begin{thm}
  There exists an autoduality
  \begin{equation*}
    ^{\vee}:\DM(k)^{\mathrm{op}}\to\DM(k)
  \end{equation*}
  making $\DM(k)$ a rigid $\Q$-linear category.

  Moreover, if $X$ is smooth and projective of dimension $d$, then
  $M(X)^{\vee}=M(X)(-d)[-2d]$.
\end{thm}

\begin{exam}
  For any (smooth) variety $X$ over $k$, the motive $M(X)^{\vee}$ is
  canonically an algebra object in $\DM(k)$. Indeed, since $\Sm/k$ is
  cartesian monoidal, $X$ carries a unique coalgebra structure given
  by the diagonal embedding and the canonical projection to
  $\spec(k)$. Thus it is an algebra object in $(\Sm/k)^{\mathrm{op}}$
  and is mapped to an algebra object in $\DM(k)$ under the
  $\otimes$-functor $^{\vee}\circ M$.

  Explicitly, the ``multiplication'' is \eg{} given by
  \begin{align*}
    M(X)^{\vee} \otimes M(X)^{\vee} &= (M(X) \otimes
    M(X))^{\vee}&&\text{autoduality}\\
    &= M(X \times X)^{\vee} &&\text{Künneth formula}\\
    &\to M(X)^{\vee}&&\text{diagonal embedding } X\to X\times X
  \end{align*}
\end{exam}

\subsection{Realization functors}
There are general sufficient conditions under which a functor
$\Sm/k\to {\cal D}$ induces a ``realization'' functor $\DM(k)\to{\cal
  D}$ (see~\cite{huber00-realization}) however we are interested only
in the following examples:
\begin{exam}
  \begin{enumerate}
  \item There is a \emph{de Rham realization} functor
    \begin{equation*}
    \omega_{\mathrm{dR}} : \DM(k) \to \D^b(k\Mod)
  \end{equation*}
  which comes with an increasing weight and a decreasing Hodge
  filtration.

  ``Realization'' here simply means that the following diagram
  commutes: 
  \begin{equation*}
    \xymatrix{
      \Sm/k \ar[r]^{M} \ar[d]_{H_{*}^{\textrm{dR}}} & \DM(k) \ar[dl]^{\omega_{dR}} \\
      \D^b(k\Mod)
    }
  \end{equation*}
  where $H_{*}^{\mathrm{dR}}$ is algebraic de Rham homology. Also the
  diagram is compatible with the filtrations.
\item Let $k$ be a subfield of $\C$ with embedding $\sigma : k \inj
  \C$. Then there is an associated \emph{Betti realization} functor
  \begin{equation*}
    \omega_{\mathrm{B},\sigma} : \DM(k) \to \D^b(\Q\Mod)
  \end{equation*}
\item Let $\ell$ be a prime. There is an \emph{$\ell$-adic
    realization} functor
  \begin{equation*}
    \omega_{\ell} : \DM(k) \to \D^b(\Q_{\ell}\Mod)
  \end{equation*}
  This is a \emph{geometric} realization thus endowed with an action
  of the Galois group.
\end{enumerate}
\end{exam}

\begin{exam}\label{exam:9-hodge-realization}
  For any embedding $\sigma:k\inj \C$ there is an isomorphism of the
  Betti and de Rham functors after scalar extension to $\C$ (the
  ``period'' or ``comparison'' isomorphism)
  \begin{equation*}
    \omega_{\mathrm{dR}} \otimes \C \cong \omega_{\mathrm{B},\sigma} \otimes \C
  \end{equation*}
  From this one deduces a \emph{Hodge realization} functor
  \begin{equation*}
    \omega_{\mathrm{H}} : \DM(k) \to \D^b(\mathrm{MHC}(k)) \text{ TODO: Target category}
  \end{equation*}

  It is not hard to see that $\omega_{\mathrm{H}}(\Q(1)) =
  \Q(1)_{\mathrm{H}}$ (see TODO: Reference to example in one of the
  previous chapters).
\end{exam}

\subsection{Motivic cohomology}

\begin{defn}
  Let $X$ be a smooth variety over $k$. Then the \emph{motivic
    cohomology of $X$ with $\Q$-coefficients} is
  \begin{equation*}
    H^{i,n}(X, \Q) := \Hom_{\DM(k)}(M(X), \Q(n)[i]), \qquad i,n\in\Z
  \end{equation*}
\end{defn}

\begin{thm}
  Let $X$ be a smooth variety over $k$. Then
  \begin{equation*}
    H^{i,n}(X, \Q) = K_{2n-i}(X)^{(n)}_{\Q}
  \end{equation*}
  where the right hand side denotes the subspace of weight $n$ (with
  respect to the Adams operator) of algebraic $K$-theory tensored with
  $\Q$.
\end{thm}

\begin{exam}\label{exam:9-hom-base}
  It follows that $\Hom_{\DM(k)}(\Q(m),\Q(n))=0$ whenever $n<m$, and
  $=\Q$ if $m=n$.
\end{exam}

\begin{exam}
  Let $X = \spec(k)$. Then the claim that 
  \begin{equation*}
    \text{(B-S)}_{k}:\quad H^{i,n}(\spec(k), \Q) = K_{2n-i}(k)^{(n)}_{\Q}=0,\qquad \forall n \geq 0, \forall i < 0
  \end{equation*}
  is known as the \emph{Beilinson-Soulé vanishing conjecture for
    $k$}. It plays a strikingly important role in the theory of mixed
  (Tate) motives as we will see a little bit in the next section. It
  is not known to hold except in some special cases, among which is
  the only one we're interested in:
\end{exam}

\begin{thm}[Borel]\label{thm:9-borel}
  Let $k = \Q$. Then
  \begin{align*}
    K_{2n-1}(\Q)_{\Q}^{(n)} &= K_{2n-1}(k)_{\Q} = \left\{ \begin{array}{ll}
        \Q^{\times} \otimes \Q &:n=1 \\
        \Q &:\textrm{$n\geq 3$ odd} \\
        0 &:\textrm{otherwise}
      \end{array} \right.\\
    K_{2n}(\Q)_{\Q} &= 0 \qquad \forall n>0
  \end{align*}
  Hence (B-S)$_{\Q}$ is true.
\end{thm}

The term $\Q^{\times}\otimes\Q$ is easily computed. Indeed, the maps
\begin{eqnarray*}
  \Q^{\times}\otimes \Q&\leftrightarrow&\oplus_{p\text{ prime}}\Q\\
  x\otimes y&\mapsto&y\cdot(v_{p}(x))_{p}\\
  q\otimes z&\mapsfrom&z\in \Q\stackrel{q}{\inj}\oplus_{p}\Q
\end{eqnarray*}
define a $Q$-linear bijection ($v_{p}$ is the $p$-valuation).
\begin{rem}
  In fact, Borel proved a similar formula for all number fields $k$;
  in particular, (B-S)$_{k}$ is true.
\end{rem}
\section{Mixed Tate motives}
\label{sec:9-tate}
\subsection{Mixed Tate motives over $\Q$}

\begin{defn}
  A \emph{mixed Tate motive over $k$} is an iterated extension in
  $\DM(k)$ of motives of the form $\Q(n), n \in \Z$.
\end{defn}
This means that a mixed Tate motive $N$ fits into a distinguished triangle
\begin{equation*}
  N'\to N\to \Q(n)\to^{+},
\end{equation*}
where $N'$ is a mixed Tate motive and $n\in\Z$.
\begin{defn}
  \begin{enumerate}
  \item The \emph{category of mixed Tate motives over $k$}, $\MTM(k)$,
    is the full subcategory of $\DM(k)$ spanned by mixed Tate motives.
  \item The \emph{triangulated category of mixed Tate motives over
      $k$}, $\DTM(k)$, is the full triangulated subcategory of
    $\DM(k)$ generated by mixed Tate motives.
  \end{enumerate}
\end{defn}

\begin{exam}
  Hence $\MTM(k) \subset \DTM(k)$ as full subcategories. If
  (B-S)$_{k}$ is satisfied, it is easy to see---using~\ref{exam:9-gm}
  and the next result~\ref{thm:9-mtmk}---
  that $\Gm\in \DTM(k)$ but $\notin\MTM(k)$.%Write ses N\to M(G_m)\to
                                %Q(n) in MTM. last arrow is 0 unless
                                %n=0 by B-S. Hence n=0 and last arrow
                                %is projection onto first factor. Thus
                                %we get N=Q(1)[1] but this is not MTM
                                %by the same argument.
\end{exam}

The following theorem is a formal consequence of how the sets
$\Hom_{\DM(k)}(\Q(m)[i],\Q(n)[j])$ look like for different
$m,n,i,j\in\Z$---assuming the Beilinson-Soulé vanishing conjecture
(see~\cite{levine92-tatemotives}):
\begin{thm}\label{thm:9-mtmk}
  If (B-S)$_k$, then
  \begin{enumerate}
  \item\label{thm:9-abelian} There exists a $t$-structure on $\DTM(k)$ whose heart is
    $\MTM(k)$. $\MTM(k)$ is abelian and closed under extensions.
  \item\label{thm:9-tensor} The $\otimes$-structure restricts to $\MTM(k)$ and makes it
    rigid. More generally, the $t$-structure is compatible with the
    tensor structure.
  \item There exists a finite, functorial, increasing $2\Z$-filtration
    $W$ on $\DTM(k)$ which induces an exact filtration on
    $\MTM(k)$. Moreover,
    \begin{equation*}
      \gr_{2n}^W M = \oplus_{i=1}^N \Q(-n), \qquad \gr_{2n}^W \otimes \gr_{2m}^W = \gr_{2(n+m)}^W
    \end{equation*}
  \item\label{thm:9-fiber} \begin{align*}
      \omega_W:\MTM(k)&\to\Q\Mod\\
      N &\mapsto \oplus_{n \in \Z}\Hom_{\MTM(k)}(\Q(-n), \gr_{2n}^WN)
      \end{align*}
      is an exact, faithful $\otimes$-functor.
  \item\label{thm:9-ext} The $\Ext$-groups satisfy
    \begin{eqnarray*}
      \Ext^1_{\MTM(k)}(N,N') & = & \Hom_{\DTM(k)}(N,N'[1]) \\
      \Ext^2_{\MTM(k)}(N,N') & \inj & \Hom_{\DTM(k)}(N,N'[2])
    \end{eqnarray*}
  \end{enumerate}
\end{thm}

\begin{cor}\label{cor:9-mtm-tannaka}
  $\MTM(\Q)$ is a neutral Tannakian category with fiber functor
  $\omega_{W}$, and
  \begin{align*}
    \Ext^1_{\MTM(\Q)}(\Q(0), \Q(n)) &= \left\{ \begin{array}{ll}
        \Q^{\times} \otimes \Q &:n = 1 \\
        \Q &:n\geq 3 \textrm{ odd} \\
        0 &:\textrm{otherwise}
      \end{array} \right.\\
    \Ext^2_{\MTM(\Q)}(\Q(0), \Q(n)) &= 0
  \end{align*}
\end{cor}
\begin{proof} By Borel's theorem~\ref{thm:9-borel}, $\Q$ satisfies the
  Beilinson-Soulé vanishing conjecture hence the theorem applies.  By
  parts~\ref{thm:9-abelian} and \ref{thm:9-tensor}, $\MTM(\Q)$ is a
  rigid abelian $\otimes$-category. By~\ref{thm:9-fiber}, $\omega_{W}$
  is a fiber functor, and by Example~\ref{exam:9-hom-base},
  $\Hom_{\MTM(k)}(\Q(0),\Q(0))=\Q$.

  The computation of the $\Ext$-groups follows from
  part~\ref{thm:9-ext} of the theorem together with~\ref{thm:9-borel}.
\end{proof}

\begin{rem}
  The theorem exemplifies the idea we mentioned in the introduction,
  namely that one should be able to recover $\MM(k)$ from $\DM(k)$ via
  the construction of a $t$-structure. The theorem shows that this is
  possible in the case of mixed \emph{Tate} motives.

  There is a canonical functor $\D^{b}(\MTM(k))\to \DTM(k)$ which in
  general is not expected to be an equivalence. It is one however in
  the case of number fields. TODO: reference
\end{rem}
\subsection{Mixed Tate motives over $\Z$}

At this stage we could apply the Tannakian theory developed in the
last chapter and get a fundamental group associated to
$(\MTM(\Q),\omega_{W})$. However, this fundamental group would be
``too big'' for our purposes; in particular, if we want to apply the
results from the last chapter (as we do), 
we should ensure that the groups $\Ext^1(\Q(0), \Q(n))$ are
finite-dimensional, in particular that $\Ext^1(\Q(0), \Q(1))$ is. The
restriction to motives ``defined over $\Z$'' is intended to achieve
exactly this.

Let us first describe explicitly the elements of the group
$\Ext^1_{\MTM(k)}(\Q(0), \Q(1))$.
\begin{defn}
  Let $f : Y \to X$ be a morphism of smooth varieties. This defines a
  complex 
  \begin{equation*}
    [Y]\to[X]
  \end{equation*}
  concentrated in degrees 1 and 0 in $K^{b}(\SmCor(k))$. Its image in
  $\DM(k)$ is denoted by $M(X,Y)$ and is called the \emph{relative
    motive associated to $f$}.
\end{defn}

\begin{rem}
  Every such $f$ induces a distinguished triangle
  \begin{equation*}
    M(X)\to M(X,Y)\to M(Y)[1]\to^{+}%the degree 1 map actually comes
                                %with a minus sign
  \end{equation*}
  in $\DM(k)$.
\end{rem}
\begin{exam}
  Let $t \in k^* / \{ 1 \}$ and consider the embedding
  \begin{equation*}
    \{ 1,t \} \inj \Pspace^1 \setminus \{0, \infty\}=\Gm.
  \end{equation*}
  The associated motive gives rise to the distinguished triangle
  \begin{equation*}
    M(\Gm)\to M(\Gm,\{1,t\})\to (\Q(0)\oplus\Q(0))[1]\to^{+}
  \end{equation*}
  It's not difficult to deduce from this another distinguished triangle
  \begin{equation*}
    \tilde{M}(\Gm)\to M(\Gm,\{1,t\})\to \Q(0)[1]\to^{+}
  \end{equation*}%use base point 1 for the two outer objects and
                 %apply 9-lemma for triangulated categories
  and by applying the shift functor:
  \begin{equation*}
    \Q(1)\to M(\Gm,\{1,t\})[-1]\to \Q(0)\to^{+}.
  \end{equation*}
  By the theorem, this induces a short exact sequence
  \begin{equation*}
    0\to\Q(1)\to M(\Gm,\{1,t\})[-1]\to \Q(0)\to 0,
  \end{equation*}
  hence we obtain an element $K(t)\in\Ext^1_{\MTM(k)}(\Q(0), \Q(1))$,
  called the \emph{Kummer extension associated with
    $t$}. $M(\Gm,\{1,t\})[-1]\in\MTM(k)$ is called the \emph{Kummer
    motive associated with $t$}.

  The association
  \begin{align*}
    k^{*}\otimes \Q&\to \Ext^{1}_{\MTM(k)}(\Q(0),\Q(1))\\
    t\otimes 1&\mapsto K(t)\\
  \end{align*}
  induces a $Q$-vector space isomorphism.

  Finally, let us consider the realizations of $K(t)$. The Hodge
  realization of this extension was described in chapter (TODO:
  Pre-Alpbach). For the étale realization fix a prime $\ell$. Then
  $\omega_{\ell}(K(t))$ arises from an extension
  \begin{equation}
    0\to \Z_{\ell}(1)\to K_{t,\ell}\to \Z_{\ell}\to 0\label{eq:9-kummer-ladic}
  \end{equation}
  by tensoring with $\Q_{\ell}$. (\ref{eq:9-kummer-ladic}) in turn
  arises from a projective system of extensions
  \begin{equation*}
    0\to \mu_{\ell^{n}}(\overline{k})\to K_{t,\ell^{n}}\stackrel{f_{n}}{\to} \Z/\ell^{n}\Z\to 0,
  \end{equation*}
  where $\mu_{\ell^{n}}(\overline{k})$ is the group of $\ell^{n}$-th
  roots of 1 in the algebraic closure of $k$. Such an extension is
  given by a $\mu_{\ell^{n}}(\overline{k})$-torsor (corresponding to
  $f_{n}^{-1}(1)$); in this case it is the set of $\ell^{n}$-th roots
  of $t$ in $\overline{k}$.

  Now, let $v$ be a finite place with residue characteristic distinct
  from $\ell$. Then $\omega_{\ell}(K(t))$ is unramified at $v$ if and
  only if $t\in {\cal O}_{(v)}^{\times}$.
\end{exam}

Having described the elements of the infinite dimensional $\Q$-vector
space $\Ext^{1}_{\MTM(\Q)}(\Q(0),\Q(1))$ as $\Q$-linear combinations
of Kummer extensions we will simply remove all Kummer motives from
$\MTM(\Q)$ to obtain $\MTM(\Z)$:
\begin{defn}
  \begin{enumerate}
  \item We say a motive $N\in\MTM(\Q)$ is \emph{defined over $\Z$} if
    for all primes $p$ there exists a prime $\ell\neq p$
    such that $\omega_{\ell}(N)$ is unramified at $p$.
  \item The \emph{category of mixed Tate motives over $\Z$},
    $\MTM(\Z)$ is the full subcategory of $\MTM(\Q)$ spanned by
    motives defined over $\Z$.
  \end{enumerate}
\end{defn}

\begin{rem}\label{rem:9-mtmZ}
\begin{itemize}
\item The category $\MTM(\Z)$ is a Tannakian subcategory of $\MTM(\Q)$.
\item The $\Ext$-groups now look as follows:
  \begin{align*}
    \Ext^1_{\MTM(\Z)}(\Q(0), \Q(n)) &= \left\{ \begin{array}{ll}
        \Q &:n\geq 3 \textrm{ odd} \\
        0 &:\textrm{otherwise}
      \end{array} \right.\\
    \Ext^2_{\MTM(\Z)}(\Q(0), \Q(n)) &= 0
  \end{align*}
\end{itemize}
\end{rem}

\begin{exam}
  Let $X$ be a smooth scheme over $k$ such that $M(X) \in
  \MTM(\Q)$, and assume that $X$ has good reduction at all
  primes. Then $M(X) \in \MTM(\Z)$.
\end{exam}

\section{Fundamental group and periods}
Now we are ready to apply the results from the last chapter to mixed
Tate motives over $\Z$.

\subsection{Tannakian structure}
We saw in~\ref{cor:9-mtm-tannaka} that $\MTM(\Q)$ is a neutral Tannaka
category with fiber functor $\omega_{W}$; by~\ref{rem:9-mtmZ},
$\MTM(\Z)$ is also a neutral Tannaka category with the restriction of
$\omega_{W}$ as fiber functor. Hence we may apply (TODO: Tannaka main
theorem) from the last chapter to obtain an algebraic group $G_M$ over
$\Q$ such that
\begin{equation}\label{eq:9-mtmZ-gm}
  \mathrm{Rep}(G_M) = (\MTM(\Z), \omega_{W}).
\end{equation}
By Theorem~\ref{thm:9-mtmk} and (TODO: last chapter, semidirect
product), $G_{M}$ is of the form
\begin{equation*}
  G_{M}=\Gm\ltimes U_{M}
\end{equation*}
with $U_{M}$ a pro-unipotent group over $\Q$. Moreover,
by~\ref{rem:9-mtmZ} and (TODO: last chapter, ring of functions), the
ring of functions of $U_{M}$ is a graded Hopf algebra of the form
\begin{equation}\label{eq:9-hopfalgebra}
  T(\oplus_{n>0}\Q\cdot e_{2n+1}),
\end{equation}
$\Q\cdot e_{2n+1}$ being a one-dimensional $\Q$-vector space in degree
$2n+1$, corresponding to $\Ext^{1}_{\MTM(\Z)}(\Q(0),\Q(2n+1))$.

The Hodge realization functor $\omega_{H}$
(\ref{exam:9-hodge-realization}) induces a $\otimes$-functor which
makes the following diagram commutative:
\begin{equation}
\xymatrix{
\MTM(\Q) \ar[rr]^{\omega_H} \ar[dr]_{\omega_{W}} && \MTH(\Q) \ar[dl]^{\omega_{dR}} \\
&\Q\Mod}\label{eq:9-group-morphism}
\end{equation}
Here, $\MTH(\Q)$ denotes the category of mixed Hodge structures over
$\Q$ of Tate type defined in (TODO: chapter 6). Commutativity follows
from the fact that $\omega_{W}$ on $\MTM(\Q)$ is the same as
$\omega_{\mathrm{dR}}$.

(\ref{eq:9-group-morphism}) induces a morphism of fundamental groups
by Tannaka duality:
\begin{equation*}
  \omega^{H}:G_{H}\to G_{M},
\end{equation*}
where $G_{H}$ is the fundamental group associated to
$(\MTH(\Q),\omega_{\mathrm{dR}})$ (see (TODO: last chapter, example)).

\begin{prop}
  $\omega^{H}$ is an epimorphism.
\end{prop}
\begin{proof}
  It suffices to prove surjectivity on the pro-unipotent factors
  $U_{H}\to U_{M}$ and by Lemma (TODO: last chapter) it suffices to
  prove this on $H_{1}$ hence on the abelianization of the associated
  graded pro-Lie algebras ${\cal U}_{H}^{\mathrm{ab}}\to{\cal
    U}_{M}^{\mathrm{ab}}$. But in degree $n$ this is dual to the
  canonical map
  \begin{equation*}
    \Ext^{1}_{\MTM(\Z)}(\Q(0),\Q(n))\to \Ext^{1}_{\MTH(\Q)}(\Q(0)_{H},\Q(n)_{H}),
  \end{equation*}
  see~\cite[A.15]{deligne-goncharov05}. Injectivity of this map
  follows from injectivity of the ``regulator map'' in
  $K$-theory. (TODO: Give reference)%TODO: Check this argument.
\end{proof}

\begin{cor}\label{cor:9-mixed-hodge-tate-realization}
  $\omega_{H}$ is fully faithful and its image is closed under taking
  subquotients.
\end{cor}
\begin{proof}
  By the Proposition, we can write
  \begin{equation*}
    \mathrm{Rep}(G_M) = \left\{ \phi \in \mathrm{Rep}(G_H) \mid \textrm{$\phi$ factors through $G_M$} \right\}    
  \end{equation*}
  The claim is now clear by~(\ref{eq:9-mtmZ-gm}) and the corresponding
  equivalence for $G_{H}$.
\end{proof}

\subsection{Upper bounds on periods}

Let $X$ be a variety over $\Q$. Complex conjugation $\mathrm{c}$
defines an involution on the $\C$-points of $X$ and hence on the
singular homology of $X(\C)$. This induces an involutive automorphism
of the $\otimes$-functor 
\begin{equation*}
\omega_{\mathrm{B}}:\MTM(\Z) \to \D^b(\Q\Mod)
\end{equation*}
(this requires an understanding of the construction of
$\omega_{\textrm{B}}$ TODO: The details are rather long; should I
write them down? Also: Isn't there an easier way to see
this?)%we want to prove that omega c = c omega and actually
% we are going to prove this on whole DM. By the
% universality of the steps in the construction of DM
% it suffices to prove functoriality on SmCor. Here
% one has to see how singular homology is extended to
% SmCor by Huber in 2.1.6 (proof on page 779). So fix
% elementary correspondence alpha from X to Y, both
% smooth varieties. (we will in the following discuss
% cohomology; homology is obtained by dualization
% on DM) Then the image of alpha will be the composition tilde{R}'(Y)
% to tilde{R}'(alpha) to tilde{R}'(X) and we have to show
% functoriality with respect to both arrows. This means that two
% squares have to commute, where the horizontal arrows are these ones
% and the vertical ones are the ones induced by c on tilde{R}' (this
% is a colimit, thus the arrow is induced by a morphism of directed
% systems). This is clear for the second arrow since this "is" simply
% functoriality of betti cohomology with respect to c. For the first
% one has to see the definition of "covariant functoriality" in
% 2.1.10; but it also  boils down to the same functoriality.
, in other words we get an element $\varepsilon$ of order 2 in
$G_{\mathrm{B}}$, the fundamental group of
$(\MTM(\Z),\omega_{\mathrm{B}})$ (that $\omega_{\mathrm{B}}$ is indeed a fiber
functor follows from the factorization
$\omega_{\mathrm{B}}=\omega_{\mathrm{B}}\circ\omega_{\mathrm{H}}$
together with Corollary~\ref{cor:9-mixed-hodge-tate-realization} and
Example (TODO: reference to example in last chapter, where it is
stated/shown that Betti realization on MTH(Q) is fiber
functor)). %Indeed, both functors in the factorization are tensor,
           %faithful and exact
$\varepsilon$ induces an involutive action $\tilde{c}$ on
$\underline{\mathrm{Isom}}_{\Q}^{\otimes}(\omega_{\mathrm{dR}},
\omega_{\mathrm{B}})$ and we claim that it makes the following square
commute:
\begin{equation*}
\xymatrix{
\underline{\mathrm{Isom}}_{\Q}^{\otimes}(\omega_{\mathrm{dR}}, \omega_{\mathrm{B}}) \ar[r]^{\tilde{c}} & \underline{\mathrm{Isom}}_{\Q}^{\otimes}(\omega_{\mathrm{dR}}, \omega_{\mathrm{B}}) \\
\spec \C \ar[r]^c \ar[u]^{\mathrm{comp}} & \spec \C \ar[u]^{\mathrm{comp}}}
\end{equation*}
In other words, we claim that the following square commutes:
\begin{equation*}
\xymatrix{\omega_{\mathrm{dR}}\otimes_{\Q}\C\ar[r]^{\mathrm{comp}}&\omega_{\mathrm{B}}\otimes_{\Q}\C\ar[d]^{\varepsilon}\\
&\omega_{\mathrm{B}}\otimes_{\Q}\C\\
\ar[uu]^{\mathrm{can}}\omega_{\mathrm{dR}}\otimes_{\Q}\C\otimes_{\mathrm{Id},\C,c}\C\ar[r]_{\mathrm{comp}\otimes\C}&\omega_{\mathrm{B}}\otimes_{\Q}\C\otimes_{\mathrm{Id},\C,c}\C\ar[u]_{\mathrm{can}}}
\end{equation*}
But this can be checked on varieties %Homology
% is cohomology \circ dual, so it suffices to prove it for cohomology.
% And every tensor triangulated Q-linear realization functor on DM is
% uniquely determined on its objects by the corresponding morphism on
% varieties.
and, taking into account the definition of $\mathrm{comp}$, one
gets for $X$ a smooth variety over $\Q$:
\begin{equation*}
\xymatrix{H^{*}_{\mathrm{dR}}(X)\otimes_{\Q}\C\ar[r]^{\int}&H^{*}_{B}(X_{\C}^{\mathrm{an}},\C)\ar[d]^{\varepsilon}&H^{*}_{B}(X_{\C}^{\mathrm{an}},\Q)\otimes_{\Q}\C\ar[d]^{\varepsilon}\ar[l]_{\mathrm{can}}\\
&H^{*}_{\mathrm{B}}(X_{\C}^{\mathrm{an}},\C)&\ar[l]_{\mathrm{can}}H^{*}_{B}(X_{\C}^{\mathrm{an}},\Q)\otimes\C\\
\ar[uu]^{\mathrm{can}}H^{*}_{\mathrm{dR}}(X)\otimes_{\Q}\C\otimes_{\mathrm{Id},\C,c}\C\ar[r]_{\int\otimes\C}
&H^{*}_{\mathrm{B}}(X_{\C}^{\mathrm{an}},\C)\otimes_{\mathrm{Id},\C,c}\C\ar[u]^{\psi}&\ar[l]_{\mathrm{can}}H^{*}_{\mathrm{B}}(X_{\C}^{\mathrm{an}},\Q)\otimes\C\otimes_{\mathrm{Id},\C,c}\C\ar[u]_{\mathrm{can}}
}  
\end{equation*}
Here, $H^{*}_{dR}$ is algebraic deRham cohomology over $\Q$,
$H^{*}_{B}$ is singular cohomology and $\psi$ is the $\C$-linear map
which takes a function $f:C_{p}(X_{\C}^{\mathrm{an}})\to\C$ on the
singular group to $c\circ f$. This makes the lower square on the right
commute. Clearly also the upper square on the right is commutative. It
remains to check commutativity of the left part of the diagram, which
is the following equality
\begin{equation*}
  \int_{\gamma}c^{*}\theta=\int_{c\circ\gamma}\theta=\overline{\int_{\gamma}\theta}=\int_{\gamma}\overline{\theta}
\end{equation*}
for algebraic $p$-forms $\theta$ on $X$ over $\Q$, and $p$-cycles
$\gamma$ on $X_{\C}^{\mathrm{an}}$. This follows from the fact that
$\theta$ is defined over $\Q$.

Now, we are in the setting where we can apply the (TODO: reference to
last chapter) and we deduce that the real periods,
$P^c_{\mathrm{comp},+}$, are a graded quotient of
\begin{equation}\label{eq:9-graded-algebra-upper-bound}
  \Q[t^2] \otimes
  T\left(\bigoplus_{n > 0} \Q\cdot e_{2n+1}\right),
\end{equation}
where $t$ has degree 1 (cf~(\ref{eq:9-hopfalgebra})). To get a bound
on the graded components of $P^c_{\mathrm{comp},+}$, we
compute the Poincaré series
of~(\ref{eq:9-graded-algebra-upper-bound}) over $\Q$; it is given by:
\begin{eqnarray*}
\varphi(x) & = & \frac{1}{1-x^2} \cdot\sum_{i \geq 0}
\left(
\frac{x^3}{1-x^2}\right)^{i} \\
& = & \frac{1}{1-x^2} \cdot \frac{1}{1-\frac{x^3}{1-x^2}} \\
& = & \frac{1}{1-x^2} \cdot \frac{1-x^2}{1-x^2-x^3} \\
& = & \frac{1}{1-x^2-x^3}.
\end{eqnarray*}
Setting 
\begin{equation*}
D_n = D_{n-2} + D_{n-3}, \qquad D_0 = D_2 = 1, D_1 = 0,
\end{equation*}
we thus conclude
\begin{equation*}
d_n := \dim_{\Q} \gr_n P^c_{\mathrm{comp},+}\leq D_n,\quad\text{ for all }n.
\end{equation*}

